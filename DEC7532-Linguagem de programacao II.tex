\documentclass[12pt]{article}
\usepackage[brazil]{babel}
\usepackage{graphicx,t1enc,wrapfig,amsmath,float}
\usepackage{framed,fancyhdr}
\usepackage{multirow}
\usepackage{longtable}
\usepackage{array}
\newcolumntype{L}[1]{>{\raggedright\let\newline\\\arraybackslash\hspace{0pt}}m{#1}}
\newcolumntype{C}[1]{>{\centering\let\newline\\\arraybackslash\hspace{0pt}}m{#1}}
\newcolumntype{R}[1]{>{\raggedleft\let\newline\\\arraybackslash\hspace{0pt}}m{#1}}
%%%%%%%%%%%%%
\oddsidemargin -0.5cm
\evensidemargin -0.5cm
\textwidth 17.5cm
\topmargin -1.5cm
\textheight 22cm
%%%%%%%%%%%% 

%\pagestyle{empty}

\newcommand{\semestre}{2018.2}

\newcommand{\disciplina}{LINGUAGEM DE PROGRAMAÇÃO II}
\newcommand{\codigo}{DEC7532}


%%%%%%%%%%%%%%%%%%%%%%%%%%%%%%%%%%%%%%%%%%%%%%%%%%%%%%%
%%%%%%%%%%%%% CRETIDOS
\newcommand{\creditosT}{1}
\newcommand{\creditosP}{3}

%%%%%%%%%%%%%%%%%%%%%%%%%%%%%%%%%%%%%%%%%%%%%%%%%%%%%%%
%%%%%%%%%%%%%% REQUISITOS
\newcommand{\requisitoA}{}
\newcommand{\requisitoB}{}
\newcommand{\requisitoC}{}

%%%%%%%%%%%%%%%%%%%%%%%%%%%%%%%%%%%%%%%%%%%%%%%%%%%%%%%
%%%%%%%%%%%%%%% Atende aos Cursos
\newcommand{\cursoA}{Graduação em Engenharia de Computação \\ \hline}
\newcommand{\cursoB}{}%Graduação em Tecnologias da Informação e Comunicação \\ \hline}
\newcommand{\cursoC}{}%Graduação em Engenharia de Energia \\ \hline}

%%%%%%%%%%%%%%%%%%%%%%%%%%%%%%%%%%%%%%%%%%%%%%%%%%%%%%%%
%%%%%%%%%% EMENTA
\newcommand{\ementa}{
%Fundamentos de uma linguagem orientada por objetos: nomes, variáveis, tipos de dados, operadores, expressões, estruturas de controle de fluxo, regras de escopo. Decomposição de problemas por objetos. Encapsulamento. Classes: Abstrata, Derivada, Genérica, Aninhada e Agregada. Mecanismo de herança. Polimorfismo. Interfaces. Objetos Componentes. Interface Gráfica com o Usuário (GUI). Sistemas Orientados por Eventos. Mecanismo de resposta à eventos. Persistência: streams, entrada e saída de dados.

Fundamentos do paradigma Orientado à Objetos. Classes e métodos. Encapsulamento, herança e polimorfismo. Modelagem e solução de problemas utilizando os conceitos de orientação a objetos, decomposição por objetos e tipos abstratos de dados. Interface gráfica com usuário (GUI). Análise dos aspectos tecnológicos complementares à solução de problemas (programação orientada a eventos, persistência de dados e objetos, tratamento de exceções).
 \\ \hline
}


\begin{document}


%%%%%%%%%%%%%%%%%%%%%%%%%%%%%%%%%%%%%%%%%%%%%%%%%%%%%%%%%%%%%
\input cabecalho.tex


%%%%%%%%%%%%%%%%%%%%%%%%%%%%%%%%%%%%%%%%%%%%%%%%%%%%%%%%%%%%%
\begin{longtable}{|C{0.11\textwidth}|C{0.29\textwidth}|C{0.09\textwidth}|C{0.09\textwidth}|C{0.15\textwidth}|C{0.158\textwidth}|} \hline
%
\multicolumn{6}{|l|}{{\bf I. IDENTIFICAÇÃO DA DISCIPLINA}} \\ \hline
%
\multirow{3}*{{\small CÓDIGO}} & \multirow{3}*{NOME DA DISCIPLINA} &\multicolumn{2}{c|}{{\small N$^\circ$ DE HORAS-AULA }} & {{\small TOTAL DE}} & \multirow{3}*{{\small MODALIDADE}} \\ 
%
& & \multicolumn{2}{c|}{\small SEMANAIS}  & {\small HORAS-AULA} & \\ \cline{3-4}
%
& & {\tiny TEÓRICAS} & {\tiny PRÁTICAS} & {\small SEMESTRAIS} & \\ \hline
% codigo da disciplina carga horaria: teorica - pratica e total
{\bf \small \codigo} & {\bf \small \disciplina } & {\bf \creditosT} & {\bf \creditosP} & {\bf 72} & Presencial\\ \hline
\end{longtable}


%%%%%%%%%%%%%%%%%%%%%%%%%%%%%%%%%%%%%%%%%%%%%%%%%%%%%%%%%%%%%%
\begin{longtable}{|C{0.12\textwidth}|L{0.736\textwidth}|C{0.12\textwidth}|} \hline
%
\multicolumn{3}{|l|}{{\bf II. PRÉ-REQUISITO(S)}} \\ \hline
%
CÓDIGO & NOME DA DISCIPLINA & CURSO \\ \hline	
%
\requisitoA
\requisitoB
\requisitoC
\end{longtable}


%%%%%%%%%%%%%%%%%%%%%%%%%%%%%%%%%%%%%%%%%%%%%%%%%%%%%%%%%%%%%%
\begin{longtable}{|L{1.025\textwidth}|} \hline
%
{\bf III. CURSO(S) PARA O(S) QUAL(IS) A DISCIPLINA É OFERECIDA } \\ \hline
%
\cursoA 
\cursoB
\cursoC

\end{longtable}

%%%%%%%%%%%%%%%%%%%%%%%%%%%%%%%%%%%%%%%%%%%%%%%%%%%%%%%%%%%%%%
\begin{longtable}{|L{1.025\textwidth}|} \hline
%
{\bf IV. EMENTA } \\ \hline
%
\ementa
\end{longtable}

%\newpage



%%%%%%%%%%%%%%%%%%%%%%%%%%%%%%%%%%%%%%%%%%%%%%%%%%%%%%%%%%%%%%%
\begin{longtable}{|L{1.025\textwidth}|} \hline
%
{\bf V. OBJETIVOS } \\ \hline
%
Objetivos Gerais:\\
 O aluno ao final desta disciplina deverá ser capaz de transpor algoritmos, tal como apreendido em lógica de programação, para uma linguagem de programação sob o paradigma da programação orientada por objetos.
\\
\\
\\
\\
\\
\\
\\
Objetivos Específicos:\\
\begin{itemize}
\item Compreender a motivação para a adoção do paradigma de orientação por objetos. 
\item Conhecer os principais pilares da orientação à objetos, bem como classes, abstração, herança e polimorfismo. 
\item Decompor problemas segundo o conceito de orientação à objetos. 
\item Implementar o conceito segundo os aspectos da orientação à objetos.
\end{itemize}

\\ \hline
\end{longtable}


%%%%%%%%%%%%%%%%%%%%%%%%%%%%%%%%%%%%%%%%%%%%%%%%%%%%%%%%%%%%%%%
\begin{longtable}{|L{1.025\textwidth}|} \hline
%
{\bf VI. CONTEÚDO PROGRAMÁTICO } \\ \hline
UNIDADE 1:\\
Apresentação da disciplina e Apresentação do plano de ensino.\\
Linguagem C para C++ e Histórico\\
Utilização da biblioteca padrão do C++ e diferenças da programação em Linguagem C\\
Compilação, flags de compilação e depuração de código. \\
Ferramentas para detecção de memoryleak e stackoverflow\\
Alocação dinâmica de memória em C++ e inputs/pipe do SO.\\
conceito   de   namespace,  std::string,   ::stringstream,   ::vector,   ::pair,   ::ifstream,   ::ofstream, e outros necessários da std.\\
\\

UNIDADE 2:\\
Motivação para a adoção do paradigma de orientação por objetos.\\
Programação Estruturada X Orientada à Objetos.\\
Decomposição de problemas por objetos. \\
Operadores Relacionais, Operadores Aritméticos, Operadores Lógicos.\\
Composição de Operadores de Atribuição.\\
Classes,   definição,   Variáveis  Membros  (Atributos),   Métodos  ou   Funções  (Comportamento),   Método Construtor, Modificadores de Acesso, Declaração e Instanciação de Objetos.\\
Classes, Tipo Abstrato de Dados, Encapsulamento e Identidade de Objetos. \\
Comparação do Operador new com a Alocação Dinâmica de Memória (Alocação Dinâmica de Tipo Abstratos de Dados).\\
Herança, Classes Abstratas, Métodos Abstratos, Hierarquia de Classes, Classe Ancestral e Classe Derivada, Redefinição de Comportamentos Ancestrais. \\
Polimorfismo.\\
\\

UNIDADE 3:\\
Interface com o Usuário em Sistemas Orientados por Eventos.\\
Classes Genéricas. Interfaces Genéricas. Métodos Genéricos.\\ Entrada e Saída de dados em Interfaces com o Usuário (Entrada e Saída de Dados para Console).\\
Persistência de Objetos, Streams de Entrada e Saída de Dados para meios persistentes. \\
Serialização de Objetos (Interface Serializable).\\
Atividades práticas: desenvolvimento de trabalho.
\\ \hline
\end{longtable} 

\newpage

%%%%%%%%%%%%%%%%%%%%%%%%%%%%%%%%%%%%%%%%%%%%%%%%%%%%%%%%%%%%%%%
\begin{longtable}{|L{1.025\textwidth}|} \hline
%
{\bf VII. BIBLIOGRAFIA BÁSICA} \\ \hline
\begin{enumerate}
%
\item LARMAN, Graig. Utilizando UML e padrões: uma introdução à análise e ao projeto orientado a objetos e ao desenvolvimento interativo. 3. ed. Porto Alegre: Bookman, 2007. 
\item DEITEL, H. M.; DEITEL, P.J. C++ Como Programar. 5a. edição. Pearson, 2006. 
\item DEITEL, H. M.; DEITEL, P.J. Java. Como Programar. 6a. edição. Pearson, 2005.

\end{enumerate}
 \\ \hline
\end{longtable}


%\newpage

%%%%%%%%%%%%%%%%%%%%%%%%%%%%%%%%%%%%%%%%%%%%%%%%%%%%%%%%%%%%%%%
\begin{longtable}{|L{1.025\textwidth}|} \hline
%
{\bf VIII. BIBLIOGRAFIA COMPLEMENTAR} \\ \hline
\begin{enumerate}
\item  BOOCH, Grady. Object-Oriented Analysis and Design with Applications (3rd Edition), Addison Wesly, 2007. 
\item GAMMA, Erich; HELM, Richard; JOHNSON, Ralph; VLISSIDES, John. Padrões de Projeto: soluções reutilizáveis de software orientado a objetos. Porto Alegre: Bookman, 2000. (18) 
\item KRUCHTEN, Philippe. Introdução ao RUP Rational Unified Process. Rio de Janeiro: Ciência Moderna. 2003. 
\item MCLAUGHLIN, Brett; POLLICE, Gary; WEST, David. Use a Cabeça! Análise e projeto orientado ao objeto. São Paulo: Alta Books, 2007 
\item BORATTI, Isaias Camilo. Programação Orientada a Objetos em Java. Visual Books, 2007.

\end{enumerate}
 \\ \hline
\end{longtable}


\input aprovacao.tex


\end{document}
