\documentclass[12pt]{article}
\usepackage[brazil]{babel}
\usepackage{graphicx,t1enc,wrapfig,amsmath,float}
\usepackage{framed,fancyhdr}
\usepackage{multirow}
\usepackage{longtable}
\usepackage{array}
\newcolumntype{L}[1]{>{\raggedright\let\newline\\\arraybackslash\hspace{0pt}}m{#1}}
\newcolumntype{C}[1]{>{\centering\let\newline\\\arraybackslash\hspace{0pt}}m{#1}}
\newcolumntype{R}[1]{>{\raggedleft\let\newline\\\arraybackslash\hspace{0pt}}m{#1}}
%%%%%%%%%%%%%
\oddsidemargin -0.5cm
\evensidemargin -0.5cm
\textwidth 17.5cm
\topmargin -1.5cm
\textheight 22cm
%%%%%%%%%%%% 

%\pagestyle{empty}

\newcommand{\semestre}{2018.2}

\newcommand{\disciplina}{PESQUISA OPERACIONAL}
\newcommand{\codigo}{DEC7524}


%%%%%%%%%%%%%%%%%%%%%%%%%%%%%%%%%%%%%%%%%%%%%%%%%%%%%%%
%%%%%%%%%%%%% CRETIDOS
\newcommand{\creditosT}{4}
\newcommand{\creditosP}{0}

%%%%%%%%%%%%%%%%%%%%%%%%%%%%%%%%%%%%%%%%%%%%%%%%%%%%%%%
%%%%%%%%%%%%%% REQUISITOS
\newcommand{\requisitoA}{DEC7142 & Cálculo Numérico em Computadores & ENE\\ \hline}
\newcommand{\requisitoB}{}
\newcommand{\requisitoC}{}

%%%%%%%%%%%%%%%%%%%%%%%%%%%%%%%%%%%%%%%%%%%%%%%%%%%%%%%
%%%%%%%%%%%%%%% Atende aos Cursos
\newcommand{\cursoA}{Graduação em Engenharia de Computação \\ \hline}
\newcommand{\cursoB}{}%Graduação em Tecnologias da Informação e Comunicação \\ \hline}
\newcommand{\cursoC}{Graduação em Engenharia de Energia \\ \hline}

%%%%%%%%%%%%%%%%%%%%%%%%%%%%%%%%%%%%%%%%%%%%%%%%%%%%%%%%
%%%%%%%%%% EMENTA
\newcommand{\ementa}{
Introdução à pesquisa operacional. Modelagem com programação linear. Método Simplex e Análise de Sensibilidade. Dualidade e análise pós-otimização. Problema de transporte e suas variantes. Otimização em redes. Programação linear avançada. Programação de metas. Programação linear inteira.
 \\ \hline
}


\begin{document}


%%%%%%%%%%%%%%%%%%%%%%%%%%%%%%%%%%%%%%%%%%%%%%%%%%%%%%%%%%%%%
\input cabecalho.tex



%%%%%%%%%%%%%%%%%%%%%%%%%%%%%%%%%%%%%%%%%%%%%%%%%%%%%%%%%%%%%
\begin{longtable}{|C{0.11\textwidth}|C{0.29\textwidth}|C{0.09\textwidth}|C{0.09\textwidth}|C{0.15\textwidth}|C{0.158\textwidth}|} \hline
%
\multicolumn{6}{|l|}{{\bf I. IDENTIFICAÇÃO DA DISCIPLINA}} \\ \hline
%
\multirow{3}*{{\small CÓDIGO}} & \multirow{3}*{NOME DA DISCIPLINA} &\multicolumn{2}{c|}{{\small N$^\circ$ DE HORAS-AULA }} & {{\small TOTAL DE}} & \multirow{3}*{{\small MODALIDADE}} \\ 
%
& & \multicolumn{2}{c|}{\small SEMANAIS}  & {\small HORAS-AULA} & \\ \cline{3-4}
%
& & {\tiny TEÓRICAS} & {\tiny PRÁTICAS} & {\small SEMESTRAIS} & \\ \hline
% codigo da disciplina carga horaria: teorica - pratica e total
{\bf \small \codigo} & {\bf \small \disciplina } & {\bf \creditosT} & {\bf \creditosP} & {\bf 72} & Presencial\\ \hline
\end{longtable}


%%%%%%%%%%%%%%%%%%%%%%%%%%%%%%%%%%%%%%%%%%%%%%%%%%%%%%%%%%%%%%
\begin{longtable}{|C{0.12\textwidth}|L{0.736\textwidth}|C{0.12\textwidth}|} \hline
%
\multicolumn{3}{|l|}{{\bf II. PRÉ-REQUISITO(S)}} \\ \hline
%
CÓDIGO & NOME DA DISCIPLINA & CURSO \\ \hline	
%
\requisitoA
\requisitoB
\requisitoC
\end{longtable}


%%%%%%%%%%%%%%%%%%%%%%%%%%%%%%%%%%%%%%%%%%%%%%%%%%%%%%%%%%%%%%
\begin{longtable}{|L{1.025\textwidth}|} \hline
%
{\bf III. CURSO(S) PARA O(S) QUAL(IS) A DISCIPLINA É OFERECIDA } \\ \hline
%
\cursoA 
\cursoB
\cursoC

\end{longtable}

%%%%%%%%%%%%%%%%%%%%%%%%%%%%%%%%%%%%%%%%%%%%%%%%%%%%%%%%%%%%%%
\begin{longtable}{|L{1.025\textwidth}|} \hline
%
{\bf IV. EMENTA } \\ \hline
%
\ementa
\end{longtable}

%\newpage



%%%%%%%%%%%%%%%%%%%%%%%%%%%%%%%%%%%%%%%%%%%%%%%%%%%%%%%%%%%%%%%
\begin{longtable}{|L{1.025\textwidth}|} \hline
%
{\bf V. OBJETIVOS } \\ \hline
%
Preparar os alunos para trabalhar com modelos reais que aparecem na indústria e nos diversos campos da ciência. O aluno deverá ser capaz de produzir modelos para variados problemas de otimização e ser capaz de resolvê-los de forma ótima com os métodos estudados na disciplina.
\\ \hline
\end{longtable}


%%%%%%%%%%%%%%%%%%%%%%%%%%%%%%%%%%%%%%%%%%%%%%%%%%%%%%%%%%%%%%%
\begin{longtable}{|L{1.025\textwidth}|} \hline
%
{\bf VI. CONTEÚDO PROGRAMÁTICO } \\ \hline
UNIDADE 1: Introdução à Pesquisa Operacional\\
\\
UNIDADE 2: Programação Linear\\
Modelagem com programação linear\\
Método Simplex \\
Análise de sensibilidade\\
Dualidade e análise pós-otimização\\
\\
UNIDADE 3: Problema de transporte e otimização em redes\\
\\
UNIDADE 4: Programação linear inteira\\
\\
UNIDADE 5: Programação de metas\\

\\ \hline
\end{longtable} 

%\newpage

%%%%%%%%%%%%%%%%%%%%%%%%%%%%%%%%%%%%%%%%%%%%%%%%%%%%%%%%%%%%%%%
\begin{longtable}{|L{1.025\textwidth}|} \hline
%
{\bf VII. BIBLIOGRAFIA BÁSICA} \\ \hline
\begin{enumerate}
%
\item GOLDBARG, Marco Cesar; LUNA, Henrique Pacca L. Otimização combinatória e programação linear: modelos e algoritmos. 2. ed. rev. a atual. Rio de Janeiro: Elsevier: Campus, 2005. 518 p. 
\item TAHA, Hamdy A. Pesquisa Operacional, 8. ed. Prentice Hall, 2008. 
\item HILLIER , Frederick S; LIEBERMAN, Gerald J. Introdução À Pesquisa Operacional. 8. ed. Bookman, 2010.
\end{enumerate}
 \\ \hline
\end{longtable}


%\newpage

%%%%%%%%%%%%%%%%%%%%%%%%%%%%%%%%%%%%%%%%%%%%%%%%%%%%%%%%%%%%%%%
\begin{longtable}{|L{1.025\textwidth}|} \hline
%
{\bf VIII. BIBLIOGRAFIA COMPLEMENTAR} \\ \hline
\begin{enumerate}
\item DANTZIG, G.B. and THAPA, M.N., 2006. Linear Programming 1: Introduction. Springer-Verlag New York.
\item Wolsey L. A. Integer Programming, Hardcover, 1998. 
\item YANASSE, Horacio H; ARENALES, Marcos; MORABITO, Reinaldo; ARMENTANO, Vinícius A. Pesquisa Operacional - Modelagem e Algoritmos, 1. ed. Elsevier - Campus, 2006. 
\item MOREIRA, Daniel A. Pesquisa Operacional - Curso Introdutório, 2. ed. Cengage Learning, 2011. 
\item SILVA, Ermes M. Pesquisa Operacional - Para os Cursos de Administração e Engenharia. 4. ed. Atlas, 2010.



\end{enumerate}
 \\ \hline
\end{longtable}


\input aprovacao.tex


\end{document}
