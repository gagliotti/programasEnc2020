\documentclass[12pt]{article}
\usepackage[brazil]{babel}
\usepackage{graphicx,t1enc,wrapfig,amsmath,float}
\usepackage{framed,fancyhdr}
\usepackage{multirow}
\usepackage{longtable}
\usepackage{array}
\newcolumntype{L}[1]{>{\raggedright\let\newline\\\arraybackslash\hspace{0pt}}m{#1}}
\newcolumntype{C}[1]{>{\centering\let\newline\\\arraybackslash\hspace{0pt}}m{#1}}
\newcolumntype{R}[1]{>{\raggedleft\let\newline\\\arraybackslash\hspace{0pt}}m{#1}}
%%%%%%%%%%%%%
\oddsidemargin -0.5cm
\evensidemargin -0.5cm
\textwidth 17.5cm
\topmargin -1.5cm
\textheight 22cm
%%%%%%%%%%%% 

%\pagestyle{empty}

\newcommand{\semestre}{2018.2}

\newcommand{\disciplina}{SEMINÁRIOS TÉCNICOS CIENTÍFICOS}
\newcommand{\codigo}{DEC7554}


%%%%%%%%%%%%%%%%%%%%%%%%%%%%%%%%%%%%%%%%%%%%%%%%%%%%%%%
%%%%%%%%%%%%% CRETIDOS
\newcommand{\creditosT}{2}
\newcommand{\creditosP}{0}

%%%%%%%%%%%%%%%%%%%%%%%%%%%%%%%%%%%%%%%%%%%%%%%%%%%%%%%
%%%%%%%%%%%%%% REQUISITOS
\newcommand{\requisitoA}{& 3600horas & ENC\\ \hline}
\newcommand{\requisitoB}{}
\newcommand{\requisitoC}{}

%%%%%%%%%%%%%%%%%%%%%%%%%%%%%%%%%%%%%%%%%%%%%%%%%%%%%%%
%%%%%%%%%%%%%%% Atende aos Cursos
\newcommand{\cursoA}{Graduação em Engenharia de Computação \\ \hline}
\newcommand{\cursoB}{}%Graduação em Tecnologias da Informação e Comunicação \\ \hline}
\newcommand{\cursoC}{}

%%%%%%%%%%%%%%%%%%%%%%%%%%%%%%%%%%%%%%%%%%%%%%%%%%%%%%%%
%%%%%%%%%% EMENTA
\newcommand{\ementa}{
Apresentação de seminários Orientação e planejamento do seminário Noções de oratória: apresentação oral de trabalhos e seminários Uso de ferramentas de apresentação Título, resumo, introdução, pesquisa bibliográfica, metodologia, resultados, discussão, conclusões, literatura, tabelas, figuras, elaboração de projetos de pesquisa, produção e apresentação oral, pôsteres.
\\ \hline
}


\begin{document}

%%%%%%%%%%%%%%%%%%%%%%%%%%%%%%%%%%%%%%%%%%%%%%%%%%%%%%%%%%%%%

\input cabecalho.tex


%%%%%%%%%%%%%%%%%%%%%%%%%%%%%%%%%%%%%%%%%%%%%%%%%%%%%%%%%%%%%
\begin{longtable}{|C{0.11\textwidth}|C{0.29\textwidth}|C{0.09\textwidth}|C{0.09\textwidth}|C{0.15\textwidth}|C{0.158\textwidth}|} \hline
%
\multicolumn{6}{|l|}{{\bf I. IDENTIFICAÇÃO DA DISCIPLINA}} \\ \hline
%
\multirow{3}*{{\small CÓDIGO}} & \multirow{3}*{NOME DA DISCIPLINA} &\multicolumn{2}{c|}{{\small N$^\circ$ DE HORAS-AULA }} & {{\small TOTAL DE}} & \multirow{3}*{{\small MODALIDADE}} \\ 
%
& & \multicolumn{2}{c|}{\small SEMANAIS}  & {\small HORAS-AULA} & \\ \cline{3-4}
%
& & {\tiny TEÓRICAS} & {\tiny PRÁTICAS} & {\small SEMESTRAIS} & \\ \hline
% codigo da disciplina carga horaria: teorica - pratica e total
{\bf \small \codigo} & {\bf \small \disciplina } & {\bf \creditosT} & {\bf \creditosP} & {\bf 36} & Presencial\\ \hline
\end{longtable}


%%%%%%%%%%%%%%%%%%%%%%%%%%%%%%%%%%%%%%%%%%%%%%%%%%%%%%%%%%%%%%
\begin{longtable}{|C{0.12\textwidth}|L{0.736\textwidth}|C{0.12\textwidth}|} \hline
%
\multicolumn{3}{|l|}{{\bf II. PRÉ-REQUISITO(S)}} \\ \hline
%
CÓDIGO & NOME DA DISCIPLINA & CURSO \\ \hline	
%
\requisitoA
\requisitoB
\requisitoC
\end{longtable}


%%%%%%%%%%%%%%%%%%%%%%%%%%%%%%%%%%%%%%%%%%%%%%%%%%%%%%%%%%%%%%
\begin{longtable}{|L{1.025\textwidth}|} \hline
%
{\bf III. CURSO(S) PARA O(S) QUAL(IS) A DISCIPLINA É OFERECIDA } \\ \hline
%
\cursoA 
\cursoB
\cursoC

\end{longtable}

%%%%%%%%%%%%%%%%%%%%%%%%%%%%%%%%%%%%%%%%%%%%%%%%%%%%%%%%%%%%%%
\begin{longtable}{|L{1.025\textwidth}|} \hline
%
{\bf IV. EMENTA } \\ \hline
%
\ementa
\end{longtable}

%\newpage



%%%%%%%%%%%%%%%%%%%%%%%%%%%%%%%%%%%%%%%%%%%%%%%%%%%%%%%%%%%%%%%
\begin{longtable}{|L{1.025\textwidth}|} \hline
%
{\bf V. OBJETIVOS } \\ \hline
Esta disciplina tem como objetivo proporcionar ambiente para o desenvolvimento de técnicas e procedimentos para apresentação e defesa de trabalhos técnicos e científicos preferencialmente aqueles sendo elaborados pelas disciplinas que convirjam para a qualificação do TCC I e apresentação do TCC II, como forma de verificar o andamento da produção científica e tecnológica formais para obtenção de título.
\\ \hline
\end{longtable}


%%%%%%%%%%%%%%%%%%%%%%%%%%%%%%%%%%%%%%%%%%%%%%%%%%%%%%%%%%%%%%%
\begin{longtable}{|L{1.025\textwidth}|} \hline
%
{\bf VI. CONTEÚDO PROGRAMÁTICO } \\ \hline
A disciplina segue o regulamento do Curso de Engenharia de Computação aprovado no colegiado
\\ \hline
\end{longtable} 

\newpage


%%%%%%%%%%%%%%%%%%%%%%%%%%%%%%%%%%%%%%%%%%%%%%%%%%%%%%%%%%%%%%%
\begin{longtable}{|L{1.025\textwidth}|} \hline
%
{\bf VII. BIBLIOGRAFIA BÁSICA} \\ \hline

\begin{enumerate}
\item FREIRE, Patrícia de Sá. Aumente a qualidade e quantidade de suas publicações científicas:manual para elaboração de projetos e artigos científicos. Curitiba: CRV, 2013. 87 p. ISBN 9788580428155. 
\item GIL, Antonio Carlos. Como elaborar projetos de pesquisa. 5. ed. São Paulo: Atlas, 2010. 184p. ISBN 9788522458233. 
\item   WAZLAWICK, Raul Sidnei. Metodologia de pesquisa para ciência da computação. Rio de Janeiro: Elsevier, 2009. 159p. ISBN 9788535234107.

\end{enumerate}
 \\ \hline
\end{longtable}


%\newpage

%%%%%%%%%%%%%%%%%%%%%%%%%%%%%%%%%%%%%%%%%%%%%%%%%%%%%%%%%%%%%%%
\begin{longtable}{|L{1.025\textwidth}|} \hline
%
{\bf VIII. BIBLIOGRAFIA COMPLEMENTAR} \\ \hline
\begin{enumerate}
\item PÁDUA, Elisabete Matallo Marchesini de. Metodologia da pesquisa: abordagem teórico-prática. 17. ed. Campinas: Papirus, 2011. 127p. (COLEÇÃO MAGISTÉRIO FORMAÇÃO E TRABALHO PEDAGÓGICO.). ISBN 8530806077. 
\item MICHALISZYN, Mario Sergio; TOMASINI, Ricardo. Pesquisa: orientações e normas para elaboração de projetos, monografias e artigos científicos. 6.ed. Rio de Janeiro: Vozes, 2011. 215 p. ISBN 9788532631619. 97 
\item RAMPAZZO, Lino. Metodologia científica: [para alunos dos cursos de graduação e pósgraduação].7. ed. São Paulo: Loyola, 2013. 154 p. ISBN 9788515024988. 
\item FIGUEIREDO, Nebia Maria Almeida de. Método e metodologia na pesquisa cientítica. 3.ed.-. São Caetano do Sul: Yendis, 2008. xvi, 239 p. ISBN 9788577280858 (broch.).. 
\item ABNT NBR 6023:2002 Informação e documentação - Referências - Elaboração 30/08/2002 . Disponível no acervo virtual http://www.abntcolecao.com.br/.
\end{enumerate}
 \\ \hline
\end{longtable}


\input aprovacao.tex


\end{document}
