\documentclass[12pt]{article}
\usepackage[brazil]{babel}
\usepackage{graphicx,t1enc,wrapfig,amsmath,float}
\usepackage{framed,fancyhdr}
\usepackage{multirow}
\usepackage{longtable}
\usepackage{array}
\newcolumntype{L}[1]{>{\raggedright\let\newline\\\arraybackslash\hspace{0pt}}m{#1}}
\newcolumntype{C}[1]{>{\centering\let\newline\\\arraybackslash\hspace{0pt}}m{#1}}
\newcolumntype{R}[1]{>{\raggedleft\let\newline\\\arraybackslash\hspace{0pt}}m{#1}}
%%%%%%%%%%%%%
\oddsidemargin -0.5cm
\evensidemargin -0.5cm
\textwidth 17.5cm
\topmargin -1.5cm
\textheight 22cm
%%%%%%%%%%%% 

%\pagestyle{empty}

\newcommand{\semestre}{2018.2}

\newcommand{\disciplina}{COMUNICAÇÃO DE DADOS}
\newcommand{\codigo}{DEC7548}


%%%%%%%%%%%%%%%%%%%%%%%%%%%%%%%%%%%%%%%%%%%%%%%%%%%%%%%
%%%%%%%%%%%%% CRETIDOS
\newcommand{\creditosT}{2}
\newcommand{\creditosP}{2}

%%%%%%%%%%%%%%%%%%%%%%%%%%%%%%%%%%%%%%%%%%%%%%%%%%%%%%%
%%%%%%%%%%%%%% REQUISITOS
\newcommand{\requisitoA}{}
\newcommand{\requisitoB}{}
\newcommand{\requisitoC}{}

%%%%%%%%%%%%%%%%%%%%%%%%%%%%%%%%%%%%%%%%%%%%%%%%%%%%%%%
%%%%%%%%%%%%%%% Atende aos Cursos
\newcommand{\cursoA}{Graduação em Engenharia de Computação \\ \hline}
\newcommand{\cursoB}{}%Graduação em Tecnologias da Informação e Comunicação \\ \hline}
\newcommand{\cursoC}{}

%%%%%%%%%%%%%%%%%%%%%%%%%%%%%%%%%%%%%%%%%%%%%%%%%%%%%%%%
%%%%%%%%%% EMENTA
\newcommand{\ementa}{
Fundamentos de comunicação de dados. Formas de transmissão de dados. Modulação por amplitude, ângulo e pulso. Demodulação. Modulação e Transmissão Digital. Meios de transmissão de dados. Detecção e correção de erros. Interfaces de comunicação de dados. Padronização de comunicação de dados.
\\ \hline
}


\begin{document}

%%%%%%%%%%%%%%%%%%%%%%%%%%%%%%%%%%%%%%%%%%%%%%%%%%%%%%%%%%%%%

\input cabecalho.tex


%%%%%%%%%%%%%%%%%%%%%%%%%%%%%%%%%%%%%%%%%%%%%%%%%%%%%%%%%%%%%
\begin{longtable}{|C{0.11\textwidth}|C{0.29\textwidth}|C{0.09\textwidth}|C{0.09\textwidth}|C{0.15\textwidth}|C{0.158\textwidth}|} \hline
%
\multicolumn{6}{|l|}{{\bf I. IDENTIFICAÇÃO DA DISCIPLINA}} \\ \hline
%
\multirow{3}*{{\small CÓDIGO}} & \multirow{3}*{NOME DA DISCIPLINA} &\multicolumn{2}{c|}{{\small N$^\circ$ DE HORAS-AULA }} & {{\small TOTAL DE}} & \multirow{3}*{{\small MODALIDADE}} \\ 
%
& & \multicolumn{2}{c|}{\small SEMANAIS}  & {\small HORAS-AULA} & \\ \cline{3-4}
%
& & {\tiny TEÓRICAS} & {\tiny PRÁTICAS} & {\small SEMESTRAIS} & \\ \hline
% codigo da disciplina carga horaria: teorica - pratica e total
{\bf \small \codigo} & {\bf \small \disciplina } & {\bf \creditosT} & {\bf \creditosP} & {\bf 72} & Presencial\\ \hline
\end{longtable}


%%%%%%%%%%%%%%%%%%%%%%%%%%%%%%%%%%%%%%%%%%%%%%%%%%%%%%%%%%%%%%
\begin{longtable}{|C{0.12\textwidth}|L{0.736\textwidth}|C{0.12\textwidth}|} \hline
%
\multicolumn{3}{|l|}{{\bf II. PRÉ-REQUISITO(S)}} \\ \hline
%
CÓDIGO & NOME DA DISCIPLINA & CURSO \\ \hline	
%
\requisitoA
\requisitoB
\requisitoC
\end{longtable}


%%%%%%%%%%%%%%%%%%%%%%%%%%%%%%%%%%%%%%%%%%%%%%%%%%%%%%%%%%%%%%
\begin{longtable}{|L{1.025\textwidth}|} \hline
%
{\bf III. CURSO(S) PARA O(S) QUAL(IS) A DISCIPLINA É OFERECIDA } \\ \hline
%
\cursoA 
\cursoB
\cursoC

\end{longtable}

%%%%%%%%%%%%%%%%%%%%%%%%%%%%%%%%%%%%%%%%%%%%%%%%%%%%%%%%%%%%%%
\begin{longtable}{|L{1.025\textwidth}|} \hline
%
{\bf IV. EMENTA } \\ \hline
%
\ementa
\end{longtable}

\newpage



%%%%%%%%%%%%%%%%%%%%%%%%%%%%%%%%%%%%%%%%%%%%%%%%%%%%%%%%%%%%%%%
\begin{longtable}{|L{1.025\textwidth}|} \hline
%
{\bf V. OBJETIVOS } \\ \hline
Objetivos Gerais:\\ 
Esta disciplina tem como objetivo abordar os principais conceitos envolvidos na comunicação de dados com ênfase à camada física do modelo OSI.\\
\\
Objetivos Específicos: \\
O aluno ao final do curso deve possuir habilidades para:
\begin{itemize}
\item Introduzir fundamentos de comunicação de dados em nível de camada física;
\item Discutir fundamentos de transmissão analógica e digital;
\item Abordar métodos de codificação e correção erros em comunicação de dados;
\item Abordar aplicações de comunicação de dados para sistemas embarcados
\end{itemize}
\\ \hline
\end{longtable}


%%%%%%%%%%%%%%%%%%%%%%%%%%%%%%%%%%%%%%%%%%%%%%%%%%%%%%%%%%%%%%%
\begin{longtable}{|L{1.025\textwidth}|} \hline
%
{\bf VI. CONTEÚDO PROGRAMÁTICO } \\ \hline
Unidade 1: Fundamentos de Comunicação de Dados\\
Introdução à Comunicação de Dados;\\
Dados e Sinais;\\
Transmissão Digital;\\
Transmissão Analógica;\\
\\
Unidade 2: Métodos de Comunicação de Dados\\
Multiplexação e Espalhamento;\\
Meios de Transmissão;\\
\\
Unidade 3: Camada de Enlace\\
Detecção e Correção de Erros;\\
Padrões de Comunicação;\\
Controle de Enlace de Dados;\\
\\
Unidade 4: Comunicação de Dados para Sistemas Embarcados\\
Introdução a Comunicação de Dados para Sistemas Embarcados\\
Protocolos de Comunicação para Sistemas Embarcados\\
\\ \hline
\end{longtable} 

\newpage


%%%%%%%%%%%%%%%%%%%%%%%%%%%%%%%%%%%%%%%%%%%%%%%%%%%%%%%%%%%%%%%
\begin{longtable}{|L{1.025\textwidth}|} \hline
%
{\bf VII. BIBLIOGRAFIA BÁSICA} \\ \hline

\begin{enumerate}
\item FOROUZAN, Behrouz A.; FEGAN, Sophia Chung; GRIESI, Ariovaldo. Comunicação de dados e redes de computadores. 4. ed. São Paulo: McGraw Hill, 2008. 1134 p. ISBN 9788586804885.
\item KUROSE, James F.; ROSS, Keith W. Redes de computadores e a Internet: uma abordagem top-down. 5. ed. São Paulo: Pearson Addison Wesley, 2010. xxiii, 614 p. ISBN 9788588639973.
\item TANENBAUM, Andrew S. Redes de computadores. Rio de Janeiro: Elsevier, Campus, c2003. xx, 945 p. ISBN 9788535211856.
\end{enumerate}
 \\ \hline
\end{longtable}


%\newpage

%%%%%%%%%%%%%%%%%%%%%%%%%%%%%%%%%%%%%%%%%%%%%%%%%%%%%%%%%%%%%%%
\begin{longtable}{|L{1.025\textwidth}|} \hline
%
{\bf VIII. BIBLIOGRAFIA COMPLEMENTAR} \\ \hline
\begin{enumerate}
\item MARIN, Paulo S. Cabeamento estruturado: desvendando cada passo : do projeto à instalação . 4. ed. rev. e atual. São Paulo: Érica, 2014. 336 p. ISBN 9788536502076.
\item STALLINGS, William. Redes e sistemas de comunicação de dados. Rio de Janeiro: Elsevier, c2005. xvi, 449 p. ISBN 9788535217315.
\item COVER, T. M.; THOMAS, Joy A. Elements of information theory. 2nd ed. Hoboken: Wiley Interscience, 2006. xxi, 748 p. ISBN 9780471241959.
Número de chamada: 519.72 C873e 2.ed.
\item MEDEIROS, Julio Cesar de O. Princípios de telecomunicações: teoria e prática. 4. ed. rev. São Paulo: Érica, 2014. 320 p. ISBN 9788536500331.
\item GOMES, Alcides Tadeu. Telecomunicações: transmissão e recepção AM/FM. 21. ed. São Paulo: Érica, 2014. 415 p. ISBN 9788571940734.
%
\end{enumerate}
 \\ \hline
\end{longtable}


\input aprovacao.tex


\end{document}
