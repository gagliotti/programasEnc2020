\documentclass[12pt]{article}
\usepackage[brazil]{babel}
\usepackage{graphicx,t1enc,wrapfig,amsmath,float}
\usepackage{framed,fancyhdr}
\usepackage{multirow}
\usepackage{longtable}
\usepackage{array}
\newcolumntype{L}[1]{>{\raggedright\let\newline\\\arraybackslash\hspace{0pt}}m{#1}}
\newcolumntype{C}[1]{>{\centering\let\newline\\\arraybackslash\hspace{0pt}}m{#1}}
\newcolumntype{R}[1]{>{\raggedleft\let\newline\\\arraybackslash\hspace{0pt}}m{#1}}
%%%%%%%%%%%%%
\oddsidemargin -0.5cm
\evensidemargin -0.5cm
\textwidth 17.5cm
\topmargin -1.5cm
\textheight 22cm
%%%%%%%%%%%% 

%\pagestyle{empty}

\newcommand{\semestre}{2018.2}

\newcommand{\disciplina}{SISTEMAS DIGITAIS EMBARCADOS}
\newcommand{\codigo}{DEC7560}


%%%%%%%%%%%%%%%%%%%%%%%%%%%%%%%%%%%%%%%%%%%%%%%%%%%%%%%
%%%%%%%%%%%%% CRETIDOS
\newcommand{\creditosT}{0}
\newcommand{\creditosP}{4}

%%%%%%%%%%%%%%%%%%%%%%%%%%%%%%%%%%%%%%%%%%%%%%%%%%%%%%%
%%%%%%%%%%%%%% REQUISITOS
\newcommand{\requisitoA}{}
\newcommand{\requisitoB}{}
\newcommand{\requisitoC}{}

%%%%%%%%%%%%%%%%%%%%%%%%%%%%%%%%%%%%%%%%%%%%%%%%%%%%%%%
%%%%%%%%%%%%%%% Atende aos Cursos
\newcommand{\cursoA}{Graduação em Engenharia de Computação \\ \hline}
\newcommand{\cursoB}{}%Graduação em Tecnologias da Informação e Comunicação \\ \hline}
\newcommand{\cursoC}{}

%%%%%%%%%%%%%%%%%%%%%%%%%%%%%%%%%%%%%%%%%%%%%%%%%%%%%%%%
%%%%%%%%%% EMENTA
\newcommand{\ementa}{
Projeto de hardware com microcontroladores. Interface com dispositivos de armazenamento, RAM, Flash e IDE. Interface com periféricos mais comuns, displays de cristal líquido e teclado. Interface com sistemas analógicos. Redes de comunicação, CAN, LIN, RS485 e I2C. Redes wireless WIFI e Bluetooth. Desenvolvimento de software de tempo real baseado em diagramas de estado. Programação na linguagem C em sistemas operacionais de tempo real para microcontroladores ($\mu$COS II).

\\ \hline
}


\begin{document}


%%%%%%%%%%%%%%%%%%%%%%%%%%%%%%%%%%%%%%%%%%%%%%%%%%%%%%%%%%%%%
\begin{longtable}{|C{0.2\textwidth}|C{0.8\textwidth}|} \hline
%
\multirow{6}*{\includegraphics[scale=0.5]{UFSC-foto.jpg}} &\\
%
&{\bf UNIVERSIDADE FEDERAL DE SANTA CATARINA}\hfill\\
%
&{\bf Centro de Ciências, Tecnologias e Saúde} \\
%
&{\bf Departamento de Computação}\\
%
&{\bf PROGRAMA DE ENSINO}\\
%
& \\ \hline

%\multicolumn{2}{|c|}{{\bf SEMESTRE \semestre}}\\ \hline
\end{longtable}


%%%%%%%%%%%%%%%%%%%%%%%%%%%%%%%%%%%%%%%%%%%%%%%%%%%%%%%%%%%%%
\begin{longtable}{|C{0.11\textwidth}|C{0.29\textwidth}|C{0.09\textwidth}|C{0.09\textwidth}|C{0.15\textwidth}|C{0.158\textwidth}|} \hline
%
\multicolumn{6}{|l|}{{\bf I. IDENTIFICAÇÃO DA DISCIPLINA}} \\ \hline
%
\multirow{3}*{{\small CÓDIGO}} & \multirow{3}*{NOME DA DISCIPLINA} &\multicolumn{2}{c|}{{\small N$^\circ$ DE HORAS-AULA }} & {{\small TOTAL DE}} & \multirow{3}*{{\small MODALIDADE}} \\ 
%
& & \multicolumn{2}{c|}{\small SEMANAIS}  & {\small HORAS-AULA} & \\ \cline{3-4}
%
& & {\tiny TEÓRICAS} & {\tiny PRÁTICAS} & {\small SEMESTRAIS} & \\ \hline
% codigo da disciplina carga horaria: teorica - pratica e total
{\bf \small \codigo} & {\bf \small \disciplina } & {\bf \creditosT} & {\bf \creditosP} & {\bf 72} & Presencial\\ \hline
\end{longtable}


%%%%%%%%%%%%%%%%%%%%%%%%%%%%%%%%%%%%%%%%%%%%%%%%%%%%%%%%%%%%%%
\begin{longtable}{|C{0.12\textwidth}|L{0.736\textwidth}|C{0.12\textwidth}|} \hline
%
\multicolumn{3}{|l|}{{\bf II. PRÉ-REQUISITO(S)}} \\ \hline
%
CÓDIGO & NOME DA DISCIPLINA & CURSO \\ \hline	
%
\requisitoA
\requisitoB
\requisitoC
\end{longtable}


%%%%%%%%%%%%%%%%%%%%%%%%%%%%%%%%%%%%%%%%%%%%%%%%%%%%%%%%%%%%%%
\begin{longtable}{|L{1.025\textwidth}|} \hline
%
{\bf III. CURSO(S) PARA O(S) QUAL(IS) A DISCIPLINA É OFERECIDA } \\ \hline
%
\cursoA 
\cursoB
\cursoC

\end{longtable}

%%%%%%%%%%%%%%%%%%%%%%%%%%%%%%%%%%%%%%%%%%%%%%%%%%%%%%%%%%%%%%
\begin{longtable}{|L{1.025\textwidth}|} \hline
%
{\bf IV. EMENTA } \\ \hline
%
\ementa
\end{longtable}

\newpage



%%%%%%%%%%%%%%%%%%%%%%%%%%%%%%%%%%%%%%%%%%%%%%%%%%%%%%%%%%%%%%%
\begin{longtable}{|L{1.025\textwidth}|} \hline
%
{\bf V. OBJETIVOS } \\ \hline

Objetivo Geral: 
\begin{itemize}
\item Capacitar o aluno a projetar um sistema eletrônico que possua um microcontrolador o qual deve controlar os demais elementos do sistema.
\item  Capacitar o aluno a identificar os requisitos de um projeto de sistema embarcado
\item Realizar o particionamento entre software e hardware
\item Escolher as ferramentas de desenvolvimento
\item Capacitar o aluno a fazer uso avançado de dispositivos como memórias, compreendendo a sua forma de interface (paralela, serial (i2c, spi, etc), dispositivos de comunicação ( rádio, wireless, etc.). 
\item Capacitar o aluno a compreender o que é um sistema de tempo real, quando é necessário um sistema com estas características e como escrever software para estes.
\end{itemize}
\\ \hline
\end{longtable}


%%%%%%%%%%%%%%%%%%%%%%%%%%%%%%%%%%%%%%%%%%%%%%%%%%%%%%%%%%%%%%%
\begin{longtable}{|L{1.025\textwidth}|} \hline
%
{\bf VI. CONTEÚDO PROGRAMÁTICO } \\ \hline

UNIDADE 1: Microcontroladores [20 horas-aula]\\
  Realizar um estudo de caso de diferentes famílias de microcontroladores e sua adequação a um sistema embarcado\\
\\
UNIDADE 2: Dipositivos de memória [20 horas-aula]\\
Interfaceamento com dispositivos de armazenamento de dados\\
Memória SRAM (paralela), memória EEPROM (serial), memória tipo cartão SD, memória FRAM, memória EEPROM (paralela)\\
Experiências práticas com os tipos de dispositivos, criação de aplicações para realizar operações sobre memórias (sistema de arquivos). \\
\\
UNIDADE 3: Interface com dispositivos de entrada e saída [8 horas-aula]\\
Displays LCD texto, displays LCD gráficos\\
teclado matricial\\
teclado PS/2, AT\\
\\
UNIDADE 4: Interface com sistemas analógicos [8]\\
 Conversores AD e DA\\
 Sensores analógicos (temperatura, luz, cor)\\
 Experiências práticas com sistemas analógicos, captura e transmissão de dados\\
 Experiências práticas com saída analógica, controle de intensidade de luz, gerador de funções\\
\\
UNIDADE 5: Comunicação de dados [8 horas-aula ]\\
 Módulos de rádio\\
 Experiências usando módulos de comunicação de dados\\
\\ \hline
\end{longtable} 

%\newpage

%%%%%%%%%%%%%%%%%%%%%%%%%%%%%%%%%%%%%%%%%%%%%%%%%%%%%%%%%%%%%%%
\begin{longtable}{|L{1.025\textwidth}|} \hline
%
{\bf VII. BIBLIOGRAFIA BÁSICA} \\ \hline
\begin{enumerate}
\item CATSOULIS, John. Designing embedded hardware. 2nd ed. Sebastopol: O'Reilly, 2005. xvi, 377 p. ISBN 9780596007553. 
\item LABROSSE, Jean J. Embedded systems building blocks: complete and ready-to-use modules in C. 2nd ed. Lawrence: CMP Books, CRC Press, c2000. xxii, 611 p. ISBN 0879306041. 
\item  WHITE, Elecia. Making embedded systems. Sebastopol: O'Reilly, 2011. xiv, 310 p. ISBN 9781449302146.
\end{enumerate}
 \\ \hline
\end{longtable}


%\newpage

%%%%%%%%%%%%%%%%%%%%%%%%%%%%%%%%%%%%%%%%%%%%%%%%%%%%%%%%%%%%%%%
\begin{longtable}{|L{1.025\textwidth}|} \hline
%
{\bf VIII. BIBLIOGRAFIA COMPLEMENTAR} \\ \hline
\begin{enumerate}
\item OLIVEIRA, André Schneider de; ANDRADE, Fernando Souza de. Sistemas embarcados: hardware e firmware na prática. 2. ed. São Paulo: Érica, 2010. 316 p. ISBN 9788536501055. 
\item SIMON, David E. An embedded software primer. Boston: Addison Wesley, c1999. xix, 424 p. ISBN 020161569X. 
\item  FERREIRA, José Manuel Martins. Introdução ao projecto com sistemas digitais e microcontroladores. Porto: FEUP, 1998. 371 p. ISBN 9727520324.
\item  SOUSA, Daniel Rodrigues de. Microcontroladores ARM7: (Philips - Família LPC213x) : o poder dos 32 bits : teoria e prática. 1. ed. São Paulo: Érica, c2006. 280 p. ISBN 9788536501208.
\item  LI, Qing; YAO, Caroline. Real-time concepts for embedded systems. Boca Raton: CMP Books, 2003. xii, 294 p. ISBN 9781578201242
%
\end{enumerate}
 \\ \hline
\end{longtable}


\input aprovacao.tex


\end{document}
