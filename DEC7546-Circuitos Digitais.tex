\documentclass[12pt]{article}
\usepackage[brazil]{babel}
\usepackage{graphicx,t1enc,wrapfig,amsmath,float}
\usepackage{framed,fancyhdr}
\usepackage{multirow}
\usepackage{longtable}
\usepackage{array}
\newcolumntype{L}[1]{>{\raggedright\let\newline\\\arraybackslash\hspace{0pt}}m{#1}}
\newcolumntype{C}[1]{>{\centering\let\newline\\\arraybackslash\hspace{0pt}}m{#1}}
\newcolumntype{R}[1]{>{\raggedleft\let\newline\\\arraybackslash\hspace{0pt}}m{#1}}
%%%%%%%%%%%%%
\oddsidemargin -0.5cm
\evensidemargin -0.5cm
\textwidth 17.5cm
\topmargin -1.5cm
\textheight 22cm
%%%%%%%%%%%% 

%\pagestyle{empty}

\newcommand{\semestre}{2018.2}

\newcommand{\disciplina}{CIRCUITOS DIGITAIS}
\newcommand{\codigo}{DEC7546}


%%%%%%%%%%%%%%%%%%%%%%%%%%%%%%%%%%%%%%%%%%%%%%%%%%%%%%%
%%%%%%%%%%%%% CRETIDOS
\newcommand{\creditosT}{4}
\newcommand{\creditosP}{0}

%%%%%%%%%%%%%%%%%%%%%%%%%%%%%%%%%%%%%%%%%%%%%%%%%%%%%%%
%%%%%%%%%%%%%% REQUISITOS
\newcommand{\requisitoA}{}
\newcommand{\requisitoB}{}
\newcommand{\requisitoC}{}

%%%%%%%%%%%%%%%%%%%%%%%%%%%%%%%%%%%%%%%%%%%%%%%%%%%%%%%
%%%%%%%%%%%%%%% Atende aos Cursos
\newcommand{\cursoA}{Graduação em Engenharia de Computação \\ \hline}
\newcommand{\cursoB}{}%Graduação em Tecnologias da Informação e Comunicação \\ \hline}
\newcommand{\cursoC}{}

%%%%%%%%%%%%%%%%%%%%%%%%%%%%%%%%%%%%%%%%%%%%%%%%%%%%%%%%
%%%%%%%%%% EMENTA
\newcommand{\ementa}{
Sistemas Numéricos. Álgebra de Boole (teoremas). Portas lógicas. Circuitos combinacionais. Técnicas de minimização de hardware. Implementação de dispositivos elementares de memória (latchs e flip-flops). Circuitos Sequenciais. Implementação de módulos básicos. Ambiente de simulação.
\\ \hline
}


\begin{document}


%%%%%%%%%%%%%%%%%%%%%%%%%%%%%%%%%%%%%%%%%%%%%%%%%%%%%%%%%%%%%
\input cabecalho.tex


%%%%%%%%%%%%%%%%%%%%%%%%%%%%%%%%%%%%%%%%%%%%%%%%%%%%%%%%%%%%%
\begin{longtable}{|C{0.11\textwidth}|C{0.29\textwidth}|C{0.09\textwidth}|C{0.09\textwidth}|C{0.15\textwidth}|C{0.158\textwidth}|} \hline
%
\multicolumn{6}{|l|}{{\bf I. IDENTIFICAÇÃO DA DISCIPLINA}} \\ \hline
%
\multirow{3}*{{\small CÓDIGO}} & \multirow{3}*{NOME DA DISCIPLINA} &\multicolumn{2}{c|}{{\small N$^\circ$ DE HORAS-AULA }} & {{\small TOTAL DE}} & \multirow{3}*{{\small MODALIDADE}} \\ 
%
& & \multicolumn{2}{c|}{\small SEMANAIS}  & {\small HORAS-AULA} & \\ \cline{3-4}
%
& & {\tiny TEÓRICAS} & {\tiny PRÁTICAS} & {\small SEMESTRAIS} & \\ \hline
% codigo da disciplina carga horaria: teorica - pratica e total
{\bf \small \codigo} & {\bf \small \disciplina } & {\bf \creditosT} & {\bf \creditosP} & {\bf 72} & Presencial\\ \hline
\end{longtable}


%%%%%%%%%%%%%%%%%%%%%%%%%%%%%%%%%%%%%%%%%%%%%%%%%%%%%%%%%%%%%%
\begin{longtable}{|C{0.12\textwidth}|L{0.736\textwidth}|C{0.12\textwidth}|} \hline
%
\multicolumn{3}{|l|}{{\bf II. PRÉ-REQUISITO(S)}} \\ \hline
%
CÓDIGO & NOME DA DISCIPLINA & CURSO \\ \hline	
%
\requisitoA
\requisitoB
\requisitoC
\end{longtable}


%%%%%%%%%%%%%%%%%%%%%%%%%%%%%%%%%%%%%%%%%%%%%%%%%%%%%%%%%%%%%%
\begin{longtable}{|L{1.025\textwidth}|} \hline
%
{\bf III. CURSO(S) PARA O(S) QUAL(IS) A DISCIPLINA É OFERECIDA } \\ \hline
%
\cursoA 
\cursoB
\cursoC

\end{longtable}

%%%%%%%%%%%%%%%%%%%%%%%%%%%%%%%%%%%%%%%%%%%%%%%%%%%%%%%%%%%%%%
\begin{longtable}{|L{1.025\textwidth}|} \hline
%
{\bf IV. EMENTA } \\ \hline
%
\ementa
\end{longtable}

\newpage



%%%%%%%%%%%%%%%%%%%%%%%%%%%%%%%%%%%%%%%%%%%%%%%%%%%%%%%%%%%%%%%
\begin{longtable}{|L{1.025\textwidth}|} \hline
%
{\bf V. OBJETIVOS } \\ \hline
%
\begin{itemize}
\item Representar equações lógicas, efetuar simplificações por mapas de Karnaugh;
\item Implementar funções lógicas utilizando portas lógicas;
\item Projetar circuitos eletrônicos fazendo dos principais dispositivos;
\item Compreender o funcionamento de registradores, memórias e fazer associações em série e em paralelo;
\item Conhecer o funcionamento interno dos principais dispositivos.
\end{itemize}

\\ \hline
\end{longtable}


%%%%%%%%%%%%%%%%%%%%%%%%%%%%%%%%%%%%%%%%%%%%%%%%%%%%%%%%%%%%%%%
\begin{longtable}{|L{1.025\textwidth}|} \hline
%
{\bf VI. CONTEÚDO PROGRAMÁTICO } \\ \hline\\
UNIDADE 1: Sistema de numeração e códigos especiais 
\begin{itemize}
    \item Sistema numérico decimal;
    \item Sistema decimal, binário, hexadecimal, conversão de bases;
    \item Operações Aritméticas básicas;
    \item Representação de números negativos.
\end{itemize}

UNIDADE 2: Álgebra de Boole 
\begin{itemize}
    \item Representar funções lógicas por meio de equações;
    \item Realizar simplificações aplicando teoremas fundamentais e mapas K (minimização);
    \item Implementar funções lógicas através de portas lógicas.
\end{itemize}

UNIDADE 3: Circuitos Combinacionais Básicos\\
Estudar os dispositivos fundamentais:
\begin{itemize}
    \item  Multiplexadores, demultiplexadores;
    \item Decodificadores, comparadores e codificadores.
\end{itemize}

UNIDADE 4: Somadores 
\begin{itemize}
    \item Circuitos aritméticos somadores;
    \item Circuitos aritméticos subtratores.
\end{itemize}

UNIDADE 5: Circuitos Sequenciais
\begin{itemize}
    \item Latches, flipflops;
    \item Máquinas de estado.
\end{itemize}

UNIDADE 6: Registradores 
\begin{itemize}
    \item Série, paralelo, associação;
    \item CIs.
\end{itemize}

UNIDADE 7: Contadores
\begin{itemize}
    \item  Up, Down, reversível;
    \item  Síncrono, assíncrono, sequencia não natural.
\end{itemize}

UNIDADE 8: Memória 
\begin{itemize}
    \item Tipos de memória e seu funcionamento interno;
    \item Associação de memória.
\end{itemize}
\\ \hline
\end{longtable} 



%\newpage

%%%%%%%%%%%%%%%%%%%%%%%%%%%%%%%%%%%%%%%%%%%%%%%%%%%%%%%%%%%%%%%
\begin{longtable}{|L{1.025\textwidth}|} \hline
%
{\bf VII. BIBLIOGRAFIA BÁSICA} \\ \hline
\begin{enumerate}
%
\item TOCCI, RONALD J.; WIDMER, NEAL S.; MOSS, GREGORY L. Sistemas Digitais: Princípios e Aplicações 11ª edição. São Paulo: Pearson. 
\item  BIGNELL, James; DONOVAN, Robert. Eletrônica digital. São Paulo: Cengage Learning, 2010. xviii, 648 p. ISBN 9788522107452
%
\item D'AMORE, ROBERTO, VHDL - Descriçao e Sintese de Circuitos Digitais, LTC, ISBN: 8521620543, ISBN-13: 9788521620549, 2ª edição, 2012.

\end{enumerate}
 \\ \hline
\end{longtable}


%\newpage

%%%%%%%%%%%%%%%%%%%%%%%%%%%%%%%%%%%%%%%%%%%%%%%%%%%%%%%%%%%%%%%
\begin{longtable}{|L{1.025\textwidth}|} \hline
%
{\bf VIII. BIBLIOGRAFIA COMPLEMENTAR} \\ \hline
\begin{enumerate}

\item FERREIRA, José Manuel Martins. Introdução ao projeto com sistemas digitais e microcontroladores. Porto: FEUP, 1998. 371 p. ISBN 9727520324 
\item WILSON, Peter. The circuit designer's companion. 3rd ed. Amsterdam: Elsevier, 2012. xv, 439 p. ISBN 9780080971384
\item PEDRONI, Volnei A. Eletrônica digital moderna e VHDL. Rio de Janeiro: Elsevier, c2010. 619 p. ISBN 9788535234657
\item IDOETA, Ivan V.; CAPUANO, Francisco G. Elementos de eletrônica digital. 41. ed. rev. e atual. São Paulo: Livros Erica Ed., c2012. 544 p. ISBN 9788571940192 
\item ARAUJO, Celso de; CRUZ, Eduardo Cesar Alves; CHOUERI JUNIOR, Salomão. Eletrônica digital. 1. ed. São Paulo: Érica, c2014. 168 p. (Série Eixos Controle e processos industriais). ISBN 9788536508177.
%
\end{enumerate}
 \\ \hline
\end{longtable}


\input aprovacao.tex


\end{document}
