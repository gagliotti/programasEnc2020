\documentclass[12pt]{article}
\usepackage[brazil]{babel}
\usepackage{graphicx,t1enc,wrapfig,amsmath,float}
\usepackage{framed,fancyhdr}
\usepackage{multirow}
\usepackage{longtable}
\usepackage{array}
\newcolumntype{L}[1]{>{\raggedright\let\newline\\\arraybackslash\hspace{0pt}}m{#1}}
\newcolumntype{C}[1]{>{\centering\let\newline\\\arraybackslash\hspace{0pt}}m{#1}}
\newcolumntype{R}[1]{>{\raggedleft\let\newline\\\arraybackslash\hspace{0pt}}m{#1}}
%%%%%%%%%%%%%
\oddsidemargin -0.5cm
\evensidemargin -0.5cm
\textwidth 17.5cm
\topmargin -1.5cm
\textheight 22cm
%%%%%%%%%%%% 

%\pagestyle{empty}

\newcommand{\semestre}{2018.2}

\newcommand{\disciplina}{INTELIGÊNCIA ARTIFICIAL}
\newcommand{\codigo}{DEC7133}


%%%%%%%%%%%%%%%%%%%%%%%%%%%%%%%%%%%%%%%%%%%%%%%%%%%%%%%
%%%%%%%%%%%%% CRETIDOS
\newcommand{\creditosT}{2}
\newcommand{\creditosP}{2}

%%%%%%%%%%%%%%%%%%%%%%%%%%%%%%%%%%%%%%%%%%%%%%%%%%%%%%%
%%%%%%%%%%%%%% REQUISITOS
\newcommand{\requisitoA}{DEC7121 & FUNDAMENTOS MATEMÁTICOS PARA COMPUTAÇÃO & TIC\\ \hline }
\newcommand{\requisitoB}{CIT7584 & ESTRUTURA DE DADOS E ALGORITMOS & TIC\\ \hline}
\newcommand{\requisitoC}{}

%%%%%%%%%%%%%%%%%%%%%%%%%%%%%%%%%%%%%%%%%%%%%%%%%%%%%%%
%%%%%%%%%%%%%%% Atende aos Cursos
\newcommand{\cursoA}{}%Graduação em Engenharia de Computação. \\ \hline}
\newcommand{\cursoB}{Graduação em Tecnologias da Informação e Comunicação \\ \hline}
\newcommand{\cursoC}{}

%%%%%%%%%%%%%%%%%%%%%%%%%%%%%%%%%%%%%%%%%%%%%%%%%%%%%%%%
%%%%%%%%%% EMENTA
\newcommand{\ementa}{
Introdução e histórico. Teoria de problemas e sua resolução. Paradigmas da IA. Modelagem de Agentes Inteligentes. Métodos de busca. Representação e aquisição de Conhecimento. Métodos de raciocínio. Tratamento de incertezas. Sistemas especialistas. Fundamentos de: lógica nebulosa, redes neurais artificiais e algoritmos genéticos.
 \\ \hline
}




\begin{document}


%%%%%%%%%%%%%%%%%%%%%%%%%%%%%%%%%%%%%%%%%%%%%%%%%%%%%%%%%%%%%
\input cabecalho.tex


%%%%%%%%%%%%%%%%%%%%%%%%%%%%%%%%%%%%%%%%%%%%%%%%%%%%%%%%%%%%%
\begin{longtable}{|C{0.11\textwidth}|C{0.29\textwidth}|C{0.09\textwidth}|C{0.09\textwidth}|C{0.15\textwidth}|C{0.158\textwidth}|} \hline
%
\multicolumn{6}{|l|}{{\bf I. IDENTIFICAÇÃO DA DISCIPLINA}} \\ \hline
%
\multirow{3}*{{\small CÓDIGO}} & \multirow{3}*{NOME DA DISCIPLINA} &\multicolumn{2}{c|}{{\small N$^\circ$ DE HORAS-AULA }} & {{\small TOTAL DE}} & \multirow{3}*{{\small MODALIDADE}} \\ 
%
& & \multicolumn{2}{c|}{\small SEMANAIS}  & {\small HORAS-AULA} & \\ \cline{3-4}
%
& & {\tiny TEÓRICAS} & {\tiny PRÁTICAS} & {\small SEMESTRAIS} & \\ \hline
% codigo da disciplina carga horaria: teorica - pratica e total
{\bf \small \codigo} & {\bf \small \disciplina } & {\bf \creditosT} & {\bf \creditosP} & {\bf 72} & Presencial\\ \hline
\end{longtable}


%%%%%%%%%%%%%%%%%%%%%%%%%%%%%%%%%%%%%%%%%%%%%%%%%%%%%%%%%%%%%%
\begin{longtable}{|C{0.12\textwidth}|L{0.736\textwidth}|C{0.12\textwidth}|} \hline
%
\multicolumn{3}{|l|}{{\bf II. PRÉ-REQUISITO(S)}} \\ \hline
%
CÓDIGO & NOME DA DISCIPLINA & CURSO \\ \hline	
%
\requisitoA
\requisitoB
\requisitoC
\end{longtable}


%%%%%%%%%%%%%%%%%%%%%%%%%%%%%%%%%%%%%%%%%%%%%%%%%%%%%%%%%%%%%%
\begin{longtable}{|L{1.025\textwidth}|} \hline
%
{\bf III. CURSO(S) PARA O(S) QUAL(IS) A DISCIPLINA É OFERECIDA } \\ \hline
%
\cursoA 
\cursoB
\cursoC

\end{longtable}

%%%%%%%%%%%%%%%%%%%%%%%%%%%%%%%%%%%%%%%%%%%%%%%%%%%%%%%%%%%%%%
\begin{longtable}{|L{1.025\textwidth}|} \hline
%
{\bf IV. EMENTA } \\ \hline
%
\ementa
\end{longtable}

\newpage



%%%%%%%%%%%%%%%%%%%%%%%%%%%%%%%%%%%%%%%%%%%%%%%%%%%%%%%%%%%%%%%
\begin{longtable}{|L{1.025\textwidth}|} \hline
%
{\bf V. OBJETIVOS } \\ \hline
%
Objetivo Geral:\\

Fornecer subsídios ao aluno para que ele possa conhecer as técnicas da inteligência Artificial.\\
\\
Objetivos Específicos:
\begin{itemize}  
\item Caracterizar a inteligência artificial e suas aplicações;
\item Demonstrar os principais paradigmas da inteligência artificial;
\item Conhecer as principais técnicas da inteligência Artificial e suas aplicações na solução de problemas.
\end{itemize}

\\ \hline
\end{longtable}


%%%%%%%%%%%%%%%%%%%%%%%%%%%%%%%%%%%%%%%%%%%%%%%%%%%%%%%%%%%%%%%
\begin{longtable}{|L{1.025\textwidth}|} \hline
%
{\bf VI. CONTEÚDO PROGRAMÁTICO } \\ \hline

Unidade 1. Introdução e inovações tecnológicas com IA.\\
           a. O que é inteligência artificial\\
           b. Evolução histórica da inteligência artificial (Material no Moodle)\\
           c. Problemas tratados em inteligência artificial\\
           d. Domínios de aplicação da inteligência artificial\\
\\
Unidade 2. Paradigmas da IA\\
                  Simbólica, Conexionista, Evolucionista e Híbrido\\
\\
Unidade 3. Teoria de problemas e sua resolução.\\
a. Teoria de problemas\\
b. Características de problemas\\
c. Estratégias para resolver problemas\\
d. Exemplos de problemas clássicos\\
\\
Unidade 4. Métodos de buscas\\
a. Busca heurística  \\
b. Método de busca Cega\\
c. Método de busca competitiva\\
         \\
Unidade 5. Representação e aquisição de Conhecimento.\\
                  a. Métodos de raciocínio \\
 b. Raciocínio progressivo e regressivo (Regras)\\
\\
Unidade 6. Sistemas especialistas\\
           a. Definição de sistemas especialistas\\
           b. Estrutura de um sistema especialista\\
           c. Exemplos de sistemas especialistas\\
\\
Unidade 7. Modelagem de Agentes Inteligentes\\
          a. Definição de agente \\
          b. Propriedades de um agente inteligente\\
          c. Características de sistemas multiagentes \\
          d. Exemplos de SMA com interação entre agentes\\
\\
Unidade 8. Redes neurais artificiais \\
       a. Caracterização de RNA. \\
       b. Principais Arquiteturas de RNA.\\
       c. Aprendizado supervisionado e não supervisionado\\
       d. Exemplos de modelos de Redes Neurais Artificiais\\
  \\
Unidade  9. Fundamentos de Algoritmos genéticos\\
a. Caracterização de Algoritmos genéticos\\
b. Exemplos de aplicações com Algoritmos genéticos\\
\\
Unidade 10. Lógica nebulosa\\
          a. Características e aplicações\\
\\ \hline
\end{longtable} 



%%%%%%%%%%%%%%%%%%%%%%%%%%%%%%%%%%%%%%%%%%%%%%%%%%%%%%%%%%%%%%%
\begin{longtable}{|L{1.025\textwidth}|} \hline
%
{\bf VII. BIBLIOGRAFIA BÁSICA} \\ \hline
\begin{enumerate}
%
\item HAYKIN, Simon. Redes Neurais: princípios e prática. 2. ed. Porto Alegre: Bookman, 2001.
\item COPPIN, Ben. Inteligência artificial. Rio de Janeiro: LTC, c2010. xxv, 636 p.
\item RUSSELL, S.; NORVIG, P. Inteligência artificial. 2. ed. Rio de Janeiro: Campus, 2004.
%

\end{enumerate}
 \\ \hline
\end{longtable}


\newpage

%%%%%%%%%%%%%%%%%%%%%%%%%%%%%%%%%%%%%%%%%%%%%%%%%%%%%%%%%%%%%%%
\begin{longtable}{|L{1.025\textwidth}|} \hline
%
{\bf VIII. BIBLIOGRAFIA COMPLEMENTAR} \\ \hline
\begin{enumerate}
\item BARRETO, J. M. Inteligência artificial: uma abordagem híbrida. Editora PPP, 2001.
\item BITTENCOURT, G. Inteligência artificial: ferramentas e teorias. Florianópolis: Editora da UFSC, 1998.
\item BRAGA, A.P; CARVALHO, A.P.L.; LUDERMIR, T.B. Redes Neuras Artificiais, Ed. Editora LTC, 2007.
\item LINDEN, R., Algoritmos Genéticos - Uma Importante Ferramenta da Inteligência Computacional, Ed. Brasport, 2ª Ed. 2008.
\item ROSA, J.L.G, Fundamentação da Inteligência Artificial, Editora LTG,2011.

%
\end{enumerate}
 \\ \hline
\end{longtable}


\input aprovacao.tex


\end{document}
