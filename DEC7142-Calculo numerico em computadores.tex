\documentclass[12pt]{article}
\usepackage[brazil]{babel}
\usepackage{graphicx,t1enc,wrapfig,amsmath,float}
\usepackage{framed,fancyhdr}
\usepackage{multirow}
\usepackage{longtable}
\usepackage{array}
\newcolumntype{L}[1]{>{\raggedright\let\newline\\\arraybackslash\hspace{0pt}}m{#1}}
\newcolumntype{C}[1]{>{\centering\let\newline\\\arraybackslash\hspace{0pt}}m{#1}}
\newcolumntype{R}[1]{>{\raggedleft\let\newline\\\arraybackslash\hspace{0pt}}m{#1}}
%%%%%%%%%%%%%
\oddsidemargin -0.5cm
\evensidemargin -0.5cm
\textwidth 17.5cm
\topmargin -1.5cm
\textheight 22cm
%%%%%%%%%%%% 

%\pagestyle{empty}

\newcommand{\semestre}{2018.2}

\newcommand{\disciplina}{CÁLCULO NUMÉRICO EM COMPUTADORES}
\newcommand{\codigo}{DEC7142}


%%%%%%%%%%%%%%%%%%%%%%%%%%%%%%%%%%%%%%%%%%%%%%%%%%%%%%%
%%%%%%%%%%%%% CRETIDOS
\newcommand{\creditosT}{2}
\newcommand{\creditosP}{2}

%%%%%%%%%%%%%%%%%%%%%%%%%%%%%%%%%%%%%%%%%%%%%%%%%%%%%%%
%%%%%%%%%%%%%% REQUISITOS
\newcommand{\requisitoA}{FQM7104 & ÁLGEBRA LINEAR & ENE\\ \hline }
\newcommand{\requisitoB}{FQM7106 & CÁLCULO IV & ENE \\ \hline}
\newcommand{\requisitoC}{DEC7143 & Lógica de Programação & ENE\\ \hline}

%%%%%%%%%%%%%%%%%%%%%%%%%%%%%%%%%%%%%%%%%%%%%%%%%%%%%%%
%%%%%%%%%%%%%%% Atende aos Cursos
\newcommand{\cursoA}{Graduação em Engenharia de Computação \\ \hline}
\newcommand{\cursoB}{}%Graduação em Tecnologias da Informação e Comunicação \\ \hline}
\newcommand{\cursoC}{Graduação em Engenharia de Energia \\ \hline}

%%%%%%%%%%%%%%%%%%%%%%%%%%%%%%%%%%%%%%%%%%%%%%%%%%%%%%%%
%%%%%%%%%% EMENTA
\newcommand{\ementa}{
Sistemas de numeração e erros numéricos. Resolução de equações não lineares transcendentais e polinomiais. Resolução de Sistemas Lineares e não lineares. Aproximações de funções por séries. Ajuste de curvas a dados experimentais. Integração numérica. Resolução numérica de equações e sistemas de equações diferenciais ordinárias.
 \\ \hline
}


\begin{document}


%%%%%%%%%%%%%%%%%%%%%%%%%%%%%%%%%%%%%%%%%%%%%%%%%%%%%%%%%%%%%
\input cabecalho.tex


%%%%%%%%%%%%%%%%%%%%%%%%%%%%%%%%%%%%%%%%%%%%%%%%%%%%%%%%%%%%%
\begin{longtable}{|C{0.11\textwidth}|C{0.29\textwidth}|C{0.09\textwidth}|C{0.09\textwidth}|C{0.15\textwidth}|C{0.158\textwidth}|} \hline
%
\multicolumn{6}{|l|}{{\bf I. IDENTIFICAÇÃO DA DISCIPLINA}} \\ \hline
%
\multirow{3}*{{\small CÓDIGO}} & \multirow{3}*{NOME DA DISCIPLINA} &\multicolumn{2}{c|}{{\small N$^\circ$ DE HORAS-AULA }} & {{\small TOTAL DE}} & \multirow{3}*{{\small MODALIDADE}} \\ 
%
& & \multicolumn{2}{c|}{\small SEMANAIS}  & {\small HORAS-AULA} & \\ \cline{3-4}
%
& & {\tiny TEÓRICAS} & {\tiny PRÁTICAS} & {\small SEMESTRAIS} & \\ \hline
% codigo da disciplina carga horaria: teorica - pratica e total
{\bf \small \codigo} & {\bf \small \disciplina } & {\bf \creditosT} & {\bf \creditosP} & {\bf 72} & Presencial\\ \hline
\end{longtable}


%%%%%%%%%%%%%%%%%%%%%%%%%%%%%%%%%%%%%%%%%%%%%%%%%%%%%%%%%%%%%%
\begin{longtable}{|C{0.12\textwidth}|L{0.736\textwidth}|C{0.12\textwidth}|} \hline
%
\multicolumn{3}{|l|}{{\bf II. PRÉ-REQUISITO(S)}} \\ \hline
%
CÓDIGO & NOME DA DISCIPLINA & CURSO \\ \hline	
%
\requisitoA
\requisitoB
\requisitoC
\end{longtable}


%%%%%%%%%%%%%%%%%%%%%%%%%%%%%%%%%%%%%%%%%%%%%%%%%%%%%%%%%%%%%%
\begin{longtable}{|L{1.025\textwidth}|} \hline
%
{\bf III. CURSO(S) PARA O(S) QUAL(IS) A DISCIPLINA É OFERECIDA } \\ \hline
%
\cursoA 
\cursoB
\cursoC

\end{longtable}

%%%%%%%%%%%%%%%%%%%%%%%%%%%%%%%%%%%%%%%%%%%%%%%%%%%%%%%%%%%%%%
\begin{longtable}{|L{1.025\textwidth}|} \hline
%
{\bf IV. EMENTA } \\ \hline
%
\ementa
\end{longtable}

\newpage



%%%%%%%%%%%%%%%%%%%%%%%%%%%%%%%%%%%%%%%%%%%%%%%%%%%%%%%%%%%%%%%
\begin{longtable}{|L{1.025\textwidth}|} \hline
%
{\bf V. OBJETIVOS } \\ \hline
%
Objetivo Geral:\\

Tornar o aluno apto a utilizar recursos computacionais nas soluções de problemas de cálculo que envolvam métodos numéricos.\\
\\
Objetivos Específicos: 
\begin{itemize}
\item Identificar os erros que afetam os resultados numéricos fornecidos por máquinas digitais;
\item Resolver equações por métodos numéricos iterativos.
\item Conhecer as propriedades básicas dos polinômios e determinar as raízes das equações polinomiais.
\item  Resolver sistemas de equações lineares por métodos diretos e iterativos.
\item Conhecer e usar o método dos mínimos quadrados para o ajuste polinomial e não polinomial.
\item Conhecer e utilizar a técnica de interpolação polinomial para a aproximação de funções.
\item Efetuar integração por meio de métodos numéricos.
\item Resolver equações e sistemas de equações diferenciais ordinárias através de métodos numéricos.
\item Elaborar algoritmos correspondentes a todos os métodos numéricos abordados e implementá-los.
\end{itemize}
\\ \hline
\end{longtable}


%%%%%%%%%%%%%%%%%%%%%%%%%%%%%%%%%%%%%%%%%%%%%%%%%%%%%%%%%%%%%%%
\begin{longtable}{|L{1.025\textwidth}|} \hline
%
{\bf VI. CONTEÚDO PROGRAMÁTICO } \\ \hline
Conteúdo Teórico seguido de Conteúdo Prático com desenvolvimento de algoritmos.\\
\\
UNIDADE 1: Algoritmos e erros\\
Aritmética de ponto flutuante.\\
Erro absoluto e erro relativo.\\
Estabilidade de algoritmos numéricos e condicionamento.\\
\\
UNIDADE 2: Zeros de funções\\
Localização de raízes de funções.\\
Métodos de partição: Bissecção e Falsa-Posição.\\
Métodos iterativos: Aproximações sucessivas (convergência), Newton e Secante.\\
\\
UNIDADE 3: Sistemas Lineares e não Lineares\\
Resolução de Sistemas Lineares (Aspectos Computacionais).\\
Métodos Diretos: Eliminação Gaussiana e Decomposição LU.\\
Métodos iterativos: Gauss-Seidel.\\
Método de Newton e variantes.\\
\\
UNIDADE 4: Aproximação de funções\\
Ajuste de curvas pelo método dos Mínimos Quadrados.\\
Interpolação polinomial\\
\\
UNIDADE 5: Integração numérica\\
Método dos Trapézios e Simpson\\
Quadratura Gaussiana\\
\\
UNIDADE 6: Equações diferenciais ordinárias\\
Resolução numérica de equações e sistemas de equações diferenciais ordinárias. \\
Métodos de passo simples.\\
Métodos de Runge-Kutta.\\

\\ \hline
\end{longtable} 

%\newpage

%%%%%%%%%%%%%%%%%%%%%%%%%%%%%%%%%%%%%%%%%%%%%%%%%%%%%%%%%%%%%%%
\begin{longtable}{|L{1.025\textwidth}|} \hline
%
{\bf VII. BIBLIOGRAFIA BÁSICA} \\ \hline
\begin{enumerate}
%
\item BURDEN, Richard L.; FAIRES, J. Douglas. Análise numérica. São Paulo: CENGAGE Learning, c2008. xiii, 721 p. ISBN 9788522106011.
\item RUGGIERO, Marcia A. Gomes; LOPES, Vera Lucia da Rocha. Cálculo Numérico: aspectos teóricos e computacionais. 2. ed. São Paulo: Makron Books, 1996. 406 p. 
\item PRESS, William H. Numerical recipes: the art of scientific computing. 3. ed. New York: Cambridge, 66 2007. 1235p. 
\end{enumerate}
 \\ \hline
\end{longtable}


\newpage

%%%%%%%%%%%%%%%%%%%%%%%%%%%%%%%%%%%%%%%%%%%%%%%%%%%%%%%%%%%%%%%
\begin{longtable}{|L{1.025\textwidth}|} \hline
%
{\bf VIII. BIBLIOGRAFIA COMPLEMENTAR} \\ \hline
\begin{enumerate}
\item CHENEY, E. W; KINCAID, David. Numerical mathematics and computing. 7th ed. Pacific Grove: Thomson Brooks/Cole, c2013. 763 p. ISBN 9781133491811.
\item FRANCO, Neide Maria Bertoldi. Cálculo numérico. São Paulo: Pearson Prentice Hall, 2007. xii, 505 p. ISBN 9788576010872.
\item CHAPRA, Steven C.; CANALE, Raymond P. Métodos numéricos para engenharia. 5. ed. São Paulo: McGraw Hill, 2008. xxi, 809 p. ISBN 9788586804878.
\item PUGA, Leila Zardo; TÁRCIA, José Henrique Mendes; PAZ, Álvaro Puga. Cálculo numérico. 2. ed. São Paulo: LCTE, 2012. 176 p. ISBN 9788585908157.
\item BURIAN, Reinaldo; LIMA, Antonio Carlos de; HETEM JUNIOR, Annibal. Cálculo numérico. Rio de Janeiro: LTC, c2007. xii, 153 p. (Fundamentos de informática). ISBN 9788521615620.
%
\end{enumerate}
 \\ \hline
\end{longtable}


\input aprovacao.tex


\end{document}
