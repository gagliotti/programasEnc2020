\documentclass[12pt]{article}
\usepackage[brazil]{babel}
\usepackage{graphicx,t1enc,wrapfig,amsmath,float}
\usepackage{framed,fancyhdr}
\usepackage{multirow}
\usepackage{longtable}
\usepackage{array}
\newcolumntype{L}[1]{>{\raggedright\let\newline\\\arraybackslash\hspace{0pt}}m{#1}}
\newcolumntype{C}[1]{>{\centering\let\newline\\\arraybackslash\hspace{0pt}}m{#1}}
\newcolumntype{R}[1]{>{\raggedleft\let\newline\\\arraybackslash\hspace{0pt}}m{#1}}
%%%%%%%%%%%%%
\oddsidemargin -0.5cm
\evensidemargin -0.5cm
\textwidth 17.5cm
\topmargin -1.5cm
\textheight 22cm
%%%%%%%%%%%% 

%\pagestyle{empty}

\newcommand{\semestre}{2018.2}

\newcommand{\disciplina}{LINGUAGEM DE PROGRAMAÇÃO I}
\newcommand{\codigo}{DEC0012}


%%%%%%%%%%%%%%%%%%%%%%%%%%%%%%%%%%%%%%%%%%%%%%%%%%%%%%%
%%%%%%%%%%%%% CRETIDOS
\newcommand{\creditosT}{2}
\newcommand{\creditosP}{4}

%%%%%%%%%%%%%%%%%%%%%%%%%%%%%%%%%%%%%%%%%%%%%%%%%%%%%%%
%%%%%%%%%%%%%% REQUISITOS
\newcommand{\requisitoA}{}
\newcommand{\requisitoB}{}
\newcommand{\requisitoC}{}

%%%%%%%%%%%%%%%%%%%%%%%%%%%%%%%%%%%%%%%%%%%%%%%%%%%%%%%
%%%%%%%%%%%%%%% Atende aos Cursos
\newcommand{\cursoA}{Graduação em Engenharia de Computação. \\ \hline}
\newcommand{\cursoB}{}
\newcommand{\cursoC}{}

%%%%%%%%%%%%%%%%%%%%%%%%%%%%%%%%%%%%%%%%%%%%%%%%%%%%%%%%
%%%%%%%%%% EMENTA
\newcommand{\ementa}{

Algoritmos e lógica de programação. Formas de representação de algoritmos. Programação estruturada, linguagens de programação e ambientes de programação. Variáveis: nomeação, declaração, inicialização, tipos de dados. Expressões: expressões aritméticas, expressão literal, expressões lógicas, expressões relacionais. Estruturas de Controle de Fluxo: linear, condicional, repetição. Estruturas de Dados Simples: vetores, matrizes, registros. Arquitetura de programa mínimo: paradigmas, regras de escopo, funções, modularização. Ponteiros e Alocação dinâmica. Funções: definição, declaração, tipos de passagem de parâmetro. Entrada e Saída de Dados: arquivos, acesso sequencial, acesso direto.

 \\ \hline
}




\begin{document}


%%%%%%%%%%%%%%%%%%%%%%%%%%%%%%%%%%%%%%%%%%%%%%%%%%%%%%%%%%%%%
\begin{longtable}{|C{0.2\textwidth}|C{0.8\textwidth}|} \hline
%
\multirow{6}*{\includegraphics[scale=0.5]{UFSC-foto.jpg}} &\\
%
&{\bf UNIVERSIDADE FEDERAL DE SANTA CATARINA}\hfill\\
%
&{\bf Centro de Ciências, Tecnologias e Saúde} \\
%
&{\bf Departamento de Computação}\\
%
&{\bf PROGRAMA DE ENSINO}\\
%
& \\ \hline

%\multicolumn{2}{|c|}{{\bf SEMESTRE \semestre}}\\ \hline
\end{longtable}


%%%%%%%%%%%%%%%%%%%%%%%%%%%%%%%%%%%%%%%%%%%%%%%%%%%%%%%%%%%%%
\begin{longtable}{|C{0.11\textwidth}|C{0.29\textwidth}|C{0.09\textwidth}|C{0.09\textwidth}|C{0.15\textwidth}|C{0.158\textwidth}|} \hline
%
\multicolumn{6}{|l|}{{\bf I. IDENTIFICAÇÃO DA DISCIPLINA}} \\ \hline
%
\multirow{3}*{{\small CÓDIGO}} & \multirow{3}*{NOME DA DISCIPLINA} &\multicolumn{2}{c|}{{\small N$^\circ$ DE HORAS-AULA }} & {{\small TOTAL DE}} & \multirow{3}*{{\small MODALIDADE}} \\ 
%
& & \multicolumn{2}{c|}{\small SEMANAIS}  & {\small HORAS-AULA} & \\ \cline{3-4}
%
& & {\tiny TEÓRICAS} & {\tiny PRÁTICAS} & {\small SEMESTRAIS} & \\ \hline
% codigo da disciplina carga horaria: teorica - pratica e total
{\bf \small \codigo} & {\bf \small \disciplina } & {\bf \creditosT} & {\bf \creditosP} & {\bf 108} & Presencial\\ \hline
\end{longtable}


%%%%%%%%%%%%%%%%%%%%%%%%%%%%%%%%%%%%%%%%%%%%%%%%%%%%%%%%%%%%%%
\begin{longtable}{|C{0.12\textwidth}|L{0.736\textwidth}|C{0.12\textwidth}|} \hline
%
\multicolumn{3}{|l|}{{\bf II. PRÉ-REQUISITO(S)}} \\ \hline
%
CÓDIGO & NOME DA DISCIPLINA \\ \hline	
%
\requisitoA
\requisitoB
\requisitoC
\end{longtable}


%%%%%%%%%%%%%%%%%%%%%%%%%%%%%%%%%%%%%%%%%%%%%%%%%%%%%%%%%%%%%%
\begin{longtable}{|L{1.025\textwidth}|} \hline
%
{\bf III. CURSO(S) PARA O(S) QUAL(IS) A DISCIPLINA É OFERECIDA } \\ \hline
%
\cursoA 
\cursoB
\cursoC

\end{longtable}

%%%%%%%%%%%%%%%%%%%%%%%%%%%%%%%%%%%%%%%%%%%%%%%%%%%%%%%%%%%%%%
\begin{longtable}{|L{1.025\textwidth}|} \hline
%
{\bf IV. EMENTA } \\ \hline
%
\ementa
\end{longtable}

\newpage



%%%%%%%%%%%%%%%%%%%%%%%%%%%%%%%%%%%%%%%%%%%%%%%%%%%%%%%%%%%%%%%
\begin{longtable}{|L{1.025\textwidth}|} \hline
%
{\bf V. OBJETIVOS } \\ \hline
%

Objetivos Gerais: O aluno ao final desta disciplina deverá ser capaz de transpor um algoritmo, tal como apreendido em lógica de programação, para uma linguagem de programação sob o paradigma da programação estruturada.

\\

Objetivos Específicos:
\begin{enumerate}
\item  Domínio do Contexto Científico e Tecnológico em Linguagem de Programação.
\item Utilização de Ferramentas e Técnicas de Programação.
\item Domínio do Paradigma Entrada.
\item Processamento e Saída de Dados.

\end{enumerate}

\\ \hline
\end{longtable}


%%%%%%%%%%%%%%%%%%%%%%%%%%%%%%%%%%%%%%%%%%%%%%%%%%%%%%%%%%%%%%%
\begin{longtable}{|L{1.025\textwidth}|} \hline
%
{\bf VI. CONTEÚDO PROGRAMÁTICO } \\ \hline

UNIDADE 1: \\
Introdução ao paradigma da programação estruturada. \\
Conceituação de elementos básicos da linguagem de programação. \\
Estruturas de controle de fluxo. \\
Arquitetura de programas. \\
\\

UNIDADE 2: \\
Estruturas de dados simples. \\
Variáveis compostas. \\
Variáveis homogenias: vetores e matrizes. \\
Variáveis heterogenias. \\
\\

UNIDADE 3 \\
Funções, chamada de funções, passagem de parâmetros. \\
Ponteiros. \\
Alocação de Memória. Alocação Estática. Alocação Dinâmica. \\
Processamento de Strings. \\
Entrada e Saída de dados. Arquivos e sistemas de arquivo.\\

\\ \hline
\end{longtable} 





%%%%%%%%%%%%%%%%%%%%%%%%%%%%%%%%%%%%%%%%%%%%%%%%%%%%%%%%%%%%%%%
\begin{longtable}{|L{1.025\textwidth}|} \hline
%
{\bf VII. BIBLIOGRAFIA BÁSICA} \\ \hline
\begin{enumerate}
%
\item MIZRAHI, Victorine Viviane. Treinamento em linguagem C. 2. ed. São Paulo: PearsonPrentice Hall, c2008. xxii, 405 p. ISBN 9758576051916.
\item FORBELLONE, André Luiz Villar; EBERSPACHER, Henri Frederico. Lógica deprogramação: a construção de algoritmos e estruturas de dados. 3. ed. São Paulo:Prentice-Hall, Pearson, 2005. xii, 218 p. ISBN 8576050242. 
\item FEOFILOFF, Paulo. Algoritmos em linguagem C. Rio de Janeiro: Elsevier, c2009. xv,208 p. ISBN 9788535232493.

%

\end{enumerate}
 \\ \hline
\end{longtable}


%\newpage

%%%%%%%%%%%%%%%%%%%%%%%%%%%%%%%%%%%%%%%%%%%%%%%%%%%%%%%%%%%%%%%
\begin{longtable}{|L{1.025\textwidth}|} \hline
%
{\bf VIII. BIBLIOGRAFIA COMPLEMENTAR} \\ \hline
\begin{enumerate}

\item SEDGEWICK, Robert; WAYNE, Kevin Daniel. Algorithms. 4th ed. Upper Saddle River:Addison Wesley, c2011. xii, 955 p. ISBN 9780321573513.
\item ASCENCIO, Ana Fernanda Gomes; CAMPOS, Edilene Aparecida Veneruchi de.Fundamentos da programação de computadores: algoritmos, Pascal, C/C++ e Java. 2. ed.São Paulo: Pearson Prentice Hall, 2008. 434 p. ISBN 9788576051480.
\item FARRER, Harry et al. Algoritmos estruturados. 3. ed. Rio de Janeiro: LTC, c1999. 284 p.(Programação estruturada de computadores). ISBN 9788521611806.
\item SCHILDT, Herbert. C, completo e total. 3. ed. rev. e atual. São Paulo: Pearson Educationdo Brasil, 2006. xx,827 p. ISBN 9788534605953.
\item LOUDON, Kyle. Mastering algorithms with C. 1st ed. Sebastopol: O’Reilly, 1999. xvii,540 p. ISBN 9781565924536.


%
\end{enumerate}
 \\ \hline
\end{longtable}


\input aprovacao.tex


\end{document}
