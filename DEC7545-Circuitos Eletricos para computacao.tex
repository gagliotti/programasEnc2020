\documentclass[12pt]{article}
\usepackage[brazil]{babel}
\usepackage{graphicx,t1enc,wrapfig,amsmath,float}
\usepackage{framed,fancyhdr}
\usepackage{multirow}
\usepackage{longtable}
\usepackage{array}
\newcolumntype{L}[1]{>{\raggedright\let\newline\\\arraybackslash\hspace{0pt}}m{#1}}
\newcolumntype{C}[1]{>{\centering\let\newline\\\arraybackslash\hspace{0pt}}m{#1}}
\newcolumntype{R}[1]{>{\raggedleft\let\newline\\\arraybackslash\hspace{0pt}}m{#1}}
%%%%%%%%%%%%%
\oddsidemargin -0.5cm
\evensidemargin -0.5cm
\textwidth 17.5cm
\topmargin -1.5cm
\textheight 22cm
%%%%%%%%%%%% 

%\pagestyle{empty}

\newcommand{\semestre}{2018.2}

\newcommand{\disciplina}{CIRCUITOS ELÉTRICOS PARA COMPUTAÇÃO}
\newcommand{\codigo}{DEC7545}


%%%%%%%%%%%%%%%%%%%%%%%%%%%%%%%%%%%%%%%%%%%%%%%%%%%%%%%
%%%%%%%%%%%%% CRETIDOS
\newcommand{\creditosT}{4}
\newcommand{\creditosP}{0}

%%%%%%%%%%%%%%%%%%%%%%%%%%%%%%%%%%%%%%%%%%%%%%%%%%%%%%%
%%%%%%%%%%%%%% REQUISITOS
\newcommand{\requisitoA}{}
\newcommand{\requisitoB}{}
\newcommand{\requisitoC}{}

%%%%%%%%%%%%%%%%%%%%%%%%%%%%%%%%%%%%%%%%%%%%%%%%%%%%%%%
%%%%%%%%%%%%%%% Atende aos Cursos
\newcommand{\cursoA}{Graduação em Engenharia de Computação \\ \hline}
\newcommand{\cursoB}{}%Graduação em Tecnologias da Informação e Comunicação \\ \hline}
\newcommand{\cursoC}{}

%%%%%%%%%%%%%%%%%%%%%%%%%%%%%%%%%%%%%%%%%%%%%%%%%%%%%%%%
%%%%%%%%%% EMENTA
\newcommand{\ementa}{
Conceitos básicos, unidades, leis fundamentais; resistência; fontes ideais independentes e dependentes em redes resistivas; amplificador operacional ideal; técnicas de análise de circuitos em corrente contínua, indutância e capacitância; resposta de circuitos RL e RC de primeira ordem; respostas natural e a um degrau de circuitos RLC; circuitos de corrente alternada; introdução a eletrônica; diodos; transistor de efeito de campo; transistor de junção bipolar;
\\ \hline
}


\begin{document}


%%%%%%%%%%%%%%%%%%%%%%%%%%%%%%%%%%%%%%%%%%%%%%%%%%%%%%%%%%%%%
\input cabecalho.tex



%%%%%%%%%%%%%%%%%%%%%%%%%%%%%%%%%%%%%%%%%%%%%%%%%%%%%%%%%%%%%
\begin{longtable}{|C{0.11\textwidth}|C{0.29\textwidth}|C{0.09\textwidth}|C{0.09\textwidth}|C{0.15\textwidth}|C{0.158\textwidth}|} \hline
%
\multicolumn{6}{|l|}{{\bf I. IDENTIFICAÇÃO DA DISCIPLINA}} \\ \hline
%
\multirow{3}*{{\small CÓDIGO}} & \multirow{3}*{NOME DA DISCIPLINA} &\multicolumn{2}{c|}{{\small N$^\circ$ DE HORAS-AULA }} & {{\small TOTAL DE}} & \multirow{3}*{{\small MODALIDADE}} \\ 
%
& & \multicolumn{2}{c|}{\small SEMANAIS}  & {\small HORAS-AULA} & \\ \cline{3-4}
%
& & {\tiny TEÓRICAS} & {\tiny PRÁTICAS} & {\small SEMESTRAIS} & \\ \hline
% codigo da disciplina carga horaria: teorica - pratica e total
{\bf \small \codigo} & {\bf \small \disciplina } & {\bf \creditosT} & {\bf \creditosP} & {\bf 72} & Presencial\\ \hline
\end{longtable}


%%%%%%%%%%%%%%%%%%%%%%%%%%%%%%%%%%%%%%%%%%%%%%%%%%%%%%%%%%%%%%
\begin{longtable}{|C{0.12\textwidth}|L{0.736\textwidth}|C{0.12\textwidth}|} \hline
%
\multicolumn{3}{|l|}{{\bf II. PRÉ-REQUISITO(S)}} \\ \hline
%
CÓDIGO & NOME DA DISCIPLINA & CURSO \\ \hline	
%
\requisitoA
\requisitoB
\requisitoC
\end{longtable}


%%%%%%%%%%%%%%%%%%%%%%%%%%%%%%%%%%%%%%%%%%%%%%%%%%%%%%%%%%%%%%
\begin{longtable}{|L{1.025\textwidth}|} \hline
%
{\bf III. CURSO(S) PARA O(S) QUAL(IS) A DISCIPLINA É OFERECIDA } \\ \hline
%
\cursoA 
\cursoB
\cursoC

\end{longtable}

%%%%%%%%%%%%%%%%%%%%%%%%%%%%%%%%%%%%%%%%%%%%%%%%%%%%%%%%%%%%%%
\begin{longtable}{|L{1.025\textwidth}|} \hline
%
{\bf IV. EMENTA } \\ \hline
%
\ementa
\end{longtable}

\newpage



%%%%%%%%%%%%%%%%%%%%%%%%%%%%%%%%%%%%%%%%%%%%%%%%%%%%%%%%%%%%%%%
\begin{longtable}{|L{1.025\textwidth}|} \hline
%
{\bf V. OBJETIVOS } \\ \hline
%
Objetivos Gerais: \\
Esta disciplina deverá abordar aspectos teóricos em circuitos elétricos com enfoque para eletrônica de maneira a cumprir com o perfil do egresso, como também dar ênfase a realização de circuitos através de projetos realizados extraclasse em ambiente de laboratório.
\\
Objetivos Específicos:
\begin{itemize}
\item Introduzir conceitos básicos de circuitos elétricos;
\item Discutir o conceito de fontes ideais independentes e dependentes em redes resistivas;
\item Discutir o conceito de amplificador operacional ideal;
\item Discutir técnicas de análise e características de circuitos em corrente contínua;
\item Discutir técnicas de análise e características de circuitos de corrente alternada;
\item Discutir dispositivos eletrônicos como diodo, transistor de efeito de campo e junção bipolar.
\end{itemize}

\\ \hline
\end{longtable}


%%%%%%%%%%%%%%%%%%%%%%%%%%%%%%%%%%%%%%%%%%%%%%%%%%%%%%%%%%%%%%%
\begin{longtable}{|L{1.025\textwidth}|} \hline
%
{\bf VI. CONTEÚDO PROGRAMÁTICO } \\ \hline

Elementos de Circuitos\\
Circuitos Resistivos Simples\\
Técnicas de análise de circuitos\\
Indutância e Capacitância\\
Resposta de Circuitos RL e RC de primeira ordem\\
Respostas Natural e a um degrau de circuitos RLC\\
Análise do Regime permanente senoidal\\
Amplificadores operacionais\\
Diodos\\
Transistor de junção bipolar\\
Transistor de efeito de campo\\
\\ \hline
\end{longtable} 



\newpage

%%%%%%%%%%%%%%%%%%%%%%%%%%%%%%%%%%%%%%%%%%%%%%%%%%%%%%%%%%%%%%%
\begin{longtable}{|L{1.025\textwidth}|} \hline
%
{\bf VII. BIBLIOGRAFIA BÁSICA} \\ \hline
\begin{enumerate}
%
\item THOMAS, Roland E.; ROSA, Albert J.; TOUSSAINT, Gregory J. Análise e projeto de circuitos elétricos lineares. 6th ed. Porto Alegre: Bookman, 2011. xii, 816 p. ISBN 9788577807876. 
\item NILSSON, James William; RIEDEL, Susan A. Circuitos elétricos. 6. ed Rio de Janeiro (RJ): LTC, c2003. 656p. 
\item Sedra; Smith, ''Microeletrônica'', Pearson, 2007.
%

\end{enumerate}
 \\ \hline
\end{longtable}


%\newpage

%%%%%%%%%%%%%%%%%%%%%%%%%%%%%%%%%%%%%%%%%%%%%%%%%%%%%%%%%%%%%%%
\begin{longtable}{|L{1.025\textwidth}|} \hline
%
{\bf VIII. BIBLIOGRAFIA COMPLEMENTAR} \\ \hline
\begin{enumerate}

\item ALEXANDER, CHARLES K.; SADIKU, MATTHEW - FUNDAMENTOS DE CIRCUITOS ELETRICOS - MCGRAW HILL - ARTMED, 2008, ISBN: 8586804975, ISBN-13: 9788586804977 
\item EDMINISTER, Joseph A. Circuitos elétricos: reedição da edição clássica. São Paulo: Makron: McGraw-Hill, c1991. 585p. 
\item JOHNSON, D.E, J.L. Hilburn, J.R. Johnson, Fundamentos de análise de circuitos elétricos, 4ª Ed., Editora Prentice-Hall do Brasil, 1994. 
\item RAZAVI, BEHZAD, - FUNDAMENTOS DE MICROELETRONICA - LTC, 2010, ISBN: 8521617321, ISBN-13: 9788521617327 
\item DORF, RICHARD; SVOBODA, JAMES A. - INTRODUÇAO AOS CIRCUITOS ELETRICOS - LTC, 2008, ISBN: 8521615825, ISBN-13: 9788521615828 

%
\end{enumerate}
 \\ \hline
\end{longtable}


\input aprovacao.tex


\end{document}
