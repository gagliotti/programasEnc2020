\documentclass[12pt]{article}
\usepackage[brazil]{babel}
\usepackage{graphicx,t1enc,wrapfig,amsmath,float}
\usepackage{framed,fancyhdr}
\usepackage{multirow}
\usepackage{longtable}
\usepackage{array}
\newcolumntype{L}[1]{>{\raggedright\let\newline\\\arraybackslash\hspace{0pt}}m{#1}}
\newcolumntype{C}[1]{>{\centering\let\newline\\\arraybackslash\hspace{0pt}}m{#1}}
\newcolumntype{R}[1]{>{\raggedleft\let\newline\\\arraybackslash\hspace{0pt}}m{#1}}
%%%%%%%%%%%%%
\oddsidemargin -0.5cm
\evensidemargin -0.5cm
\textwidth 17.5cm
\topmargin -1.5cm
\textheight 22cm
%%%%%%%%%%%% 

%\pagestyle{empty}

\newcommand{\semestre}{2018.2}

\newcommand{\disciplina}{SISTEMAS DISTRIBUÍDOS}
\newcommand{\codigo}{DEC7558}


%%%%%%%%%%%%%%%%%%%%%%%%%%%%%%%%%%%%%%%%%%%%%%%%%%%%%%%
%%%%%%%%%%%%% CRETIDOS
\newcommand{\creditosT}{4}
\newcommand{\creditosP}{0}

%%%%%%%%%%%%%%%%%%%%%%%%%%%%%%%%%%%%%%%%%%%%%%%%%%%%%%%
%%%%%%%%%%%%%% REQUISITOS
\newcommand{\requisitoA}{}
\newcommand{\requisitoB}{}
\newcommand{\requisitoC}{}

%%%%%%%%%%%%%%%%%%%%%%%%%%%%%%%%%%%%%%%%%%%%%%%%%%%%%%%
%%%%%%%%%%%%%%% Atende aos Cursos
\newcommand{\cursoA}{Graduação em Engenharia de Computação \\ \hline}
\newcommand{\cursoB}{}%Graduação em Tecnologias da Informação e Comunicação \\ \hline}
\newcommand{\cursoC}{}

%%%%%%%%%%%%%%%%%%%%%%%%%%%%%%%%%%%%%%%%%%%%%%%%%%%%%%%%
%%%%%%%%%% EMENTA
\newcommand{\ementa}{
Fundamentos de Sistemas Distribuídos: Arquitetura de Sistemas Distribuídos, Comunicação entre Processos, Comunicação em Grupo, Objetos Distribuídos. Sistemas Par-a-Par, Sincronização: relógios físicos, relógios lógicos e estados globais. Coordenação, Exclusão Mútua Distribuída. Transação Distribuída, Detecção e Prevensão de Deadlock Distribuído, Tolerância à Falta.

\\ \hline
}


\begin{document}


%%%%%%%%%%%%%%%%%%%%%%%%%%%%%%%%%%%%%%%%%%%%%%%%%%%%%%%%%%%%%
\input cabecalho.tex

%%%%%%%%%%%%%%%%%%%%%%%%%%%%%%%%%%%%%%%%%%%%%%%%%%%%%%%%%%%%%
\begin{longtable}{|C{0.11\textwidth}|C{0.29\textwidth}|C{0.09\textwidth}|C{0.09\textwidth}|C{0.15\textwidth}|C{0.158\textwidth}|} \hline
%
\multicolumn{6}{|l|}{{\bf I. IDENTIFICAÇÃO DA DISCIPLINA}} \\ \hline
%
\multirow{3}*{{\small CÓDIGO}} & \multirow{3}*{NOME DA DISCIPLINA} &\multicolumn{2}{c|}{{\small N$^\circ$ DE HORAS-AULA }} & {{\small TOTAL DE}} & \multirow{3}*{{\small MODALIDADE}} \\ 
%
& & \multicolumn{2}{c|}{\small SEMANAIS}  & {\small HORAS-AULA} & \\ \cline{3-4}
%
& & {\tiny TEÓRICAS} & {\tiny PRÁTICAS} & {\small SEMESTRAIS} & \\ \hline
% codigo da disciplina carga horaria: teorica - pratica e total
{\bf \small \codigo} & {\bf \small \disciplina } & {\bf \creditosT} & {\bf \creditosP} & {\bf 72} & Presencial\\ \hline
\end{longtable}


%%%%%%%%%%%%%%%%%%%%%%%%%%%%%%%%%%%%%%%%%%%%%%%%%%%%%%%%%%%%%%
\begin{longtable}{|C{0.12\textwidth}|L{0.736\textwidth}|C{0.12\textwidth}|} \hline
%
\multicolumn{3}{|l|}{{\bf II. PRÉ-REQUISITO(S)}} \\ \hline
%
CÓDIGO & NOME DA DISCIPLINA & CURSO \\ \hline	
%
\requisitoA
\requisitoB
\requisitoC
\end{longtable}


%%%%%%%%%%%%%%%%%%%%%%%%%%%%%%%%%%%%%%%%%%%%%%%%%%%%%%%%%%%%%%
\begin{longtable}{|L{1.025\textwidth}|} \hline
%
{\bf III. CURSO(S) PARA O(S) QUAL(IS) A DISCIPLINA É OFERECIDA } \\ \hline
%
\cursoA 
\cursoB
\cursoC

\end{longtable}

%%%%%%%%%%%%%%%%%%%%%%%%%%%%%%%%%%%%%%%%%%%%%%%%%%%%%%%%%%%%%%
\begin{longtable}{|L{1.025\textwidth}|} \hline
%
{\bf IV. EMENTA } \\ \hline
%
\ementa
\end{longtable}

%\newpage



%%%%%%%%%%%%%%%%%%%%%%%%%%%%%%%%%%%%%%%%%%%%%%%%%%%%%%%%%%%%%%%
\begin{longtable}{|L{1.025\textwidth}|} \hline
%
{\bf V. OBJETIVOS } \\ \hline

Objetivo Geral: \\

Habilitar o aluno a projetar e desenvolver sistemas computacionais de natureza distribuída, bem como reconhecer as principais características e algoritmos em um sistema distribuído.\\
\\
Objetivos Específicos:
\begin{itemize}
\item Familiarizar o aluno com o modelo distribuído de computação;
\item Apresentar os principais conceitos envolvidos no projeto e no desenvolvimento de sistemas distribuídos;
\item Capacitar o aluno a utilizar ferramentas para o desenvolvimento de algoritmos e sistemas distribuídos.
\end{itemize}
\\ \hline
\end{longtable}


%%%%%%%%%%%%%%%%%%%%%%%%%%%%%%%%%%%%%%%%%%%%%%%%%%%%%%%%%%%%%%%
\begin{longtable}{|L{1.025\textwidth}|} \hline
%
{\bf VI. CONTEÚDO PROGRAMÁTICO } \\ \hline

Conteúdo Teórico seguido de Conteúdo Prático com desenvolvimento de problemas em computador: \\
\\
UNIDADE 1: Introdução [4 horas-aula]\\
Conceitos de sistemas distribuídos\\
Comunicação em redes de computadores\\
Computação cliente-servidor\\
Definição de sistemas distribuídos\\
Tipos de sistemas distribuídos\\
Exemplos de sistemas distribuídos\\
\\
UNIDADE 2: Processos em Sistemas Distribuídos [8 horas-aula]\\
Processos e threads\\
Processos cliente-servidor\\
Virtualização\\
Migração de código\\
\\
UNIDADE 3: Comunicação entre processos distribuídos [22 horas-aula]\\
Protocolos de rede em camadas\\
Comunicação cliente-servidor\\
Sockets\
Chamada remota de procedimento\\
Invocação remota de método\\
Comunicação em grupo\\
Comunicação par a par\\
\\
UNIDADE 4:  Concorrência e sincronização [18 horas-aula]\\
Sincronização de relógios\\
Algoritmos para exclusão mútua\\
Algoritmos de eleição\\
Algoritmos de acordo\\
Transações distribuídas\\
\\
UNIDADE 5: Sistema de arquivos distribuídos [8 horas-aula]\\
Arquiteturas\\
Nomeação\\
Sincronização\\
Consistência e replicação\\
Memória Compartilhada Distribuída  \\
\\
UNIDADE 6: Tolerância a Faltas [4 horas-aula]\\
Definição\\
Segurança de Funcionamento \\
Classificação e Semântica de Faltas\\
Fases da Tolerância a Faltas\\
Técnicas de Replicação \\
\\
UNIDADE 7: Estudos de caso de sistemas distribuídos [8 horas-aula]\\
Computação em Grid/Cluster\\
Web Services/DPWS\\
Computação em nuvem\\
Internet of Things\\
Deep Web\\
Docker/Kubernetes\\

\\ \hline
\end{longtable} 



%\newpage

%%%%%%%%%%%%%%%%%%%%%%%%%%%%%%%%%%%%%%%%%%%%%%%%%%%%%%%%%%%%%%%
\begin{longtable}{|L{1.025\textwidth}|} \hline
%
{\bf VII. BIBLIOGRAFIA BÁSICA} \\ \hline
\begin{enumerate}
%
\item COULOURIS, George; DOLLIMORE, Jean; KINDBERG, Tim. Sistemas Distribuídos conceitos e projetos. 4a. Ed. Editora Bookman, 2007. 
\item STEVENS, W. Richard; FENNER, Bill; RUDOFF, Andrew M.. Programação de Rede UNIX. API para sockets de rede. 3a. Ed. Editora Artmed, 2005. 
\item TANENBAUM, Andrew S.; Maarten Van Steen. Sistemas Distribuídos: princípios e paradigmas. 2a. Ed. Editora Pearson, 2007.
\end{enumerate}
 \\ \hline
\end{longtable}


\newpage

%%%%%%%%%%%%%%%%%%%%%%%%%%%%%%%%%%%%%%%%%%%%%%%%%%%%%%%%%%%%%%%
\begin{longtable}{|L{1.025\textwidth}|} \hline
%
{\bf VIII. BIBLIOGRAFIA COMPLEMENTAR} \\ \hline
\begin{enumerate}
\item DANTAS, Mário. Computação Distribuída de Alto Desempenho. Axcel Books, 2005. 
\item DEITEL, H. M.; DEITEL, P.J. Java: como programar. 6. ed. São Paulo: Pearson, 2005. 
\item KSHEMKALYAN, Ajay D., SINGHAL, Mukesh. Distributed Computing: Principles, Algorithms, and Systems. Cambridge University Press, 2011.
\item BARNES, David J; KÖLLING, Michael. Programação orientada a objetos com Java. São Paulo: Pearson, Prentice-Hall, c2004. xviii, 368 p. ISBN 8576050129.
\item BORATTI, Isaias Camilo. Programação orientada a objetos em JAVA. Florianópolis: Visual Books, 2007. 310p. ISBN 9788575021996.
%
\end{enumerate}
 \\ \hline
\end{longtable}


\input aprovacao.tex


\end{document}
