\documentclass[12pt]{article}
\usepackage[brazil]{babel}
\usepackage{graphicx,t1enc,wrapfig,amsmath,float}
\usepackage{framed,fancyhdr}
\usepackage{multirow}
\usepackage{longtable}
\usepackage{array}
\newcolumntype{L}[1]{>{\raggedright\let\newline\\\arraybackslash\hspace{0pt}}m{#1}}
\newcolumntype{C}[1]{>{\centering\let\newline\\\arraybackslash\hspace{0pt}}m{#1}}
\newcolumntype{R}[1]{>{\raggedleft\let\newline\\\arraybackslash\hspace{0pt}}m{#1}}
%%%%%%%%%%%%%
\oddsidemargin -0.5cm
\evensidemargin -0.5cm
\textwidth 17.5cm
\topmargin -1.5cm
\textheight 22cm
%%%%%%%%%%%% 

%\pagestyle{empty}

\newcommand{\semestre}{2018.2}

\newcommand{\disciplina}{REDES DE COMPUTADORES}
\newcommand{\codigo}{DEC7557}


%%%%%%%%%%%%%%%%%%%%%%%%%%%%%%%%%%%%%%%%%%%%%%%%%%%%%%%
%%%%%%%%%%%%% CRETIDOS
\newcommand{\creditosT}{3}
\newcommand{\creditosP}{1}

%%%%%%%%%%%%%%%%%%%%%%%%%%%%%%%%%%%%%%%%%%%%%%%%%%%%%%%
%%%%%%%%%%%%%% REQUISITOS
\newcommand{\requisitoA}{}
\newcommand{\requisitoB}{}
\newcommand{\requisitoC}{}

%%%%%%%%%%%%%%%%%%%%%%%%%%%%%%%%%%%%%%%%%%%%%%%%%%%%%%%
%%%%%%%%%%%%%%% Atende aos Cursos
\newcommand{\cursoA}{Graduação em Engenharia de Computação \\ \hline}
\newcommand{\cursoB}{}%Graduação em Tecnologias da Informação e Comunicação \\ \hline}
\newcommand{\cursoC}{}

%%%%%%%%%%%%%%%%%%%%%%%%%%%%%%%%%%%%%%%%%%%%%%%%%%%%%%%%
%%%%%%%%%% EMENTA
\newcommand{\ementa}{
Introdução e principais conceitos. Modelos de referência (OSI e TCP/IP). A camada de aplicação: principais protocolos e aplicações. Camada de transporte: o protocolo UDP, TCP e controle de congestionamento. Camada de rede: o protocolo IP e algoritmos de roteamento.
Camada de enlace: principais protocolos e padrões IEEE. Introdução à administração de Redes de Computadores. Redes na Automação Industrial.

\\ \hline
}


\begin{document}


%%%%%%%%%%%%%%%%%%%%%%%%%%%%%%%%%%%%%%%%%%%%%%%%%%%%%%%%%%%%%
\input cabecalho.tex


%%%%%%%%%%%%%%%%%%%%%%%%%%%%%%%%%%%%%%%%%%%%%%%%%%%%%%%%%%%%%
\begin{longtable}{|C{0.11\textwidth}|C{0.29\textwidth}|C{0.09\textwidth}|C{0.09\textwidth}|C{0.15\textwidth}|C{0.158\textwidth}|} \hline
%
\multicolumn{6}{|l|}{{\bf I. IDENTIFICAÇÃO DA DISCIPLINA}} \\ \hline
%
\multirow{3}*{{\small CÓDIGO}} & \multirow{3}*{NOME DA DISCIPLINA} &\multicolumn{2}{c|}{{\small N$^\circ$ DE HORAS-AULA }} & {{\small TOTAL DE}} & \multirow{3}*{{\small MODALIDADE}} \\ 
%
& & \multicolumn{2}{c|}{\small SEMANAIS}  & {\small HORAS-AULA} & \\ \cline{3-4}
%
& & {\tiny TEÓRICAS} & {\tiny PRÁTICAS} & {\small SEMESTRAIS} & \\ \hline
% codigo da disciplina carga horaria: teorica - pratica e total
{\bf \small \codigo} & {\bf \small \disciplina } & {\bf \creditosT} & {\bf \creditosP} & {\bf 72} & Presencial\\ \hline
\end{longtable}


%%%%%%%%%%%%%%%%%%%%%%%%%%%%%%%%%%%%%%%%%%%%%%%%%%%%%%%%%%%%%%
\begin{longtable}{|C{0.12\textwidth}|L{0.736\textwidth}|C{0.12\textwidth}|} \hline
%
\multicolumn{3}{|l|}{{\bf II. PRÉ-REQUISITO(S)}} \\ \hline
%
CÓDIGO & NOME DA DISCIPLINA & CURSO \\ \hline	
FQM7107 & Probabilidade e Estatística & ENC \\ \hline
%
\requisitoA
\requisitoB
\requisitoC
\end{longtable}


%%%%%%%%%%%%%%%%%%%%%%%%%%%%%%%%%%%%%%%%%%%%%%%%%%%%%%%%%%%%%%
\begin{longtable}{|L{1.025\textwidth}|} \hline
%
{\bf III. CURSO(S) PARA O(S) QUAL(IS) A DISCIPLINA É OFERECIDA } \\ \hline
%
\cursoA 
\cursoB
\cursoC

\end{longtable}

%%%%%%%%%%%%%%%%%%%%%%%%%%%%%%%%%%%%%%%%%%%%%%%%%%%%%%%%%%%%%%
\begin{longtable}{|L{1.025\textwidth}|} \hline
%
{\bf IV. EMENTA } \\ \hline
%
\ementa
\end{longtable}

\newpage



%%%%%%%%%%%%%%%%%%%%%%%%%%%%%%%%%%%%%%%%%%%%%%%%%%%%%%%%%%%%%%%
\begin{longtable}{|L{1.025\textwidth}|} \hline
%
{\bf V. OBJETIVOS } \\ \hline
Objetivos Gerais: \\

O objetivo principal desta disciplina é apresentar os conceitos relacionados às arquiteturas, serviços e protocolos das Redes de Computadores.\\
\\
Objetivos Específicos:
\begin{itemize}
\item Descrever os principais aspectos de operação dos protocolos dos diferentes níveis da Arquitetura Internet.
\item Apresentar a política de endereçamento da Internet.
\item Apresentar os conceitos de gerenciamento na Internet e os protocolos associados.
\item Apresentar as principais tecnologias de redes locais sem fio.
\item Apresentar, analisar e usar tecnologias e suporte para Gerência de Redes.
\item Apresentar os principais conceitos de segurança em Redes.
\item Pesquisar sobre Tendências e Futuro em Gerência de Redes e Redes na Automação.
\end{itemize}
\\ \hline
\end{longtable}


%%%%%%%%%%%%%%%%%%%%%%%%%%%%%%%%%%%%%%%%%%%%%%%%%%%%%%%%%%%%%%%
\begin{longtable}{|L{1.025\textwidth}|} \hline
%
{\bf VI. CONTEÚDO PROGRAMÁTICO } \\ \hline
Unidade 1: Introdução às Redes de Computadores [6 horas-aula]\\
Conceitos Gerais	\\
Medidas de Desempenho\\
Camadas de protocolos e serviços\\
Topologias de redes\\
Histórico das redes de computadores e Internet\\
\\
Unidade 2: Camada de Aplicação [8 horas-aula]\\
Fundamentos das aplicações de rede\\
Principais protocolos da camada de aplicação (HTTP, FTP, SMTP)\\
Serviço de diretório da Internet (DNS)\\
\\
Unidade 3: Camada de Transporte [16 horas-aula]\\
Introdução e Serviços da camada de transporte\\
Protocolos TCP e UDP\\
Princípios do controle de congestionamento\\
\\
Unidade 4: Camada de Rede [20 horas-aula]\\
Introdução\\
Endereçamento IP\\
O protocolo IP\\
Alocação dinâmica de IPs\\
Tradução e Mapeamento de IPs\\
Roteamento na Internet\\
IPv6\\
\\
Unidade 5: Camada de enlace e redes locais [8 horas-aula]\\
Serviços oferecidos pela camada de enlace\\
Protocolos de acesso múltiplo\\
Endereçamento na camada de enlace\\
Redes Ethernet\\
\\
Unidade 6: Gerenciamento de Rede e Automação [14 horas-aula] \\
Noções de Gerenciamento de Redes\\
Gerenciando a Rede Internet.\\
Ferramentas de Gerenciamento para Internet \\
Redes Aplicadas à Automação \\
\\ \hline
\end{longtable} 



%\newpage

%%%%%%%%%%%%%%%%%%%%%%%%%%%%%%%%%%%%%%%%%%%%%%%%%%%%%%%%%%%%%%%
\begin{longtable}{|L{1.025\textwidth}|} \hline
%
{\bf VII. BIBLIOGRAFIA BÁSICA} \\ \hline
\begin{enumerate}
%
\item  KUROSE, James F.; ROSS, Keith W. Redes de computadores e a Internet: uma abordagem top-down. 5. ed. São Paulo: Pearson Addison Wesley, 2010. xxiii, 614 p. ISBN 9788588639973.
\item TANENBAUM, A.S., WETHERALL, D. J. Redes de Computadores, tradução da 5a Edição, Editora Prentice Hall Brasil, 2011. 
\item  TORRES, G. Redes de computadores. 2. ed. rev. e atual. Rio de Janeiro: Novaterra,
c2014. xxviii, 1005 p. ISBN 9788561893286.
\end{enumerate}
 \\ \hline
\end{longtable}


\newpage

%%%%%%%%%%%%%%%%%%%%%%%%%%%%%%%%%%%%%%%%%%%%%%%%%%%%%%%%%%%%%%%
\begin{longtable}{|L{1.025\textwidth}|} \hline
%
{\bf VIII. BIBLIOGRAFIA COMPLEMENTAR} \\ \hline
\begin{enumerate}
\item CARISSIMI, A. S.; ROCHOL, J.; GRANVILLE, L. Z. Redes de Computadores. Porto Alegre: Bookman, 2009. 
\item MARIN, Paulo S. Cabeamento estruturado: desvendando cada passo : do projeto à instalação. 4. ed. rev. e atual. São Paulo: Érica, 2014. 336 p. ISBN 9788536502076.
\item  STALLINGS, W. Redes e Sistemas de Comunicação de Dados, Rio de Janeiro: Elsevier. 5a. Edicao, 2005.
\item TRONCO, Tania Regina. Redes da nova geração: arquitetura de convergência das redes :
IP, telefônica e óptica. 2. ed. rev. e atual. São Paulo: Érica, 2014. 164 p. ISBN
9788536501383.
\item FOROUZAN, Behrouz A.; FEGAN, Sophia Chung; GRIESI, Ariovaldo. Comunicação
de dados e redes de computadores. 4. ed. São Paulo: McGraw Hill, 2008. 1134 p. ISBN
9788586804885.


%
\end{enumerate}
 \\ \hline
\end{longtable}


\input aprovacao.tex


\end{document}
