\documentclass[12pt]{article}
\usepackage[brazil]{babel}
\usepackage{graphicx,t1enc,wrapfig,amsmath,float}
\usepackage{framed,fancyhdr}
\usepackage{multirow}
\usepackage{longtable}
\usepackage{array}
\newcolumntype{L}[1]{>{\raggedright\let\newline\\\arraybackslash\hspace{0pt}}m{#1}}
\newcolumntype{C}[1]{>{\centering\let\newline\\\arraybackslash\hspace{0pt}}m{#1}}
\newcolumntype{R}[1]{>{\raggedleft\let\newline\\\arraybackslash\hspace{0pt}}m{#1}}
%%%%%%%%%%%%%
\oddsidemargin -0.5cm
\evensidemargin -0.5cm
\textwidth 17.5cm
\topmargin -1.5cm
\textheight 22cm
%%%%%%%%%%%% 

%\pagestyle{empty}

\newcommand{\semestre}{2018.2}

\newcommand{\disciplina}{REDES DE COMPUTADORES}
\newcommand{\codigo}{DEC7557}


%%%%%%%%%%%%%%%%%%%%%%%%%%%%%%%%%%%%%%%%%%%%%%%%%%%%%%%
%%%%%%%%%%%%% CRETIDOS
\newcommand{\creditosT}{3}
\newcommand{\creditosP}{1}

%%%%%%%%%%%%%%%%%%%%%%%%%%%%%%%%%%%%%%%%%%%%%%%%%%%%%%%
%%%%%%%%%%%%%% REQUISITOS
\newcommand{\requisitoA}{}
\newcommand{\requisitoB}{}
\newcommand{\requisitoC}{}

%%%%%%%%%%%%%%%%%%%%%%%%%%%%%%%%%%%%%%%%%%%%%%%%%%%%%%%
%%%%%%%%%%%%%%% Atende aos Cursos
\newcommand{\cursoA}{Graduação em Engenharia de Computação \\ \hline}
\newcommand{\cursoB}{}%Graduação em Tecnologias da Informação e Comunicação \\ \hline}
\newcommand{\cursoC}{}

%%%%%%%%%%%%%%%%%%%%%%%%%%%%%%%%%%%%%%%%%%%%%%%%%%%%%%%%
%%%%%%%%%% EMENTA
\newcommand{\ementa}{
Introdução e principais conceitos. Modelos de referência (OSI e TCP/IP). Camadas de aplicação, transporte, rede
e enlace. Roteamento. Administração de Redes de Computadores. Introdução à Segurança em Redes.
\\ \hline
}


\begin{document}


%%%%%%%%%%%%%%%%%%%%%%%%%%%%%%%%%%%%%%%%%%%%%%%%%%%%%%%%%%%%%
\input cabecalho.tex


%%%%%%%%%%%%%%%%%%%%%%%%%%%%%%%%%%%%%%%%%%%%%%%%%%%%%%%%%%%%%
\begin{longtable}{|C{0.11\textwidth}|C{0.29\textwidth}|C{0.09\textwidth}|C{0.09\textwidth}|C{0.15\textwidth}|C{0.158\textwidth}|} \hline
%
\multicolumn{6}{|l|}{{\bf I. IDENTIFICAÇÃO DA DISCIPLINA}} \\ \hline
%
\multirow{3}*{{\small CÓDIGO}} & \multirow{3}*{NOME DA DISCIPLINA} &\multicolumn{2}{c|}{{\small N$^\circ$ DE HORAS-AULA }} & {{\small TOTAL DE}} & \multirow{3}*{{\small MODALIDADE}} \\ 
%
& & \multicolumn{2}{c|}{\small SEMANAIS}  & {\small HORAS-AULA} & \\ \cline{3-4}
%
& & {\tiny TEÓRICAS} & {\tiny PRÁTICAS} & {\small SEMESTRAIS} & \\ \hline
% codigo da disciplina carga horaria: teorica - pratica e total
{\bf \small \codigo} & {\bf \small \disciplina } & {\bf \creditosT} & {\bf \creditosP} & {\bf 72} & Presencial\\ \hline
\end{longtable}


%%%%%%%%%%%%%%%%%%%%%%%%%%%%%%%%%%%%%%%%%%%%%%%%%%%%%%%%%%%%%%
\begin{longtable}{|C{0.12\textwidth}|L{0.736\textwidth}|C{0.12\textwidth}|} \hline
%
\multicolumn{3}{|l|}{{\bf II. PRÉ-REQUISITO(S)}} \\ \hline
%
CÓDIGO & NOME DA DISCIPLINA & CURSO \\ \hline	
%
\requisitoA
\requisitoB
\requisitoC
\end{longtable}


%%%%%%%%%%%%%%%%%%%%%%%%%%%%%%%%%%%%%%%%%%%%%%%%%%%%%%%%%%%%%%
\begin{longtable}{|L{1.025\textwidth}|} \hline
%
{\bf III. CURSO(S) PARA O(S) QUAL(IS) A DISCIPLINA É OFERECIDA } \\ \hline
%
\cursoA 
\cursoB
\cursoC

\end{longtable}

%%%%%%%%%%%%%%%%%%%%%%%%%%%%%%%%%%%%%%%%%%%%%%%%%%%%%%%%%%%%%%
\begin{longtable}{|L{1.025\textwidth}|} \hline
%
{\bf IV. EMENTA } \\ \hline
%
\ementa
\end{longtable}

\newpage



%%%%%%%%%%%%%%%%%%%%%%%%%%%%%%%%%%%%%%%%%%%%%%%%%%%%%%%%%%%%%%%
\begin{longtable}{|L{1.025\textwidth}|} \hline
%
{\bf V. OBJETIVOS } \\ \hline
Objetivos Gerais: \\

O objetivo principal desta disciplina é apresentar os conceitos relacionados às arquiteturas, serviços e protocolos das Redes de Computadores.\\
\\
Objetivos Específicos:
\begin{itemize}
\item Descrever os principais aspectos de operação dos protocolos dos diferentes níveis da Arquitetura Internet.
\item Apresentar a política de endereçamento da Internet.
\item Apresentar os conceitos de gerenciamento na Internet e os protocolos associados.
\item Apresentar as principais tecnologias de redes locais sem fio.
\item Apresentar, analisar e usar tecnologias e suporte para Gerência de Redes.
\item Apresentar os principais conceitos de segurança em Redes.
\item Pesquisar sobre Tendências e Futuro em Administração e Gerência de Redes.
\end{itemize}
\\ \hline
\end{longtable}


%%%%%%%%%%%%%%%%%%%%%%%%%%%%%%%%%%%%%%%%%%%%%%%%%%%%%%%%%%%%%%%
\begin{longtable}{|L{1.025\textwidth}|} \hline
%
{\bf VI. CONTEÚDO PROGRAMÁTICO } \\ \hline
Unidade 1: Introdução às Redes de Computadores [6 horas-aula]\\
Conceitos Gerais	\\
Medidas de Desempenho\\
Camadas de protocolos e serviços\\
Histórico das redes de computadores e Internet\\
Topologias de redes\\
\\
Unidade 2: Camada de Aplicação [8 horas-aula]\\
Fundamentos das aplicações de rede\\
Principais protocolos da camada de aplicação (HTTP, FTP, SMTP)\\
Serviço de diretório da Internet (DNS)\\
\\
Unidade 3: Camada de Transporte [16 horas-aula]\\
Introdução e Serviços da camada de transporte\\
Protocolos TCP e UDP\\
Princípios do controle de congestionamento\\
\\
Unidade 4: Camada de Rede [24 horas-aula]\\
Introdução\\
Endereçamento IP\\
O protocolo IP\\
Alocação dinâmica de IPs\\
Tradução e Mapeamento de IPs\\
Roteamento na Internet\\
IPv6\\
\\
Unidade 5: Camada de enlace e redes locais [8 horas-aula]\\
Serviços oferecidos pela camada de enlace\\
Protocolos de acesso múltiplo\\
Endereçamento na camada de enlace\\
Redes Ethernet\\
\\
Unidade 6: Gerenciamento de Rede na Internet [10 horas-aula] \\
Noções de Gerenciamento de Redes\\
Gerenciando a Rede Internet.\\
Ferramentas de Gerenciamento para Internet \\
\\ \hline
\end{longtable} 



%\newpage

%%%%%%%%%%%%%%%%%%%%%%%%%%%%%%%%%%%%%%%%%%%%%%%%%%%%%%%%%%%%%%%
\begin{longtable}{|L{1.025\textwidth}|} \hline
%
{\bf VII. BIBLIOGRAFIA BÁSICA} \\ \hline
\begin{enumerate}
%
\item TANENBAUM, A.S., WETHERALL, D. J. Redes de Computadores, tradução da 5a Edição, Editora Prentice Hall Brasil, 2011. 
\item  KUROSE, James F.; ROSS, Keith W. Redes de computadores e a Internet: uma abordagem top-down. 5. ed. São Paulo: Pearson Addison Wesley, 2010. xxiii, 614 p. ISBN 9788588639973.
\item  MORAES, A. F. Redes sem Fio: Instalação, Configuração e Segurança, São Paulo: Editora Érica, 2012.

\end{enumerate}
 \\ \hline
\end{longtable}


\newpage

%%%%%%%%%%%%%%%%%%%%%%%%%%%%%%%%%%%%%%%%%%%%%%%%%%%%%%%%%%%%%%%
\begin{longtable}{|L{1.025\textwidth}|} \hline
%
{\bf VIII. BIBLIOGRAFIA COMPLEMENTAR} \\ \hline
\begin{enumerate}
\item CARISSIMI, A. S.; ROCHOL, J.; GRANVILLE, L. Z. Redes de Computadores. Porto Alegre: Bookman, 2009. 
\item KUMAR, A., MANJUNATH, D. e KURI, J., Wireless Networking. Morgan Kaufmann, 2008. 
\item  SOARES, Luiz Fernando Gomes; LEMOS, Guido; COLCHER, Sergio. Redes de Computadores: Das LANs, MANs e WANs, às Redes ATM. Rio de Janeiro: Editora Campus, 1995. 
\item  STALLINGS, W. Redes e Sistemas de Comunicação de Dados, Rio de Janeiro: Elsevier. 5a. Edicao, 2005.
\item TORRES, Gabriel. Redes de computadores. 2. ed. rev. e atual. Rio de Janeiro: Novaterra, c2014. xxviii, 1005 p.


%
\end{enumerate}
 \\ \hline
\end{longtable}


\input aprovacao.tex


\end{document}
