\documentclass[12pt]{article}
\usepackage[brazil]{babel}
\usepackage{graphicx,t1enc,wrapfig,amsmath,float}
\usepackage{framed,fancyhdr}
\usepackage{multirow}
\usepackage{longtable}
\usepackage{array}
\newcolumntype{L}[1]{>{\raggedright\let\newline\\\arraybackslash\hspace{0pt}}m{#1}}
\newcolumntype{C}[1]{>{\centering\let\newline\\\arraybackslash\hspace{0pt}}m{#1}}
\newcolumntype{R}[1]{>{\raggedleft\let\newline\\\arraybackslash\hspace{0pt}}m{#1}}
%%%%%%%%%%%%%
\oddsidemargin -0.5cm
\evensidemargin -0.5cm
\textwidth 17.5cm
\topmargin -1.5cm
\textheight 22cm
%%%%%%%%%%%% 

%\pagestyle{empty}

\newcommand{\semestre}{2018.2}

\newcommand{\disciplina}{INTELIGÊNCIA ARTIFICIAL II}
\newcommand{\codigo}{DEC7542}


%%%%%%%%%%%%%%%%%%%%%%%%%%%%%%%%%%%%%%%%%%%%%%%%%%%%%%%
%%%%%%%%%%%%% CRETIDOS
\newcommand{\creditosT}{2}
\newcommand{\creditosP}{2}

%%%%%%%%%%%%%%%%%%%%%%%%%%%%%%%%%%%%%%%%%%%%%%%%%%%%%%%
%%%%%%%%%%%%%% REQUISITOS
\newcommand{\requisitoA}{}
\newcommand{\requisitoB}{}
\newcommand{\requisitoC}{}

%%%%%%%%%%%%%%%%%%%%%%%%%%%%%%%%%%%%%%%%%%%%%%%%%%%%%%%
%%%%%%%%%%%%%%% Atende aos Cursos
\newcommand{\cursoA}{Graduação em Engenharia de Computação \\ \hline}
\newcommand{\cursoB}{}%Graduação em Tecnologias da Informação e Comunicação \\ \hline}
\newcommand{\cursoC}{}%Graduação em Engenharia de Energia \\ \hline}

%%%%%%%%%%%%%%%%%%%%%%%%%%%%%%%%%%%%%%%%%%%%%%%%%%%%%%%%
%%%%%%%%%% EMENTA
\newcommand{\ementa}{
Introdução Inteligência Computacional. Lógica Nebulosa/Fuzzy. Conjuntos nebulosos. Tratamento de Incertezas: fuzificação e defuzificação. Raciocínio e inferência em lógica nebulosa. Algoritmos Genéticos e Programação Genética. Sistemas de Colônia de Formigas. Redes Neurais Artificiais. Aprendizado não supervisionado e supervisionado.

\\ \hline
}


\begin{document}


%%%%%%%%%%%%%%%%%%%%%%%%%%%%%%%%%%%%%%%%%%%%%%%%%%%%%%%%%%%%%
\input cabecalho.tex


%%%%%%%%%%%%%%%%%%%%%%%%%%%%%%%%%%%%%%%%%%%%%%%%%%%%%%%%%%%%%
\begin{longtable}{|C{0.11\textwidth}|C{0.29\textwidth}|C{0.09\textwidth}|C{0.09\textwidth}|C{0.15\textwidth}|C{0.158\textwidth}|} \hline
%
\multicolumn{6}{|l|}{{\bf I. IDENTIFICAÇÃO DA DISCIPLINA}} \\ \hline
%
\multirow{3}*{{\small CÓDIGO}} & \multirow{3}*{NOME DA DISCIPLINA} &\multicolumn{2}{c|}{{\small N$^\circ$ DE HORAS-AULA }} & {{\small TOTAL DE}} & \multirow{3}*{{\small MODALIDADE}} \\ 
%
& & \multicolumn{2}{c|}{\small SEMANAIS}  & {\small HORAS-AULA} & \\ \cline{3-4}
%
& & {\tiny TEÓRICAS} & {\tiny PRÁTICAS} & {\small SEMESTRAIS} & \\ \hline
% codigo da disciplina carga horaria: teorica - pratica e total
{\bf \small \codigo} & {\bf \small \disciplina } & {\bf \creditosT} & {\bf \creditosP} & {\bf 72} & Presencial\\ \hline
\end{longtable}


%%%%%%%%%%%%%%%%%%%%%%%%%%%%%%%%%%%%%%%%%%%%%%%%%%%%%%%%%%%%%%
\begin{longtable}{|C{0.12\textwidth}|L{0.736\textwidth}|C{0.12\textwidth}|} \hline
%
\multicolumn{3}{|l|}{{\bf II. PRÉ-REQUISITO(S)}} \\ \hline
%
CÓDIGO & NOME DA DISCIPLINA & CURSO \\ \hline	
%
\requisitoA
\requisitoB
\requisitoC
\end{longtable}


%%%%%%%%%%%%%%%%%%%%%%%%%%%%%%%%%%%%%%%%%%%%%%%%%%%%%%%%%%%%%%
\begin{longtable}{|L{1.025\textwidth}|} \hline
%
{\bf III. CURSO(S) PARA O(S) QUAL(IS) A DISCIPLINA É OFERECIDA } \\ \hline
%
\cursoA 
\cursoB
\cursoC

\end{longtable}

%%%%%%%%%%%%%%%%%%%%%%%%%%%%%%%%%%%%%%%%%%%%%%%%%%%%%%%%%%%%%%
\begin{longtable}{|L{1.025\textwidth}|} \hline
%
{\bf IV. EMENTA } \\ \hline
%
\ementa
\end{longtable}

%\newpage



%%%%%%%%%%%%%%%%%%%%%%%%%%%%%%%%%%%%%%%%%%%%%%%%%%%%%%%%%%%%%%%
\begin{longtable}{|L{1.025\textwidth}|} \hline
%
{\bf V. OBJETIVOS } \\ \hline
%
Objetivo Geral: \\
Capacitar o aluno para o desenvolvimento e aplicação de métodos matemáticos e técnicas algorítmicas da Inteligência Artificial que se utilizam de modelos conexionistas, evolucionários e de inspiração biológica.\\
\\
Objetivos Específicos:\\
Apresentar os conceitos de lógica nebulosa;\\
Apresentar os conceitos de redes neurais artificiais;\\
Apresentar os conceitos de computação evolucionária;\\
Apresentar os conceitos de algoritmos baseados em enxames;\\
Apresentar os conceitos de máquinas de vetores de suporte;\\
Desenvolver exercícios com lógica nebulosa, redes neurais, computação evolucionária e algoritmos baseados em enxame e máquinas de vetores de suporte.\\

\\ \hline
\end{longtable}

%\newpage
%%%%%%%%%%%%%%%%%%%%%%%%%%%%%%%%%%%%%%%%%%%%%%%%%%%%%%%%%%%%%%%
\begin{longtable}{|L{1.025\textwidth}|} \hline
%
{\bf VI. CONTEÚDO PROGRAMÁTICO } \\ \hline

Conteúdo Teórico seguido de Conteúdo Prático com desenvolvimento de problemas em computador: \\
\\
UNIDADE 1: Introdução [2 horas-aula]\\
Uma Breve História da Inteligência Artificial/Computacional\\
Conceitos da Inteligência Computacional\\
Aplicações da Inteligência Computacional\\
\\
UNIDADE 2: Lógica Nebulosa/Fuzzy [16 horas-aula]\\
Introdução\\
Conjuntos Nebulosos\\
Tratamento de Incertezas\\
Sistema de Inferência\\
Raciocínio e Incertezas em Lógica Nebulosa\\
\\
UNIDADE 3: Redes Neurais Artificiais [22 horas-aula]\\
Introdução\\
Aprendizado Supervisionado e não Supervisionado\\
Redes Perceptron\\
Perceptron Multicamadas\\
Redes Auto-Organizáveis\\
Sistemas Neurofuzzy\\
\\
UNIDADE 4: Computação Evolucionária [16 horas-aula]\\
Introdução\\
Algoritmos Genéticos\\
Outros algoritmos Evolucionários\\
\\
UNIDADE 5: Tópicos em Inteligência Computacional [16 horas-aula]\\
Introdução\\
Fundamentos de Inteligência Coletiva\\
Otimização Baseada em Colônias de Formigas\\
Otimização por Enxames de Partículas\\
Máquinas de Vetores de Suporte\\

\\ \hline
\end{longtable} 

\newpage

%%%%%%%%%%%%%%%%%%%%%%%%%%%%%%%%%%%%%%%%%%%%%%%%%%%%%%%%%%%%%%%
\begin{longtable}{|L{1.025\textwidth}|} \hline
%
{\bf VII. BIBLIOGRAFIA BÁSICA} \\ \hline
\begin{enumerate}
%
\item RUSSELL, S.; NORVIG, P. Inteligência Artificial. 2 ed. Editora Campus. 2004. 
\item COPPIN, Ben. Inteligência artificial. Rio de Janeiro: LTC, c2010. xxv, 636 p. ISBN 9788521617297.
\item HAYKIN, Simon. Redes Neurais: princípios e prática. Bookman, 2a. Ed., 2001.

\end{enumerate}
 \\ \hline
\end{longtable}


%\newpage

%%%%%%%%%%%%%%%%%%%%%%%%%%%%%%%%%%%%%%%%%%%%%%%%%%%%%%%%%%%%%%%
\begin{longtable}{|L{1.025\textwidth}|} \hline
%
{\bf VIII. BIBLIOGRAFIA COMPLEMENTAR} \\ \hline
\begin{enumerate}
\item  ROSA, J.L.G. Fundamentação da Inteligência Artificial. Editora LTC,2011. 
\item BRAGA, A. P.; CARVALHO, A. P. L. F.; LUDERMIR, T. B. Redes Neurais Artificiais - teoria e aplicações. 2a ed. Editora LTC, 2007. 
\item NASCIMENTO JÚNIOR, Cairo Lúcio; YONEYAMA, Takashi. Inteligência artificial: em controle e automação. São Paulo: FAPESP, c2000. vii, 218 p. ISBN 9788521203100.
\item FACELI, Katti et al. Inteligência artificial: uma abordagem de aprendizado por máquina. Rio de Janeiro: LTC, c2011. xvi, 378 p. ISBN 9788521618805.
\item SIMÕES, M. G,; SHAW, I. S. Controle e Modelagem Fuzzy. 2a ed. Editora Blucher, 2007.

\end{enumerate}
 \\ \hline
\end{longtable}


\input aprovacao.tex


\end{document}
