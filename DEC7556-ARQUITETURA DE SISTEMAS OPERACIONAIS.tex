\documentclass[12pt]{article}
\usepackage[brazil]{babel}
\usepackage{graphicx,t1enc,wrapfig,amsmath,float}
\usepackage{framed,fancyhdr}
\usepackage{multirow}
\usepackage{longtable}
\usepackage{array}
\newcolumntype{L}[1]{>{\raggedright\let\newline\\\arraybackslash\hspace{0pt}}m{#1}}
\newcolumntype{C}[1]{>{\centering\let\newline\\\arraybackslash\hspace{0pt}}m{#1}}
\newcolumntype{R}[1]{>{\raggedleft\let\newline\\\arraybackslash\hspace{0pt}}m{#1}}
%%%%%%%%%%%%%
\oddsidemargin -0.5cm
\evensidemargin -0.5cm
\textwidth 17.5cm
\topmargin -1.5cm
\textheight 22cm
%%%%%%%%%%%% 

%\pagestyle{empty}

\newcommand{\semestre}{2018.2}

\newcommand{\disciplina}{ARQUITETURA DE SISTEMAS OPERACIONAIS}
\newcommand{\codigo}{DEC7556}


%%%%%%%%%%%%%%%%%%%%%%%%%%%%%%%%%%%%%%%%%%%%%%%%%%%%%%%
%%%%%%%%%%%%% CRETIDOS
\newcommand{\creditosT}{2}
\newcommand{\creditosP}{2}

%%%%%%%%%%%%%%%%%%%%%%%%%%%%%%%%%%%%%%%%%%%%%%%%%%%%%%%
%%%%%%%%%%%%%% REQUISITOS
\newcommand{\requisitoA}{DEC0006 & Estrutura de Dados & ENC\\}
\newcommand{\requisitoB}{DEC7123 & Organização e Arquitetura de Computadores I & ENC\\ \hline}
\newcommand{\requisitoC}{}

%%%%%%%%%%%%%%%%%%%%%%%%%%%%%%%%%%%%%%%%%%%%%%%%%%%%%%%
%%%%%%%%%%%%%%% Atende aos Cursos
\newcommand{\cursoA}{Graduação em Engenharia de Computação \\ \hline}
\newcommand{\cursoB}{}%Graduação em Tecnologias da Informação e Comunicação \\ \hline}
\newcommand{\cursoC}{}

%%%%%%%%%%%%%%%%%%%%%%%%%%%%%%%%%%%%%%%%%%%%%%%%%%%%%%%%
%%%%%%%%%% EMENTA
\newcommand{\ementa}{
Introdução, histórico; Organização Interna dos Sistemas operacionais; Chamadas de Sistema; Gerenciamento de Processos; Gerenciamento de Memória; Gerenciamento de Dispositivos de Entrada e Saída; Sistemas de Arquivos; Proteção e Segurança em Sistemas Operacionais; Estudos de caso de Sistemas Operacionais.
\\ \hline
}


\begin{document}

%%%%%%%%%%%%%%%%%%%%%%%%%%%%%%%%%%%%%%%%%%%%%%%%%%%%%%%%%%%%%

\input cabecalho.tex


%%%%%%%%%%%%%%%%%%%%%%%%%%%%%%%%%%%%%%%%%%%%%%%%%%%%%%%%%%%%%
\begin{longtable}{|C{0.11\textwidth}|C{0.29\textwidth}|C{0.09\textwidth}|C{0.09\textwidth}|C{0.15\textwidth}|C{0.158\textwidth}|} \hline
%
\multicolumn{6}{|l|}{{\bf I. IDENTIFICAÇÃO DA DISCIPLINA}} \\ \hline
%
\multirow{3}*{{\small CÓDIGO}} & \multirow{3}*{NOME DA DISCIPLINA} &\multicolumn{2}{c|}{{\small N$^\circ$ DE HORAS-AULA }} & {{\small TOTAL DE}} & \multirow{3}*{{\small MODALIDADE}} \\ 
%
& & \multicolumn{2}{c|}{\small SEMANAIS}  & {\small HORAS-AULA} & \\ \cline{3-4}
%
& & {\tiny TEÓRICAS} & {\tiny PRÁTICAS} & {\small SEMESTRAIS} & \\ \hline
% codigo da disciplina carga horaria: teorica - pratica e total
{\bf \small \codigo} & {\bf \small \disciplina } & {\bf \creditosT} & {\bf \creditosP} & {\bf 72} & Presencial\\ \hline
\end{longtable}


%%%%%%%%%%%%%%%%%%%%%%%%%%%%%%%%%%%%%%%%%%%%%%%%%%%%%%%%%%%%%%
\begin{longtable}{|C{0.12\textwidth}|L{0.736\textwidth}|C{0.12\textwidth}|} \hline
%
\multicolumn{3}{|l|}{{\bf II. PRÉ-REQUISITO(S)}} \\ \hline
%
CÓDIGO & NOME DA DISCIPLINA & CURSO \\ \hline	
%
\requisitoA
\requisitoB
\requisitoC
\end{longtable}


%%%%%%%%%%%%%%%%%%%%%%%%%%%%%%%%%%%%%%%%%%%%%%%%%%%%%%%%%%%%%%
\begin{longtable}{|L{1.025\textwidth}|} \hline
%
{\bf III. CURSO(S) PARA O(S) QUAL(IS) A DISCIPLINA É OFERECIDA } \\ \hline
%
\cursoA 
\cursoB
\cursoC

\end{longtable}

%%%%%%%%%%%%%%%%%%%%%%%%%%%%%%%%%%%%%%%%%%%%%%%%%%%%%%%%%%%%%%
\begin{longtable}{|L{1.025\textwidth}|} \hline
%
{\bf IV. EMENTA } \\ \hline
%
\ementa
\end{longtable}

\newpage



%%%%%%%%%%%%%%%%%%%%%%%%%%%%%%%%%%%%%%%%%%%%%%%%%%%%%%%%%%%%%%%
\begin{longtable}{|L{1.025\textwidth}|} \hline
%
{\bf V. OBJETIVOS } \\ \hline
Objetivo Geral: \\

Esta disciplina tem como objetivo explorar os principais conceitos, arquiteturas e características internas dos sistemas operacionais. \\
\\
Objetivos Específicos:
\begin{itemize}
\item Apresentar os conceitos, finalidades e exemplos de sistemas operacionais;
\item Abordar conceitos sobre gerência de processos, memória, entrada e saída e sistemas de arquivos;
\item Fazer com que o discente obtenha conhecimento sobre as várias técnicas empregadas no projeto e implementação de um sistema operacional;
\item Implementar algoritmos para simular partes de um sistema operacional como a gerência de processos, gerência de memória e sistemas de arquivos.
\end{itemize}
\\ \hline
\end{longtable}


%%%%%%%%%%%%%%%%%%%%%%%%%%%%%%%%%%%%%%%%%%%%%%%%%%%%%%%%%%%%%%%
\begin{longtable}{|L{1.025\textwidth}|} \hline
%
{\bf VI. CONTEÚDO PROGRAMÁTICO } \\ \hline
Conteúdo Teórico seguido de Conteúdo Prático com desenvolvimento de problemas em computador: \\
\\
UNIDADE 1: Introdução \\% [4 horas-aula]\\
Definição e Características de um Sistema Operacional\\
Estrutura de um Sistema Operacional\\
Serviços do Sistema Operacional\\
Chamadas de Sistemas\\
Projeto e Implementação do Sistema Operacional\\
Mecanismos e Políticas\\
Implementação\\
Estrutura do Sistema Operacional\\
Monolíticos\\
Camadas\\
Microkernels\\
Módulos\\
Máquinas virtuais\\
Cliente-sevidor\\
\\
UNIDADE 2: Gerência de processos\\ %[28h-aula]\\
Conceito de Processos\\
Estados de um Processo\\
Bloco de Controle de Processos\\
Escalonamento de Processos\\
Troca de contexto\\
Criação de Processos\\
Threads\\
Motivação para o uso de Threads\\
Modelos de Múltiplas Threads\\
Bibliotecas de Threads\\
Posix Threads - Pthreads\\
Windows Threads\\
Threads em Java\\
Aspectos do uso de Threads\\
Escalonamento de processos\\
Ciclos de CPU e ES (Entrada e Saída)\\
Conceitos de Preempção\\
Algoritmos de Escalonamento\\
First Come, First Served - FCFS\\
Shortest Job First - SJF\\
Escalonamento por Prioridade\\
Round-Robin\\
Filas Multinível\\
Escalonamento de Threads\\
Escalonamento em Múltiplos processadores\\
Programação concorrente\\
Sincronização de processos\\
Caracterização\\
Seção Crítica\\
Hardware de Sincronismo\\
Semáforos\\
Monitores\\
Problemas Clássicos de Sincronismo\\
Comunicação entre processos\\
PIPEs\\
PIPEs nomeados\\
Memória compartilhada\\
Sockets\\
Deadlock\\
Caracterização do Deadlock\\
Grafo de Alocação de Recursos\\
Métodos para Tratamento de Deadlocks\\
Prevenção de Deadlocks\\
Detecção de Deadlock\\
Recuperação do Deadlock\\
\\
UNIDADE 3: Gerência de memória \\ %[8h-aula]\\
Carregamento absoluto e carregamento relocado\\
Alocação contígua\\
Partições fixas\\
Partições variáveis\\
Alocação não-contígua\\
Paginação\\
Segmentação\\
Segmentação paginada\\
Memória virtual\\
Paginação por Demanda\\
Algoritmos de substituição de página\\
Trashing\\
\\
UNIDADE 4: Sistemas de arquivos \\%[10h-aula]\\
Arquivos e diretórios\\
Estruturação de arquivos\\
Implementação de sistemas de arquivos\\
Alocação de espaço em disco\\
Alocação contígua\\
Alocação encadeada\\
Alocação indexada\\
Gerência de espaço livre em discos\\
Múltiplos sistemas de arquivos.\\
Sistemas de Arquivos de Rede\\
\\
UNIDADE 5: Gerência de entrada e saída %\\[12h-aula]\\
Controlador e driver de dispositivo\\
E/S programada\\
Interrupções\\
DMA (Direct Memory Access - Acesso Direto a Memória)\\
Organização de discos rígidos\\
Algoritmos de escalonamento de braço de disco		\\
Sistemas RAID (Redundant Array of Independent Disks)\\
\\
UNIDADE 6: Proteção e Segurança em Sistemas Operacionais \\%[10h]\\
Princípios de proteção\\
Matriz de acesso\\
Domínio de proteção\\
Conceitos de criptografia\\
\\ \hline
\end{longtable} 

\newpage

%%%%%%%%%%%%%%%%%%%%%%%%%%%%%%%%%%%%%%%%%%%%%%%%%%%%%%%%%%%%%%%
\begin{longtable}{|L{1.025\textwidth}|} \hline
%
{\bf VII. BIBLIOGRAFIA BÁSICA} \\ \hline

\begin{enumerate}
\item  SILBERSCHATZ, Abraham; GALVIN, Peter Baer; GAGNE; Greg. Fundamentos de Sistemas Operacionais. 8a ed. LTC, 2010. 
\item TANENBAUM, Andrew S. Sistemas Operacionais Modernos. 3a ed. Pearson, 2010. 
\item MARQUES, José Alves; FERREIRA, Paulo; RIBEIRO, Carlos; VEIGA, Luís. RODRIGUES, Rodrigo. Sistemas Operacionais. Rio de Janeiro: Editora LTC, 2011.

\end{enumerate}
 \\ \hline
\end{longtable}


%\newpage

%%%%%%%%%%%%%%%%%%%%%%%%%%%%%%%%%%%%%%%%%%%%%%%%%%%%%%%%%%%%%%%
\begin{longtable}{|L{1.025\textwidth}|} \hline
%
{\bf VIII. BIBLIOGRAFIA COMPLEMENTAR} \\ \hline
\begin{enumerate}
\item SILBERSCHATZ, Abraham; GALVIN, Peter Baer; GAGNE; Greg. Sistemas Operacionais com Java, 7a ed. Elsevier, 2008. 
\item DEITEL, Harvey M.; DEITEL, Paul J.; CHOFFNES, D. R. Sistemas Operacionais. 3a ed. Pearson, 2005. 
\item STUART, Brian L. Princípios de Sistemas Operacionais - Projetos e Aplicações. Cengage Learning, 2011. 
\item MACHADO, Francis Berenger.; MAIA, Luiz Paulo. Arquitetura de Sistemas Operacionais, 4a ed. LTC, 2007. 
%\item OLIVEIRA, Rômulo Silva de; CARISSINI, Alexandre da Silva; TOSCANI, Simão Sirineo. Sistemas operacionais. 4. ed. Porto Alegre: Bookman, 2010. xii, 374 p. (Livros didáticos ; 11). ISBN 9788577805211.
\item  TANENBAUM, Andrew S.; WOODHULL, Albert S. Sistemas Operacionais - Projeto e Implementação. 3a ed. Bookman, 2008.

%\item MAZIERO, C. Sistemas Operacionais: Conceitos e Mecanismos [recurso eletrônico]. Editora UFPR, 2019. 456 p. ISBN 978-85-7335-340-2. Disponível em: http://wiki.inf.ufpr.br/maziero/lib/exe/fetch.php?media=socm:socm-livro.pdf


\end{enumerate}
 \\ \hline
\end{longtable}



\input aprovacao.tex


\end{document}
