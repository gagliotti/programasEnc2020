\documentclass[12pt]{article}
\usepackage[brazil]{babel}
\usepackage{graphicx,t1enc,wrapfig,amsmath,float}
\usepackage{framed,fancyhdr}
\usepackage{multirow}
\usepackage{longtable}
\usepackage{array}
\newcolumntype{L}[1]{>{\raggedright\let\newline\\\arraybackslash\hspace{0pt}}m{#1}}
\newcolumntype{C}[1]{>{\centering\let\newline\\\arraybackslash\hspace{0pt}}m{#1}}
\newcolumntype{R}[1]{>{\raggedleft\let\newline\\\arraybackslash\hspace{0pt}}m{#1}}
%%%%%%%%%%%%%
\oddsidemargin -0.5cm
\evensidemargin -0.5cm
\textwidth 17.5cm
\topmargin -1.5cm
\textheight 22cm
%%%%%%%%%%%% 

%\pagestyle{empty}

\newcommand{\semestre}{2018.2}

\newcommand{\disciplina}{LINGUAGEM DE DESCRIÇÃO DE HARDWARE}
\newcommand{\codigo}{DEC7555}


%%%%%%%%%%%%%%%%%%%%%%%%%%%%%%%%%%%%%%%%%%%%%%%%%%%%%%%
%%%%%%%%%%%%% CRETIDOS
\newcommand{\creditosT}{2}
\newcommand{\creditosP}{2}

%%%%%%%%%%%%%%%%%%%%%%%%%%%%%%%%%%%%%%%%%%%%%%%%%%%%%%%
%%%%%%%%%%%%%% REQUISITOS
\newcommand{\requisitoA}{}
\newcommand{\requisitoB}{}
\newcommand{\requisitoC}{}

%%%%%%%%%%%%%%%%%%%%%%%%%%%%%%%%%%%%%%%%%%%%%%%%%%%%%%%
%%%%%%%%%%%%%%% Atende aos Cursos
\newcommand{\cursoA}{Graduação em Engenharia de Computação \\ \hline}
\newcommand{\cursoB}{}%Graduação em Tecnologias da Informação e Comunicação \\ \hline}
\newcommand{\cursoC}{}

%%%%%%%%%%%%%%%%%%%%%%%%%%%%%%%%%%%%%%%%%%%%%%%%%%%%%%%%
%%%%%%%%%% EMENTA
\newcommand{\ementa}{
Desenvolvimento de projetos em Alto Nível através de Linguagens de Descrição de Hardware (VHDL, Verilog), Máquina Finita de Estados, RTL (RegisterTransferLevel), em dispositivos como Field Programmable Gate Array (FPGA). São considerados Co-Projetos de Hardware/Software em ambientes System-on-Chip enfocando CORE e IP para o re-uso de sistemas. Para esses estudos são considerados os usos das ferramentas EDA (Eletronic Design Automation) da Xilinx e Altera. Considerações sobre: co-projeto de hardware/software; engenharia de software para o sistema; questões de sincronização de clock; protocolo de comunicação; escalonamento; RTOS (Real Time Operating System); validação e verificação; tolerância à falhas; programando sensores e atuadores; simulação, ferramentas EDA, ambiente distribuído.
\\ \hline
}


\begin{document}

%%%%%%%%%%%%%%%%%%%%%%%%%%%%%%%%%%%%%%%%%%%%%%%%%%%%%%%%%%%%%

\input cabecalho.tex


%%%%%%%%%%%%%%%%%%%%%%%%%%%%%%%%%%%%%%%%%%%%%%%%%%%%%%%%%%%%%
\begin{longtable}{|C{0.11\textwidth}|C{0.29\textwidth}|C{0.09\textwidth}|C{0.09\textwidth}|C{0.15\textwidth}|C{0.158\textwidth}|} \hline
%
\multicolumn{6}{|l|}{{\bf I. IDENTIFICAÇÃO DA DISCIPLINA}} \\ \hline
%
\multirow{3}*{{\small CÓDIGO}} & \multirow{3}*{NOME DA DISCIPLINA} &\multicolumn{2}{c|}{{\small N$^\circ$ DE HORAS-AULA }} & {{\small TOTAL DE}} & \multirow{3}*{{\small MODALIDADE}} \\ 
%
& & \multicolumn{2}{c|}{\small SEMANAIS}  & {\small HORAS-AULA} & \\ \cline{3-4}
%
& & {\tiny TEÓRICAS} & {\tiny PRÁTICAS} & {\small SEMESTRAIS} & \\ \hline
% codigo da disciplina carga horaria: teorica - pratica e total
{\bf \small \codigo} & {\bf \small \disciplina } & {\bf \creditosT} & {\bf \creditosP} & {\bf 72} & Presencial\\ \hline
\end{longtable}


%%%%%%%%%%%%%%%%%%%%%%%%%%%%%%%%%%%%%%%%%%%%%%%%%%%%%%%%%%%%%%
\begin{longtable}{|C{0.12\textwidth}|L{0.736\textwidth}|C{0.12\textwidth}|} \hline
%
\multicolumn{3}{|l|}{{\bf II. PRÉ-REQUISITO(S)}} \\ \hline
%
CÓDIGO & NOME DA DISCIPLINA & CURSO \\ \hline	
DEC7546 & Circuitos Digitais & ENC \\ 
DEC7123 & Organização e Arquitetura de Computadores I & ENC \\ \hline
%
\requisitoA
\requisitoB
\requisitoC
\end{longtable}


%%%%%%%%%%%%%%%%%%%%%%%%%%%%%%%%%%%%%%%%%%%%%%%%%%%%%%%%%%%%%%
\begin{longtable}{|L{1.025\textwidth}|} \hline
%
{\bf III. CURSO(S) PARA O(S) QUAL(IS) A DISCIPLINA É OFERECIDA } \\ \hline
%
\cursoA 
\cursoB
\cursoC

\end{longtable}

%%%%%%%%%%%%%%%%%%%%%%%%%%%%%%%%%%%%%%%%%%%%%%%%%%%%%%%%%%%%%%
\begin{longtable}{|L{1.025\textwidth}|} \hline
%
{\bf IV. EMENTA } \\ \hline
%
%\ementa
Histórico e aspectos gerais da linguagem VHDL; Estruturas básicas da linguagem;
Componentes e esquemas de iteração; 
Subprogramas; 
Funções; 
Bibliotecas, pacotes e configurações; 
Síntese de circuitos lógicos combinacionais; 
Síntese de circuitos lógicos sequenciais; 
Conceito de circuitos síncronos e assíncronos; 
Máquinas de Estado; 
Síntese de memórias, contadores e circuitos de serialização; Conceitos de Caminho de Dados e Caminho de Controle; 
Conversão de algoritmos em processadores de propósito único; 
Projeto de sistema embarcado baseado em FPGA. \\ \hline
\end{longtable}

\newpage



%%%%%%%%%%%%%%%%%%%%%%%%%%%%%%%%%%%%%%%%%%%%%%%%%%%%%%%%%%%%%%%
\begin{longtable}{|L{1.025\textwidth}|} \hline
%
{\bf V. OBJETIVOS } \\ \hline
%Objetivo Geral: 
%\begin{itemize}
%\item Compreender o funcionamento de uma FPGA e o ciclo de desenvolvimento de Hardware utilizando esse tipo de dispositivo. 
%\item Compreender a capacidade desses dispositivos e as possibilidades de integração de Hard Cores, Soft Cores e código personalizado.
%\item Escrever código em linguagem de descrição de hardware, utilizar ferramentas de desenvolvimento e simulação
%\item Criar um pequeno projeto de dispositivo que explore os recursos de uma FPGA
%\end{itemize}

Compreender o funcionamento de um FPGA e o ciclo de desenvolvimento de Hardware utilizando esse tipo de dispositivo. 
Compreender a capacidade desses dispositivos e as possibilidades de integração de Hard Cores, Soft Cores e código personalizado. Escrever código em linguagem de descrição de hardware, utilizando ferramentas de desenvolvimento e simulação.
Criar um pequeno projeto de dispositivo que explore os recursos de um FPGA.

\\ \hline
\end{longtable}


%%%%%%%%%%%%%%%%%%%%%%%%%%%%%%%%%%%%%%%%%%%%%%%%%%%%%%%%%%%%%%%
\begin{longtable}{|L{1.025\textwidth}|} \hline
%
{\bf VI. CONTEÚDO PROGRAMÁTICO } \\ \hline
 \\
 
UNIDADE 1: Introdução\\
- Objetivos da disciplina\\
– Metodologia de projeto: pequenos sistemas digitais\\
– Metodologia de projeto: grandes sistemas digitais\\
– Visão geral de Dispositivos lógicos programáveis\\
\\

UNIDADE 2: Linguagem de Hardware e Síntese de Circuitos \\
– Introdução ao VHDL\\
– Definição de Entidade e Arquitetura\\
– Operadores Lógicos e Aritméticos\\
– Tipos de dados\\
– Configuração das máquinas dos alunos para uso da ferramenta de síntese Quartu II\\
\\

UNIDADE 3: Definição de Componentes \\
– Definição de Componentes\\
– Formas de conexão interna de componentes\\
– Exemplos e exercícios de síntese e simulação de componentes\\
\\

UNIDADE 4: Simuladores \\
– Testes em componentes criados em VHDL\\
– Ferramenta Qsim\\
– Ferramenta ModelSim\\
– Exercícios de síntese e simulação no ModelSim\\
\\

UNIDADE 5: Comandos Concorrentes\\
– O que é a Concorrência em VHDL?\\
– Comandos concorrentes básicos: WHEN – ELSE, WITH – SELECT, e PROCESS\\
– Exercícios de implementação de lógica combinacional usando os comandos concorrentes apresentados\\
\\

UNIDADE 6: Comandos Sequenciais\\
– Definição de Lista de sensibilidade\\
– Comandos sequenciais básicos: IF – ELSE, CASE – WHEN, e NULL\\
– Exercícios de implementação de lógica sequencial usando os comandos sequenciais\\
\\

UNIDADE 7: Circuitos Síncronos e Assíncronos\\
– Conceitos de sincronismo\\
– Flip-flops\\
– Registradores\\
– Máquinas de estado: Mealy, Moore e One-Hot-Encoding\\
– Exercícios de implementação de circuitos síncronos e assíncronos\\
– Exercício de implementação de um cronômetro baseado em máquina de estados\\
\\

UNIDADE 8: Esquemas de Geração\\
– Declaração de GENERIC\\
– Comando GENERATE\\
– Esquema de geração IF\\
– Comando LOOP\\
– Comando FOR\\
– Comando WHILE\\
– Exercícios de implementação de esquemas de geração\\
\\

UNIDADE 9: Declaração de tipos e principais atributos\\
– Definição de Bibliotecas\\
– Definição de Pacotes\\
– Ordem de análise na síntese\\
– Tipos e Subtipos\\
– Vetores Unidimensionais e Multidimensionais\\
– Atributos que geram novos sinais\\
– Atributos que não geram novos sinais\\
– Exercícios de implementação de tipos e subtipos\\
– Exercícios de implementação de atributos e vetores\\
\\

UNIDADE 10: Memórias e PLL\\
– Conceitos de memórias internas e externas ao FPGA\\
– Elaboração de uma Read-only memory (ROM) usando VHDL\\
– Elaboração de uma Random-acces memory (RAM) usando VHDL\\
– Uso de templates da ferramenta Quartus para gerar memórias\\
– Definição de PLL\\
– Template da ferramenta Quartus II para gerar um componente PLL\\
– Exercícios de implementação de memórias\\
– Exercícios de implementação de PLL\\
\\

UNIDADE 11: Configurações\\
– Uso de configuração\\
– Exemplo de caso prático do uso de configuração\\
– Exercícios de implementação baseados no conceito de Configurações\\
\\

UNIDADE 12: Variáveis\\
– Conceito de variáveis\\
– Atribuição de valores em variáveis\\
– Diferenças entre Sinais e Variáveis\\
– Exercícios de implementação utilizando o conceito de variáveis\\
\\

UNIDADE 13: Instruções auxiliares\\
– ASSERT\\
– ALIAS\\
– PACKAGE\\
– FUNCTION\\
– PROCEDURE\\
– Exercício de implementação com instruções auxiliares\\
\\

UNIDADE 14: Projetos de circuitos\\
– Registrador de deslocamento com Data-Load\\
– Deboucer para chave\\
– Temporizador\\
– Exercício de implementação dos projetos de circuitos\\
\\

UNIDADE 15: Projetos de circuitos\\
– Conversor Paralelo-Serial\\
– Conversor Serial-Paralelo\\
– Decodificador de endereços\\
– Conversor BCD para display de sete segmentos\\
– Exercício de implementação dos projetos de circuitos\\
\\

UNIDADE 16: Projeto de circuitos\\
– Medidores de frequência\\
– Redes neurais\\
\\

UNIDADE 17: Caminhos de Dados e Controle\\
– Definição de Caminho de Dados e Caminho de controle\\
– Definição de um processador didático\\
– Implementação do processador didático em VHDL\\
\\

UNIDADE 18: Processador didático\\
– Implementação do processador didático em VHDL\\
– Programação e teste do processador didático\\
\\

UNIDADE 19: Processadores de Propósito Único - PPU\\
– Conceito de PPU\\
– Técnica de mapeamento de algoritmos para geração de PPU\\
– Exercício de conversão de algoritmo em PPU\\
\\

UNIDADE 20: Trabalho\\
– Sorteio do Trabalho Final para cada aluno\\
– Explanação sobre cada um dos trabalhos\\
– Definição das documentações exigidas na entrega do Trabalho Final\\
\\

\hline
\end{longtable} 

\newpage


%%%%%%%%%%%%%%%%%%%%%%%%%%%%%%%%%%%%%%%%%%%%%%%%%%%%%%%%%%%%%%%
\begin{longtable}{|L{1.025\textwidth}|} \hline
%
{\bf VII. BIBLIOGRAFIA BÁSICA} \\ \hline

\begin{enumerate}
\item HAMBLEN, James O; HALL, Tyson S; FURMAN, Michael D. Rapid Prototyping of Digital Systems. Boston: Springer Science+Business Media, LLC, 2008. 
\item PEDRONI, Volnei A. Eletrônica digital moderna e VHDL. Rio de Janeiro: Elsevier, c2010. 619 p. ISBN 9788535234657. 
\item D’AMORE, Roberto. VHDL: descrição e síntese de circuitos digitais. 2. ed. Rio de Janeiro: LTC, c2012. xiii, 292 p. ISBN 9788521620549.


\end{enumerate}
 \\ \hline
\end{longtable}


%\newpage

%%%%%%%%%%%%%%%%%%%%%%%%%%%%%%%%%%%%%%%%%%%%%%%%%%%%%%%%%%%%%%%
\begin{longtable}{|L{1.025\textwidth}|} \hline
%
{\bf VIII. BIBLIOGRAFIA COMPLEMENTAR} \\ \hline
\begin{enumerate}

\item CHU, Pong P. FPGA prototyping by VHDL examples: Xilinx Spartan-3 version.
Hoboken, N.J.: Wiley-Interscience, [2008] 1 recurso online (xxv, 440 p ISBN 9780470231623 (e-book). Disponível em: <https://doi.org/10.1002/9780470231630>.
\item CHU, Pong P., RTL hardware design using VHDL: coding for efficiency, portability, and scalability. Hoboken, N.J.: Wiley-Interscience, c2006. 1 online resource (xxiii, 66 ISBN 0471720925 (alk. paper)).
\item The Designer’s Guide to VHDL 3 edição, Volume 3, Peter J. Ashenden ISBN: 978-0-12-088785-9 (disponível no science direct).
\item FERREIRA, José Manuel Martins. Introdução ao projeto com sistemas digitais e microcontroladores. Porto: FEUP, 1998. 371 p. ISBN 9727520324.
\item WILSON, Peter. The circuit designer’s companion. 3rd ed. Amsterdam: Elsevier, 2012. xv, 439 p. ISBN 9780080971384.
\item Free Range VHDL - Livro disponível gratuitamente em http://www.freerangefactory.org/ 
\item The Designer's Guide to VHDL 3 edição, Volume 3, Peter J. Ashenden ISBN: 978-0-12-088785-9 (disponível no science direct).
\end{enumerate}
 \\ \hline
\end{longtable}


\input aprovacao.tex


\end{document}
