\documentclass[12pt]{article}
\usepackage[brazil]{babel}
\usepackage{graphicx,t1enc,wrapfig,amsmath,float}
\usepackage{framed,fancyhdr}
\usepackage{multirow}
\usepackage{longtable}
\usepackage{array}
\newcolumntype{L}[1]{>{\raggedright\let\newline\\\arraybackslash\hspace{0pt}}m{#1}}
\newcolumntype{C}[1]{>{\centering\let\newline\\\arraybackslash\hspace{0pt}}m{#1}}
\newcolumntype{R}[1]{>{\raggedleft\let\newline\\\arraybackslash\hspace{0pt}}m{#1}}
%%%%%%%%%%%%%
\oddsidemargin -0.5cm
\evensidemargin -0.5cm
\textwidth 17.5cm
\topmargin -1.5cm
\textheight 22cm
%%%%%%%%%%%% 

%\pagestyle{empty}

\newcommand{\semestre}{2020.1} 

\newcommand{\disciplina}{INTRODUÇÃO À ENGENHARIA DE COMPUTAÇÃO}
\newcommand{\codigo}{DEC7070}


%%%%%%%%%%%%%%%%%%%%%%%%%%%%%%%%%%%%%%%%%%%%%%%%%%%%%%%
%%%%%%%%%%%%% CRETIDOS
\newcommand{\creditosT}{1}
\newcommand{\creditosP}{1}

%%%%%%%%%%%%%%%%%%%%%%%%%%%%%%%%%%%%%%%%%%%%%%%%%%%%%%%
%%%%%%%%%%%%%% REQUISITOS
\newcommand{\requisitoA}{}
\newcommand{\requisitoB}{}
\newcommand{\requisitoC}{}

%%%%%%%%%%%%%%%%%%%%%%%%%%%%%%%%%%%%%%%%%%%%%%%%%%%%%%%
%%%%%%%%%%%%%%% Atende aos Cursos
\newcommand{\cursoA}{Graduação em Engenharia de Computação. \\ \hline}
\newcommand{\cursoC}{}

%%%%%%%%%%%%%%%%%%%%%%%%%%%%%%%%%%%%%%%%%%%%%%%%%%%%%%%%
%%%%%%%%%% EMENTA
\newcommand{\ementa}{

Perfil do profissional da computação. Campo de atuação. Ética profissional. Regulamentação profissional. Estrutura e objetivos do curso. Procedimento de matrícula. Histórico e evolução dos computadores. Introdução à computação. Características básicas dos computadores: hardware e software. Componentes básicos dos computadores: memória, unidade central de processamento, entrada e saída. Modelo de von Neumann.

 \\ \hline
}




\begin{document}


%%%%%%%%%%%%%%%%%%%%%%%%%%%%%%%%%%%%%%%%%%%%%%%%%%%%%%%%%%%%%
\begin{longtable}{|C{0.2\textwidth}|C{0.8\textwidth}|} \hline
%
\multirow{6}*{\includegraphics[scale=0.5]{UFSC-foto.jpg}} &\\
%
&{\bf UNIVERSIDADE FEDERAL DE SANTA CATARINA}\hfill\\
%
&{\bf Centro de Ciências, Tecnologias e Saúde} \\
%
&{\bf Departamento de Computação}\\
%
&{\bf PROGRAMA DE ENSINO}\\
%
& \\ \hline

%\multicolumn{2}{|c|}{{\bf SEMESTRE \semestre}}\\ \hline
\end{longtable}


%%%%%%%%%%%%%%%%%%%%%%%%%%%%%%%%%%%%%%%%%%%%%%%%%%%%%%%%%%%%%
\begin{longtable}{|C{0.11\textwidth}|C{0.29\textwidth}|C{0.09\textwidth}|C{0.09\textwidth}|C{0.15\textwidth}|C{0.158\textwidth}|} \hline
%
\multicolumn{6}{|l|}{{\bf I. IDENTIFICAÇÃO DA DISCIPLINA}} \\ \hline
%
\multirow{3}*{{\small CÓDIGO}} & \multirow{3}*{NOME DA DISCIPLINA} &\multicolumn{2}{c|}{{\small N$^\circ$ DE HORAS-AULA }} & {{\small TOTAL DE}} & \multirow{3}*{{\small MODALIDADE}} \\ 
%
& & \multicolumn{2}{c|}{\small SEMANAIS}  & {\small HORAS-AULA} & \\ \cline{3-4}
%
& & {\tiny TEÓRICAS} & {\tiny PRÁTICAS} & {\small SEMESTRAIS} & \\ \hline
% codigo da disciplina carga horaria: teorica - pratica e total
{\bf \small \codigo} & {\bf \small \disciplina } & {\bf \creditosT} & {\bf \creditosP} & {\bf 36} & Presencial\\ \hline
\end{longtable}


%%%%%%%%%%%%%%%%%%%%%%%%%%%%%%%%%%%%%%%%%%%%%%%%%%%%%%%%%%%%%%
\begin{longtable}{|C{0.12\textwidth}|L{0.736\textwidth}|C{0.12\textwidth}|} \hline
%
\multicolumn{3}{|l|}{{\bf II. PRÉ-REQUISITO(S)}} \\ \hline
%
CÓDIGO & NOME DA DISCIPLINA & CURSO \\ \hline	
%
\requisitoA
\requisitoB
\requisitoC
\end{longtable}


%%%%%%%%%%%%%%%%%%%%%%%%%%%%%%%%%%%%%%%%%%%%%%%%%%%%%%%%%%%%%%
\begin{longtable}{|L{1.025\textwidth}|} \hline
%
{\bf III. CURSO(S) PARA O(S) QUAL(IS) A DISCIPLINA É OFERECIDA } \\ \hline
%
\cursoA 

\end{longtable}

%%%%%%%%%%%%%%%%%%%%%%%%%%%%%%%%%%%%%%%%%%%%%%%%%%%%%%%%%%%%%%
\begin{longtable}{|L{1.025\textwidth}|} \hline
%
{\bf IV. EMENTA } \\ \hline
%
\ementa
\end{longtable}

\newpage



%%%%%%%%%%%%%%%%%%%%%%%%%%%%%%%%%%%%%%%%%%%%%%%%%%%%%%%%%%%%%%%
\begin{longtable}{|L{1.025\textwidth}|} \hline
%
{\bf V. OBJETIVOS } \\ \hline
%
Objetivo Geral:

Fornecer ao aluno ingressante no curso de Engenharia de Computação uma visão geral acerca das principais áreas de atuação, competências, habilidades e o perfil do egresso do profissional de Engenharia de Computação. \\

Objetivos Específicos:
\begin{enumerate}

\item Fornecer aos alunos uma visão dos cursos de graduação em Engenharia de Computação: estrutura curricular, ênfases, mercado de atuação, etc;
\item Capacitar o aluno a conhecer a estrutura básica de um computador, seu funcionamento e aplicações; 
\item Permitir ao aluno ter uma visão crítica sobre as áreas de atuação e a relação entre elas.

\end{enumerate}

\\ \hline
\end{longtable}


%%%%%%%%%%%%%%%%%%%%%%%%%%%%%%%%%%%%%%%%%%%%%%%%%%%%%%%%%%%%%%%
\begin{longtable}{|L{1.025\textwidth}|} \hline
%
{\bf VI. CONTEÚDO PROGRAMÁTICO } \\ \hline

UNIDADE 1: Introdução [4 horas-aula] \\
Áreas de atuação em computação \\
Regulamentação da profissão \\
Ética profissional \\
Engenharia: ser engenheiro \\
Projetos em Engenharia \\
Sobre a Universidade Federal de Santa Catarina \\
Estrutura do Curso de Engenharia de Computação da UFSC \\\\

UNIDADE 2: História da Computação [4 horas-aula] \\
Introdução à Computação \\
Histórico e evolução da Computação \\
Aspectos futurísticos da computação \\\\

UNIDADE 3: Estrutura de Computadores [8 horas-aula] \\
Evolução dos computadores \\
Estrutura Interna (memória, unidade de processamento, barramentos) \\
Hardware versus software \\
Modelos computacionais (von Neumann e Harvard) \\\\

UNIDADE 4: Projetos de Engenharia: Experimentos com Sistemas Microcontrolados [16 horas-aula] \\
Introdução ao Arduino \\
Simulação de Circuitos Elétricos \\
Programação em Arduino \\
Experimentos com Arduino \\

\\ \hline
\end{longtable} 

%%%%%%%%%%%%%%%%%%%%%%%%%%%%%%%%%%%%%%%%%%%%%%%%%%%%%%%%%%%%%%%
\begin{longtable}{|L{1.025\textwidth}|} \hline
%
{\bf VII. BIBLIOGRAFIA BÁSICA} \\ \hline
\begin{enumerate}
%

\item BROOKSHEAR, J. Glenn. Ciência da Computação – uma visão abrangente. 11ª ed. Bookman, 2013.
\item ARAUJO, Celso de; CRUZ, Eduardo C. A.; JUNIOR, Salomão C. Eletrônica Digital. Editora Érika, 2013.
\item BAZZO, Walter Antonio; PEREIRA, Luiz Teixeira do Vale. Introdução à Engenharia – conceitos, ferramentas e comportamentos. Editora da UFSC, 2006.

%

\end{enumerate}
 \\ \hline
\end{longtable}


%\newpage

%%%%%%%%%%%%%%%%%%%%%%%%%%%%%%%%%%%%%%%%%%%%%%%%%%%%%%%%%%%%%%%
\begin{longtable}{|L{1.025\textwidth}|} \hline
%
{\bf VIII. BIBLIOGRAFIA COMPLEMENTAR} \\ \hline
\begin{enumerate}

\item IDOETA, Ivan Valeije; CAPUANO, Francisco Gabriel; Elementos de Eletrônica Digital. 41ª ed. Editora Érika, 2013.
\item MONTEIRO, M. A. Introdução à organização de computadores. 5. Ed. Rio de Janeiro: LTC, 2007.
%\item HENNESSY, John L.; PATTERSON, David A. Arquitetura de Computadores – uma abordagem quantitativa. 4ª ed. Editora Campus, 2008.
\item PATTERSON, David A.; HENNESSY, John L. Arquitetura de computadores: uma abordagem quantitativa. 5. ed. Rio de Janeiro: Elsevier, c2014. xxv, 435 [200] p. ISBN 9788535261226.
\item BROCKMAN, Jay B. Introdução à engenharia: modelagem e solução de problemas. Rio de Janeiro: LTC, c2010. xvii, 294 p. ISBN 9788521617266.
\item BIGNELL, James; DONOVAN, Robert. Eletrônica digital. São Paulo: Cengage Learning, 2010. xviii, 648 p. ISBN 9788522107452
%
\end{enumerate}
 \\ \hline
\end{longtable}


\input aprovacao.tex


\end{document}
