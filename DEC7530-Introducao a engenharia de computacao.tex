\documentclass[12pt]{article}
\usepackage[brazil]{babel}
\usepackage{graphicx,t1enc,wrapfig,amsmath,float}
\usepackage{framed,fancyhdr}
\usepackage{multirow}
\usepackage{longtable}
\usepackage{array}
\newcolumntype{L}[1]{>{\raggedright\let\newline\\\arraybackslash\hspace{0pt}}m{#1}}
\newcolumntype{C}[1]{>{\centering\let\newline\\\arraybackslash\hspace{0pt}}m{#1}}
\newcolumntype{R}[1]{>{\raggedleft\let\newline\\\arraybackslash\hspace{0pt}}m{#1}}
%%%%%%%%%%%%%
\oddsidemargin -0.5cm
\evensidemargin -0.5cm
\textwidth 17.5cm
\topmargin -1.5cm
\textheight 22cm
%%%%%%%%%%%% 

%\pagestyle{empty}

\newcommand{\semestre}{2018.2}

\newcommand{\disciplina}{INTRODUÇÃO À ENGENHARIA DE COMPUTAÇÃO}
\newcommand{\codigo}{DEC7530}


%%%%%%%%%%%%%%%%%%%%%%%%%%%%%%%%%%%%%%%%%%%%%%%%%%%%%%%
%%%%%%%%%%%%% CRETIDOS
\newcommand{\creditosT}{3}
\newcommand{\creditosP}{1}

%%%%%%%%%%%%%%%%%%%%%%%%%%%%%%%%%%%%%%%%%%%%%%%%%%%%%%%
%%%%%%%%%%%%%% REQUISITOS
\newcommand{\requisitoA}{}
\newcommand{\requisitoB}{}
\newcommand{\requisitoC}{}

%%%%%%%%%%%%%%%%%%%%%%%%%%%%%%%%%%%%%%%%%%%%%%%%%%%%%%%
%%%%%%%%%%%%%%% Atende aos Cursos
\newcommand{\cursoA}{Graduação em Engenharia de Computação \\ \hline}
\newcommand{\cursoB}{}%Graduação em Tecnologias da Informação e Comunicação \\ \hline}
\newcommand{\cursoC}{}%Graduação em Engenharia de Energia \\ \hline}

%%%%%%%%%%%%%%%%%%%%%%%%%%%%%%%%%%%%%%%%%%%%%%%%%%%%%%%%
%%%%%%%%%% EMENTA
\newcommand{\ementa}{
Perfil do profissional da computação. Campo de atuação. Ética profissional. Regulamentação profissional. Estrutura e objetivos do curso. Histórico e evolução dos computadores. Introdução à computação. Características básicas dos computadores: hardware e software. Componentes básicos dos computadores: memória, unidade central de processamento, entrada e saída. Modelo de von Neumann. Software básico e programas aplicativos. Sistemas de numeração: representação numérica e conversão de base.
 \\ \hline
}


\begin{document}


%%%%%%%%%%%%%%%%%%%%%%%%%%%%%%%%%%%%%%%%%%%%%%%%%%%%%%%%%%%%%
\input cabecalho.tex



%%%%%%%%%%%%%%%%%%%%%%%%%%%%%%%%%%%%%%%%%%%%%%%%%%%%%%%%%%%%%
\begin{longtable}{|C{0.11\textwidth}|C{0.29\textwidth}|C{0.09\textwidth}|C{0.09\textwidth}|C{0.15\textwidth}|C{0.158\textwidth}|} \hline
%
\multicolumn{6}{|l|}{{\bf I. IDENTIFICAÇÃO DA DISCIPLINA}} \\ \hline
%
\multirow{3}*{{\small CÓDIGO}} & \multirow{3}*{NOME DA DISCIPLINA} &\multicolumn{2}{c|}{{\small N$^\circ$ DE HORAS-AULA }} & {{\small TOTAL DE}} & \multirow{3}*{{\small MODALIDADE}} \\ 
%
& & \multicolumn{2}{c|}{\small SEMANAIS}  & {\small HORAS-AULA} & \\ \cline{3-4}
%
& & {\tiny TEÓRICAS} & {\tiny PRÁTICAS} & {\small SEMESTRAIS} & \\ \hline
% codigo da disciplina carga horaria: teorica - pratica e total
{\bf \small \codigo} & {\bf \small \disciplina } & {\bf \creditosT} & {\bf \creditosP} & {\bf 72} & Presencial\\ \hline
\end{longtable}


%%%%%%%%%%%%%%%%%%%%%%%%%%%%%%%%%%%%%%%%%%%%%%%%%%%%%%%%%%%%%%
\begin{longtable}{|C{0.12\textwidth}|L{0.736\textwidth}|C{0.12\textwidth}|} \hline
%
\multicolumn{3}{|l|}{{\bf II. PRÉ-REQUISITO(S)}} \\ \hline
%
CÓDIGO & NOME DA DISCIPLINA & CURSO \\ \hline	
%
\requisitoA
\requisitoB
\requisitoC
\end{longtable}


%%%%%%%%%%%%%%%%%%%%%%%%%%%%%%%%%%%%%%%%%%%%%%%%%%%%%%%%%%%%%%
\begin{longtable}{|L{1.025\textwidth}|} \hline
%
{\bf III. CURSO(S) PARA O(S) QUAL(IS) A DISCIPLINA É OFERECIDA } \\ \hline
%
\cursoA 
\cursoB
\cursoC

\end{longtable}

%%%%%%%%%%%%%%%%%%%%%%%%%%%%%%%%%%%%%%%%%%%%%%%%%%%%%%%%%%%%%%
\begin{longtable}{|L{1.025\textwidth}|} \hline
%
{\bf IV. EMENTA } \\ \hline
%
\ementa
\end{longtable}

%\newpage



%%%%%%%%%%%%%%%%%%%%%%%%%%%%%%%%%%%%%%%%%%%%%%%%%%%%%%%%%%%%%%%
\begin{longtable}{|L{1.025\textwidth}|} \hline
%
{\bf V. OBJETIVOS } \\ \hline
%
Objetivo Geral: \\
Fornecer ao aluno ingressante no curso de Engenharia de Computação uma visão geral acerca das principais áreas de atuação, competências, habilidades e o perfil do egresso do profissional de Engenharia de Computação.\\
\\
Objetivos Específicos:
\begin{itemize}
\item Fornecer aos alunos uma visão dos cursos de graduação em Engenharia de Computação: estrutura curricular, ênfases, mercado de atuação, etc;
\item Capacitar o aluno a conhecer a estrutura básica de um computador, seu funcionamento e aplicações;
\item Permitir ao aluno ter uma visão crítica sobre as áreas de atuação e a relação entre elas.
\end{itemize}
\\ \hline
\end{longtable}


%%%%%%%%%%%%%%%%%%%%%%%%%%%%%%%%%%%%%%%%%%%%%%%%%%%%%%%%%%%%%%%
\begin{longtable}{|L{1.025\textwidth}|} \hline
%
{\bf VI. CONTEÚDO PROGRAMÁTICO } \\ \hline
Conteúdo Teórico seguido de Conteúdo Prático com desenvolvimento de problemas em computador: \\
\\
UNIDADE 1: Introdução [4 horas-aula]\\
Introdução à Computação\\
Histórico da Computação\\
Sobre a Universidade Federal de Santa Catarina\\
Estrutura do Curso de Engenharia de Computação da UFSC\\
\\
UNIDADE 2: Atuação Profissional [4 horas-aula]\\
Áreas de atuação em computação\\
Regulamentação da profissão\\
Ética profissional\\
Engenharia: ser engenheiro\\
Projetos em Engenharia\\
\\
UNIDADE 3: Estrutura de Computadores [12 horas-aula]\\
Evolução dos computadores\\
Estrutura Interna (memória, unidade de processamento, barramentos)\\
Sistemas de Numeração (base binária, base octal e base hexadecimal)\\
Conversão de base\\
Hardware versus software\\
Introdução ao software básico e sistemas operacionais\\
Programação em linguagem de montagem\\
\\
UNIDADE 4: Experimentos com Sistemas Microcontrolados [16 horas-aula]\\
Introdução ao Arduino\\
Simulação de Circuitos Elétricos\\
Programação em Arduino\\
Experimentos com Arduino\\
\\
UNIDADE 5: Experimentos com Robôs Móveis [16 horas-aula]\\
Introdução a robótica\\
Programação de robôs móveis\\
\\
UNIDADE 6: Jogos e Programação Scratch [20 horas-aula]\\
Introdução a Jogos digitais\\
Programação em Scratch\\
Experimentos com Scratch e realidade aumentada.\\
\\ \hline
\end{longtable} 

%\newpage

%%%%%%%%%%%%%%%%%%%%%%%%%%%%%%%%%%%%%%%%%%%%%%%%%%%%%%%%%%%%%%%
\begin{longtable}{|L{1.025\textwidth}|} \hline
%
{\bf VII. BIBLIOGRAFIA BÁSICA} \\ \hline
\begin{enumerate}
%
\item CAPRON, H. L.; JOHNSON, J. A. Introdução à informática. São Paulo: Ed. Pearson, 2004.
\item NORTON, Peter. Introdução à informática. São Paulo: Ed. Pearson, 2004. 
\item Mokarzel,Fabio/Som. Introdução à Ciência da Computação. São Paulo.Ed. Campus/Elsevier. 2008
\end{enumerate}
 \\ \hline
\end{longtable}


%\newpage

%%%%%%%%%%%%%%%%%%%%%%%%%%%%%%%%%%%%%%%%%%%%%%%%%%%%%%%%%%%%%%%
\begin{longtable}{|L{1.025\textwidth}|} \hline
%
{\bf VIII. BIBLIOGRAFIA COMPLEMENTAR} \\ \hline
\begin{enumerate}
\item Unplugged. (2013). Computer Science Unplugged. Disponível em: http://csunplugged.org/projects/. Acesso em 25/01/2016. 
\item MONTEIRO, M. A. Introdução à organização de computadores. 5. Ed. Rio de Janeiro: LTC, 2007. 
\item MURDOCCA, M.J.; HEURING V.P. Introdução à arquitetura de computadores. Rio de Janeiro: Campus, 2001.
\item BAZZO, Walter Antonio; PEREIRA, Luiz Teixeira do Vale. Introdução à engenharia: conceitos, ferramentas e comportamentos. 3. ed. Florianópolis: Ed. da UFSC, 2010. 251 p. (Série didática). ISBN 9788532805898.
\item BROCKMAN, Jay B. Introdução à engenharia: modelagem e solução de problemas. Rio de Janeiro: LTC, c2010. xvii, 294 p. ISBN 9788521617266.
\end{enumerate}
 \\ \hline
\end{longtable}


\input aprovacao.tex


\end{document}
