\documentclass[12pt]{article}
\usepackage[brazil]{babel}
\usepackage{graphicx,t1enc,wrapfig,amsmath,float}
\usepackage{framed,fancyhdr}
\usepackage{multirow}
\usepackage{longtable}
\usepackage{array}
\newcolumntype{L}[1]{>{\raggedright\let\newline\\\arraybackslash\hspace{0pt}}m{#1}}
\newcolumntype{C}[1]{>{\centering\let\newline\\\arraybackslash\hspace{0pt}}m{#1}}
\newcolumntype{R}[1]{>{\raggedleft\let\newline\\\arraybackslash\hspace{0pt}}m{#1}}
%%%%%%%%%%%%%
\oddsidemargin -0.5cm
\evensidemargin -0.5cm
\textwidth 17.5cm
\topmargin -1.5cm
\textheight 22cm
%%%%%%%%%%%% 

%\pagestyle{empty}

\newcommand{\semestre}{2018.2}

\newcommand{\disciplina}{ENGENHARIA DE SOFTWARE II}
\newcommand{\codigo}{DEC7130}


%%%%%%%%%%%%%%%%%%%%%%%%%%%%%%%%%%%%%%%%%%%%%%%%%%%%%%%
%%%%%%%%%%%%% CRETIDOS
\newcommand{\creditosT}{3}
\newcommand{\creditosP}{1}

%%%%%%%%%%%%%%%%%%%%%%%%%%%%%%%%%%%%%%%%%%%%%%%%%%%%%%%
%%%%%%%%%%%%%% REQUISITOS
\newcommand{\requisitoA}{DEC7124 & ENGENHARIA DE SOFTWARE I & TIC \\ \hline}
\newcommand{\requisitoB}{}
\newcommand{\requisitoC}{}

%%%%%%%%%%%%%%%%%%%%%%%%%%%%%%%%%%%%%%%%%%%%%%%%%%%%%%%
%%%%%%%%%%%%%%% Atende aos Cursos
\newcommand{\cursoB}{Graduação em Engenharia de Computação \\ \hline}
\newcommand{\cursoA}{Graduação em Tecnologias da Informação e Comunicação \\ \hline}
\newcommand{\cursoC}{}

%%%%%%%%%%%%%%%%%%%%%%%%%%%%%%%%%%%%%%%%%%%%%%%%%%%%%%%%
%%%%%%%%%% EMENTA
\newcommand{\ementa}{
Evolução da prática de desenvolvimento de software; Critérios de qualidade de artefatos de software; 94 modelos de ciclo de vida; metodologias de desenvolvimento de software; manutenção de software; engenharia reversa; modelagem formal de sistemas; abordagens voltadas ao reuso de software; teste de software; gerenciamento do processo de produção de software e técnicas de apoio ao gerenciamento do processo de produção de software; apoio automatizado ao desenvolvimento de software.
 \\ \hline
}


\begin{document}
%%%%%%%%%%%%%%%%%%%%%%%%%%%%%%%%%%%%%%%%%%%%%%%%%%%%%%%%%%%%%
\input cabecalho.tex



%%%%%%%%%%%%%%%%%%%%%%%%%%%%%%%%%%%%%%%%%%%%%%%%%%%%%%%%%%%%%
\begin{longtable}{|C{0.11\textwidth}|C{0.29\textwidth}|C{0.09\textwidth}|C{0.09\textwidth}|C{0.15\textwidth}|C{0.158\textwidth}|} \hline
%
\multicolumn{6}{|l|}{{\bf I. IDENTIFICAÇÃO DA DISCIPLINA}} \\ \hline
%
\multirow{3}*{{\small CÓDIGO}} & \multirow{3}*{NOME DA DISCIPLINA} &\multicolumn{2}{c|}{{\small N$^\circ$ DE HORAS-AULA }} & {{\small TOTAL DE}} & \multirow{3}*{{\small MODALIDADE}} \\ 
%
& & \multicolumn{2}{c|}{\small SEMANAIS}  & {\small HORAS-AULA} & \\ \cline{3-4}
%
& & {\tiny TEÓRICAS} & {\tiny PRÁTICAS} & {\small SEMESTRAIS} & \\ \hline
% codigo da disciplina carga horaria: teorica - pratica e total
{\bf \small \codigo} & {\bf \small \disciplina } & {\bf \creditosT} & {\bf \creditosP} & {\bf 72} & Presencial\\ \hline
\end{longtable}


%%%%%%%%%%%%%%%%%%%%%%%%%%%%%%%%%%%%%%%%%%%%%%%%%%%%%%%%%%%%%%
\begin{longtable}{|C{0.12\textwidth}|L{0.736\textwidth}|C{0.12\textwidth}|} \hline
%
\multicolumn{3}{|l|}{{\bf II. PRÉ-REQUISITO(S)}} \\ \hline
%
CÓDIGO & NOME DA DISCIPLINA & CURSO \\ \hline	
%
\requisitoA
\requisitoB
\requisitoC
\end{longtable}


%%%%%%%%%%%%%%%%%%%%%%%%%%%%%%%%%%%%%%%%%%%%%%%%%%%%%%%%%%%%%%
\begin{longtable}{|L{1.025\textwidth}|} \hline
%
{\bf III. CURSO(S) PARA O(S) QUAL(IS) A DISCIPLINA É OFERECIDA } \\ \hline
%
\cursoA 
\cursoB
\cursoC

\end{longtable}

%%%%%%%%%%%%%%%%%%%%%%%%%%%%%%%%%%%%%%%%%%%%%%%%%%%%%%%%%%%%%%
\begin{longtable}{|L{1.025\textwidth}|} \hline
%
{\bf IV. EMENTA } \\ \hline
%
\ementa
\end{longtable}

%\newpage



%%%%%%%%%%%%%%%%%%%%%%%%%%%%%%%%%%%%%%%%%%%%%%%%%%%%%%%%%%%%%%%
\begin{longtable}{|L{1.025\textwidth}|} \hline
%
{\bf V. OBJETIVOS } \\ \hline
%
Objetivo Geral:\\

Fornecer subsídios ao aluno para que ele possa compreender os processos de desenvolvimento, implementação e manutenção de software.\\
\\
Objetivos Específicos:\\
\begin{itemize}
\item O aluno ao final do curso deve possuir habilidades para:
\item Definir engenharia de software explicitando seus conceitos e objetivos; 
\item Conhecer e aplicar o conceito destinado aos processos de software; 
\item Conhecer os modelos de ciclo de vida;
\item Entender o que é um software de qualidade e conhecer as métricas existentes.
\end{itemize}
\\ \hline
\end{longtable}


%%%%%%%%%%%%%%%%%%%%%%%%%%%%%%%%%%%%%%%%%%%%%%%%%%%%%%%%%%%%%%%
\begin{longtable}{|L{1.025\textwidth}|} \hline
%
{\bf VI. CONTEÚDO PROGRAMÁTICO } \\ \hline
Parte I: Fundamentos em Engenharia de Software\\
Visão geral da engenharia de software\\
Desenvolvimento de processo de software\\
Verificação e Validação de software\\
\\
Parte II: Avaliação de Software\\
Manutenção de software e suas variações\\
Evolução de software e sistemas legados\\
Qualidade e certificações\\
\\
Parte III: Gerenciamento\\
Gerenciamento da qualidade\\
Gerenciamento de projeto de software\\
Aprimoramento dos processos
\\ \hline
\end{longtable} 





%%%%%%%%%%%%%%%%%%%%%%%%%%%%%%%%%%%%%%%%%%%%%%%%%%%%%%%%%%%%%%%
\begin{longtable}{|L{1.025\textwidth}|} \hline
%
{\bf VII. BIBLIOGRAFIA BÁSICA} \\ \hline
\begin{enumerate}
%
\item BECK, K. Programação extrema (xp) explicada: acolha as mudanças. Porto Alegre: Bookman, 2004. 182p. 
\item GAMMA, E. ET AL. Padrões de projeto: soluções reutilizáveis de software orientado a objetos. Porto Alegre: Bookman, 2000. 364p 
\item SOMMERVILLE, I. Engenharia de software. 8. ed. São Paulo: Pearson Addison-Wesley, 2007. xiv, 552 p.

\end{enumerate}
 \\ \hline
\end{longtable}


\newpage

%%%%%%%%%%%%%%%%%%%%%%%%%%%%%%%%%%%%%%%%%%%%%%%%%%%%%%%%%%%%%%%
\begin{longtable}{|L{1.025\textwidth}|} \hline
%
{\bf VIII. BIBLIOGRAFIA COMPLEMENTAR} \\ \hline
\begin{enumerate}
\item BOOCH, G.; RUMBAUGH, J.; JACOBSON, I. UML - Guia do Usuário. 2. ed. Rio de Janeiro: Campus, 2006. 
\item LARMAN, C. Utilizando UML e padrões: uma introdução à análise e ao projetos orientados a objetos e ao desenvolvimento interativo. 3. ed. Porto Alegre: Bookman, 2007. 
\item MENDES, E.; MOSLEY, N. Web Engineering. New York: Springer, 2007. 
\item PAULA FILHO, W. Engenharia de software: fundamentos, métodos e padrões. 3.ed. Rio de Janeiro: LTC, c2009. 
\item PRESSMAN, Roger S. Engenharia de software: uma abordagem profissional.7. ed. Porto Alegre: AMGH, 2011. 780 p.

%
\end{enumerate}
 \\ \hline
\end{longtable}


\input aprovacao.tex


\end{document}
