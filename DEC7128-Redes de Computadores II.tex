\documentclass[12pt]{article}
\usepackage[brazil]{babel}
\usepackage{graphicx,t1enc,wrapfig,amsmath,float}
\usepackage{framed,fancyhdr}
\usepackage{multirow}
\usepackage{longtable}
\usepackage{array}
\newcolumntype{L}[1]{>{\raggedright\let\newline\\\arraybackslash\hspace{0pt}}m{#1}}
\newcolumntype{C}[1]{>{\centering\let\newline\\\arraybackslash\hspace{0pt}}m{#1}}
\newcolumntype{R}[1]{>{\raggedleft\let\newline\\\arraybackslash\hspace{0pt}}m{#1}}
%%%%%%%%%%%%%
\oddsidemargin -0.5cm
\evensidemargin -0.5cm
\textwidth 17.5cm
\topmargin -1.5cm
\textheight 22cm
%%%%%%%%%%%% 

%\pagestyle{empty}

\newcommand{\semestre}{2018.2}

\newcommand{\disciplina}{REDES DE COMPUTADORES II}
\newcommand{\codigo}{DEC7128}


%%%%%%%%%%%%%%%%%%%%%%%%%%%%%%%%%%%%%%%%%%%%%%%%%%%%%%%
%%%%%%%%%%%%% CRETIDOS
\newcommand{\creditosT}{2}
\newcommand{\creditosP}{2}

%%%%%%%%%%%%%%%%%%%%%%%%%%%%%%%%%%%%%%%%%%%%%%%%%%%%%%%
%%%%%%%%%%%%%% REQUISITOS
\newcommand{\requisitoA}{DEC7126 & REDES DE COMPUTADORES I & TIC \\ \hline}
\newcommand{\requisitoB}{}
\newcommand{\requisitoC}{}

%%%%%%%%%%%%%%%%%%%%%%%%%%%%%%%%%%%%%%%%%%%%%%%%%%%%%%%
%%%%%%%%%%%%%%% Atende aos Cursos
\newcommand{\cursoB}{}%Graduação em Engenharia de Computação. \\ \hline}
\newcommand{\cursoA}{Graduação em Tecnologias da Informação e Comunicação \\ \hline}
\newcommand{\cursoC}{}

%%%%%%%%%%%%%%%%%%%%%%%%%%%%%%%%%%%%%%%%%%%%%%%%%%%%%%%%
%%%%%%%%%% EMENTA
\newcommand{\ementa}{
Introdução. Redes sem fio e redes móveis. Roteamento. Administração de Redes de Computadores. Segurança. Aplicações.
 \\ \hline
}




\begin{document}


%%%%%%%%%%%%%%%%%%%%%%%%%%%%%%%%%%%%%%%%%%%%%%%%%%%%%%%%%%%%%
\input cabecalho.tex

%%%%%%%%%%%%%%%%%%%%%%%%%%%%%%%%%%%%%%%%%%%%%%%%%%%%%%%%%%%%%
\begin{longtable}{|C{0.11\textwidth}|C{0.29\textwidth}|C{0.09\textwidth}|C{0.09\textwidth}|C{0.15\textwidth}|C{0.158\textwidth}|} \hline
%
\multicolumn{6}{|l|}{{\bf I. IDENTIFICAÇÃO DA DISCIPLINA}} \\ \hline
%
\multirow{3}*{{\small CÓDIGO}} & \multirow{3}*{NOME DA DISCIPLINA} &\multicolumn{2}{c|}{{\small N$^\circ$ DE HORAS-AULA }} & {{\small TOTAL DE}} & \multirow{3}*{{\small MODALIDADE}} \\ 
%
& & \multicolumn{2}{c|}{\small SEMANAIS}  & {\small HORAS-AULA} & \\ \cline{3-4}
%
& & {\tiny TEÓRICAS} & {\tiny PRÁTICAS} & {\small SEMESTRAIS} & \\ \hline
% codigo da disciplina carga horaria: teorica - pratica e total
{\bf \small \codigo} & {\bf \small \disciplina } & {\bf \creditosT} & {\bf \creditosP} & {\bf 72} & Presencial\\ \hline
\end{longtable}


%%%%%%%%%%%%%%%%%%%%%%%%%%%%%%%%%%%%%%%%%%%%%%%%%%%%%%%%%%%%%%
\begin{longtable}{|C{0.12\textwidth}|L{0.736\textwidth}|C{0.12\textwidth}|} \hline
%
\multicolumn{3}{|l|}{{\bf II. PRÉ-REQUISITO(S)}} \\ \hline
%
CÓDIGO & NOME DA DISCIPLINA & CURSO \\ \hline	
%
\requisitoA
\requisitoB
\requisitoC
\end{longtable}


%%%%%%%%%%%%%%%%%%%%%%%%%%%%%%%%%%%%%%%%%%%%%%%%%%%%%%%%%%%%%%
\begin{longtable}{|L{1.025\textwidth}|} \hline
%
{\bf III. CURSO(S) PARA O(S) QUAL(IS) A DISCIPLINA É OFERECIDA } \\ \hline
%
\cursoA 
\cursoB
\cursoC

\end{longtable}

%%%%%%%%%%%%%%%%%%%%%%%%%%%%%%%%%%%%%%%%%%%%%%%%%%%%%%%%%%%%%%
\begin{longtable}{|L{1.025\textwidth}|} \hline
%
{\bf IV. EMENTA } \\ \hline
%
\ementa
\end{longtable}

%\newpage



%%%%%%%%%%%%%%%%%%%%%%%%%%%%%%%%%%%%%%%%%%%%%%%%%%%%%%%%%%%%%%%
\begin{longtable}{|L{1.025\textwidth}|} \hline
%
{\bf V. OBJETIVOS } \\ \hline
%
Objetivo Geral:\\

Apresentar os principais conceitos relativos à Rede Internet, analisar e elucidar os assuntos relacionados a Administração e Gerência de Redes.\\
\\
Objetivos Específicos: \\
\begin{itemize}
\item Descrever os principais aspectos de operação dos protocolos dos diferentes níveis da Arquitetura Internet.
\item Apresentar a política de endereçamento da Internet.
\item Apresentar os conceitos de gerenciamento na Internet e os protocolos associados.
\item Apresentar as principais tecnologias de redes locais sem fio.
\item Apresentar, analisar e usar tecnologias e suporte para Gerência de Redes.
\item Apresentar os principais conceitos de segurança em Redes.
\item Pesquisar sobre Tendências e Futuro em Administração e Gerência de Redes.
\end{itemize}

\\ \hline
\end{longtable}


%%%%%%%%%%%%%%%%%%%%%%%%%%%%%%%%%%%%%%%%%%%%%%%%%%%%%%%%%%%%%%%
\begin{longtable}{|L{1.025\textwidth}|} \hline
%
{\bf VI. CONTEÚDO PROGRAMÁTICO } \\ \hline
UNIDADE 1 - Introdução [2 ha]\\
Evolução da Rede Internet ao longo do mundo\\
Situação atual\\
Níveis da Arquitetura Internet\\
\\
UNIDADE 2 - Redes sem fio e redes móveis [16 ha]\\
Serviços oferecidos pela camada de enlace\\
Redes sem fio IEEE 802.11\\
Redes móveis IEEE 802.15.4 e Bluetooth\\
Redes de telefonia móvel (celulares)\\
\\
UNIDADE 3 - Nível de Redes e seus conceitos [22 ha]\\
Roteamento na Internet\\
IPv6\\
\\
UNIDADE 4 - Gerenciamento de Redes na Internet [24 ha]\\
Noções de Gerenciamento de Redes\\
Gerenciando a Rede Internet.\\
Ferramentas de Gerenciamento para Internet\\
\\
UNIDADE 5 - Gerenciamento de Segurança [8 ha]\\
Noções de segurança em redes\\
Princípios da criptografia\\
Integridade de mensagem e autenticação\\
Estudo de casos
\\ \hline
\end{longtable} 





%%%%%%%%%%%%%%%%%%%%%%%%%%%%%%%%%%%%%%%%%%%%%%%%%%%%%%%%%%%%%%%
\begin{longtable}{|L{1.025\textwidth}|} \hline
%
{\bf VII. BIBLIOGRAFIA BÁSICA} \\ \hline
\begin{enumerate}
%
\item KUROSE, James F; ROSS, Keith W. Redes de computadores e a Internet: uma abordagem topdow. 5. ed. São Paulo: Addison Wesley, 2010.
\item TANENBAUM, Andrew S. Redes de computadores. 4. ed. Rio de Janeiro: Elsevier, 2003.
\item COMER, Douglas. Interligação em rede com TCP/IP. Volume 1: princípios, protocolos e arquitetura. Rio de Janeiro: Campus, 2006.
%

\end{enumerate}
 \\ \hline
\end{longtable}


%\newpage

%%%%%%%%%%%%%%%%%%%%%%%%%%%%%%%%%%%%%%%%%%%%%%%%%%%%%%%%%%%%%%%
\begin{longtable}{|L{1.025\textwidth}|} \hline
%
{\bf VIII. BIBLIOGRAFIA COMPLEMENTAR} \\ \hline
\begin{enumerate}
\item CARISSIMI, A. S.; ROCHOL, J.; GRANVILLE, L. Z. Redes de Computadores. Porto Alegre: Bookman, 2009.
\item SOARES, Luiz Fernando Gomes; LEMOS, Guido; COLCHER, Sergio. Redes de Computadores: Das LANs, MANs e WANs, às Redes ATM. Rio de Janeiro: Editora Campus, 1995.
\item STALLINGS, W. Redes e Sistemas de Comunicação de Dados, Rio de Janeiro: Elsevier. 5. Edicao, 2005.
\item TORRES, Gabriel. Redes de Computadores. Rio de Janeiro: Nova Terra, 2009. 
\item MARIN, Paulo S. Cabeamento estruturado: desvendando cada passo : do projeto à instalação. 4. ed. rev. e atual. São Paulo: Érica, 2014. 336 p.
%
\end{enumerate}
 \\ \hline
\end{longtable}


\input aprovacao.tex


\end{document}
