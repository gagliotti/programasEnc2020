\documentclass[12pt]{article}
\usepackage[brazil]{babel}
\usepackage{graphicx,t1enc,wrapfig,amsmath,float}
\usepackage{framed,fancyhdr}
\usepackage{multirow}
\usepackage{longtable}
\usepackage{array}
\newcolumntype{L}[1]{>{\raggedright\let\newline\\\arraybackslash\hspace{0pt}}m{#1}}
\newcolumntype{C}[1]{>{\centering\let\newline\\\arraybackslash\hspace{0pt}}m{#1}}
\newcolumntype{R}[1]{>{\raggedleft\let\newline\\\arraybackslash\hspace{0pt}}m{#1}}
%%%%%%%%%%%%%
\oddsidemargin -0.5cm
\evensidemargin -0.5cm
\textwidth 17.5cm
\topmargin -1.5cm
\textheight 22cm
%%%%%%%%%%%% 

%\pagestyle{empty}

\newcommand{\semestre}{2018.2}

\newcommand{\disciplina}{BANCO DE DADOS II}
\newcommand{\codigo}{DEC7134}


%%%%%%%%%%%%%%%%%%%%%%%%%%%%%%%%%%%%%%%%%%%%%%%%%%%%%%%
%%%%%%%%%%%%% CRETIDOS
\newcommand{\creditosT}{2}
\newcommand{\creditosP}{2}

%%%%%%%%%%%%%%%%%%%%%%%%%%%%%%%%%%%%%%%%%%%%%%%%%%%%%%%
%%%%%%%%%%%%%% REQUISITOS
\newcommand{\requisitoA}{DEC7129 & BANCO DE DADOS I &TIC \\ \hline }
\newcommand{\requisitoB}{}%CIT7584 & ESTRUTURA DE DADOS E ALGORITMOS \\ \hline}
\newcommand{\requisitoC}{}

%%%%%%%%%%%%%%%%%%%%%%%%%%%%%%%%%%%%%%%%%%%%%%%%%%%%%%%
%%%%%%%%%%%%%%% Atende aos Cursos
\newcommand{\cursoA}{}%Graduação em Engenharia de Computação \\ \hline}
\newcommand{\cursoB}{Graduação em Tecnologias da Informação e Comunicação \\ \hline}
\newcommand{\cursoC}{}

%%%%%%%%%%%%%%%%%%%%%%%%%%%%%%%%%%%%%%%%%%%%%%%%%%%%%%%%
%%%%%%%%%% EMENTA
\newcommand{\ementa}{
SQL embutida: instruções estáticas e dinâmicas, cursores. Processamento de consultas: otimização algébrica; plano de execução. Transações: definição, propriedades, estados. Recuperação de falhas: categorias de falhas, gerência de buffer, técnicas de recuperação. Controle de concorrência. Noções básicas de bancos de dados distribuídos. Tópicos avançados em Banco de Dados.
 \\ \hline
}




\begin{document}


%%%%%%%%%%%%%%%%%%%%%%%%%%%%%%%%%%%%%%%%%%%%%%%%%%%%%%%%%%%%%
\input cabecalho.tex

%%%%%%%%%%%%%%%%%%%%%%%%%%%%%%%%%%%%%%%%%%%%%%%%%%%%%%%%%%%%%
\begin{longtable}{|C{0.11\textwidth}|C{0.29\textwidth}|C{0.09\textwidth}|C{0.09\textwidth}|C{0.15\textwidth}|C{0.158\textwidth}|} \hline
%
\multicolumn{6}{|l|}{{\bf I. IDENTIFICAÇÃO DA DISCIPLINA}} \\ \hline
%
\multirow{3}*{{\small CÓDIGO}} & \multirow{3}*{NOME DA DISCIPLINA} &\multicolumn{2}{c|}{{\small N$^\circ$ DE HORAS-AULA }} & {{\small TOTAL DE}} & \multirow{3}*{{\small MODALIDADE}} \\ 
%
& & \multicolumn{2}{c|}{\small SEMANAIS}  & {\small HORAS-AULA} & \\ \cline{3-4}
%
& & {\tiny TEÓRICAS} & {\tiny PRÁTICAS} & {\small SEMESTRAIS} & \\ \hline
% codigo da disciplina carga horaria: teorica - pratica e total
{\bf \small \codigo} & {\bf \small \disciplina } & {\bf \creditosT} & {\bf \creditosP} & {\bf 72} & Presencial\\ \hline
\end{longtable}


%%%%%%%%%%%%%%%%%%%%%%%%%%%%%%%%%%%%%%%%%%%%%%%%%%%%%%%%%%%%%%
\begin{longtable}{|C{0.12\textwidth}|L{0.736\textwidth}|C{0.12\textwidth}|} \hline
%
\multicolumn{3}{|l|}{{\bf II. PRÉ-REQUISITO(S)}} \\ \hline
%
CÓDIGO & NOME DA DISCIPLINA & CURSO \\ \hline	
%
\requisitoA
\requisitoB
\requisitoC
\end{longtable}


%%%%%%%%%%%%%%%%%%%%%%%%%%%%%%%%%%%%%%%%%%%%%%%%%%%%%%%%%%%%%%
\begin{longtable}{|L{1.025\textwidth}|} \hline
%
{\bf III. CURSO(S) PARA O(S) QUAL(IS) A DISCIPLINA É OFERECIDA } \\ \hline
%
\cursoA 
\cursoB
\cursoC

\end{longtable}

%%%%%%%%%%%%%%%%%%%%%%%%%%%%%%%%%%%%%%%%%%%%%%%%%%%%%%%%%%%%%%
\begin{longtable}{|L{1.025\textwidth}|} \hline
%
{\bf IV. EMENTA } \\ \hline
%
\ementa
\end{longtable}

\newpage



%%%%%%%%%%%%%%%%%%%%%%%%%%%%%%%%%%%%%%%%%%%%%%%%%%%%%%%%%%%%%%%
\begin{longtable}{|L{1.025\textwidth}|} \hline
%
{\bf V. OBJETIVOS } \\ \hline
%
Objetivo Geral:\\

Prover ao aluno conhecimentos que possibilitem um entendimento sólido sobre banco de dados de modo que este possa atuar no gerenciamento, na manutenção e no desenvolvimento de soluções de banco de dados.\\
\\
Objetivos Específicos 
\begin{itemize}
\item Apresentar as principais estruturas de um banco de dados;
\item Apresentar os recursos existentes para o suporte ao desenvolvimento de aplicações que manipulem informações disponíveis em um banco de dados;
\item Abordar os conceitos que possibilitem o gerenciamento de banco de dados;
\item Apresentar as novas tecnologias na área de banco de dados.
\end{itemize}

\\ \hline
\end{longtable}


%%%%%%%%%%%%%%%%%%%%%%%%%%%%%%%%%%%%%%%%%%%%%%%%%%%%%%%%%%%%%%%
\begin{longtable}{|L{1.025\textwidth}|} \hline
%
{\bf VI. CONTEÚDO PROGRAMÁTICO } \\ \hline
Conteúdo Teórico seguido de Conteúdo Prático com foco no estudo das estruturas que suportam um banco de dados, na manutenção de banco de dados através e na apresentação das novas tecnologias dessa área: \\
\\
UNIDADE 1: SQL avançada [8 horas-aula] \\
- SQL embutida\\
- SQL dinâmica\\
- Procedimento Armazenado\\
\\
UNIDADE 2: Organização física de banco de dados [8 horas-aula] \\
- Armazenamento e estrutura de arquivos\\
- Indexação e hashing\\
\\
UNIDADE 3: Processamento de consultas [8 horas-aula]\\
- Visão geral\\
- Custo da consulta \\
- Otimização de consultas\\
\\
UNIDADE 4: Gerenciamento de transações [4 horas-aula]\\
- Definição de transações\\
- Propriedades e estados\\
\\
UNIDADE 5: Recuperação de falhas (Sistema de recuperação) [4 horas-aula] \\
- Classificação das falhas\\
- Gerenciamento de buffer\\
- Técnicas de recuperação\\
\\
UNIDADE 6: Controle de concorrência [4 horas-aula] \\
- Protocolos\\
- Tratamento de impasse\\
- Concorrência em estruturas de índices\\
\\
UNIDADE 7: Noções básicas de banco de dados distribuído [4 horas-aula]\\
- Banco de dados paralelos\\
- Banco de dados homogêneos e heterogêneos\\
- Armazenamento de dados distribuídos\\
- Transações distribuídas\\
\\
UNIDADE 8: Tópicos avançados em banco de dados [10 horas-aula]\\
\\
UNIDADE 9: Seminários [10 horas-aula]
\\ \hline
\end{longtable} 



%%%%%%%%%%%%%%%%%%%%%%%%%%%%%%%%%%%%%%%%%%%%%%%%%%%%%%%%%%%%%%%
\begin{longtable}{|L{1.025\textwidth}|} \hline
%
{\bf VII. BIBLIOGRAFIA BÁSICA} \\ \hline
\begin{enumerate}
%
\item SILBERSCHATZ, A.; KORTH, H. F.; SUDARSHAN, S. Sistema de bancos de dados. 5. ed. Rio de Janeiro: Elsevier, 2006. 
\item ELMASRI, R; NAVATHE, S. B. Sistemas de banco de dados. 6. ed. São Paulo: Pearson Addison Wesley, 2011. 
\item DATE, C. J. Introdução a sistemas de bancos de dados. 8. ed. Rio de Janeiro: Elsevier, 2004.
\end{enumerate}
 \\ \hline
\end{longtable}


\newpage

%%%%%%%%%%%%%%%%%%%%%%%%%%%%%%%%%%%%%%%%%%%%%%%%%%%%%%%%%%%%%%%
\begin{longtable}{|L{1.025\textwidth}|} \hline
%
{\bf VIII. BIBLIOGRAFIA COMPLEMENTAR} \\ \hline
\begin{enumerate}
\item GARCIA-MOLINA, Hector; ULLMAN, Jeffrey D.; WIDOM, Jennifer. Database systems: The 108 complete book. 2. ed. Prentice Hall, 2008. 
\item HOFFER, Jeffrey A.; PRESCOTT, Mary; TOPI, Heikki. Modern database management. 9. ed. Prentice Hall, 2008. 
\item O'NEIL, P.; O'NEIL, E. Database: principles, programming, and performance. 2. ed. Califórnia: Morgan Kaufmann, 2001. 
\item RAMAKRISHNAN, Raghu; GEHRKE, Johannes. Sistemas de gerenciamento de banco de dados. 3. ed. São Paulo: McGraw-Hill Medical, 2008. 
\item ULLMAN, J. D.; WIDOM, J. A. First course in database systems. New Jersey: Prentice-Hall, 2007.

%
\end{enumerate}
 \\ \hline
\end{longtable}


\input aprovacao.tex


\end{document}
