\documentclass[12pt]{article}
\usepackage[brazil]{babel}
\usepackage{graphicx,t1enc,wrapfig,amsmath,float}
\usepackage{framed,fancyhdr}
\usepackage{multirow}
\usepackage{longtable}
\usepackage{array}
\newcolumntype{L}[1]{>{\raggedright\let\newline\\\arraybackslash\hspace{0pt}}m{#1}}
\newcolumntype{C}[1]{>{\centering\let\newline\\\arraybackslash\hspace{0pt}}m{#1}}
\newcolumntype{R}[1]{>{\raggedleft\let\newline\\\arraybackslash\hspace{0pt}}m{#1}}
%%%%%%%%%%%%%
\oddsidemargin -0.5cm
\evensidemargin -0.5cm
\textwidth 17.5cm
\topmargin -1.5cm
\textheight 22cm
%%%%%%%%%%%% 

%\pagestyle{empty}

\newcommand{\semestre}{2018.2}

\newcommand{\disciplina}{SISTEMAS OPERACIONAIS}
\newcommand{\codigo}{DEC7131}


%%%%%%%%%%%%%%%%%%%%%%%%%%%%%%%%%%%%%%%%%%%%%%%%%%%%%%%
%%%%%%%%%%%%% CRETIDOS
\newcommand{\creditosT}{2}
\newcommand{\creditosP}{2}

%%%%%%%%%%%%%%%%%%%%%%%%%%%%%%%%%%%%%%%%%%%%%%%%%%%%%%%
%%%%%%%%%%%%%% REQUISITOS
\newcommand{\requisitoA}{CIT7244 & ESTRUTURA DE COMPUTADORES & TIC \\ \hline}
\newcommand{\requisitoB}{CIT7584 & ESTRUTURA DE DADOS E ALGORITMOS & TIC\\ \hline}
\newcommand{\requisitoC}{}

%%%%%%%%%%%%%%%%%%%%%%%%%%%%%%%%%%%%%%%%%%%%%%%%%%%%%%%
%%%%%%%%%%%%%%% Atende aos Cursos
\newcommand{\cursoB}{}%Graduação em Engenharia de Computação. \\ \hline}
\newcommand{\cursoA}{Graduação em Tecnologias da Informação e Comunicação \\ \hline}
\newcommand{\cursoC}{}

%%%%%%%%%%%%%%%%%%%%%%%%%%%%%%%%%%%%%%%%%%%%%%%%%%%%%%%%
%%%%%%%%%% EMENTA
\newcommand{\ementa}{
Histórico e evolução dos sistemas operacionais. Arquitetura de sistemas operacionais. Gerenciamento de processos. Gerenciamento de memória. Gerenciamento de dispositivos de entrada e saída. Sistemas de arquivos. Segurança em sistemas operacionais. Estudos de caso.
 \\ \hline
}


\begin{document}
%%%%%%%%%%%%%%%%%%%%%%%%%%%%%%%%%%%%%%%%%%%%%%%%%%%%%%%%%%%%%
\input cabecalho.tex


%%%%%%%%%%%%%%%%%%%%%%%%%%%%%%%%%%%%%%%%%%%%%%%%%%%%%%%%%%%%%
\begin{longtable}{|C{0.11\textwidth}|C{0.29\textwidth}|C{0.09\textwidth}|C{0.09\textwidth}|C{0.15\textwidth}|C{0.158\textwidth}|} \hline
%
\multicolumn{6}{|l|}{{\bf I. IDENTIFICAÇÃO DA DISCIPLINA}} \\ \hline
%
\multirow{3}*{{\small CÓDIGO}} & \multirow{3}*{NOME DA DISCIPLINA} &\multicolumn{2}{c|}{{\small N$^\circ$ DE HORAS-AULA }} & {{\small TOTAL DE}} & \multirow{3}*{{\small MODALIDADE}} \\ 
%
& & \multicolumn{2}{c|}{\small SEMANAIS}  & {\small HORAS-AULA} & \\ \cline{3-4}
%
& & {\tiny TEÓRICAS} & {\tiny PRÁTICAS} & {\small SEMESTRAIS} & \\ \hline
% codigo da disciplina carga horaria: teorica - pratica e total
{\bf \small \codigo} & {\bf \small \disciplina } & {\bf \creditosT} & {\bf \creditosP} & {\bf 72} & Presencial\\ \hline
\end{longtable}


%%%%%%%%%%%%%%%%%%%%%%%%%%%%%%%%%%%%%%%%%%%%%%%%%%%%%%%%%%%%%%
\begin{longtable}{|C{0.12\textwidth}|L{0.736\textwidth}|C{0.12\textwidth}|} \hline
%
\multicolumn{3}{|l|}{{\bf II. PRÉ-REQUISITO(S)}} \\ \hline
%
CÓDIGO & NOME DA DISCIPLINA & CURSO \\ \hline	
%
\requisitoA
\requisitoB
\requisitoC
\end{longtable}


%%%%%%%%%%%%%%%%%%%%%%%%%%%%%%%%%%%%%%%%%%%%%%%%%%%%%%%%%%%%%%
\begin{longtable}{|L{1.025\textwidth}|} \hline
%
{\bf III. CURSO(S) PARA O(S) QUAL(IS) A DISCIPLINA É OFERECIDA } \\ \hline
%
\cursoA 
\cursoB
\cursoC

\end{longtable}

%%%%%%%%%%%%%%%%%%%%%%%%%%%%%%%%%%%%%%%%%%%%%%%%%%%%%%%%%%%%%%
\begin{longtable}{|L{1.025\textwidth}|} \hline
%
{\bf IV. EMENTA } \\ \hline
%
\ementa
\end{longtable}

%\newpage



%%%%%%%%%%%%%%%%%%%%%%%%%%%%%%%%%%%%%%%%%%%%%%%%%%%%%%%%%%%%%%%
\begin{longtable}{|L{1.025\textwidth}|} \hline
%
{\bf V. OBJETIVOS } \\ \hline
%
Objetivo Geral:\\

Definir conceitos básicos e avançados de sistemas operacionais proporcionando aos discentes um conhecimento abrangente sobre o tema. Ao final da disciplina, o discente estará apto a reconhecer as principais características existentes em sistemas operacionais, bem como ser capaz de escolher um sistema operacional adequado para determinados tipos de aplicações. \\
\\
Objetivos Específicos:\\
\begin{itemize}
\item Apresentar os conceitos, finalidades e exemplos de sistemas operacionais;
\item Abordar conceitos sobre gerência de processos, memória, entrada e saída e sistemas de arquivos;
\item Fazer com que o discente obtenha conhecimento sobre as várias técnicas empregadas no projeto e implementação de um sistema operacional;
\item Implementar algoritmos para simular partes de um sistema operacional como a gerência de processos, gerência de memória e sistemas de arquivos.
\end{itemize}
\\ \hline
\end{longtable}


%%%%%%%%%%%%%%%%%%%%%%%%%%%%%%%%%%%%%%%%%%%%%%%%%%%%%%%%%%%%%%%
\begin{longtable}{|L{1.025\textwidth}|} \hline
%
{\bf VI. CONTEÚDO PROGRAMÁTICO } \\ \hline
UNIDADE1: Introdução [4 horas-aula]\\
Definição e Características de um Sistema Operacional\\
Estrutura de um Sistema Operacional\\
Serviços do Sistema Operacional\\
Chamadas de Sistemas\\
Projeto e Implementação do Sistema Operacional\\
Mecanismos e Políticas\\
Implementação\\
Estrutura do Sistema Operacional\\
Monolíticos\\
Camadas\\
Microkernels\\
Módulos\\
Máquinas virtuais\\
Cliente-servidor\\
\\

UNIDADE 2: Gerência de processos [32h-aula]\\
Conceito de Processos\\
Estados de um Processo\\
Bloco de Controle de Processos\\
Escalonamento de Processos\\
Troca de contexto\\
Criação de Processos\\
Comunicação entre Processos\\
Threads\\
Motivação para o uso de Threads\\
Modelos de Múltiplas Threads\\
Bibliotecas de Threads\\
Posix Threads - Pthreads\\
Windows Threads\\
Threads em Java\\
Aspectos do uso de Threads\\
Escalonamento de processos\\
Ciclos de CPU e ES (Entrada e Saída)\\
Conceitos de Preempção\\
Algoritmos de Escalonamento\\
First Come, First Served - FCFS\\
Shortest Job First - SJF
Escalonamento por Prioridade\\
Round-Robin\\
Filas Multinível\\
Escalonamento de Threads\\
Escalonamento em Múltiplos processadores\\
Programação concorrente\\
Sincronização de processos\\
Caracterização\\
Seção Crítica\\
Hardware de Sincronismo\\
Semáforos\\
Monitores\\
Problemas Clássicos de Sincronismo\\
Deadlock\\
Caracterização do Deadlock\\
Grafo de Alocação de Recursos\\
Métodos para Tratamento de Deadlocks\\
Prevenção de Deadlocks\\
Detecção de Deadlock\\
Recuperação do Deadlock\\
\\
UNIDADE 3: Gerência de memória [12h-aula]\\
Carregamento absoluto e carregamento relocado\\
Alocação contígua\\
Partições fixas\\
Partições variáveis\\
Alocação não-contígua\\
Paginação\\
Segmentação\\
Segmentação paginada\\
Memória virtual\\
Paginação por Demanda\\
Algoritmos de substituição de página\\
Trashing\\
\\
UNIDADE 4: Sistemas de arquivos [12h-aula]\\
Arquivos e diretórios\\
Estruturação de arquivos\\
Segurança em sistemas de arquivos\\
Implementação de sistemas de arquivos\\
Alocação de espaço em disco\\
Alocação contígua\\
Alocação encadeada\\
Alocação indexada\\
Gerência de espaço livre em discos\\
Múltiplos sistemas de arquivos.\\
Sistemas de Arquivos de Rede\\
\\
UNIDADE 5: Gerência de entrada e saída [8h-aula]\\
Controlador e driver de dispositivo\\
E/S programada\\
Interrupções\\
DMA (Direct Memory Access - Acesso Direto a Memória)\\
Organização de discos rígidos\\
Algoritmos de escalonamento de braço de disco		\\
Sistemas RAID (Redundant Array of Independent Disks)\\
\\
UNIDADE 6: Proteção e Segurança em Sistemas Operacionais [4h]\\
Princípios de proteção\\
Matriz de acesso\\
Domínio de proteção\\
Conceitos de criptografia\\
Recuperação do Deadlock\\


\\ \hline
\end{longtable} 





%%%%%%%%%%%%%%%%%%%%%%%%%%%%%%%%%%%%%%%%%%%%%%%%%%%%%%%%%%%%%%%
\begin{longtable}{|L{1.025\textwidth}|} \hline
%
{\bf VII. BIBLIOGRAFIA BÁSICA} \\ \hline
\begin{enumerate}
%
\item SILBERSCHATZ, Abraham; GALVIN, Peter Baer; GAGNE; Greg. Fundamentos de Sistemas Operacionais. 8 ed. LTC, 2009.
\item TANENBAUM, Andrew S. Sistemas Operacionais Modernos. 3 ed. Pearson, 2010.
\item TANENBAUM, Andrew S.; WOODHULL, Albert S. Sistemas Operacionais - Projeto e Implementação. 3 ed. Bookman, 2008.
\end{enumerate}
 \\ \hline
\end{longtable}


\newpage

%%%%%%%%%%%%%%%%%%%%%%%%%%%%%%%%%%%%%%%%%%%%%%%%%%%%%%%%%%%%%%%
\begin{longtable}{|L{1.025\textwidth}|} \hline
%
{\bf VIII. BIBLIOGRAFIA COMPLEMENTAR} \\ \hline
\begin{enumerate}
\item DEITEL, H. M; DEITEL, P. J.; CHOFFNES, D. R. Sistemas Operacionais. 3 ed. Pearson, 2005.
\item MACHADO, Francis Berenger.; MAIA, Luiz Paulo. Arquitetura de Sistemas Operacionais. LTC, 2004.
\item OLIVEIRA, R. S. de, TOSCANI, S. S., CARISSIMI, A. da S., Sistemas Operacionais, 4 ed. Sagra Luzzatto, 2010.
\item SILBERSCHATZ, Abraham; GALVIN, Peter Baer; GAGNE; Greg. Sistemas Operacionais com Java, 5 ed. Elsevier, 2006.
\item STUART, Brian L. Princípios de Sistemas Operacionais - Projetos e Aplicações. Cengage Learning, 2010.

%
\end{enumerate}
 \\ \hline
\end{longtable}


\input aprovacao.tex


\end{document}
