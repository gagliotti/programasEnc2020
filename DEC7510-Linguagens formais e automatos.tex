\documentclass[12pt]{article}
\usepackage[brazil]{babel}
\usepackage{graphicx,t1enc,wrapfig,amsmath,float}
\usepackage{framed,fancyhdr}
\usepackage{multirow}
\usepackage{longtable}
\usepackage{array}
\newcolumntype{L}[1]{>{\raggedright\let\newline\\\arraybackslash\hspace{0pt}}m{#1}}
\newcolumntype{C}[1]{>{\centering\let\newline\\\arraybackslash\hspace{0pt}}m{#1}}
\newcolumntype{R}[1]{>{\raggedleft\let\newline\\\arraybackslash\hspace{0pt}}m{#1}}
%%%%%%%%%%%%%
\oddsidemargin -0.5cm
\evensidemargin -0.5cm
\textwidth 17.5cm
\topmargin -1.5cm
\textheight 22cm
%%%%%%%%%%%% 

%\pagestyle{empty}

\newcommand{\semestre}{2018.2}

\newcommand{\disciplina}{LINGUAGENS FORMAIS E AUTÔMATOS}
\newcommand{\codigo}{DEC7510}


%%%%%%%%%%%%%%%%%%%%%%%%%%%%%%%%%%%%%%%%%%%%%%%%%%%%%%%
%%%%%%%%%%%%% CRETIDOS
\newcommand{\creditosT}{2}
\newcommand{\creditosP}{2}

%%%%%%%%%%%%%%%%%%%%%%%%%%%%%%%%%%%%%%%%%%%%%%%%%%%%%%%
%%%%%%%%%%%%%% REQUISITOS
\newcommand{\requisitoA}{}
\newcommand{\requisitoB}{}
\newcommand{\requisitoC}{}

%%%%%%%%%%%%%%%%%%%%%%%%%%%%%%%%%%%%%%%%%%%%%%%%%%%%%%%
%%%%%%%%%%%%%%% Atende aos Cursos
\newcommand{\cursoA}{Graduação em Engenharia de Computação \\ \hline}
\newcommand{\cursoB}{}%Graduação em Tecnologias da Informação e Comunicação \\ \hline}
\newcommand{\cursoC}{}%Graduação em Engenharia de Energia \\ \hline}

%%%%%%%%%%%%%%%%%%%%%%%%%%%%%%%%%%%%%%%%%%%%%%%%%%%%%%%%
%%%%%%%%%% EMENTA
\newcommand{\ementa}{
Conceitos Centrais: Símbolos, Alfabeto, Strings e Linguagem. Linguagens Regulares. Expressões Regulares. Automatos Finitos e Expressões Regulares. Propriedades das Linguagens Regulares. Linguagens Livres de Contexto. Automato de Pilha. Introdução a Máquinas de Turing.
 \\ \hline
}


\begin{document}


\input cabecalho.tex

%%%%%%%%%%%%%%%%%%%%%%%%%%%%%%%%%%%%%%%%%%%%%%%%%%%%%%%%%%%%%
\begin{longtable}{|C{0.11\textwidth}|C{0.29\textwidth}|C{0.09\textwidth}|C{0.09\textwidth}|C{0.15\textwidth}|C{0.158\textwidth}|} \hline
%
\multicolumn{6}{|l|}{{\bf I. IDENTIFICAÇÃO DA DISCIPLINA}} \\ \hline
%
\multirow{3}*{{\small CÓDIGO}} & \multirow{3}*{NOME DA DISCIPLINA} &\multicolumn{2}{c|}{{\small N$^\circ$ DE HORAS-AULA }} & {{\small TOTAL DE}} & \multirow{3}*{{\small MODALIDADE}} \\ 
%
& & \multicolumn{2}{c|}{\small SEMANAIS}  & {\small HORAS-AULA} & \\ \cline{3-4}
%
& & {\tiny TEÓRICAS} & {\tiny PRÁTICAS} & {\small SEMESTRAIS} & \\ \hline
% codigo da disciplina carga horaria: teorica - pratica e total
{\bf \small \codigo} & {\bf \small \disciplina } & {\bf \creditosT} & {\bf \creditosP} & {\bf 72} & Presencial\\ \hline
\end{longtable}


%%%%%%%%%%%%%%%%%%%%%%%%%%%%%%%%%%%%%%%%%%%%%%%%%%%%%%%%%%%%%%
\begin{longtable}{|C{0.12\textwidth}|L{0.736\textwidth}|C{0.12\textwidth}|} \hline
%
\multicolumn{3}{|l|}{{\bf II. PRÉ-REQUISITO(S)}} \\ \hline
%
CÓDIGO & NOME DA DISCIPLINA & CURSO \\ \hline	
%
\requisitoA
\requisitoB
\requisitoC
\end{longtable}


%%%%%%%%%%%%%%%%%%%%%%%%%%%%%%%%%%%%%%%%%%%%%%%%%%%%%%%%%%%%%%
\begin{longtable}{|L{1.025\textwidth}|} \hline
%
{\bf III. CURSO(S) PARA O(S) QUAL(IS) A DISCIPLINA É OFERECIDA } \\ \hline
%
\cursoA 
\cursoB
\cursoC

\end{longtable}

%%%%%%%%%%%%%%%%%%%%%%%%%%%%%%%%%%%%%%%%%%%%%%%%%%%%%%%%%%%%%%
\begin{longtable}{|L{1.025\textwidth}|} \hline
%
{\bf IV. EMENTA } \\ \hline
%
\ementa
\end{longtable}

%\newpage



%%%%%%%%%%%%%%%%%%%%%%%%%%%%%%%%%%%%%%%%%%%%%%%%%%%%%%%%%%%%%%%
\begin{longtable}{|L{1.025\textwidth}|} \hline
%
{\bf V. OBJETIVOS } \\ \hline
%
Apresentar os principais métodos de tratamento sintático de linguagens lineares abstratas, com a respectiva associação às linguagens típicas da ciência da computação. Estudar formalismos operacionais, axiomáticos e denotacionais e sua aplicação em compiladores, interpretadores e em ciência da computação em geral.
\\ \hline
\end{longtable}


%%%%%%%%%%%%%%%%%%%%%%%%%%%%%%%%%%%%%%%%%%%%%%%%%%%%%%%%%%%%%%%
\begin{longtable}{|L{1.025\textwidth}|} \hline
%
{\bf VI. CONTEÚDO PROGRAMÁTICO } \\ \hline
Unidade I: Autômatos Finitos Determinísticos. Definição de um Autômato Finito Determinístico. Como um DFA processa Strings. Notação formal para DFAs, Tabela de Transição. Estendendo a Função de Transição para Strings. A linguagem definida por um DFA. Exercícios.\\
\\
Unidade II: Autômatos Finitos Não-Determinísticos. Uma visão informal. Definição. Função de Transição Estendida. A linguagem definida por uma NFA. Equivalência entre Autômato Finito Determinístico e Não-Determinístico. Exercícios.\\
\\
Unidade III: Autômatos Finitos de Transição Vazia. Uso da Transição Epsilon (Vazia). Notação Formal. Fechamento. Epsilons. Função de Transição Estendida. Eliminação de Transições Epsilons.\\
\\
Unidade IV: Expressões Regulares. Operadores. Construção de Expressões Regulares. Precedência entre operadores. Autômatos Finitos e Expressões Regulares. Conversão de DFA para Expressões Regulares. Conversão de Expressões Regulares em Autômatos. Exercícios.\\
\\
Unidade V: Linguagens Livres de Contexto. Definição. Gramáticas. Derivações à esquerda e à direita. Linguagem descrita por uma gramática. Formas sentenciais. Árvores de derivação. Inferência, derivação e árvores gramaticais. Ambiguidades. Aplicações. Exercícios.\\
\\
Unidade VI: A Máquina de Turing. Notação para máquina de Turing. Descrição instantânea para máquina de Turing. Diagramas de transição para máquinas de Turing. A linguagem da máquina de Turing.

\\ \hline
\end{longtable} 

%\newpage

%%%%%%%%%%%%%%%%%%%%%%%%%%%%%%%%%%%%%%%%%%%%%%%%%%%%%%%%%%%%%%%
\begin{longtable}{|L{1.025\textwidth}|} \hline
%
{\bf VII. BIBLIOGRAFIA BÁSICA} \\ \hline
\begin{enumerate}
%
\item HOPCROFT, John E.; ULLMAN, Jeffrey D.; MOTWANI, Rajeev. Introdução à teoria de autômatos, linguagens e computação. Rio de Janeiro: Elsevier, 2003. x, 560p. ISBN 0-201-02988-X. 
\item RAMOS, Marcus Vinícius Midena; JOSÉ NETO, João; VEGA, Ítalo Santiago. Linguagens formais: teoria, modelagem e implementação. Porto Alegre: Bookman, 2009. 656 p. ISBN 9788577804535. 
\item SIPSER, Michael. Introdução à teoria da computação. São Paulo: Cengage Learning, c2007. xxi, 459 p. ISBN 9788522104994.
\end{enumerate}
 \\ \hline
\end{longtable}


\newpage

%%%%%%%%%%%%%%%%%%%%%%%%%%%%%%%%%%%%%%%%%%%%%%%%%%%%%%%%%%%%%%%
\begin{longtable}{|L{1.025\textwidth}|} \hline
%
{\bf VIII. BIBLIOGRAFIA COMPLEMENTAR} \\ \hline
\begin{enumerate}
\item HOPCROFT, John E.; MOTWANI, Rajeev; ULLMAN, Jeffrey D. Introduction to automata theory, languages, and computation. 3nd ed. Boston: Addison Wesley, 2007. xvii, 535p. ISBN 0-321455363 
\item AHO, Alfred V. et al. Compiladores: princípios, técnicas e ferramentas. 2. ed. São Paulo: Pearson Addison Wesley, c2008. x,634 p. ISBN 9788588639249. 
\item PRICE, Ana Maria de Alencar; TOSCANI, Simão Sirineo. Implementação de linguagens de programação: compiladores. 3. ed. Porto Alegre: Bookman, 2008. 195, [1] p. (Série livros didáticos ; ISBN 9788577803484

\item HOPCROFT, John; MOTWANI, Rajeev; ULLMAN, Jefferey, Introdução à Teoria de Autômatos, Linguagens e Computação, Elsevier; Edição: 1a, 2002, ISBN-10: 8535210725, ISBN-13: 978-8535210729

\item Paulo B. Menezes; Linguagens Formais e Autômatos - Volume 3, Bookman; Edição: 6a, 2010, ISBN-10: 8577807657, ISBN-13: 978-8577807659

\end{enumerate}
 \\ \hline
\end{longtable}


\input aprovacao.tex


\end{document}
