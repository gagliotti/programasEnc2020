\documentclass[12pt]{article}
\usepackage[brazil]{babel}
\usepackage{graphicx,t1enc,wrapfig,amsmath,float}
\usepackage{framed,fancyhdr}
\usepackage{multirow}
\usepackage{longtable}
\usepackage{array}
\newcolumntype{L}[1]{>{\raggedright\let\newline\\\arraybackslash\hspace{0pt}}m{#1}}
\newcolumntype{C}[1]{>{\centering\let\newline\\\arraybackslash\hspace{0pt}}m{#1}}
\newcolumntype{R}[1]{>{\raggedleft\let\newline\\\arraybackslash\hspace{0pt}}m{#1}}
%%%%%%%%%%%%%
\oddsidemargin -0.5cm
\evensidemargin -0.5cm
\textwidth 17.5cm
\topmargin -1.5cm
\textheight 22cm
%%%%%%%%%%%% 

%\pagestyle{empty}

\newcommand{\semestre}{2018.2}

\newcommand{\disciplina}{SISTEMAS DE AQUISIÇÃO DE SINAIS}
\newcommand{\codigo}{DEC7561}


%%%%%%%%%%%%%%%%%%%%%%%%%%%%%%%%%%%%%%%%%%%%%%%%%%%%%%%
%%%%%%%%%%%%% CRETIDOS
\newcommand{\creditosT}{0}
\newcommand{\creditosP}{4}

%%%%%%%%%%%%%%%%%%%%%%%%%%%%%%%%%%%%%%%%%%%%%%%%%%%%%%%
%%%%%%%%%%%%%% REQUISITOS
\newcommand{\requisitoA}{}
\newcommand{\requisitoB}{}
\newcommand{\requisitoC}{}

%%%%%%%%%%%%%%%%%%%%%%%%%%%%%%%%%%%%%%%%%%%%%%%%%%%%%%%
%%%%%%%%%%%%%%% Atende aos Cursos
\newcommand{\cursoA}{Graduação em Engenharia de Computação \\ \hline}
\newcommand{\cursoB}{}%Graduação em Tecnologias da Informação e Comunicação \\ \hline}
\newcommand{\cursoC}{}

%%%%%%%%%%%%%%%%%%%%%%%%%%%%%%%%%%%%%%%%%%%%%%%%%%%%%%%%
%%%%%%%%%% EMENTA
\newcommand{\ementa}{
Característica dos sensores. Princípio físico dos sensores. Sensores óticos. Circuitos de interface. Detectores de movimento. Sensores de posição, deslocamento e nível. Sensores de aceleração e velocidade. Sensor de Força. Sensor de Pressão. Sensores de fluxo e acústico. Sensor de umidade. Detector de luz. Detectores de radiação. Sensores de temperatura. Sensores químicos. Circuitos amostradores. Conversores Analógicos Digitais. Conversores Digitais Analógicos.

\\ \hline
}


\begin{document}


%%%%%%%%%%%%%%%%%%%%%%%%%%%%%%%%%%%%%%%%%%%%%%%%%%%%%%%%%%%%%

\input cabecalho.tex


%%%%%%%%%%%%%%%%%%%%%%%%%%%%%%%%%%%%%%%%%%%%%%%%%%%%%%%%%%%%%
\begin{longtable}{|C{0.11\textwidth}|C{0.29\textwidth}|C{0.09\textwidth}|C{0.09\textwidth}|C{0.15\textwidth}|C{0.158\textwidth}|} \hline
%
\multicolumn{6}{|l|}{{\bf I. IDENTIFICAÇÃO DA DISCIPLINA}} \\ \hline
%
\multirow{3}*{{\small CÓDIGO}} & \multirow{3}*{NOME DA DISCIPLINA} &\multicolumn{2}{c|}{{\small N$^\circ$ DE HORAS-AULA }} & {{\small TOTAL DE}} & \multirow{3}*{{\small MODALIDADE}} \\ 
%
& & \multicolumn{2}{c|}{\small SEMANAIS}  & {\small HORAS-AULA} & \\ \cline{3-4}
%
& & {\tiny TEÓRICAS} & {\tiny PRÁTICAS} & {\small SEMESTRAIS} & \\ \hline
% codigo da disciplina carga horaria: teorica - pratica e total
{\bf \small \codigo} & {\bf \small \disciplina } & {\bf \creditosT} & {\bf \creditosP} & {\bf 72} & Presencial\\ \hline
\end{longtable}


%%%%%%%%%%%%%%%%%%%%%%%%%%%%%%%%%%%%%%%%%%%%%%%%%%%%%%%%%%%%%%
\begin{longtable}{|C{0.12\textwidth}|L{0.736\textwidth}|C{0.12\textwidth}|} \hline
%
\multicolumn{3}{|l|}{{\bf II. PRÉ-REQUISITO(S)}} \\ \hline
%
CÓDIGO & NOME DA DISCIPLINA & CURSO \\ \hline	
%
\requisitoA
\requisitoB
\requisitoC
\end{longtable}


%%%%%%%%%%%%%%%%%%%%%%%%%%%%%%%%%%%%%%%%%%%%%%%%%%%%%%%%%%%%%%
\begin{longtable}{|L{1.025\textwidth}|} \hline
%
{\bf III. CURSO(S) PARA O(S) QUAL(IS) A DISCIPLINA É OFERECIDA } \\ \hline
%
\cursoA 
\cursoB
\cursoC

\end{longtable}

%%%%%%%%%%%%%%%%%%%%%%%%%%%%%%%%%%%%%%%%%%%%%%%%%%%%%%%%%%%%%%
\begin{longtable}{|L{1.025\textwidth}|} \hline
%
{\bf IV. EMENTA } \\ \hline
%
\ementa
\end{longtable}

\newpage



%%%%%%%%%%%%%%%%%%%%%%%%%%%%%%%%%%%%%%%%%%%%%%%%%%%%%%%%%%%%%%%
\begin{longtable}{|L{1.025\textwidth}|} \hline
%
{\bf V. OBJETIVOS } \\ \hline

Objetivos Gerais:\\
Esta disciplina deverá explorar os fundamentos, conceitos, mecanismos e técnicas que permitam a reconstrução de um contexto através da leitura de grandezas físicas e de sinais.

Objetivos Específicos: 
\begin{itemize}
\item Introduzir conceitos básicos de sensores;
\item Discutir assuntos relacionados a hardware para aquisição de sinais;
\item Discutir softwares para simulação e aquisição de sinais;
\item Discutir técnicas de análise e projeto de sistemas de aquisição de sinais
\end{itemize}
\\ \hline
\end{longtable}


%%%%%%%%%%%%%%%%%%%%%%%%%%%%%%%%%%%%%%%%%%%%%%%%%%%%%%%%%%%%%%%
\begin{longtable}{|L{1.025\textwidth}|} \hline
%
{\bf VI. CONTEÚDO PROGRAMÁTICO } \\ \hline
Conteúdo teórico: \\
Introdução a Sistemas de Aquisição de Dados;\\
Sensores;\\
Amplificadores Operacionais para condicionamento de sinais;\\
Filtros Analógicos;\\
Software para simulação e aquisição de dados;\\
Conversores Analógicos/Digitais e Digitais/Analógicos;
\\ \hline
\end{longtable} 

%\newpage


%%%%%%%%%%%%%%%%%%%%%%%%%%%%%%%%%%%%%%%%%%%%%%%%%%%%%%%%%%%%%%%
\begin{longtable}{|L{1.025\textwidth}|} \hline
%
{\bf VII. BIBLIOGRAFIA BÁSICA} \\ \hline
\begin{enumerate}
\item Jacob Fraden. Handbook of Modern Sensors: Physics. Designs, and Applications. Springer. 2010. ISBN-10. 1441964657. 
\item FILHO, Sidnei Noceti. Filtros Seletores de Sinais, 3.ed. UFSC, 2010. ISBN-13: 9788532804952.
\item  Emily Gertz. Patrick Di Justo. Environmental Monitoring with Arduino Building Simple Devices to Collect Data About the World Around Us. Make. 2012. ISBN-10:1449310567.
\end{enumerate}
 \\ \hline
\end{longtable}


\newpage

%%%%%%%%%%%%%%%%%%%%%%%%%%%%%%%%%%%%%%%%%%%%%%%%%%%%%%%%%%%%%%%
\begin{longtable}{|L{1.025\textwidth}|} \hline
%
{\bf VIII. BIBLIOGRAFIA COMPLEMENTAR} \\ \hline
\begin{enumerate}
\item Nikolay V. Kirianakl. Sergey Y. Yurish. Nestor 0. Shpak, Vadim P Deynega. Data Acquisition and Signal Processingfor Smart Sensors. Wiley. 2002. ISBN-10: 0470843179. 
\item  THOMAZINI, Daniel; URBANO, Pedro. Sensores Industriais, 8.ed. Érica, 2011. ISBN: 8536500719.
\item  Robert King. Introduction to Data Acquisition with LabView, McGraw-Hill, 2012, ISBN-10 0073385875. 
\item  Charles D Spencer, Digital Design for Computer Data Acquisition, Cambridge University Press. 2009. ISBN-10:0521102553.
\item  WEBSTER, John G.; CLARK, John W. Medical instrumentation: application and design. 4th. ed. Hoboken, N.J.: Wiley, c2010 xix, 713p. ISBN 9780471676003



%
\end{enumerate}
 \\ \hline
\end{longtable}


\input aprovacao.tex


\end{document}
