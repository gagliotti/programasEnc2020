\documentclass[12pt]{article}
\usepackage[brazil]{babel}
\usepackage{graphicx,t1enc,wrapfig,amsmath,float}
\usepackage{framed,fancyhdr}
\usepackage{multirow}
\usepackage{longtable}
\usepackage{array}
\newcolumntype{L}[1]{>{\raggedright\let\newline\\\arraybackslash\hspace{0pt}}m{#1}}
\newcolumntype{C}[1]{>{\centering\let\newline\\\arraybackslash\hspace{0pt}}m{#1}}
\newcolumntype{R}[1]{>{\raggedleft\let\newline\\\arraybackslash\hspace{0pt}}m{#1}}
%%%%%%%%%%%%%
\oddsidemargin -0.5cm
\evensidemargin -0.5cm
\textwidth 17.5cm
\topmargin -1.5cm
\textheight 22cm
%%%%%%%%%%%% 

%\pagestyle{empty}

\newcommand{\semestre}{2018.2}

\newcommand{\disciplina}{PROJETO DE SISTEMAS UBIQUOS}
\newcommand{\codigo}{DEC7564}


%%%%%%%%%%%%%%%%%%%%%%%%%%%%%%%%%%%%%%%%%%%%%%%%%%%%%%%
%%%%%%%%%%%%% CRETIDOS
\newcommand{\creditosT}{0}
\newcommand{\creditosP}{4}

%%%%%%%%%%%%%%%%%%%%%%%%%%%%%%%%%%%%%%%%%%%%%%%%%%%%%%%
%%%%%%%%%%%%%% REQUISITOS
\newcommand{\requisitoA}{}
\newcommand{\requisitoB}{}
\newcommand{\requisitoC}{}

%%%%%%%%%%%%%%%%%%%%%%%%%%%%%%%%%%%%%%%%%%%%%%%%%%%%%%%
%%%%%%%%%%%%%%% Atende aos Cursos
\newcommand{\cursoA}{Graduação em Engenharia de Computação \\ \hline}
\newcommand{\cursoB}{}%Graduação em Tecnologias da Informação e Comunicação \\ \hline}
\newcommand{\cursoC}{}

%%%%%%%%%%%%%%%%%%%%%%%%%%%%%%%%%%%%%%%%%%%%%%%%%%%%%%%%
%%%%%%%%%% EMENTA
\newcommand{\ementa}{
Fundamentos de Sistemas Ubíquos e Pervasivos: computação móvel e embarcada, computação sensível ao contexto e descoberta de serviços. Fundamentos de sistemas operacionais embarcados, sistemas autônomos e reconfiguráveis. Tecnologias de Sistemas ubíquos: Middleware para sistemas ubíquos, Redes de Sensores Sem Fio, Identificação por Rádio Frequência (RFID), FlexRay, TinyOs, Android, Bluetooth.
\\ \hline
}


\begin{document}

%%%%%%%%%%%%%%%%%%%%%%%%%%%%%%%%%%%%%%%%%%%%%%%%%%%%%%%%%%%%%

\input cabecalho.tex


%%%%%%%%%%%%%%%%%%%%%%%%%%%%%%%%%%%%%%%%%%%%%%%%%%%%%%%%%%%%%
\begin{longtable}{|C{0.11\textwidth}|C{0.29\textwidth}|C{0.09\textwidth}|C{0.09\textwidth}|C{0.15\textwidth}|C{0.158\textwidth}|} \hline
%
\multicolumn{6}{|l|}{{\bf I. IDENTIFICAÇÃO DA DISCIPLINA}} \\ \hline
%
\multirow{3}*{{\small CÓDIGO}} & \multirow{3}*{NOME DA DISCIPLINA} &\multicolumn{2}{c|}{{\small N$^\circ$ DE HORAS-AULA }} & {{\small TOTAL DE}} & \multirow{3}*{{\small MODALIDADE}} \\ 
%
& & \multicolumn{2}{c|}{\small SEMANAIS}  & {\small HORAS-AULA} & \\ \cline{3-4}
%
& & {\tiny TEÓRICAS} & {\tiny PRÁTICAS} & {\small SEMESTRAIS} & \\ \hline
% codigo da disciplina carga horaria: teorica - pratica e total
{\bf \small \codigo} & {\bf \small \disciplina } & {\bf \creditosT} & {\bf \creditosP} & {\bf 72} & Presencial\\ \hline
\end{longtable}


%%%%%%%%%%%%%%%%%%%%%%%%%%%%%%%%%%%%%%%%%%%%%%%%%%%%%%%%%%%%%%
\begin{longtable}{|C{0.12\textwidth}|L{0.736\textwidth}|C{0.12\textwidth}|} \hline
%
\multicolumn{3}{|l|}{{\bf II. PRÉ-REQUISITO(S)}} \\ \hline
%
CÓDIGO & NOME DA DISCIPLINA & CURSO \\ \hline	
%
\requisitoA
\requisitoB
\requisitoC
\end{longtable}


%%%%%%%%%%%%%%%%%%%%%%%%%%%%%%%%%%%%%%%%%%%%%%%%%%%%%%%%%%%%%%
\begin{longtable}{|L{1.025\textwidth}|} \hline
%
{\bf III. CURSO(S) PARA O(S) QUAL(IS) A DISCIPLINA É OFERECIDA } \\ \hline
%
\cursoA 
\cursoB
\cursoC

\end{longtable}

%%%%%%%%%%%%%%%%%%%%%%%%%%%%%%%%%%%%%%%%%%%%%%%%%%%%%%%%%%%%%%
\begin{longtable}{|L{1.025\textwidth}|} \hline
%
{\bf IV. EMENTA } \\ \hline
%
\ementa
\end{longtable}

\newpage



%%%%%%%%%%%%%%%%%%%%%%%%%%%%%%%%%%%%%%%%%%%%%%%%%%%%%%%%%%%%%%%
\begin{longtable}{|L{1.025\textwidth}|} \hline
%
{\bf V. OBJETIVOS } \\ \hline
Objetivo Geral: \\
Habilitar o aluno a projetar e desenvolver sistemas computacionais ubíquos, bem como reconhecer as principais características e tecnologias de sistemas ubíquos e pervasivos.\\
\\
Objetivos Específicos:\\
Denvolvimento de projetos para sistemas ubíquos.

\begin{itemize}
\item Familiarizar o aluno com o modelo sistemas distribuídos para computação ubíqua;
\item  Apresentar os principais conceitos envolvidos no projeto e no desenvolvimento de sistemas ubíquos;
\item  Capacitar o aluno no desenvolvimento de projetos para sistemas ubíquos.
\end{itemize}
\\ \hline
\end{longtable}


%%%%%%%%%%%%%%%%%%%%%%%%%%%%%%%%%%%%%%%%%%%%%%%%%%%%%%%%%%%%%%%
\begin{longtable}{|L{1.025\textwidth}|} \hline
%
{\bf VI. CONTEÚDO PROGRAMÁTICO } \\ \hline
Conteúdo Teórico seguido de Conteúdo Prático com desenvolvimento de problemas em computador: \\
\\
UNIDADE 1: Fundamentos de Sistemas Ubíquos e Pervasivos [4 horas-aula]\\
Conceitos de sistemas ubíquos e pervasivos\\
Exemplos de sistemas  ubíquos\\
Computação móvel e embarcada\\
Computação sensível ao contexto e descoberta de serviços\\
\\
UNIDADE 2: Projeto de Sistemas Ubíquos [8 horas-aula]\\
Definição dos projetos de Sistemas Ubíquos.\\
\\
UNIDADE 3: Orientação de Projeto de Sistemas Ubíquos [44 horas-aula]\\
Orientação de projeto.\\
Experimentação e análise.\\
\\
UNIDADE 4:  Defesa de projeto [16 horas-aula]\\
Escrita de artigo no formato do Simpósio Brasileiro de Engenharia de Sistemas\\ Computacionais ou Simpósio Brasileiro de Computação Ubíqua ou Pervasiva.\\
Defesa de projeto com banca de avaliadores.\\
\\ \hline
\end{longtable} 

\newpage


%%%%%%%%%%%%%%%%%%%%%%%%%%%%%%%%%%%%%%%%%%%%%%%%%%%%%%%%%%%%%%%
\begin{longtable}{|L{1.025\textwidth}|} \hline
%
{\bf VII. BIBLIOGRAFIA BÁSICA} \\ \hline
\begin{enumerate}
\item COULOURIS, George; DOLLIMORE, Jean; KINDBERG, Tim. Sistemas Distribuídos conceitos e projetos. 4a. Ed. Editora Bookman, 2007. 
\item LEE, Valentino; SCHNEIDER, Heather; SCHELL, Robbie. Aplicações móveis: arquitetura, projeto e desenvolvimento. São Paulo: Pearson Makron Books, 2005. xx, 328 p. ISBN 8534615403 (broch.). 
\item ALLEN, Sarah; GRAUPERA, Vidal; LUNDRIGAN, Lee. Desenvolvimento profissional multiplataforma para smartphone: iPhone, Android, Windows mobile e BlackBerry. Rio de Janeiro: Alta Books, 2012. xvi, 264 p. ISBN 9788576086611.
\end{enumerate}
 \\ \hline
\end{longtable}


%\newpage

%%%%%%%%%%%%%%%%%%%%%%%%%%%%%%%%%%%%%%%%%%%%%%%%%%%%%%%%%%%%%%%
\begin{longtable}{|L{1.025\textwidth}|} \hline
%
{\bf VIII. BIBLIOGRAFIA COMPLEMENTAR} \\ \hline
\begin{enumerate}
\item C Dargie, Waltenegus., Poellabauer, Chirtian; Fundamentals of Wireless Sensor Networks: Theory and Practice (Wireless Communications and Mobile Computing). 
\item DEITEL, H. M.; DEITEL, P.J. Java: como programar. 6. ed. São Paulo: Pearson, 2005.
\item LECHETA, Ricardo R. Google Android: aprenda a criar aplicações para dispositivos móveis com Android SDK. 3. ed. São Paulo: Novatec, 2013. 824 p. ISBN 9788575223444.
\item TANENBAUM, Andrew S.; STEEN, Maarten van. Sistemas distribuídos: princípios e paradgmas. 2. ed. São Paulo: Pearson Prentice Hall, 2007. x, 402 p. ISBN 978-85-7605-142-8 (broch.).
\item FALUDI, Robert. Building wireless sensor networks. Sebastopol: O'Reilly, 2010. xviii, 300 p. ISBN 9780596807733.
%
\end{enumerate}
 \\ \hline
\end{longtable}


\input aprovacao.tex


\end{document}
