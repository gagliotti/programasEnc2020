\documentclass[12pt]{article}
\usepackage[brazil]{babel}
\usepackage{graphicx,t1enc,wrapfig,amsmath,float}
\usepackage{framed,fancyhdr}
\usepackage{multirow}
\usepackage{longtable}
\usepackage{array}
\newcolumntype{L}[1]{>{\raggedright\let\newline\\\arraybackslash\hspace{0pt}}m{#1}}
\newcolumntype{C}[1]{>{\centering\let\newline\\\arraybackslash\hspace{0pt}}m{#1}}
\newcolumntype{R}[1]{>{\raggedleft\let\newline\\\arraybackslash\hspace{0pt}}m{#1}}
%%%%%%%%%%%%%
\oddsidemargin -0.5cm
\evensidemargin -0.5cm
\textwidth 17.5cm
\topmargin -1.5cm
\textheight 22cm
%%%%%%%%%%%% 

%\pagestyle{empty}

\newcommand{\semestre}{2018.2}

\newcommand{\disciplina}{MICROCONTROLADORES E MICROPROCESSADOS}
\newcommand{\codigo}{DEC7511}


%%%%%%%%%%%%%%%%%%%%%%%%%%%%%%%%%%%%%%%%%%%%%%%%%%%%%%%
%%%%%%%%%%%%% CRETIDOS
\newcommand{\creditosT}{2}
\newcommand{\creditosP}{2}

%%%%%%%%%%%%%%%%%%%%%%%%%%%%%%%%%%%%%%%%%%%%%%%%%%%%%%%
%%%%%%%%%%%%%% REQUISITOS
\newcommand{\requisitoA}{}
\newcommand{\requisitoB}{}
\newcommand{\requisitoC}{}

%%%%%%%%%%%%%%%%%%%%%%%%%%%%%%%%%%%%%%%%%%%%%%%%%%%%%%%
%%%%%%%%%%%%%%% Atende aos Cursos
\newcommand{\cursoA}{Graduação em Engenharia de Computação \\ \hline}
\newcommand{\cursoB}{}%Graduação em Tecnologias da Informação e Comunicação \\ \hline}
\newcommand{\cursoC}{}%Graduação em Engenharia de Energia \\ \hline}

%%%%%%%%%%%%%%%%%%%%%%%%%%%%%%%%%%%%%%%%%%%%%%%%%%%%%%%%
%%%%%%%%%% EMENTA
\newcommand{\ementa}{
Microprocessadores: introdução histórica; estrutura básica de um microprocessador;microprocessadores comercialmente disponíveis; memórias; controladores; computadores;microcontroladores; operações de entrada/saída. Microcontroladores: arquiteturas típicas de um microcontrolador e seus registradores; arquiteturas; exemplos de microcontroladores comerciais; instruções; programação; mapa de memória, portas de entrada e saída; modulo temporizador; contadores; interrupções, conversão analógico-digital; acesso à memória; barramentos padrões; dispositivos periféricos; ferramentas de programação, simulação e depuração. Aplicações de microcontroladores e microprocessadores. Projetos de sistemas práticos com microcontroladores.
 \\ \hline
}


\begin{document}


%%%%%%%%%%%%%%%%%%%%%%%%%%%%%%%%%%%%%%%%%%%%%%%%%%%%%%%%%%%%%
\input cabecalho.tex



%%%%%%%%%%%%%%%%%%%%%%%%%%%%%%%%%%%%%%%%%%%%%%%%%%%%%%%%%%%%%
\begin{longtable}{|C{0.11\textwidth}|C{0.29\textwidth}|C{0.09\textwidth}|C{0.09\textwidth}|C{0.15\textwidth}|C{0.158\textwidth}|} \hline
%
\multicolumn{6}{|l|}{{\bf I. IDENTIFICAÇÃO DA DISCIPLINA}} \\ \hline
%
\multirow{3}*{{\small CÓDIGO}} & \multirow{3}*{NOME DA DISCIPLINA} &\multicolumn{2}{c|}{{\small N$^\circ$ DE HORAS-AULA }} & {{\small TOTAL DE}} & \multirow{3}*{{\small MODALIDADE}} \\ 
%
& & \multicolumn{2}{c|}{\small SEMANAIS}  & {\small HORAS-AULA} & \\ \cline{3-4}
%
& & {\tiny TEÓRICAS} & {\tiny PRÁTICAS} & {\small SEMESTRAIS} & \\ \hline
% codigo da disciplina carga horaria: teorica - pratica e total
{\bf \small \codigo} & {\bf \small \disciplina } & {\bf \creditosT} & {\bf \creditosP} & {\bf 72} & Presencial\\ \hline
\end{longtable}


%%%%%%%%%%%%%%%%%%%%%%%%%%%%%%%%%%%%%%%%%%%%%%%%%%%%%%%%%%%%%%
\begin{longtable}{|C{0.12\textwidth}|L{0.736\textwidth}|C{0.12\textwidth}|} \hline
%
\multicolumn{3}{|l|}{{\bf II. PRÉ-REQUISITO(S)}} \\ \hline
%
CÓDIGO & NOME DA DISCIPLINA & CURSO \\ \hline	
%
\requisitoA
\requisitoB
\requisitoC
\end{longtable}


%%%%%%%%%%%%%%%%%%%%%%%%%%%%%%%%%%%%%%%%%%%%%%%%%%%%%%%%%%%%%%
\begin{longtable}{|L{1.025\textwidth}|} \hline
%
{\bf III. CURSO(S) PARA O(S) QUAL(IS) A DISCIPLINA É OFERECIDA } \\ \hline
%
\cursoA 
\cursoB
\cursoC

\end{longtable}

%%%%%%%%%%%%%%%%%%%%%%%%%%%%%%%%%%%%%%%%%%%%%%%%%%%%%%%%%%%%%%
\begin{longtable}{|L{1.025\textwidth}|} \hline
%
{\bf IV. EMENTA } \\ \hline
%
\ementa
\end{longtable}

%\newpage



%%%%%%%%%%%%%%%%%%%%%%%%%%%%%%%%%%%%%%%%%%%%%%%%%%%%%%%%%%%%%%%
\begin{longtable}{|L{1.025\textwidth}|} \hline
%
{\bf V. OBJETIVOS } \\ \hline
%
Objetivo Geral: \\

Explorar a arquitetura de microprocessadores e microcontroladores, suas unidades funcionais internas, interfaceamento com  seus periféricos e linguagem de programação.\\
\\
Objetivos Específicos: 
\begin{itemize}
\item Conhecer a arquitetura interna dos microprocessadores e microcontroladores
\item Estudar os modelos e diferenças entre as famílias de microprocessadores e microcontroladores
\item Desenvolver sistemas embarcados baseados em microcontroladores
\item Trabalhar com periféricos de entrada e saída
\item Utilizar software de desenvolvimento e simulação de sistemas embarcados
\item Desenvolver um projeto completo de sistema embarcado
\end{itemize}
\\ \hline
\end{longtable}


%%%%%%%%%%%%%%%%%%%%%%%%%%%%%%%%%%%%%%%%%%%%%%%%%%%%%%%%%%%%%%%
\begin{longtable}{|L{1.025\textwidth}|} \hline
%
{\bf VI. CONTEÚDO PROGRAMÁTICO } \\ \hline
UNIDADE 1 - INTRODUÇÃO [8 ha]\\
Apresentação da disciplina\\
Evolução dos microprocessadores\\
\\
UNIDADE 2 - Introdução a Sistemas Embarcados [8 ha]\\
Definições\\
Aplicações de sistemas embarcados\\
Diferença entre microprocessadores e microcontroladores\\
Fabricantes\\
\\
UNIDADE 3 - Arquitetura de Microcontroladores e  Linguagem Assembly [12 ha]\\
Microcontroladores da família PIC\\
Estudo da arquitetura de microcontrolador\\
Desenvolvimento de programas em assembly para microcontroladores\\
\\
UNIDADE 4 - Microcontroladores - Linguagem Ce Periféricos[28 ha]\\
Desenvolvimento de programa em C para microcontroladores\\
Tipos de dados\\
Entrada e saída\\ 
Estruturas de repetição\\
Estrutura de seleção\\
Subrotinas\\
Interrupção\\
Display de 7 segmentos\\
Timers\\
Conversor Analógico/Digital\\
Display LCD\\
Memórias EEPROM e FLASH\\
Barramentos (I2C, SPI, USB)\\
\\
UNIDADE 5 - Projetos de sistemas embarcados [16 ha]\\
Projeto de Hardware\\
Projeto de software\\
Projeto de sistema embarcado eficiente\\
Desenvolvimento de projeto original
\\ \hline
\end{longtable} 

%\newpage

%%%%%%%%%%%%%%%%%%%%%%%%%%%%%%%%%%%%%%%%%%%%%%%%%%%%%%%%%%%%%%%
\begin{longtable}{|L{1.025\textwidth}|} \hline
%
{\bf VII. BIBLIOGRAFIA BÁSICA} \\ \hline
\begin{enumerate}
%
\item CATSOULIS, John. Designing embedded hardware. 2nd ed. Sebastopol: O'Reilly, 2005. xvi, 377 p. ISBN 9780596007553. 
\item STALLINGS, W. - Arquitetura e Organização de Computadores - 5a. Ed., Pearson/Prentice Hall, 2002. 
\item SOUZA, David José de. Desbravando o PIC/ ampliado e atualizado para PIC16F628A. 12. ed. São Paulo: Érica, 2014. 268 p. ISBN 9788571948679.
\end{enumerate}
 \\ \hline
\end{longtable}


%\newpage

%%%%%%%%%%%%%%%%%%%%%%%%%%%%%%%%%%%%%%%%%%%%%%%%%%%%%%%%%%%%%%%
\begin{longtable}{|L{1.025\textwidth}|} \hline
%
{\bf VIII. BIBLIOGRAFIA COMPLEMENTAR} \\ \hline
\begin{enumerate}
\item FERREIRA, José Manuel Martins. Introdução ao projeto com sistemas digitais e microcontroladores. Porto: FEUP, 1998. 371 p. ISBN 9727520324 
\item KLEINJOHANN, Bernd; KLEINJOHANN, Lisa; WOLF, Wayne. Distributed Embedded Systems: Design, Middleware and Resources. Boston: Springer Science+Business Media, LLC, 2008. 
\item PATTERSON, D. A.; HENNESSY, P. Organização e Projeto de Computadores. Editora Campus (Elsevier), 2005. 
\item PARHAMI, Behrooz. Arquitetura de computadores: de microprocessadores a supercomputadores. São Paulo: McGraw Hill, 2008 xvi, 560 p. ISBN 9788577260256. 
\item MALVINO, Albert Paul; BATES, David J. Eletrônica. 7. ed. Porto Alegre: AMGH, 2007. v. ISBN 9788577260225 (v.1).
\end{enumerate}
 \\ \hline
\end{longtable}


\input aprovacao.tex


\end{document}
