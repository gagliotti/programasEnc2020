\documentclass[12pt]{article}
\usepackage[brazil]{babel}
\usepackage{graphicx,t1enc,wrapfig,amsmath,float}
\usepackage{framed,fancyhdr}
\usepackage{multirow}
\usepackage{longtable}
\usepackage{array}
\newcolumntype{L}[1]{>{\raggedright\let\newline\\\arraybackslash\hspace{0pt}}m{#1}}
\newcolumntype{C}[1]{>{\centering\let\newline\\\arraybackslash\hspace{0pt}}m{#1}}
\newcolumntype{R}[1]{>{\raggedleft\let\newline\\\arraybackslash\hspace{0pt}}m{#1}}
%%%%%%%%%%%%%
\oddsidemargin -0.5cm
\evensidemargin -0.5cm
\textwidth 17.5cm
\topmargin -1.5cm
\textheight 22cm
%%%%%%%%%%%% 

%\pagestyle{empty}

\newcommand{\semestre}{2018.2}

\newcommand{\disciplina}{MODELAGEM E SIMULAÇÃO}
\newcommand{\codigo}{DEC7523}


%%%%%%%%%%%%%%%%%%%%%%%%%%%%%%%%%%%%%%%%%%%%%%%%%%%%%%%
%%%%%%%%%%%%% CRETIDOS
\newcommand{\creditosT}{2}
\newcommand{\creditosP}{2}

%%%%%%%%%%%%%%%%%%%%%%%%%%%%%%%%%%%%%%%%%%%%%%%%%%%%%%%
%%%%%%%%%%%%%% REQUISITOS
\newcommand{\requisitoA}{FQM7107 & Probabilidade e Estatística & ENC\\}
\newcommand{\requisitoB}{DEC0012 & Linguagem de Programação I & ENC\\}
\newcommand{\requisitoC}{}

%%%%%%%%%%%%%%%%%%%%%%%%%%%%%%%%%%%%%%%%%%%%%%%%%%%%%%%
%%%%%%%%%%%%%%% Atende aos Cursos
\newcommand{\cursoA}{Graduação em Engenharia de Computação \\ \hline}
\newcommand{\cursoB}{}%Graduação em Tecnologias da Informação e Comunicação \\ \hline}
%\newcommand{\cursoC}{Graduação em Engenharia de Energia \\ \hline}

%%%%%%%%%%%%%%%%%%%%%%%%%%%%%%%%%%%%%%%%%%%%%%%%%%%%%%%%
%%%%%%%%%% EMENTA
\newcommand{\ementa}{
Introdução à simulação. Propriedades e classificação dos modelos de simulação. Geração de números aleatórios. Geração e teste. Simulação de sistemas discretos. Verificação e validação de modelos. Técnicas estatísticas para análise de dados e de resultados de modelos de simulação. Modelagem e Simulação de sistemas de computação. Avaliação de desempenho de sistemas.
 \\ \hline
}


\begin{document}


%%%%%%%%%%%%%%%%%%%%%%%%%%%%%%%%%%%%%%%%%%%%%%%%%%%%%%%%%%%%%
\input cabecalho.tex



%%%%%%%%%%%%%%%%%%%%%%%%%%%%%%%%%%%%%%%%%%%%%%%%%%%%%%%%%%%%%
\begin{longtable}{|C{0.11\textwidth}|C{0.29\textwidth}|C{0.09\textwidth}|C{0.09\textwidth}|C{0.15\textwidth}|C{0.158\textwidth}|} \hline
%
\multicolumn{6}{|l|}{{\bf I. IDENTIFICAÇÃO DA DISCIPLINA}} \\ \hline
%
\multirow{3}*{{\small CÓDIGO}} & \multirow{3}*{NOME DA DISCIPLINA} &\multicolumn{2}{c|}{{\small N$^\circ$ DE HORAS-AULA }} & {{\small TOTAL DE}} & \multirow{3}*{{\small MODALIDADE}} \\ 
%
& & \multicolumn{2}{c|}{\small SEMANAIS}  & {\small HORAS-AULA} & \\ \cline{3-4}
%
& & {\tiny TEÓRICAS} & {\tiny PRÁTICAS} & {\small SEMESTRAIS} & \\ \hline
% codigo da disciplina carga horaria: teorica - pratica e total
{\bf \small \codigo} & {\bf \small \disciplina } & {\bf \creditosT} & {\bf \creditosP} & {\bf 72} & Presencial\\ \hline
\end{longtable}


%%%%%%%%%%%%%%%%%%%%%%%%%%%%%%%%%%%%%%%%%%%%%%%%%%%%%%%%%%%%%%
\begin{longtable}{|C{0.12\textwidth}|L{0.736\textwidth}|C{0.12\textwidth}|} \hline
%
\multicolumn{3}{|l|}{{\bf II. PRÉ-REQUISITO(S)}} \\ \hline
%
CÓDIGO & NOME DA DISCIPLINA & CURSO \\ \hline	
%
\requisitoA
\requisitoB \hline
\requisitoC
\end{longtable}


%%%%%%%%%%%%%%%%%%%%%%%%%%%%%%%%%%%%%%%%%%%%%%%%%%%%%%%%%%%%%%
\begin{longtable}{|L{1.025\textwidth}|} \hline
%
{\bf III. CURSO(S) PARA O(S) QUAL(IS) A DISCIPLINA É OFERECIDA } \\ \hline
%
\cursoA 
\cursoB
\cursoC

\end{longtable}

%%%%%%%%%%%%%%%%%%%%%%%%%%%%%%%%%%%%%%%%%%%%%%%%%%%%%%%%%%%%%%
\begin{longtable}{|L{1.025\textwidth}|} \hline
%
{\bf IV. EMENTA } \\ \hline
%
\ementa
\end{longtable}

%\newpage



%%%%%%%%%%%%%%%%%%%%%%%%%%%%%%%%%%%%%%%%%%%%%%%%%%%%%%%%%%%%%%%
\begin{longtable}{|L{1.025\textwidth}|} \hline
%
{\bf V. OBJETIVOS } \\ \hline
%
\begin{itemize}
\item \textbf{Objetivo Geral:} Proporcionar aos alunos um conjunto de conhecimentos teóricos e
práticos sobre as técnicas e métodos associados à modelagem analítica e simulação de
sistemas.
\item \textbf{Objetivos Específicos:}
\begin{itemize}
\item desenvolver os conceitos de modelagem e simulação de sistemas computacionais;
\item aprofundar os conceitos relacionados a modelagem e simulação discreta;
\item capacitar os alunos a modelar sistemas discretos em uma ferramenta de simulação;
\item desenvolver um projeto de simulação discreta com os alunos;
\item avaliar o desempenho de diferentes cenários de sistemas simulados.
\end{itemize}
\end{itemize}
\\ \hline
\end{longtable}


%%%%%%%%%%%%%%%%%%%%%%%%%%%%%%%%%%%%%%%%%%%%%%%%%%%%%%%%%%%%%%%
\begin{longtable}{|L{1.025\textwidth}|} \hline
%
{\bf VI. CONTEÚDO PROGRAMÁTICO } \\ \hline
Conteúdo Teórico seguido de Conteúdo Prático com desenvolvimento de simulações no computador:\\
UNIDADE 1: Introdução à simulação\\
- Introdução à simulação\\
- Propriedades e classificação dos modelos de simulação\\
- Simulação de sistemas de computação\\
- Simulação de Sistemas Contínuos\\
\\
UNIDADE 2: Ferramentas matemáticas de auxílio à simulação \\
- Geração de números aleatórios\\
- Noções básicas em teoria dos números\\
- Geração e teste\\
- Distribuições clássicas contínuas e discretas\\
\\
UNIDADE 3: Simulação de Sistemas Discretos\\
- Simulação de sistemas discretos\\
- Técnicas estatísticas para análise de dados e de resultados de modelos de simulação\\
- Simulação de sistemas simples de filas\\
- Verificação e validação de modelos discretos\\
- Uso de ferramenta de simulação para modelagem de sistemas discretos\\
- Avaliação de desempenho em modelos de simulação
\\

\hline
\end{longtable} 

\newpage

%%%%%%%%%%%%%%%%%%%%%%%%%%%%%%%%%%%%%%%%%%%%%%%%%%%%%%%%%%%%%%%
\begin{longtable}{|L{1.025\textwidth}|} \hline
%
{\bf VII. BIBLIOGRAFIA BÁSICA} \\ \hline
\begin{enumerate}
%
\item FREITAS FILHO, Paulo José de. Introdução à modelagem e simulação de sistemas com aplicações em Arena. 2. ed. Florianópolis: Visual Books, 2008. 372 p. ISBN 9788575022283.
\item BROCKMAN, Jay B. Introdução à engenharia: modelagem e solução de problemas. Rio de Janeiro: LTC, c2010. xvii, 294 p. ISBN 9788521617266.
\item SOUZA, Antonio Carlos Zambroni de; PINHEIRO, Carlos Alberto Murari. Introdução à modelagem, análise e simulação de: sistemas dinâmicos. Rio de Janeiro: Interciência, 2008. xiii, 173 p. ISBN 9788571931886
%\item AL-BEGAIN, Khalid; TELEK, Miklós; HEINDL, Armin. Analytical and Stochastic Modeling Techniques and Applications: 15th International Conference, ASMTA 2008 Nicosia, Cyprus, June 4- 6, 2008 Proceedings. Berlin: Springer-Verlag Berlin Heidelberg, 2008. (Lecture Notes in Computer Science, 0302-9743; 5055). (Online:http://dx.doi.org/10.1007/978-3-540-68982-9)

\end{enumerate}
 \\ \hline
\end{longtable}


\newpage

%%%%%%%%%%%%%%%%%%%%%%%%%%%%%%%%%%%%%%%%%%%%%%%%%%%%%%%%%%%%%%%
\begin{longtable}{|L{1.025\textwidth}|} \hline
%
{\bf VIII. BIBLIOGRAFIA COMPLEMENTAR} \\ \hline
\begin{enumerate}

\item GARCIA, Claudio. Modelagem e simulação de processos industriais e de sistemas eletromecânicos. 2 ed. rev. e ampl. São Paulo: EDUSP, 2005. 678 p. (Acadêmica ; 11). ISBN 9788531409042.
\item ALECRIM, Paulo Dias de. Simulação computacional: para redes de computadores. Rio de Janeiro: Ciência Moderna, c2009. xii, 253 p. ISBN 9788573937701. 
\item AL-BEGAIN, Khalid; TELEK, Miklós; HEINDL, Armin. Analytical and Stochastic Modeling Techniques and Applications: 15th International Conference, ASMTA 2008 Nicosia, Cyprus, June 4- 6, 2008 Proceedings. Berlin: Springer-Verlag Berlin Heidelberg, 2008. (Lecture Notes in Computer Science, 0302-9743; 5055). (Online: http://dx.doi.org/10.1007/978-3-540-68982-9)

\item HOLLAND, John M. Designing mobile autonomous robots. Amsterdam: Elsevier, 2004. xv, 335 p. ISBN 9780750676830.
\item TRIOLA, Mario F. Introdução à estatística: atualização da tecnologia. 11. ed. Rio de
Janeiro: LTC, c2013. xxviii, 707 p. ISBN 9788521622062

%\item MARIN, Jean-Michel; ROBERT, Christian P. Bayesian Core: A Practical Approach to Computational Bayesian Statistics. New York: Springer Science+Business Media, LLC, 2007. (Springer Texts in Statistics, 1431-875X). (Online: http://dx.doi.org/10.1007/978-0-387-38983-7)


\end{enumerate}
 \\ \hline
\end{longtable}


\input aprovacao.tex


\end{document}
