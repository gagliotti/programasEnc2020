\documentclass[12pt]{article}
\usepackage[brazil]{babel}
\usepackage{graphicx,t1enc,wrapfig,amsmath,float}
\usepackage{framed,fancyhdr}
\usepackage{multirow}
\usepackage{longtable}
\usepackage{array}
\newcolumntype{L}[1]{>{\raggedright\let\newline\\\arraybackslash\hspace{0pt}}m{#1}}
\newcolumntype{C}[1]{>{\centering\let\newline\\\arraybackslash\hspace{0pt}}m{#1}}
\newcolumntype{R}[1]{>{\raggedleft\let\newline\\\arraybackslash\hspace{0pt}}m{#1}}
%%%%%%%%%%%%%
\oddsidemargin -0.5cm
\evensidemargin -0.5cm
\textwidth 17.5cm
\topmargin -1.5cm
\textheight 22cm
%%%%%%%%%%%% 

%\pagestyle{empty}

\newcommand{\semestre}{2018.2}

\newcommand{\disciplina}{LÓGICA APLICADA À COMPUTAÇÃO}
\newcommand{\codigo}{DEC7502}


%%%%%%%%%%%%%%%%%%%%%%%%%%%%%%%%%%%%%%%%%%%%%%%%%%%%%%%
%%%%%%%%%%%%% CRETIDOS
\newcommand{\creditosT}{2}
\newcommand{\creditosP}{2}

%%%%%%%%%%%%%%%%%%%%%%%%%%%%%%%%%%%%%%%%%%%%%%%%%%%%%%%
%%%%%%%%%%%%%% REQUISITOS
\newcommand{\requisitoA}{}
\newcommand{\requisitoB}{}
\newcommand{\requisitoC}{}

%%%%%%%%%%%%%%%%%%%%%%%%%%%%%%%%%%%%%%%%%%%%%%%%%%%%%%%
%%%%%%%%%%%%%%% Atende aos Cursos
\newcommand{\cursoA}{Graduação em Engenharia de Computação \\ \hline}
\newcommand{\cursoB}{}%Graduação em Tecnologias da Informação e Comunicação \\ \hline}
\newcommand{\cursoC}{}%Graduação em Engenharia de Energia \\ \hline}

%%%%%%%%%%%%%%%%%%%%%%%%%%%%%%%%%%%%%%%%%%%%%%%%%%%%%%%%
%%%%%%%%%% EMENTA
\newcommand{\ementa}{
Introdução à Lógica; Lógica Proposicional - símbolos proposicionais, tabelas verdade, operadores lógicos, fórmulas bem formadas, tautologias, contradições, contingência, métodos de prova; Lógica de Predicados - sintaxe e semântica, interpretação das variáveis, funções e predicados, equivalência entre fórmulas, métodos de prova; Programação em Lógica - Introdução, cláusulas de Horn, resolvente e unificação, SLD derivação e refutação, linguagem de programação Prolog. Lógicas não Clássicas - lógica modal, de multivalores, temporal e não monotônica.
 \\ \hline
}


\begin{document}


%%%%%%%%%%%%%%%%%%%%%%%%%%%%%%%%%%%%%%%%%%%%%%%%%%%%%%%%%%%%%
\input cabecalho.tex


%%%%%%%%%%%%%%%%%%%%%%%%%%%%%%%%%%%%%%%%%%%%%%%%%%%%%%%%%%%%%
\begin{longtable}{|C{0.11\textwidth}|C{0.29\textwidth}|C{0.09\textwidth}|C{0.09\textwidth}|C{0.15\textwidth}|C{0.158\textwidth}|} \hline
%
\multicolumn{6}{|l|}{{\bf I. IDENTIFICAÇÃO DA DISCIPLINA}} \\ \hline
%
\multirow{3}*{{\small CÓDIGO}} & \multirow{3}*{NOME DA DISCIPLINA} &\multicolumn{2}{c|}{{\small N$^\circ$ DE HORAS-AULA }} & {{\small TOTAL DE}} & \multirow{3}*{{\small MODALIDADE}} \\ 
%
& & \multicolumn{2}{c|}{\small SEMANAIS}  & {\small HORAS-AULA} & \\ \cline{3-4}
%
& & {\tiny TEÓRICAS} & {\tiny PRÁTICAS} & {\small SEMESTRAIS} & \\ \hline
% codigo da disciplina carga horaria: teorica - pratica e total
{\bf \small \codigo} & {\bf \small \disciplina } & {\bf \creditosT} & {\bf \creditosP} & {\bf 72} & Presencial\\ \hline
\end{longtable}


%%%%%%%%%%%%%%%%%%%%%%%%%%%%%%%%%%%%%%%%%%%%%%%%%%%%%%%%%%%%%%
\begin{longtable}{|C{0.12\textwidth}|L{0.736\textwidth}|C{0.12\textwidth}|} \hline
%
\multicolumn{3}{|l|}{{\bf II. PRÉ-REQUISITO(S)}} \\ \hline
%
CÓDIGO & NOME DA DISCIPLINA & CURSO \\ \hline	
%
\requisitoA
\requisitoB
\requisitoC
\end{longtable}


%%%%%%%%%%%%%%%%%%%%%%%%%%%%%%%%%%%%%%%%%%%%%%%%%%%%%%%%%%%%%%
\begin{longtable}{|L{1.025\textwidth}|} \hline
%
{\bf III. CURSO(S) PARA O(S) QUAL(IS) A DISCIPLINA É OFERECIDA } \\ \hline
%
\cursoA 
\cursoB
\cursoC

\end{longtable}

%%%%%%%%%%%%%%%%%%%%%%%%%%%%%%%%%%%%%%%%%%%%%%%%%%%%%%%%%%%%%%
\begin{longtable}{|L{1.025\textwidth}|} \hline
%
{\bf IV. EMENTA } \\ \hline
%
\ementa
\end{longtable}

\newpage



%%%%%%%%%%%%%%%%%%%%%%%%%%%%%%%%%%%%%%%%%%%%%%%%%%%%%%%%%%%%%%%
\begin{longtable}{|L{1.025\textwidth}|} \hline
%
{\bf V. OBJETIVOS } \\ \hline
%
Objetivo Geral:\\
Esta disciplina tem como objetivo geral possibilitar aos alunos o uso da lógica como uma ferramenta para a formalização e dedução de problemas inerentes a computação.\\
\\
Objetivos Específicos:
\begin{itemize}
\item  Desenvolver a capacidade de raciocínio lógico para a resolução de problemas;
\item Abordar as técnicas de prova de teoremas usando os métodos de prova da lógica proposicional e de predicados;
\item Abordar as potencialidades de uma linguagem de programação em lógica;
\item Mostrar como uma linguagem lógica pode ser usada para a especificação formal de sistemas.
\end{itemize}
\\ \hline
\end{longtable}


%%%%%%%%%%%%%%%%%%%%%%%%%%%%%%%%%%%%%%%%%%%%%%%%%%%%%%%%%%%%%%%
\begin{longtable}{|L{1.025\textwidth}|} \hline
%
{\bf VI. CONTEÚDO PROGRAMÁTICO } \\ \hline
Conteúdo Teórico seguido de Conteúdo Prático com desenvolvimento de problemas em computador:\\
\\
UNIDADE 1: Introdução\\
 Apresentação da disciplina (ementa, bibliografia, metodologia e avaliações)\\
 Histórico\\
 Aplicações\\
 Revisão Álgebra / Lógica Booleana\\
 Tabela Verdade\\
 Expressões Lógicas\\
 Equivalência Expressões - Tabelas Verdade - Circuitos Digitais\\
 Modelagem Lógica
 Equivalências Lógicas e Simplificações de Expressões Lógicas\\
\\
UNIDADE 2: Lógica Proposicional\\
 Símbolos proposicionais\\
 Operadores lógicos\\
 Formulas bem formadas\\
 Tautologia e contradição\\
 Contingência\\
 Métodos de prova\\
\\
UNIDADE 3: Lógica de Predicados\\
 Sintaxe e semântica da lógica de predicados\\
 Interpretação de variáveis, funções e predicados\\
 Equivalências entre formulas\\
 Métodos de prova\\
\\
UNIDADE 4: Programação em Lógica\\
 Introdução a programação em lógica\\
 Cláusulas de Horn\\
 Resolventes e unificação\\
 SLD derivação e refutação\\
 Programação em lógica com Prolog\\
 Sintaxe e semântica de Prolog\\
 Resolução de problemas com Prolog\\
\\
UNIDADE 5: Lógicas não Clássicas\\
 Lógica modal\\
 Lógicas de multivalores\\
 Lógicas não-monotônicas\\
 Lógica temporal\\

\\ \hline
\end{longtable} 

%\newpage

%%%%%%%%%%%%%%%%%%%%%%%%%%%%%%%%%%%%%%%%%%%%%%%%%%%%%%%%%%%%%%%
\begin{longtable}{|L{1.025\textwidth}|} \hline
%
{\bf VII. BIBLIOGRAFIA BÁSICA} \\ \hline
\begin{enumerate}
%
\item GERSTING, J. L. Fundamentos Matemáticos para a Ciência da Computação. 5a ed. LTC, 2004. 
\item SOUZA, J. N. Lógica para Ciência da Computação - Uma Introdução Concisa. 2a ed. Rio de Janeiro: Editora Campus, 2008. v. 1. 223 p. 
\item FILHO, Alencar E. Iniciação a Lógica Matemática. 21a ed. São Paulo: Nobel, 2008.
\end{enumerate}
 \\ \hline
\end{longtable}


\newpage

%%%%%%%%%%%%%%%%%%%%%%%%%%%%%%%%%%%%%%%%%%%%%%%%%%%%%%%%%%%%%%%
\begin{longtable}{|L{1.025\textwidth}|} \hline
%
{\bf VIII. BIBLIOGRAFIA COMPLEMENTAR} \\ \hline
\begin{enumerate}
\item SILVA, Flávio S. C. et al. Lógica para Computação. Editora Thomson, 2006 
\item MENEZES, P. B. Matemática Discreta para Computação e Informática. 2a ed. Porto Alegre: Bookman, 2008.
\item LIPSCHUTZ, Seymour; LIPSON, Marc. Matemática discreta. 3. ed. Porto Alegre: Bookman, 2013. xi, 471 p. (Coleção Schaum). ISBN 9788565837736.
\item SCHEINERMAN, Edward R. Matemática discreta: uma introdução. São Paulo: Cengage Learning, 2011. xxiii, 573 p. ISBN 9788522107964.
\item LIPSCHUTZ, Seymour; LIPSON, Marc. Teoria e problemas de matemática discreta. 2. ed. Porto Alegre: Bookman, 2004. 511 p. (Schaum). ISBN 9788536303611 (broch.).
\end{enumerate}
 \\ \hline
\end{longtable}


\input aprovacao.tex


\end{document}
