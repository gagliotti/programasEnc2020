\documentclass[12pt]{article}
\usepackage[brazil]{babel}
\usepackage{graphicx,t1enc,wrapfig,amsmath,float}
\usepackage{framed,fancyhdr}
\usepackage{multirow}
\usepackage{longtable}
\usepackage{array}
\newcolumntype{L}[1]{>{\raggedright\let\newline\\\arraybackslash\hspace{0pt}}m{#1}}
\newcolumntype{C}[1]{>{\centering\let\newline\\\arraybackslash\hspace{0pt}}m{#1}}
\newcolumntype{R}[1]{>{\raggedleft\let\newline\\\arraybackslash\hspace{0pt}}m{#1}}
%%%%%%%%%%%%%
\oddsidemargin -0.5cm
\evensidemargin -0.5cm
\textwidth 17.5cm
\topmargin -1.5cm
\textheight 22cm
%%%%%%%%%%%% 

%\pagestyle{empty}

\newcommand{\semestre}{2018.2}

\newcommand{\disciplina}{LINGUAGEM DE PROGRAMAÇÃO I}
\newcommand{\codigo}{DEC7531}


%%%%%%%%%%%%%%%%%%%%%%%%%%%%%%%%%%%%%%%%%%%%%%%%%%%%%%%
%%%%%%%%%%%%% CRETIDOS
\newcommand{\creditosT}{0}
\newcommand{\creditosP}{4}

%%%%%%%%%%%%%%%%%%%%%%%%%%%%%%%%%%%%%%%%%%%%%%%%%%%%%%%
%%%%%%%%%%%%%% REQUISITOS
\newcommand{\requisitoA}{}
\newcommand{\requisitoB}{}
\newcommand{\requisitoC}{}

%%%%%%%%%%%%%%%%%%%%%%%%%%%%%%%%%%%%%%%%%%%%%%%%%%%%%%%
%%%%%%%%%%%%%%% Atende aos Cursos
\newcommand{\cursoA}{Graduação em Engenharia de Computação \\ \hline}
\newcommand{\cursoB}{}%Graduação em Tecnologias da Informação e Comunicação \\ \hline}
\newcommand{\cursoC}{}%Graduação em Engenharia de Energia \\ \hline}

%%%%%%%%%%%%%%%%%%%%%%%%%%%%%%%%%%%%%%%%%%%%%%%%%%%%%%%%
%%%%%%%%%% EMENTA
\newcommand{\ementa}{
Programação Estruturada: linguagens que suportam programação estruturada. Ambientes de Programação: escolha, instalação e execução. Variáveis: nomeação, declaração, inicialização, tipos de dados. Expressões: expressão aritméticas, expressão literal, expressão lógicas, expressões relacionais. Arquitetura de Programa Mínimo: paradigmas, regras de escopo, funções, modularização. Estruturas de Dados Simples: vetores, matrizes, registros. Estruturas de Controle de Fluxo: Linear, condicional, repetição. Ponteiros: definição, declaração e uso. Funções: definição, declaração, tipos de passagem de parâmetro. Alocação Dinâmica: definição, declaração e uso. Entrada e Saída de Dados: arquivos, acesso sequencial, acesso direto.
 \\ \hline
}


\begin{document}


%%%%%%%%%%%%%%%%%%%%%%%%%%%%%%%%%%%%%%%%%%%%%%%%%%%%%%%%%%%%%
\input cabecalho.tex


%%%%%%%%%%%%%%%%%%%%%%%%%%%%%%%%%%%%%%%%%%%%%%%%%%%%%%%%%%%%%
\begin{longtable}{|C{0.11\textwidth}|C{0.29\textwidth}|C{0.09\textwidth}|C{0.09\textwidth}|C{0.15\textwidth}|C{0.158\textwidth}|} \hline
%
\multicolumn{6}{|l|}{{\bf I. IDENTIFICAÇÃO DA DISCIPLINA}} \\ \hline
%
\multirow{3}*{{\small CÓDIGO}} & \multirow{3}*{NOME DA DISCIPLINA} &\multicolumn{2}{c|}{{\small N$^\circ$ DE HORAS-AULA }} & {{\small TOTAL DE}} & \multirow{3}*{{\small MODALIDADE}} \\ 
%
& & \multicolumn{2}{c|}{\small SEMANAIS}  & {\small HORAS-AULA} & \\ \cline{3-4}
%
& & {\tiny TEÓRICAS} & {\tiny PRÁTICAS} & {\small SEMESTRAIS} & \\ \hline
% codigo da disciplina carga horaria: teorica - pratica e total
{\bf \small \codigo} & {\bf \small \disciplina } & {\bf \creditosT} & {\bf \creditosP} & {\bf 72} & Presencial\\ \hline
\end{longtable}


%%%%%%%%%%%%%%%%%%%%%%%%%%%%%%%%%%%%%%%%%%%%%%%%%%%%%%%%%%%%%%
\begin{longtable}{|C{0.12\textwidth}|L{0.736\textwidth}|C{0.12\textwidth}|} \hline
%
\multicolumn{3}{|l|}{{\bf II. PRÉ-REQUISITO(S)}} \\ \hline
%
CÓDIGO & NOME DA DISCIPLINA & CURSO \\ \hline	
%
\requisitoA
\requisitoB
\requisitoC
\end{longtable}


%%%%%%%%%%%%%%%%%%%%%%%%%%%%%%%%%%%%%%%%%%%%%%%%%%%%%%%%%%%%%%
\begin{longtable}{|L{1.025\textwidth}|} \hline
%
{\bf III. CURSO(S) PARA O(S) QUAL(IS) A DISCIPLINA É OFERECIDA } \\ \hline
%
\cursoA 
\cursoB
\cursoC

\end{longtable}

%%%%%%%%%%%%%%%%%%%%%%%%%%%%%%%%%%%%%%%%%%%%%%%%%%%%%%%%%%%%%%
\begin{longtable}{|L{1.025\textwidth}|} \hline
%
{\bf IV. EMENTA } \\ \hline
%
\ementa
\end{longtable}

%\newpage



%%%%%%%%%%%%%%%%%%%%%%%%%%%%%%%%%%%%%%%%%%%%%%%%%%%%%%%%%%%%%%%
\begin{longtable}{|L{1.025\textwidth}|} \hline
%
{\bf V. OBJETIVOS } \\ \hline
%
Objetivos Gerais:\\
 O aluno ao final desta disciplina deverá ser capaz de transpor um algoritmo, tal como apreendido em lógica de programação, para uma linguagem de programação sob o paradigma da programação estruturada.
\\
\\
\\
\\
\\
Objetivos Específicos: \\
Domínio do Contexto Científico e Tecnológico em Linguagem de Programação. Utilização de Ferramentas e Técnicas de Programação. Domínio do Paradigma Entrada, Processamento e Saída de Dados.

\\ \hline
\end{longtable}


%%%%%%%%%%%%%%%%%%%%%%%%%%%%%%%%%%%%%%%%%%%%%%%%%%%%%%%%%%%%%%%
\begin{longtable}{|L{1.025\textwidth}|} \hline
%
{\bf VI. CONTEÚDO PROGRAMÁTICO } \\ \hline
Unidade 1: Introdução ao paradigma da programação estruturada. Conceituação de elementos básicos da linguagem de programação. Estruturas de controle de fluxo. Arquitetura de programas.\\
\\
Unidade 2: Estruturas de dados simples. Variáveis compostas. Variáveis homogenias: vetores e matrizes. Variáveis heterogenias.\\
\\
Unidade 3: Funções, chamada de funções, passagem de parâmetros. Ponteiros. Alocação de Memória. Alocação Estática. Alocação Dinâmica. Processamento de Strings. Entrada e Saída de dados. Arquivos e sistemas de arquivo.

\\ \hline
\end{longtable} 

%\newpage

%%%%%%%%%%%%%%%%%%%%%%%%%%%%%%%%%%%%%%%%%%%%%%%%%%%%%%%%%%%%%%%
\begin{longtable}{|L{1.025\textwidth}|} \hline
%
{\bf VII. BIBLIOGRAFIA BÁSICA} \\ \hline
\begin{enumerate}
%
\item MIZRAHI, Victorine V. Treinamento em Linguagem C, Módulo 1. Pearson, 2004. 
\item Mizrahi, V. V., Treinamento em Linguagem C - Módulo 2. Pearson, 2004. 
\item FEOFILOFF, Paulo. Algoritmos em linguagem C. Rio de Janeiro: Elsevier, c2009. 
\end{enumerate}
 \\ \hline
\end{longtable}


\newpage

%%%%%%%%%%%%%%%%%%%%%%%%%%%%%%%%%%%%%%%%%%%%%%%%%%%%%%%%%%%%%%%
\begin{longtable}{|L{1.025\textwidth}|} \hline
%
{\bf VIII. BIBLIOGRAFIA COMPLEMENTAR} \\ \hline
\begin{enumerate}
\item SEDGEWICK, Robert; WAYNE, Kevin Daniel. Algorithms. 4th ed. Upper Saddle River: Addison Wesley 
\item Robert Sedgewick, Algorithms in C, Part 5 (Graph Algorithms) Addison Wesley. 2002.
\item FARRER, Harry et al. Algoritmos estruturados. 3. ed. Rio de Janeiro: LTC, c1999. 284 p. (Programação 56 estruturada de computadores) 
\item E-book: C - Program Structure, disponível em http://www.tutorialspoint.com/cprogramming/c\_program\_structure.htm 
\item  E-book: The C Book, disponível em http://publications.gbdirect.co.uk/c\_book/

\end{enumerate}
 \\ \hline
\end{longtable}


\input aprovacao.tex


\end{document}
