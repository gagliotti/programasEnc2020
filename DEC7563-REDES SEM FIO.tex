\documentclass[12pt]{article}
\usepackage[brazil]{babel}
\usepackage{graphicx,t1enc,wrapfig,amsmath,float}
\usepackage{framed,fancyhdr}
\usepackage{multirow}
\usepackage{longtable}
\usepackage{array}
\newcolumntype{L}[1]{>{\raggedright\let\newline\\\arraybackslash\hspace{0pt}}m{#1}}
\newcolumntype{C}[1]{>{\centering\let\newline\\\arraybackslash\hspace{0pt}}m{#1}}
\newcolumntype{R}[1]{>{\raggedleft\let\newline\\\arraybackslash\hspace{0pt}}m{#1}}
%%%%%%%%%%%%%
\oddsidemargin -0.5cm
\evensidemargin -0.5cm
\textwidth 17.5cm
\topmargin -1.5cm
\textheight 22cm
%%%%%%%%%%%% 

%\pagestyle{empty}

\newcommand{\semestre}{2023.1}

\newcommand{\disciplina}{REDES SEM FIO}
\newcommand{\codigo}{DEC7563}


%%%%%%%%%%%%%%%%%%%%%%%%%%%%%%%%%%%%%%%%%%%%%%%%%%%%%%%
%%%%%%%%%%%%% CRETIDOS
\newcommand{\creditosT}{3}
\newcommand{\creditosP}{1}

%%%%%%%%%%%%%%%%%%%%%%%%%%%%%%%%%%%%%%%%%%%%%%%%%%%%%%%
%%%%%%%%%%%%%% REQUISITOS
\newcommand{\requisitoA}{DEC 7557 & Redes de Computadores & ENC \hline}
\newcommand{\requisitoB}{}
\newcommand{\requisitoC}{}

%%%%%%%%%%%%%%%%%%%%%%%%%%%%%%%%%%%%%%%%%%%%%%%%%%%%%%%
%%%%%%%%%%%%%%% Atende aos Cursos
\newcommand{\cursoA}{Graduação em Engenharia de Computação \\ \hline}
\newcommand{\cursoB}{}%Graduação em Tecnologias da Informação e Comunicação \\ \hline}
\newcommand{\cursoC}{}

%%%%%%%%%%%%%%%%%%%%%%%%%%%%%%%%%%%%%%%%%%%%%%%%%%%%%%%%
%%%%%%%%%% EMENTA
\newcommand{\ementa}{
Introdução à comunicação sem fio. Principais redes sem fio: ad hoc e infraestruturadas. Camada física (PHY): protocolos, topologias, potência de transmissão e recepção, taxas de transferência.
Protocolos da camada de enlace (MAC): controle de acesso ao meio livre de contenção, controle de acesso ao meio com contenção. Roteamento e QoS. Principais tecnologias de Low-Rate Wireless Personal Area Networks (LR-WPAN): eficiência de energia, escalabilidade (aplicações
para IoT) e confiabilidade de redes LR-WPAN. Desenvolvimento de projeto ou aplicação com redes de sensores sem fio.
\\ \hline
}


\begin{document}

%%%%%%%%%%%%%%%%%%%%%%%%%%%%%%%%%%%%%%%%%%%%%%%%%%%%%%%%%%%%%

\input cabecalho.tex


%%%%%%%%%%%%%%%%%%%%%%%%%%%%%%%%%%%%%%%%%%%%%%%%%%%%%%%%%%%%%
\begin{longtable}{|C{0.11\textwidth}|C{0.29\textwidth}|C{0.09\textwidth}|C{0.09\textwidth}|C{0.15\textwidth}|C{0.158\textwidth}|} \hline
%
\multicolumn{6}{|l|}{{\bf I. IDENTIFICAÇÃO DA DISCIPLINA}} \\ \hline
%
\multirow{3}*{{\small CÓDIGO}} & \multirow{3}*{NOME DA DISCIPLINA} &\multicolumn{2}{c|}{{\small N$^\circ$ DE HORAS-AULA }} & {{\small TOTAL DE}} & \multirow{3}*{{\small MODALIDADE}} \\ 
%
& & \multicolumn{2}{c|}{\small SEMANAIS}  & {\small HORAS-AULA} & \\ \cline{3-4}
%
& & {\tiny TEÓRICAS} & {\tiny PRÁTICAS} & {\small SEMESTRAIS} & \\ \hline
% codigo da disciplina carga horaria: teorica - pratica e total
{\bf \small \codigo} & {\bf \small \disciplina } & {\bf \creditosT} & {\bf \creditosP} & {\bf 72} & Presencial\\ \hline
\end{longtable}


%%%%%%%%%%%%%%%%%%%%%%%%%%%%%%%%%%%%%%%%%%%%%%%%%%%%%%%%%%%%%%
\begin{longtable}{|C{0.12\textwidth}|L{0.736\textwidth}|C{0.12\textwidth}|} \hline
%
\multicolumn{3}{|l|}{{\bf II. PRÉ-REQUISITO(S)}} \\ \hline
%
CÓDIGO & NOME DA DISCIPLINA & CURSO \\ \hline	
%
\requisitoA
\requisitoB
\requisitoC
\end{longtable}


%%%%%%%%%%%%%%%%%%%%%%%%%%%%%%%%%%%%%%%%%%%%%%%%%%%%%%%%%%%%%%
\begin{longtable}{|L{1.025\textwidth}|} \hline
%
{\bf III. CURSO(S) PARA O(S) QUAL(IS) A DISCIPLINA É OFERECIDA } \\ \hline
%
\cursoA 
\cursoB
\cursoC

\end{longtable}

%%%%%%%%%%%%%%%%%%%%%%%%%%%%%%%%%%%%%%%%%%%%%%%%%%%%%%%%%%%%%%
\begin{longtable}{|L{1.025\textwidth}|} \hline
%
{\bf IV. EMENTA } \\ \hline
%
\ementa
\end{longtable}

\newpage



%%%%%%%%%%%%%%%%%%%%%%%%%%%%%%%%%%%%%%%%%%%%%%%%%%%%%%%%%%%%%%%
\begin{longtable}{|L{1.025\textwidth}|} \hline
%
{\bf V. OBJETIVOS } \\ \hline
Objetivo Geral:\\

Capacitar o estudante a analisar de forma crítica os problemas de larga
escala e propor soluções empregando o uso das tecnologias de Redes Sem Fio mais apropriadas. 
\\
Objetivos Específicos:

\begin{itemize}
\item Aprofundar o conceito de arquiteturas e protocolos sem fio e os princípios básicos de operação de transmissão de dados;
\item  Dimensionar o uso de tecnologias sem fio para solução de problemas de larga escala; 
\item  Compreender o desenvolvimento de protocolos de comunicação sem fio;
\item Dominar questões relacionadas a escalabilidade, abrangência e topologias de redes de comunicação sem fio;
\item Propor soluções para projetos de redes sem fio empregando diferentes recursos tecnológicos.
\end{itemize}
\\ \hline
\end{longtable}


%%%%%%%%%%%%%%%%%%%%%%%%%%%%%%%%%%%%%%%%%%%%%%%%%%%%%%%%%%%%%%%
\begin{longtable}{|L{1.025\textwidth}|} \hline
%
{\bf VI. CONTEÚDO PROGRAMÁTICO } \\ \hline
Conteúdo Teórico seguido de Conteúdo Prático com desenvolvimento de um projeto de redes sem fio: \\
Unidade 1. Introdução à comunicação sem fio (6 horas/aula)\\
 Principais redes sem fio: ad hoc e infraestruturadas\\
 Tipos de redes sem fio\\
 Componentes básicos da arquitetura de redes sem fio: camadas, hosts, estações base e enlaces\\
 Características e desafios da transmissão sem fio\\

 Unidade 2 - Camada física (PHY) (10 horas/aula)\\
 Principais funções da camada física\\
 Tipos de transmissão e multiplexação\\
 Potência de transmissão e recepção\\
 Taxas de transferência e alcance\\
 Estudos de casos\\

 Unidade 3 - Protocolos da camada de enlace (MAC) (10 horas/aula)\\
 Principais funções da camada MAC
 Controle de acesso ao meio livre de contenção
 Controle de acesso ao meio com contenção
 Estudos de casos
\\
Unidade 4 - Roteamento e QoS (10 horas/aula)\\
 QoS em redes sem fio\\
 Principais tecnologias de Low-Rate e topologias\\
 Desafios do roteamento em redes sem fio\\
 Exemplos de protocolos de roteamento na prática: hierárquicos, por demanda, etc\\

Unidade 5 - Wireless Personal Area Networks (LR-WPAN) (18 horas/aulas)
  Eficiência de energia
  Escalabilidade e aplicações para IoT
  Nuvem e desafios de segurança
  Confiabilidade, privacidade e segurança 

 Unidade 6 - Desenvolvimento de projeto ou aplicação com redes de sensores sem fio (18 horas/aulas)



\\ \hline
\end{longtable} 

%\newpage


%%%%%%%%%%%%%%%%%%%%%%%%%%%%%%%%%%%%%%%%%%%%%%%%%%%%%%%%%%%%%%%
\begin{longtable}{|L{1.025\textwidth}|} \hline
%
{\bf VII. BIBLIOGRAFIA BÁSICA} \\ \hline
\begin{enumerate}
\item KUROSE, James F.; ROSS, Keith W. Redes de computadores e a Internet: uma abordagem top-down. 5. ed. São Paulo: Pearson Addison Wesley, 2010. xxiii, 614 p. ISBN 9788588639973.
\item FOROUZAN, Behrouz A.; FEGAN, Sophia Chung; GRIESI, Ariovaldo. Comunicação de dados e redes de computadores. 4. ed. São Paulo: McGraw Hill, 2008. 1134 p. ISBN 9788586804885. 
\item LI, Deying; CHENG, Maggie Xiaoyan. Advances in Wireless Ad Hoc and Sensor Networks. Boston: Springer-Verlag US, 2008. (Signals and Communication Technology, 1860-4862). 
\end{enumerate}
 \\ \hline
\end{longtable}


%\newpage

%%%%%%%%%%%%%%%%%%%%%%%%%%%%%%%%%%%%%%%%%%%%%%%%%%%%%%%%%%%%%%%
\begin{longtable}{|L{1.025\textwidth}|} \hline
%
{\bf VIII. BIBLIOGRAFIA COMPLEMENTAR} \\ \hline
\begin{enumerate}
\item TRONCO, Tania Regina. Redes da nova geração: arquitetura de convergência das redes : IP, telefônica e óptica. 2. ed. rev. e atual. São Paulo: Érica, 2014. 164 p. ISBN 9788536501383.
\item STALLINGS, William. Redes e sistemas de comunicação de dados. Rio de Janeiro: Elsevier, c2005. xvi, 449 p. ISBN 9788535217315. 
\item  MEDEIROS, Julio Cesar de O. Princípios de telecomunicações: teoria e prática. 4. ed. rev. São Paulo: Érica, 2014. 320 p. ISBN 9788536500331. 
\item  RAPPAPORT, Theodore S. Comunicações sem fio: princípios e práticas. 2. ed. São Paulo: Pearson Prentice Hall, 2009. xix, 409 p. ISBN 9788576051985. 
\item FALUDI, Robert. Building wireless sensor networks. Sebastopol: O'Reilly, 2010. xviii, 300 p. ISBN 9780596807733. 
%
\end{enumerate}
 \\ \hline
\end{longtable}


\input aprovacao.tex


\end{document}
