\documentclass[12pt]{article}
\usepackage[brazil]{babel}
\usepackage{graphicx,t1enc,wrapfig,amsmath,float}
\usepackage{framed,fancyhdr}
\usepackage{multirow}
\usepackage{longtable}
\usepackage{array}
\newcolumntype{L}[1]{>{\raggedright\let\newline\\\arraybackslash\hspace{0pt}}m{#1}}
\newcolumntype{C}[1]{>{\centering\let\newline\\\arraybackslash\hspace{0pt}}m{#1}}
\newcolumntype{R}[1]{>{\raggedleft\let\newline\\\arraybackslash\hspace{0pt}}m{#1}}
%%%%%%%%%%%%%
\oddsidemargin -0.5cm
\evensidemargin -0.5cm
\textwidth 17.5cm
\topmargin -1.5cm
\textheight 22cm
%%%%%%%%%%%% 

%\pagestyle{empty}

\newcommand{\semestre}{2018.2}

\newcommand{\disciplina}{REDES SEM FIO}
\newcommand{\codigo}{DEC7563}


%%%%%%%%%%%%%%%%%%%%%%%%%%%%%%%%%%%%%%%%%%%%%%%%%%%%%%%
%%%%%%%%%%%%% CRETIDOS
\newcommand{\creditosT}{3}
\newcommand{\creditosP}{1}

%%%%%%%%%%%%%%%%%%%%%%%%%%%%%%%%%%%%%%%%%%%%%%%%%%%%%%%
%%%%%%%%%%%%%% REQUISITOS
\newcommand{\requisitoA}{}
\newcommand{\requisitoB}{}
\newcommand{\requisitoC}{}

%%%%%%%%%%%%%%%%%%%%%%%%%%%%%%%%%%%%%%%%%%%%%%%%%%%%%%%
%%%%%%%%%%%%%%% Atende aos Cursos
\newcommand{\cursoA}{Graduação em Engenharia de Computação \\ \hline}
\newcommand{\cursoB}{}%Graduação em Tecnologias da Informação e Comunicação \\ \hline}
\newcommand{\cursoC}{}

%%%%%%%%%%%%%%%%%%%%%%%%%%%%%%%%%%%%%%%%%%%%%%%%%%%%%%%%
%%%%%%%%%% EMENTA
\newcommand{\ementa}{
Introdução a redes sem fio. Fundamentos de transmissão e propragação de sinal. Tipos de Antenas. Protocolos e Mecanismos de Controle: Acesso ao Meio, Topologia, Potência, Ruído e Taxa. Padronização de redes sem fio (Padrões IEEE WPAN, WLAN e WMAN). Roteamento e QoS em redes sem fio: ad hoc e infraestruturadas, Mobilidade IP, TCP móvel. Estudos de casos: redes locais, redes celulares, redes de sensores e redes veiculares.
\\ \hline
}


\begin{document}

%%%%%%%%%%%%%%%%%%%%%%%%%%%%%%%%%%%%%%%%%%%%%%%%%%%%%%%%%%%%%

\input cabecalho.tex


%%%%%%%%%%%%%%%%%%%%%%%%%%%%%%%%%%%%%%%%%%%%%%%%%%%%%%%%%%%%%
\begin{longtable}{|C{0.11\textwidth}|C{0.29\textwidth}|C{0.09\textwidth}|C{0.09\textwidth}|C{0.15\textwidth}|C{0.158\textwidth}|} \hline
%
\multicolumn{6}{|l|}{{\bf I. IDENTIFICAÇÃO DA DISCIPLINA}} \\ \hline
%
\multirow{3}*{{\small CÓDIGO}} & \multirow{3}*{NOME DA DISCIPLINA} &\multicolumn{2}{c|}{{\small N$^\circ$ DE HORAS-AULA }} & {{\small TOTAL DE}} & \multirow{3}*{{\small MODALIDADE}} \\ 
%
& & \multicolumn{2}{c|}{\small SEMANAIS}  & {\small HORAS-AULA} & \\ \cline{3-4}
%
& & {\tiny TEÓRICAS} & {\tiny PRÁTICAS} & {\small SEMESTRAIS} & \\ \hline
% codigo da disciplina carga horaria: teorica - pratica e total
{\bf \small \codigo} & {\bf \small \disciplina } & {\bf \creditosT} & {\bf \creditosP} & {\bf 72} & Presencial\\ \hline
\end{longtable}


%%%%%%%%%%%%%%%%%%%%%%%%%%%%%%%%%%%%%%%%%%%%%%%%%%%%%%%%%%%%%%
\begin{longtable}{|C{0.12\textwidth}|L{0.736\textwidth}|C{0.12\textwidth}|} \hline
%
\multicolumn{3}{|l|}{{\bf II. PRÉ-REQUISITO(S)}} \\ \hline
%
CÓDIGO & NOME DA DISCIPLINA & CURSO \\ \hline	
%
\requisitoA
\requisitoB
\requisitoC
\end{longtable}


%%%%%%%%%%%%%%%%%%%%%%%%%%%%%%%%%%%%%%%%%%%%%%%%%%%%%%%%%%%%%%
\begin{longtable}{|L{1.025\textwidth}|} \hline
%
{\bf III. CURSO(S) PARA O(S) QUAL(IS) A DISCIPLINA É OFERECIDA } \\ \hline
%
\cursoA 
\cursoB
\cursoC

\end{longtable}

%%%%%%%%%%%%%%%%%%%%%%%%%%%%%%%%%%%%%%%%%%%%%%%%%%%%%%%%%%%%%%
\begin{longtable}{|L{1.025\textwidth}|} \hline
%
{\bf IV. EMENTA } \\ \hline
%
\ementa
\end{longtable}

\newpage



%%%%%%%%%%%%%%%%%%%%%%%%%%%%%%%%%%%%%%%%%%%%%%%%%%%%%%%%%%%%%%%
\begin{longtable}{|L{1.025\textwidth}|} \hline
%
{\bf V. OBJETIVOS } \\ \hline
Objetivo Geral:\\

Capacitar o estudante a analisar de forma crítica os problemas e soluções das Redes Sem Fio na transmissão de dados em diversos tipos de aplicações. 
\\
Objetivos Específicos:

\begin{itemize}
\item Aprofundar o conceito de Arquitetura Multicamadas e os princípios básicos de operação das Redes de Computadores.
\item  Aprofundar os conceitos sobre a organização da arquitetura e os conceitos associados ao Modelo de Referência OSI e da arquitetura de protocolos TCP/IP. 
\item  Compreender as características associadas aos Meios de Transmissão mais utilizados para transferência de dados em Redes de Computadores.
\end{itemize}
\\ \hline
\end{longtable}


%%%%%%%%%%%%%%%%%%%%%%%%%%%%%%%%%%%%%%%%%%%%%%%%%%%%%%%%%%%%%%%
\begin{longtable}{|L{1.025\textwidth}|} \hline
%
{\bf VI. CONTEÚDO PROGRAMÁTICO } \\ \hline
Conteúdo Teórico seguido de Conteúdo Prático com desenvolvimento de um projeto de redes sem fio: \\
Unidade 1. Introdução às  Redes sem Fio (6 horas/aula)\\
 Desenvolvimento das redes sem fio\\
 Tipos de redes sem fio (WWAN, WMAN, WLAN, WPAN)\\
 Componentes de redes sem fio: hosts, estações base e enlaces\\
 Características : vantagens e desvantagens\\
 Acesso múltiplo por divisão de código (CDMA)\\
\\
Unidade 2. Fundamentos de transmissão e propagação do sinal (8 horas aula)\\
 Antenas (Ominidirecional, direcional e semi-direcional)\\
 Mecanismos de controle: acesso ao meio, topologia, potencia, ruído e taxa;\\
 Espalhamento espectral\\
 SNR\\
\\
Unidade 3. Padronização de Redes sem Fio (20 horas aula)*\\
 Padrões de redes WLAN (IEEE 802.11)\\
 Padrões de redes WPAN (IEEE 802.15.1 e IEEE 802.15.4)\\
 Padrões de redes WMAN (IEEE 802.16)\\

Unidade 4.  Roteamento em redes sem fio (10 horas aula)\\
 QoS em redes em sem fio\\
Mobilidade IP\\
 TCP Móvel\\
\\
Unidade 5. Diferentes tipos de redes sem fio (10 horas aula) *\\
 Estudos de casos: redes locais, redes celulares, redes de sensores e redes veiculares.\\

*desenvolvimento de prática aplicando os conceitos estudados em sala de aula.\\



\\ \hline
\end{longtable} 

%\newpage


%%%%%%%%%%%%%%%%%%%%%%%%%%%%%%%%%%%%%%%%%%%%%%%%%%%%%%%%%%%%%%%
\begin{longtable}{|L{1.025\textwidth}|} \hline
%
{\bf VII. BIBLIOGRAFIA BÁSICA} \\ \hline
\begin{enumerate}
\item KUROSE, James F.; ROSS, Keith W. Redes de computadores e a Internet: uma abordagem top-down. 5. ed. São Paulo: Pearson Addison Wesley, 2010. xxiii, 614 p. ISBN 9788588639973.
\item FOROUZAN, Behrouz A.; FEGAN, Sophia Chung; GRIESI, Ariovaldo. Comunicação de dados e redes de computadores. 4. ed. São Paulo: McGraw Hill, 2008. 1134 p. ISBN 9788586804885. 
\item LI, Deying; CHENG, Maggie Xiaoyan. Advances in Wireless Ad Hoc and Sensor Networks. Boston: Springer-Verlag US, 2008. (Signals and Communication Technology, 1860-4862). 
\end{enumerate}
 \\ \hline
\end{longtable}


%\newpage

%%%%%%%%%%%%%%%%%%%%%%%%%%%%%%%%%%%%%%%%%%%%%%%%%%%%%%%%%%%%%%%
\begin{longtable}{|L{1.025\textwidth}|} \hline
%
{\bf VIII. BIBLIOGRAFIA COMPLEMENTAR} \\ \hline
\begin{enumerate}
\item TRONCO, Tania Regina. Redes da nova geração: arquitetura de convergência das redes : IP, telefônica e óptica. 2. ed. rev. e atual. São Paulo: Érica, 2014. 164 p. ISBN 9788536501383.
\item STALLINGS, William. Redes e sistemas de comunicação de dados. Rio de Janeiro: Elsevier, c2005. xvi, 449 p. ISBN 9788535217315. 
\item  MEDEIROS, Julio Cesar de O. Princípios de telecomunicações: teoria e prática. 4. ed. rev. São Paulo: Érica, 2014. 320 p. ISBN 9788536500331. 
\item  RAPPAPORT, Theodore S. Comunicações sem fio: princípios e práticas. 2. ed. São Paulo: Pearson Prentice Hall, 2009. xix, 409 p. ISBN 9788576051985. 
\item FALUDI, Robert. Building wireless sensor networks. Sebastopol: O'Reilly, 2010. xviii, 300 p. ISBN 9780596807733. 
%
\end{enumerate}
 \\ \hline
\end{longtable}


\input aprovacao.tex


\end{document}
