\documentclass[12pt]{article}
\usepackage[brazil]{babel}
\usepackage{graphicx,t1enc,wrapfig,amsmath,float}
\usepackage{framed,fancyhdr}
\usepackage{multirow}
\usepackage{longtable}
\usepackage{array}
\newcolumntype{L}[1]{>{\raggedright\let\newline\\\arraybackslash\hspace{0pt}}m{#1}}
\newcolumntype{C}[1]{>{\centering\let\newline\\\arraybackslash\hspace{0pt}}m{#1}}
\newcolumntype{R}[1]{>{\raggedleft\let\newline\\\arraybackslash\hspace{0pt}}m{#1}}
%%%%%%%%%%%%%
\oddsidemargin -0.5cm
\evensidemargin -0.5cm
\textwidth 17.5cm
\topmargin -1.5cm
\textheight 22cm
%%%%%%%%%%%% 

%\pagestyle{empty}

\newcommand{\semestre}{2018.2}

\newcommand{\disciplina}{TRABALHO DE CONCLUSÃO DE CURSO II}
\newcommand{\codigo}{DEC7572}


%%%%%%%%%%%%%%%%%%%%%%%%%%%%%%%%%%%%%%%%%%%%%%%%%%%%%%%
%%%%%%%%%%%%% CRETIDOS
\newcommand{\creditosT}{4}
\newcommand{\creditosP}{0}

%%%%%%%%%%%%%%%%%%%%%%%%%%%%%%%%%%%%%%%%%%%%%%%%%%%%%%%
%%%%%%%%%%%%%% REQUISITOS
\newcommand{\requisitoA}{DEC7571 & TRABALHO DE CONCLUSÃO DE CURSO I
 \\ \hline}
\newcommand{\requisitoB}{}
\newcommand{\requisitoC}{}

%%%%%%%%%%%%%%%%%%%%%%%%%%%%%%%%%%%%%%%%%%%%%%%%%%%%%%%
%%%%%%%%%%%%%%% Atende aos Cursos
\newcommand{\cursoA}{Graduação em Engenharia de Computação \\ \hline}
\newcommand{\cursoB}{}%Graduação em Tecnologias da Informação e Comunicação \\ \hline}
\newcommand{\cursoC}{}

%%%%%%%%%%%%%%%%%%%%%%%%%%%%%%%%%%%%%%%%%%%%%%%%%%%%%%%%
%%%%%%%%%% EMENTA
\newcommand{\ementa}{
Divulgar as Normas para realização do TCC; Acompanhar o cumprimento das Normas para realização do TCC; Acompanhar alunos e professores orientadores em todas as fases de desenvolvimento do projeto de TCC; Desenvolver a capacidade de trabalho do aluno e a aplicação dos conhecimentos adquiridos no decorrer do curso; Incentivar a criatividade e o espírito crítico do aluno; Participar do processo de aprovação do projeto de TCC. Introduzir o aluno na prática de investigação científica.
\\ \hline
}


\begin{document}

%%%%%%%%%%%%%%%%%%%%%%%%%%%%%%%%%%%%%%%%%%%%%%%%%%%%%%%%%%%%%

\input cabecalho.tex


%%%%%%%%%%%%%%%%%%%%%%%%%%%%%%%%%%%%%%%%%%%%%%%%%%%%%%%%%%%%%
\begin{longtable}{|C{0.11\textwidth}|C{0.29\textwidth}|C{0.09\textwidth}|C{0.09\textwidth}|C{0.15\textwidth}|C{0.158\textwidth}|} \hline
%
\multicolumn{6}{|l|}{{\bf I. IDENTIFICAÇÃO DA DISCIPLINA}} \\ \hline
%
\multirow{3}*{{\small CÓDIGO}} & \multirow{3}*{NOME DA DISCIPLINA} &\multicolumn{2}{c|}{{\small N$^\circ$ DE HORAS-AULA }} & {{\small TOTAL DE}} & \multirow{3}*{{\small MODALIDADE}} \\ 
%
& & \multicolumn{2}{c|}{\small SEMANAIS}  & {\small HORAS-AULA} & \\ \cline{3-4}
%
& & {\tiny TEÓRICAS} & {\tiny PRÁTICAS} & {\small SEMESTRAIS} & \\ \hline
% codigo da disciplina carga horaria: teorica - pratica e total
{\bf \small \codigo} & {\bf \small \disciplina } & {\bf \creditosT} & {\bf \creditosP} & {\bf 72} & \\ \hline
\end{longtable}


%%%%%%%%%%%%%%%%%%%%%%%%%%%%%%%%%%%%%%%%%%%%%%%%%%%%%%%%%%%%%%
\begin{longtable}{|C{0.12\textwidth}|L{0.736\textwidth}|C{0.12\textwidth}|} \hline
%
\multicolumn{3}{|l|}{{\bf II. PRÉ-REQUISITO(S)}} \\ \hline
%
CÓDIGO & NOME DA DISCIPLINA & CURSO \\ \hline	
%
\requisitoA
\requisitoB
\requisitoC
\end{longtable}


%%%%%%%%%%%%%%%%%%%%%%%%%%%%%%%%%%%%%%%%%%%%%%%%%%%%%%%%%%%%%%
\begin{longtable}{|L{1.025\textwidth}|} \hline
%
{\bf III. CURSO(S) PARA O(S) QUAL(IS) A DISCIPLINA É OFERECIDA } \\ \hline
%
\cursoA 
\cursoB
\cursoC

\end{longtable}

%%%%%%%%%%%%%%%%%%%%%%%%%%%%%%%%%%%%%%%%%%%%%%%%%%%%%%%%%%%%%%
\begin{longtable}{|L{1.025\textwidth}|} \hline
%
{\bf IV. EMENTA } \\ \hline
%
\ementa
\end{longtable}

\newpage



%%%%%%%%%%%%%%%%%%%%%%%%%%%%%%%%%%%%%%%%%%%%%%%%%%%%%%%%%%%%%%%
\begin{longtable}{|L{1.025\textwidth}|} \hline
%
{\bf V. OBJETIVOS } \\ \hline
Objetivos Gerais: \\

Coordenar as ações de alunos e orientadores de um projeto de TCC visando à integralização de conhecimentos adquiridos ao longo da formação universitária no Curso de Engenharia de Computação.\\
\\
Objetivos Específicos: 
\begin{itemize}
\item Propiciar aos estudantes a ocasião de demonstrar o conhecimento adquirido, o aprofundamento temático e o aprimoramento da capacidade de interpretação e de crítica.
\item Propiciar aos discentes as condições necessárias para a elaboração de um estudo teórico-prático, dentro das normas técnicas que caracterizam a pesquisa científica.
\item  Oportunizar aos estudante a possibilidade de vivenciar na prática o contexto do trabalho na área de Engenharia de Computação e de adquirir experiência no processo de iniciação científica.
\item  Oportunizar aos estudantes a experiência de desenvolver, apresentar e defender seus projetos sob a égide da lei de inovação de produtos e processos e da defesa da propriedade intelectual.
\item Aprofundar os conhecimentos em uma ou mais áreas relacionadas ao curso.
\end{itemize}
\\ \hline
\end{longtable}


%%%%%%%%%%%%%%%%%%%%%%%%%%%%%%%%%%%%%%%%%%%%%%%%%%%%%%%%%%%%%%%
\begin{longtable}{|L{1.025\textwidth}|} \hline
%
{\bf VI. CONTEÚDO PROGRAMÁTICO } \\ \hline
Unidade I: Apresentação da disciplina.\\
\\
Unidade II: Cronograma da disciplina e cronograma individual.\\
\\
Unidade III: Apresentação dos projetos e acompanhamento individual\\
\\
Unidade IV: Prévias das Apresentações. \\
\\
Unidade V: Defesas e entrega da versão final. \\
\\ \hline
\end{longtable} 

\newpage


%%%%%%%%%%%%%%%%%%%%%%%%%%%%%%%%%%%%%%%%%%%%%%%%%%%%%%%%%%%%%%%
\begin{longtable}{|L{1.025\textwidth}|} \hline
%
{\bf VII. BIBLIOGRAFIA BÁSICA} \\ \hline

\begin{enumerate}
\item GIL, Antonio Carlos. Como Elaborar Projetos de Pesquisa. 5a ed. Editora Atlas, 2010. 
\item MICHALISZYN, Mario Sergio; TOMASINI, Ricardo. Pesquisa - orientações e normas para elaboração de projetos, monografias e artigos científicos. 6a ed. Editora Vozes, 2005. 
\item WASLAWICK, Raul Sidnei. Metodologia de Pesquisa para Ciência da Computação. Editora Campus Elsevier, 2009.
\end{enumerate}

 \\ \hline
\end{longtable}


%\newpage

%%%%%%%%%%%%%%%%%%%%%%%%%%%%%%%%%%%%%%%%%%%%%%%%%%%%%%%%%%%%%%%
\begin{longtable}{|L{1.025\textwidth}|} \hline
%
{\bf VIII. BIBLIOGRAFIA COMPLEMENTAR} \\ \hline
\begin{enumerate}
\item SANTOS, João Almeida; FILHO, Domingos Parra. Metodologia Científica. 2a ed. Editora Cengage Learning, 2011. 
\item MARCONI, Marina de Andrade; LAKATOS, Eva Maria. Fundamentos de Metodologia Científica. 7a ed. Editora Atlas, 2010. 
\item RAMPAZZO, Lino. Metodologia Científica para alunos de graduação e pós-graduação. Edições Loyola, 2013.
\item CARVALHO, Maria Cecilia Maringoni de (Org.). Construindo o saber: metodologia cientifica, fundamentos e tecnicas. 24. ed. Campinas: Papirus, 2014. 224 p. ISBN 9788530809119.
\item LAKATOS, Eva Maria; MARCONI, Marina de Andrade. Fundamentos de metodologia cientifica. 7. ed. São Paulo: Atlas, 2010. 297 p. ISBN 9788522457588.
%
\end{enumerate}
 \\ \hline
\end{longtable}


\input aprovacao.tex


\end{document}
