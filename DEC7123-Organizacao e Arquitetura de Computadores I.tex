\documentclass[12pt]{article}
\usepackage[brazil]{babel}
\usepackage{graphicx,t1enc,wrapfig,amsmath,float}
\usepackage{framed,fancyhdr}
\usepackage{multirow}
\usepackage{longtable}
\usepackage{array}
\newcolumntype{L}[1]{>{\raggedright\let\newline\\\arraybackslash\hspace{0pt}}m{#1}}
\newcolumntype{C}[1]{>{\centering\let\newline\\\arraybackslash\hspace{0pt}}m{#1}}
\newcolumntype{R}[1]{>{\raggedleft\let\newline\\\arraybackslash\hspace{0pt}}m{#1}}
%%%%%%%%%%%%%
\oddsidemargin -0.5cm
\evensidemargin -0.5cm
\textwidth 17.5cm
\topmargin -1.5cm
\textheight 22cm
%%%%%%%%%%%% 

%\pagestyle{empty}

\newcommand{\semestre}{2018.2}

\newcommand{\disciplina}{ORGANIZAÇÃO E ARQUITETURA DE COMPUTADORES I}
\newcommand{\codigo}{DEC7123}


%%%%%%%%%%%%%%%%%%%%%%%%%%%%%%%%%%%%%%%%%%%%%%%%%%%%%%%
%%%%%%%%%%%%% CRETIDOS
\newcommand{\creditosT}{4}
\newcommand{\creditosP}{0}

%%%%%%%%%%%%%%%%%%%%%%%%%%%%%%%%%%%%%%%%%%%%%%%%%%%%%%%
%%%%%%%%%%%%%% REQUISITOS
\newcommand{\requisitoA}{DEC7546 &Circuitos Digitais & ENC\\ \hline}
\newcommand{\requisitoB}{}
\newcommand{\requisitoC}{}

%%%%%%%%%%%%%%%%%%%%%%%%%%%%%%%%%%%%%%%%%%%%%%%%%%%%%%%
%%%%%%%%%%%%%%% Atende aos Cursos
\newcommand{\cursoA}{Graduação em Engenharia de Computação \\ \hline}
\newcommand{\cursoB}{}%Graduação em Tecnologias da Informação e Comunicação \\ \hline}
\newcommand{\cursoC}{}

%%%%%%%%%%%%%%%%%%%%%%%%%%%%%%%%%%%%%%%%%%%%%%%%%%%%%%%%
%%%%%%%%%% EMENTA
\newcommand{\ementa}{
Sistemas de numeração em ponto flutuante e números negativos; Sistemas Computacionais: hardware de um computador, software de um computador, instrução de máquina, linguagem de montagem;  Conjunto de instruções CISC e RISC; Estudo de caso: computador didático BIP I; Estudo de caso: processador didático nPD; Arquitetura ARM: famílias de processadores ARM, processador Cortex-M0; Assembly do Cortex M0; Estruturas de Controle: desvios, repetições, suporte a procedimentos e pilhas; Pipelining; Tecnologias de memórias; Entrada e Saída de dados; Interrupção e DMA; Avaliação de desempenho de sistemas computacionais. \\ \hline
}




\begin{document}


%%%%%%%%%%%%%%%%%%%%%%%%%%%%%%%%%%%%%%%%%%%%%%%%%%%%%%%%%%%%%
\input cabecalho.tex



%%%%%%%%%%%%%%%%%%%%%%%%%%%%%%%%%%%%%%%%%%%%%%%%%%%%%%%%%%%%%
\begin{longtable}{|C{0.11\textwidth}|C{0.29\textwidth}|C{0.09\textwidth}|C{0.09\textwidth}|C{0.15\textwidth}|C{0.158\textwidth}|} \hline
%
\multicolumn{6}{|l|}{{\bf I. IDENTIFICAÇÃO DA DISCIPLINA}} \\ \hline
%
\multirow{3}*{{\small CÓDIGO}} & \multirow{3}*{NOME DA DISCIPLINA} &\multicolumn{2}{c|}{{\small N$^\circ$ DE HORAS-AULA }} & {{\small TOTAL DE}} & \multirow{3}*{{\small MODALIDADE}} \\ 
%
& & \multicolumn{2}{c|}{\small SEMANAIS}  & {\small HORAS-AULA} & \\ \cline{3-4}
%
& & {\tiny TEÓRICAS} & {\tiny PRÁTICAS} & {\small SEMESTRAIS} & \\ \hline
% codigo da disciplina carga horaria: teorica - pratica e total
{\bf \small \codigo} & {\bf \small \disciplina } & {\bf \creditosT} & {\bf \creditosP} & {\bf 72} & Presencial\\ \hline
\end{longtable}


%%%%%%%%%%%%%%%%%%%%%%%%%%%%%%%%%%%%%%%%%%%%%%%%%%%%%%%%%%%%%%
\begin{longtable}{|C{0.12\textwidth}|L{0.736\textwidth}|C{0.12\textwidth}|} \hline
%
\multicolumn{3}{|l|}{{\bf II. PRÉ-REQUISITO(S)}} \\ \hline
%
CÓDIGO & NOME DA DISCIPLINA & CURSO \\ \hline	
%
\requisitoA
\requisitoB
\requisitoC
\end{longtable}


%%%%%%%%%%%%%%%%%%%%%%%%%%%%%%%%%%%%%%%%%%%%%%%%%%%%%%%%%%%%%%
\begin{longtable}{|L{1.025\textwidth}|} \hline
%
{\bf III. CURSO(S) PARA O(S) QUAL(IS) A DISCIPLINA É OFERECIDA } \\ \hline
%
\cursoA 
\cursoB
\cursoC

\end{longtable}

%%%%%%%%%%%%%%%%%%%%%%%%%%%%%%%%%%%%%%%%%%%%%%%%%%%%%%%%%%%%%%
\begin{longtable}{|L{1.025\textwidth}|} \hline
%
{\bf IV. EMENTA } \\ \hline
%
\ementa
\end{longtable}

\newpage



%%%%%%%%%%%%%%%%%%%%%%%%%%%%%%%%%%%%%%%%%%%%%%%%%%%%%%%%%%%%%%%
\begin{longtable}{|L{1.025\textwidth}|} \hline
%
{\bf V. OBJETIVOS } \\ \hline
%
Objetivo Geral:
\begin{itemize}
\item Compreender o funcionamento de processadores e microcontroladores. 
\item Compreender a capacidade desses dispositivos e as possibilidades de utilização como computadores dedicados.
\item Compreender os critérios para a escolha de processadores e microcontroladores quando estes devem ser utilizados para construir um computador, embarcado ou não.
\end{itemize}

\\ \hline
\end{longtable}


%%%%%%%%%%%%%%%%%%%%%%%%%%%%%%%%%%%%%%%%%%%%%%%%%%%%%%%%%%%%%%%
\begin{longtable}{|L{1.025\textwidth}|} \hline
%
{\bf VI. CONTEÚDO PROGRAMÁTICO } \\ \hline
UNIDADE 1 (Introdução):\\
      Apresentação da disciplina\\
      Revisão de sistemas de numeração\\
      Exercícios em sala de aula\\
\\
UNIDADE 2 (Sistemas de Numeração):\\
      Operações com frações\\
      Representação de números positivos e negativos\\
      Numeração em ponto flutuante\\
      Exercícios em sala de aula\\
\\
UNIDADE 3 (Fundamentos de circuitos digitais - Revisão):\\
      Funções lógicas\\
      Portas lógicas\\
      Circuitos Combinacionais\\ 
      Circuitos Sequenciais\\
      Tabela-verdade\\
      Maxtermos e Mintermos\\
      Exercícios em sala de aula\\
      \\
UNIDADE 4 (Sistemas Computacionais):\\
      Hardware de um computador\\
      Software de um computador\\
      Instrução de Máquina\\
      Linguagem de Montagem\\
      Conjunto de instruções CISC e RISC\\
\\
UNIDADE 5 (Estudo de caso - BIP I):\\
      Um computador Básico \\
      Conceito de Memória principal\\
      Conceito de Processador \\
      Processador didático - BIP I\\
      Definindo a arquitetura\\
      Definindo a organização \\
      Caminho de Dados \\
      Caminho de controle \\
\\
UNIDADE 6 (Implementação do BIP I):\\
      Exercício de implementação do processador BIP I em simulador de circuitos digitais (LogiSim)\\
      Apresentação da linguagem assembly do BIP I\\
      Exercício de um pequeno programa para o BIP I (LogiSim)\\
      Simulação do programa implementado (LogiSim)\\
\\
UNIDADE 7 (Estudo de caso - nPD):\\
      Organização do processador nPD\\
      Conceito de Memória no processador nPD\\
      Banco de registradores\\
      ULA\\
      Caminho de Dados\\ 
      Caminho de controle\\ 
\\
UNIDADE 8 (Assembly do nPD):\\
      Apresentação das instruções assembly do nPD\\
      Exercícios em assembly com uso do Simulador nPD\\
\\
UNIDADE 9 (Arquitetura ARM):\\
      Famílias de processadores ARM\\
      Processador Cortex-M0 \\
      Mapa de memória do Cortex M0\\
      Registradores do Cortex M0\\
\\
UNIDADE 10 (Assembly do Cortex M0):\\
      Definições relativas à Área de Programa e à Área de Dados\\
      Definição de nomes para Registradores \\
      Relacionando símbolo com valor constante\\
      Alocando memória e especificando conteúdo\\
      Reservando um bloco de memória\\
\\
UNIDADE 11 (Instruções para acesso à memória):\\
      ADR\\
      LDR e STR com offset imediato\\ 
      LDR e STR com offset dado por registradores\\ 
      LDR relativo ao PC \\
      LDM e STM \\
      PUSH e POP \\
      Exercícios relativos às instruções para acesso à memória (Simulador)\\
      \\
UNIDADE 12 (Instruções para processamento de dados):\\
      ADC, ADD, RSB, SBC, e SUB \\
      AND, ORR, EOR e BIC \\
      ASR, LSL, LSR, e ROR \\
      CMP e CMN \\
      MOV e MVN \\
      Exercícios relativos à  Instruções para processamento de dados (Simulador)\\
\\
UNIDADE 13 (Instruções para processamento de dados):\\
      MULS\\
      REV, REV16, e REVSH\\
      SXT e UXT \\
      TST\\
      Exercícios relativos à  Instruções para processamento de dados (Simulador)\\
      \\
UNIDADE 14 (Instruções de controle e desvio):\\
      B\\
      BL\\
      BX\\
      BLX \\
      Exercícios relativos à Instruções de controle e desvio (Simulador)\\
\\
UNIDADE 15 (Estruturas de Controle):\\
      Estrutura IF \\
      Estrutura IF ELSE\\ 
      Repetiçõe por Laço FOR\\ 
      Repetições por Laço WHILE\\
      Exercício relativos à  Estruturas de Controle (Simulador)\\
      Suporte à Procedimentos e Pilhas\\
\\
UNIDADE 16 (Pipelining):\\
      Processadores Monociclo e Multi-ciclo\\
      Desempenho: monociclo em relação ao pipeline\\
      Falhas em pipeline - Hazards \\
      Hazard Estrutural \\
      Hazard de Dados\\
      Hazard de Controle\\ 
\\
UNIDADE 17 (Memórias):\\
      Tecnologias \\
      Memórias Não Voláteis\\
      Memórias Voláteis \\
      Processadores e memórias\\ 
      Hierarquia de Memória\\
      Princípios básicos de Cache\\ 
      Mapeamento direto \\
      Mapeamento Associativo \\
      \\
UNIDADE 18 (Memórias):\\
      Mapeamento Associativo por Conjunto\\
      Gerência de blocos na memória cache\\
      Escrita na memória cache \\
      Múltiplas memórias cache \\
      Memória Virtual \\
      MMU \\
       Acesso à memória virtual\\
\\ 
UNIDADE 19 (Entrada e saída de Dados):\\
      Entrada e Saída de dados \\
      Tipos de Entrada e Saída \\
      Barramentos: internos e externos\\ 
      Serial Peripheral Interface - SPI \\
      Polaridade e fase do clock\\
      Inter-Integrated Circuit - I2C\\ 
      Controller Area Network - CAN \\
      Universal Asynchronous Receiver/Transmitter\\
\\
UNIDADE 20 (Interrupções e DMA):\\
      Interrupções \\
      Interrupção de Hardware\\ 
      Interrupção de Software \\
      Controlador de Interrupções\\ 
      DMA\\
\\
UNIDADE 21 (Análise de Desempenho):\\
      Desempenho - tempo de execução\\
      Tempo de execução - ciclos de relógio\\ 
      Taxa de execução \\
      Uso de Benchmark \\
      Lei de Amdahl\\
      \\
UNIDADE 22 (Trabalho):\\
      Sorteio do Trabalho Final para cada aluno\\
      Explanação sobre cada um dos trabalhos\\
      Definição das documentações exigidas na entrega do Trabalho Final\\
\\
UNIDADE 23 (Execução)\\
      Acompanhar os alunos na execução dos trabalhos\\
\hline
\end{longtable} 





%%%%%%%%%%%%%%%%%%%%%%%%%%%%%%%%%%%%%%%%%%%%%%%%%%%%%%%%%%%%%%%
\begin{longtable}{|L{1.025\textwidth}|} \hline
%
{\bf VII. BIBLIOGRAFIA BÁSICA} \\ \hline
\begin{enumerate}
%
\item STALLINGS, W. Arquitetura e organização de computadores, 8. ed. São Paulo: Prentice-Hall, 2010. 
\item TANENBAUM, Andrew. Organização estruturada de computadores. 5. ed. Rio de Janeiro: Pearson, 2006. 
\item HENNESSY, J. L.; PATTERSON, D. A. Arquitetura de computadores: uma abordagem quantitativa. Rio de Janeiro: Campus, 2003
%
\end{enumerate}
 \\ \hline
\end{longtable}


%\newpage

%%%%%%%%%%%%%%%%%%%%%%%%%%%%%%%%%%%%%%%%%%%%%%%%%%%%%%%%%%%%%%%
\begin{longtable}{|L{1.025\textwidth}|} \hline
%
{\bf VIII. BIBLIOGRAFIA COMPLEMENTAR} \\ \hline
\begin{enumerate}

\item WEBER, R.F. Fundamentos de arquitetura de computadores. 3. ed. Bookman Editora, 2008.
\item MONTEIRO, M. A. Introdução à organização de computadores. 5. Ed. Rio de Janeiro: LTC, 2007. 
\item MURDOCCA, M.J.; HEURING V.P. Introdução à arquitetura de computadores. Rio de Janeiro: Campus, 2001. 
\item CAPRON, H. L.; JOHNSON, J. A. Introdução à informática. São Paulo: Ed. Pearson, 2004. 
\item MALVINO, Albert Paul; BATES, David J. Eletrônica. 7. ed. Porto Alegre: AMGH, 2007. v. ISBN 9788577260225 (v.1).

%
\end{enumerate}
 \\ \hline
\end{longtable}


\input aprovacao.tex


\end{document}
