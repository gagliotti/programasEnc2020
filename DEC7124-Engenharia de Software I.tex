\documentclass[12pt]{article}
\usepackage[brazil]{babel}
\usepackage{graphicx,t1enc,wrapfig,amsmath,float}
\usepackage{framed,fancyhdr}
\usepackage{multirow}
\usepackage{longtable}
\usepackage{array}
\newcolumntype{L}[1]{>{\raggedright\let\newline\\\arraybackslash\hspace{0pt}}m{#1}}
\newcolumntype{C}[1]{>{\centering\let\newline\\\arraybackslash\hspace{0pt}}m{#1}}
\newcolumntype{R}[1]{>{\raggedleft\let\newline\\\arraybackslash\hspace{0pt}}m{#1}}
%%%%%%%%%%%%%
\oddsidemargin -0.5cm
\evensidemargin -0.5cm
\textwidth 17.5cm
\topmargin -1.5cm
\textheight 22cm
%%%%%%%%%%%% 

%\pagestyle{empty}

\newcommand{\semestre}{2018.2}

\newcommand{\disciplina}{ENGENHARIA DE SOFTWARE I}
\newcommand{\codigo}{DEC7124}


%%%%%%%%%%%%%%%%%%%%%%%%%%%%%%%%%%%%%%%%%%%%%%%%%%%%%%%
%%%%%%%%%%%%% CREDITOS
\newcommand{\creditosT}{2}
\newcommand{\creditosP}{2}

%%%%%%%%%%%%%%%%%%%%%%%%%%%%%%%%%%%%%%%%%%%%%%%%%%%%%%%
%%%%%%%%%%%%%% REQUISITOS
\newcommand{\requisitoA}{CIT7139 & Programação em Computadores & TIC\\ \hline}
\newcommand{\requisitoB}{}
\newcommand{\requisitoC}{}

%%%%%%%%%%%%%%%%%%%%%%%%%%%%%%%%%%%%%%%%%%%%%%%%%%%%%%%
%%%%%%%%%%%%%%% Atende aos Cursos
%\newcommand{\cursoA}{Graduação em Engenharia de Computação. \\ \hline}
\newcommand{\cursoA}{Graduação em Tecnologias da Informação e Comunicação \\ \hline}
\newcommand{\cursoC}{}
\newcommand{\cursoB}{}


%%%%%%%%%%%%%%%%%%%%%%%%%%%%%%%%%%%%%%%%%%%%%%%%%%%%%%%%
%%%%%%%%%% EMENTA
\newcommand{\ementa}{

Análise de requisitos: requisitos funcionais e requisitos não-funcionais; técnicas para levantamento e representação de requisitos, incluindo casos de uso. Modelagem orientada a objetos. Projeto orientado a objetos: técnicas para projeto; padrões de projeto, componentes e frameworks; projeto de arquitetura. Linguagem de especificação orientada a objetos. Métodos de análise e projeto orientados a objetos.

 \\ \hline
}




\begin{document}


%%%%%%%%%%%%%%%%%%%%%%%%%%%%%%%%%%%%%%%%%%%%%%%%%%%%%%%%%%%%%
\input cabecalho.tex

%%%%%%%%%%%%%%%%%%%%%%%%%%%%%%%%%%%%%%%%%%%%%%%%%%%%%%%%%%%%%
\begin{longtable}{|C{0.11\textwidth}|C{0.29\textwidth}|C{0.09\textwidth}|C{0.09\textwidth}|C{0.15\textwidth}|C{0.158\textwidth}|} \hline
%
\multicolumn{6}{|l|}{{\bf I. IDENTIFICAÇÃO DA DISCIPLINA}} \\ \hline
%
\multirow{3}*{{\small CÓDIGO}} & \multirow{3}*{NOME DA DISCIPLINA} &\multicolumn{2}{c|}{{\small N$^\circ$ DE HORAS-AULA }} & {{\small TOTAL DE}} & \multirow{3}*{{\small MODALIDADE}} \\ 
%
& & \multicolumn{2}{c|}{\small SEMANAIS}  & {\small HORAS-AULA} & \\ \cline{3-4}
%
& & {\tiny TEÓRICAS} & {\tiny PRÁTICAS} & {\small SEMESTRAIS} & \\ \hline
% codigo da disciplina carga horaria: teorica - pratica e total
{\bf \small \codigo} & {\bf \small \disciplina } & {\bf \creditosT} & {\bf \creditosP} & {\bf 72} & Presencial\\ \hline
\end{longtable}


%%%%%%%%%%%%%%%%%%%%%%%%%%%%%%%%%%%%%%%%%%%%%%%%%%%%%%%%%%%%%%
\begin{longtable}{|C{0.12\textwidth}|L{0.736\textwidth}|C{0.12\textwidth}|} \hline
%
\multicolumn{3}{|l|}{{\bf II. PRÉ-REQUISITO(S)}} \\ \hline
%
CÓDIGO & NOME DA DISCIPLINA & CURSO \\ \hline	
%
\requisitoA
\requisitoB
\requisitoC
\end{longtable}


%%%%%%%%%%%%%%%%%%%%%%%%%%%%%%%%%%%%%%%%%%%%%%%%%%%%%%%%%%%%%%
\begin{longtable}{|L{1.025\textwidth}|} \hline
%
{\bf III. CURSO(S) PARA O(S) QUAL(IS) A DISCIPLINA É OFERECIDA } \\ \hline
%
\cursoA 
\cursoB
\cursoC

\end{longtable}

%%%%%%%%%%%%%%%%%%%%%%%%%%%%%%%%%%%%%%%%%%%%%%%%%%%%%%%%%%%%%%
\begin{longtable}{|L{1.025\textwidth}|} \hline
%
{\bf IV. EMENTA } \\ \hline
%
\ementa
\end{longtable}

\newpage



%%%%%%%%%%%%%%%%%%%%%%%%%%%%%%%%%%%%%%%%%%%%%%%%%%%%%%%%%%%%%%%
\begin{longtable}{|L{1.025\textwidth}|} \hline
%
{\bf V. OBJETIVOS } \\ \hline
%
Objetivos Gerais: \\
Fornecer subsídios ao aluno para que ele possa analisar e projetar adequadamente um produto de software utilizando uma metodologia orientada a objetos.\\
 \\
Objetivos Específicos:\\

O aluno ao final do curso deve possuir habilidades para:
\begin{itemize}
\item Modelar os dados de uma organização utilizando uma notação apropriada;
\item Projetar um sistema a partir da engenharia de requisitos;
\item Analisar e projetar software através do paradigma orientado a objetos.
\end{itemize}

\\ \hline
\end{longtable}


%%%%%%%%%%%%%%%%%%%%%%%%%%%%%%%%%%%%%%%%%%%%%%%%%%%%%%%%%%%%%%%
\begin{longtable}{|L{1.025\textwidth}|} \hline
%
{\bf VI. CONTEÚDO PROGRAMÁTICO } \\ \hline

Unidade I: Fundamentos de Análise de Sistemas de Informação Orientado a Objetos\\
Conteúdo Teórico seguido de Conteúdo Prático \\
Engenharia de Requisitos:\\
Elicitação, especificação, avaliação e documentação\\
Modelagem orientada a objetos\\
\\
Unidade II: Projeto Orientado a Objetos\\
Conteúdo Teórico seguido de Conteúdo Prático com desenvolvimento de modelagem por computador. \\
Projeto Orientado a Objetos\\
Projeto da Arquitetura
\\ \hline
\end{longtable} 





%%%%%%%%%%%%%%%%%%%%%%%%%%%%%%%%%%%%%%%%%%%%%%%%%%%%%%%%%%%%%%%
\begin{longtable}{|L{1.025\textwidth}|} \hline
%
{\bf VII. BIBLIOGRAFIA BÁSICA} \\ \hline
\begin{enumerate}
%
\item BOOCH, G.; RUMBAUGH, J.; JACOBSON, I. UML - Guia do Usuário. 2. ed. Rio de Janeiro: Campus., 2006.
\item SILVA, R. P. UML2 em modelagem orientada a objetos. Florianópolis: Visual Books, 2007.
\item SOMMERVILLE, Ian. Engenharia de software. 8. ed. São Paulo: Pearson Addison-Wesley, 2007. xiv, 552 p.
%

\end{enumerate}
 \\ \hline
\end{longtable}


\newpage

%%%%%%%%%%%%%%%%%%%%%%%%%%%%%%%%%%%%%%%%%%%%%%%%%%%%%%%%%%%%%%%
\begin{longtable}{|L{1.025\textwidth}|} \hline
%
{\bf VIII. BIBLIOGRAFIA COMPLEMENTAR} \\ \hline
\begin{enumerate}

\item BEZERRA, Eduardo. Princípios de análise e projeto de sistemas com UML. Rio de Janeiro: Elsevier, 2003. 286p.
\item LARMAN, Craig. Utilizando UML e padrões: uma introdução à análise e ao projetos orientados a objetos e ao desenvolvimento interativo. 3 ed. Porto Alegre: Bookman, 2007.
\item MENDES, E.; MOSLEY, N. Web Engineering. New York: Springer, 2007.
\item PRESSMAN, Roger. Engenharia de software. 6 ed.. São Paulo: McGraw Hill, 2006. 752p.
\item WAZLAWICK, Raul Sidnei. Análise e projeto de sistemas de informação orientados a objetos. Rio de Janeiro: Campus, 2004.
%
\end{enumerate}
 \\ \hline
\end{longtable}


\input aprovacao.tex


\end{document}
