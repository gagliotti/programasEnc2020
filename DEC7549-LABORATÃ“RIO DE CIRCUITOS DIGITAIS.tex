\documentclass[12pt]{article}
\usepackage[brazil]{babel}
\usepackage{graphicx,t1enc,wrapfig,amsmath,float}
\usepackage{framed,fancyhdr}
\usepackage{multirow}
\usepackage{longtable}
\usepackage{array}
\newcolumntype{L}[1]{>{\raggedright\let\newline\\\arraybackslash\hspace{0pt}}m{#1}}
\newcolumntype{C}[1]{>{\centering\let\newline\\\arraybackslash\hspace{0pt}}m{#1}}
\newcolumntype{R}[1]{>{\raggedleft\let\newline\\\arraybackslash\hspace{0pt}}m{#1}}
%%%%%%%%%%%%%
\oddsidemargin -0.5cm
\evensidemargin -0.5cm
\textwidth 17.5cm
\topmargin -1.5cm
\textheight 22cm
%%%%%%%%%%%% 

%\pagestyle{empty}

\newcommand{\semestre}{2018.2}

\newcommand{\disciplina}{LABORATÓRIO DE CIRCUITOS DIGITAIS}
\newcommand{\codigo}{DEC7549}


%%%%%%%%%%%%%%%%%%%%%%%%%%%%%%%%%%%%%%%%%%%%%%%%%%%%%%%
%%%%%%%%%%%%% CRETIDOS
\newcommand{\creditosT}{0}
\newcommand{\creditosP}{4}

%%%%%%%%%%%%%%%%%%%%%%%%%%%%%%%%%%%%%%%%%%%%%%%%%%%%%%%
%%%%%%%%%%%%%% REQUISITOS
\newcommand{\requisitoA}{}
\newcommand{\requisitoB}{}
\newcommand{\requisitoC}{}

%%%%%%%%%%%%%%%%%%%%%%%%%%%%%%%%%%%%%%%%%%%%%%%%%%%%%%%
%%%%%%%%%%%%%%% Atende aos Cursos
\newcommand{\cursoA}{Graduação em Engenharia de Computação \\ \hline}
\newcommand{\cursoB}{}%Graduação em Tecnologias da Informação e Comunicação \\ \hline}https://www.overleaf.com/project/5ef2058d7bcf9600012c3daf
\newcommand{\cursoC}{}

%%%%%%%%%%%%%%%%%%%%%%%%%%%%%%%%%%%%%%%%%%%%%%%%%%%%%%%%
%%%%%%%%%% EMENTA
\newcommand{\ementa}{
Desenvolvimento de atividades práticas que permitam explorar os fundamentos, conceitos e técnicas relativas em circuitos digitais.
\\ \hline
}


\begin{document}

%%%%%%%%%%%%%%%%%%%%%%%%%%%%%%%%%%%%%%%%%%%%%%%%%%%%%%%%%%%%%

\input cabecalho.tex


%%%%%%%%%%%%%%%%%%%%%%%%%%%%%%%%%%%%%%%%%%%%%%%%%%%%%%%%%%%%%
\begin{longtable}{|C{0.11\textwidth}|C{0.29\textwidth}|C{0.09\textwidth}|C{0.09\textwidth}|C{0.15\textwidth}|C{0.158\textwidth}|} \hline
%
\multicolumn{6}{|l|}{{\bf I. IDENTIFICAÇÃO DA DISCIPLINA}} \\ \hline
%
\multirow{3}*{{\small CÓDIGO}} & \multirow{3}*{NOME DA DISCIPLINA} &\multicolumn{2}{c|}{{\small N$^\circ$ DE HORAS-AULA }} & {{\small TOTAL DE}} & \multirow{3}*{{\small MODALIDADE}} \\ 
%
& & \multicolumn{2}{c|}{\small SEMANAIS}  & {\small HORAS-AULA} & \\ \cline{3-4}
%
& & {\tiny TEÓRICAS} & {\tiny PRÁTICAS} & {\small SEMESTRAIS} & \\ \hline
% codigo da disciplina carga horaria: teorica - pratica e total
{\bf \small \codigo} & {\bf \small \disciplina } & {\bf \creditosT} & {\bf \creditosP} & {\bf 72} & Presencial\\ \hline
\end{longtable}


%%%%%%%%%%%%%%%%%%%%%%%%%%%%%%%%%%%%%%%%%%%%%%%%%%%%%%%%%%%%%%
\begin{longtable}{|C{0.12\textwidth}|L{0.736\textwidth}|C{0.12\textwidth}|} \hline
%
\multicolumn{3}{|l|}{{\bf II. PRÉ-REQUISITO(S)}} \\ \hline
%
CÓDIGO & NOME DA DISCIPLINA & CURSO \\ \hline	
%
\requisitoA
\requisitoB
\requisitoC
\end{longtable}


%%%%%%%%%%%%%%%%%%%%%%%%%%%%%%%%%%%%%%%%%%%%%%%%%%%%%%%%%%%%%%
\begin{longtable}{|L{1.025\textwidth}|} \hline
%
{\bf III. CURSO(S) PARA O(S) QUAL(IS) A DISCIPLINA É OFERECIDA } \\ \hline
%
\cursoA 
\cursoB
\cursoC

\end{longtable}

%%%%%%%%%%%%%%%%%%%%%%%%%%%%%%%%%%%%%%%%%%%%%%%%%%%%%%%%%%%%%%
\begin{longtable}{|L{1.025\textwidth}|} \hline
%
{\bf IV. EMENTA } \\ \hline
%
\ementa
\end{longtable}

%\newpage



%%%%%%%%%%%%%%%%%%%%%%%%%%%%%%%%%%%%%%%%%%%%%%%%%%%%%%%%%%%%%%%
\begin{longtable}{|L{1.025\textwidth}|} \hline
%
{\bf V. OBJETIVOS } \\ \hline
Objetivo Geral: \\
\\
Esta disciplina deverá abordar aspectos práticos circuitos digitais e explorando os equipamentos e componentes do mundo real.\\
\\
Objetivos Específicos:
\begin{itemize}
\item Colocar os alunos em contato com componentes eletrônicos reais;
\item Utilizar equipamentos de medição de sinais eletrônicos como multímetros, geradores de funções, fontes de alimentação e osciloscópios;
\item Montar em placa eletrônica universal circuitos digitais clássicos;
\item Estudar os componentes eletrônicos básicos da eletrônica
\item Medir e avaliar circuitos digitais
\item Projetar circuitos digitais para soluções de problemas digitais
\end{itemize}
\\ \hline
\end{longtable}


%%%%%%%%%%%%%%%%%%%%%%%%%%%%%%%%%%%%%%%%%%%%%%%%%%%%%%%%%%%%%%%
\begin{longtable}{|L{1.025\textwidth}|} \hline
%
{\bf VI. CONTEÚDO PROGRAMÁTICO } \\ \hline
UNIDADE 1: Medidas Elétricas [12 horas-aula]\\
Apresentar os conceitos fundamentais de medidas elétricas\\
Estudar e utilizar multímetro (tensão, corrente, resistência, etc)\\
Utilizar fontes de alimentação estudando suas características e cuidados\\
Utilizar geradores de função\\
Utilizar osciloscópios\\
\\
UNIDADE 2: Componentes Eletrônicos [12 horas-aula]\\
Características técnicas, comerciais e de montagem de resistores\\
Características técnicas, comerciais e de montagem de capacitores\\
Características técnicas, comerciais e de montagem de diodos\\
Características técnicas, comerciais e de montagem de transistores\\
Características técnicas, comerciais e de montagem de circuitos integrados\\
\\
UNIDADE 3: Montagem de Circuitos Digitais Combinacionais. [12 horas-aula]\\
Portas Lógicas\\
Projeto de circuitos lógicos combinacionais\\
Codificadores e decodificadores\\
\\
UNIDADE 4: Montagem de Circuitos Digitais Sequenciais. [36 horas-aula]\\
Flip-fllops\\
Registradores de deslocamento\\
Contadores\\
Multiplex/demultiplex\\
Conversor analógico/digital e digital/analógico\\
Memórias
\\ \hline
\end{longtable} 

\newpage


%%%%%%%%%%%%%%%%%%%%%%%%%%%%%%%%%%%%%%%%%%%%%%%%%%%%%%%%%%%%%%%
\begin{longtable}{|L{1.025\textwidth}|} \hline
%
{\bf VII. BIBLIOGRAFIA BÁSICA} \\ \hline

\begin{enumerate}
\item TOCCI, RONALD J.; WIDMER, NEAL S.; MOSS, GREGORY L. Sistemas Digitais: Princípios e Aplicações 11ª edição. São Paulo: Pearson. 
\item BIGNELL, James; DONOVAN, Robert. Eletrônica digital. São Paulo: Cengage Learning, 2010. xviii, 648 p. ISBN 9788522107452. 
\item IDOETA, Ivan V.; CAPUANO, Francisco G. Elementos de eletrônica digital. 41. ed. rev. e atual. São Paulo: Livros Erica Ed., c2012. 544 p. ISBN 9788571940192. 
\end{enumerate}
 \\ \hline
\end{longtable}


%\newpage

%%%%%%%%%%%%%%%%%%%%%%%%%%%%%%%%%%%%%%%%%%%%%%%%%%%%%%%%%%%%%%%
\begin{longtable}{|L{1.025\textwidth}|} \hline
%
{\bf VIII. BIBLIOGRAFIA COMPLEMENTAR} \\ \hline
\begin{enumerate}

\item FERREIRA, José Manuel Martins. Introdução ao projeto com sistemas digitais e microcontroladores. Porto: FEUP, 1998. 371 p. ISBN 9727520324 
\item WILSON, Peter. The circuit designer’s companion. 3rd ed. Amsterdam: Elsevier, 2012. xv, 439 p. ISBN 9780080971384
\item PEDRONI, Volnei A. Eletrônica digital moderna e VHDL. Rio de Janeiro: Elsevier, c2010. 619 p. ISBN 9788535234657.
\item MALVINO, Albert Paul; BATES, David J. Eletrônica. 7. ed. Porto Alegre: AMGH, 2007. v. ISBN 9788577260225 (v.1).
\item D’AMORE, Roberto. VHDL: descrição e síntese de circuitos digitais. 2. ed. Rio de Janeiro: LTC, c2012. xiii, 292 p. ISBN 9788521620549.

\end{enumerate}
 \\ \hline
\end{longtable}


\input aprovacao.tex


\end{document}

