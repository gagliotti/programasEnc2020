\documentclass[12pt]{article}
\usepackage[brazil]{babel}
\usepackage{graphicx,t1enc,wrapfig,amsmath,float}
\usepackage{framed,fancyhdr}
\usepackage{multirow}
\usepackage{longtable}
\usepackage{array}
\newcolumntype{L}[1]{>{\raggedright\let\newline\\\arraybackslash\hspace{0pt}}m{#1}}
\newcolumntype{C}[1]{>{\centering\let\newline\\\arraybackslash\hspace{0pt}}m{#1}}
\newcolumntype{R}[1]{>{\raggedleft\let\newline\\\arraybackslash\hspace{0pt}}m{#1}}
%%%%%%%%%%%%%
\oddsidemargin -0.5cm
\evensidemargin -0.5cm
\textwidth 17.5cm
\topmargin -1.5cm
\textheight 22cm
%%%%%%%%%%%% 

%\pagestyle{empty}

\newcommand{\semestre}{2018.2}

\newcommand{\disciplina}{Planejamento e Gestão de Projetos}
\newcommand{\codigo}{DEC0008}


%%%%%%%%%%%%%%%%%%%%%%%%%%%%%%%%%%%%%%%%%%%%%%%%%%%%%%%
%%%%%%%%%%%%% CRETIDOS
\newcommand{\creditosT}{4}
\newcommand{\creditosP}{2}

%%%%%%%%%%%%%%%%%%%%%%%%%%%%%%%%%%%%%%%%%%%%%%%%%%%%%%%
%%%%%%%%%%%%%% REQUISITOS
\newcommand{\requisitoA}{DEC7532 & Linguagem de Programação II}
\newcommand{\requisitoB}{}
\newcommand{\requisitoC}{}

%%%%%%%%%%%%%%%%%%%%%%%%%%%%%%%%%%%%%%%%%%%%%%%%%%%%%%%
%%%%%%%%%%%%%%% Atende aos Cursos
\newcommand{\cursoA}{Graduação em Engenharia de Computação. \\ \hline}
\newcommand{\cursoB}{Graduação em Tecnologias da Informação e Comunicação \\ \hline}
\newcommand{\cursoC}{}

%%%%%%%%%%%%%%%%%%%%%%%%%%%%%%%%%%%%%%%%%%%%%%%%%%%%%%%%
%%%%%%%%%% EMENTA
\newcommand{\ementa}{

Notação assintótica. Recorrências. Técnicas de análise de algoritmos. Listas lineares e suas generalizações: listas ordenadas, listas encadeadas, pilhas e filas. Aplicações de listas. Algoritmos de inserção, remoção e consulta. Tabelas de Espalhamento. Árvores binárias. Métodos de pesquisa e ordenação. Técnicas de implementação iterativa e recursiva de estruturas de dados. Grafos e grafos orientados. Representação de problemas com grafos.
 \\ \hline
}




\begin{document}


%%%%%%%%%%%%%%%%%%%%%%%%%%%%%%%%%%%%%%%%%%%%%%%%%%%%%%%%%%%%%
\begin{longtable}{|C{0.2\textwidth}|C{0.8\textwidth}|} \hline
%
\multirow{6}*{\includegraphics[scale=0.5]{UFSC-foto.jpg}} &\\
%
&{\bf UNIVERSIDADE FEDERAL DE SANTA CATARINA}\hfill\\
%
&{\bf Centro de Ciências, Tecnologias e Saúde} \\
%
&{\bf Departamento de Computação}\\
%
&{\bf PROGRAMA DE ENSINO}\\
%
& \\ \hline

%\multicolumn{2}{|c|}{{\bf SEMESTRE \semestre}}\\ \hline
\end{longtable}


%%%%%%%%%%%%%%%%%%%%%%%%%%%%%%%%%%%%%%%%%%%%%%%%%%%%%%%%%%%%%
\begin{longtable}{|C{0.11\textwidth}|C{0.29\textwidth}|C{0.09\textwidth}|C{0.09\textwidth}|C{0.15\textwidth}|C{0.158\textwidth}|} \hline
%
\multicolumn{6}{|l|}{{\bf I. IDENTIFICAÇÃO DA DISCIPLINA}} \\ \hline
%
\multirow{3}*{{\small CÓDIGO}} & \multirow{3}*{NOME DA DISCIPLINA} &\multicolumn{2}{c|}{{\small N$^\circ$ DE HORAS-AULA }} & {{\small TOTAL DE}} & \multirow{3}*{{\small MODALIDADE}} \\ 
%
& & \multicolumn{2}{c|}{\small SEMANAIS}  & {\small HORAS-AULA} & \\ \cline{3-4}
%
& & {\tiny TEÓRICAS} & {\tiny PRÁTICAS} & {\small SEMESTRAIS} & \\ \hline
% codigo da disciplina carga horaria: teorica - pratica e total
{\bf \small \codigo} & {\bf \small \disciplina } & {\bf \creditosT} & {\bf \creditosP} & {\bf 108} & Presencial\\ \hline
\end{longtable}


%%%%%%%%%%%%%%%%%%%%%%%%%%%%%%%%%%%%%%%%%%%%%%%%%%%%%%%%%%%%%%
\begin{longtable}{|C{0.12\textwidth}|L{0.736\textwidth}|C{0.12\textwidth}|} \hline
%
\multicolumn{3}{|l|}{{\bf II. PRÉ-REQUISITO(S)}} \\ \hline
%
CÓDIGO & NOME DA DISCIPLINA \\ \hline	
%
\requisitoA \\ \hline
\requisitoB
\requisitoC
\end{longtable}


%%%%%%%%%%%%%%%%%%%%%%%%%%%%%%%%%%%%%%%%%%%%%%%%%%%%%%%%%%%%%%
\begin{longtable}{|L{1.025\textwidth}|} \hline
%
{\bf III. CURSO(S) PARA O(S) QUAL(IS) A DISCIPLINA É OFERECIDA } \\ \hline
%
\cursoA 
\cursoB
\cursoC

\end{longtable}

%%%%%%%%%%%%%%%%%%%%%%%%%%%%%%%%%%%%%%%%%%%%%%%%%%%%%%%%%%%%%%
\begin{longtable}{|L{1.025\textwidth}|} \hline
%
{\bf IV. EMENTA } \\ \hline
%
\ementa
\end{longtable}

\newpage



%%%%%%%%%%%%%%%%%%%%%%%%%%%%%%%%%%%%%%%%%%%%%%%%%%%%%%%%%%%%%%%
\begin{longtable}{|L{1.025\textwidth}|} \hline
%
{\bf V. OBJETIVOS } \\ \hline
%
Objetivo Geral:

Abordar formalmente as estruturas de dados e as técnicas de
manipulação destas estruturas, bem como analisar métodos de pesquisa, ordenação e
representação de dados aplicando a estrutura de dados mais adequada para um dado
sistema computacional. \\

Objetivos Específicos:
\begin{itemize}
\item Identificar limites de crescimento de funções
\item Aplicar técnicas de análise de complexidade de algoritmos
\item Estudar as técnicas para estruturação de dados;
\item Analisar e conhecer os principais algoritmos de ordenação de dados;
\item Estudar técnicas de busca de dados; e
\item Implementar estruturas de dados e algoritmos de ordenação e pesquisa de dados
usando a linguagem de programação C/C++
\end{itemize}

\\ \hline
\end{longtable}


%%%%%%%%%%%%%%%%%%%%%%%%%%%%%%%%%%%%%%%%%%%%%%%%%%%%%%%%%%%%%%%
\begin{longtable}{|L{1.025\textwidth}|} \hline
%
{\bf VI. CONTEÚDO PROGRAMÁTICO } \\ \hline

Unidade 1: Introdução \\
Modelo de computação\\
Notação assintótica\\
Recorrências\\
Invariantes\\
Tipo abstrato de dados\\
\\
Unidade 2: Estruturas Lineares\\
Lista encadeada, circular e duplamente encadeada\\
Implementação de listas encadeadas\\
Pilhas\\
Filas\\
Aplicações de pilhas e filas\\
\\
Unidade 3: Algoritmos de Ordenação de Dados\\
Algoritmos de ordenação de dados\\
Algoritmos de inserção, remoção e pesquisa de dados\\
Técnicas de implementação iterativa e recursiva de estruturas de dados\\
Métodos de busca\\
\\
Unidade 4: Árvores\\
Árvore binária (conceitos e aplicações)\\
Árvores balanceadas\\
\\
Unidade 5: Tabela de Espalhamento\\
Função de espalhamento\\
Tratamento de colisões\\
\\
Unidade 5: Grafos\\
Conceitos de grafos\\
Problemas sobre grafos
\\ \hline
\end{longtable} 





%%%%%%%%%%%%%%%%%%%%%%%%%%%%%%%%%%%%%%%%%%%%%%%%%%%%%%%%%%%%%%%
\begin{longtable}{|L{1.025\textwidth}|} \hline
%
{\bf VII. BIBLIOGRAFIA BÁSICA} \\ \hline
\begin{enumerate}
\item CORMEN, Thomas H. et al. Algoritmos: teoria e prática. Rio de Janeiro: Elsevier, 2002. xvii, 916 p. ISBN 9788535209266.
\item FEOFILOFF, Paulo. Algoritmos em linguagem C. Rio de Janeiro: Elsevier, c2009. xv, 208 p. ISBN 9788535232493.
\item ZIVIANI, Nivio. Projeto de algoritmos: com implementações em Pascal e C. 3. ed. rev. e ampl. São Paulo: Cengage Learning, c2011. xx, 639 p. ISBN 9788522110506
\end{enumerate}
 \\ \hline
\end{longtable}


%\newpage

%%%%%%%%%%%%%%%%%%%%%%%%%%%%%%%%%%%%%%%%%%%%%%%%%%%%%%%%%%%%%%%
\begin{longtable}{|L{1.025\textwidth}|} \hline
%
{\bf VIII. BIBLIOGRAFIA COMPLEMENTAR} \\ \hline
\begin{enumerate}

\item CELES FILHO, Waldemar; CERQUEIRA, Renato; RANGEL, José Lucas. Introdução a
estruturas de dados: Introdução a estruturas de dados : com técnicas de programação em C. Rio de Janeiro: Elsevier, 2004. xiv, 294 p. ISBN 9788535212280.
\item TENENBAUM, Aaron M.; LANGSAM, Yedidyah; AUGENSTEIN, Moshe. Estruturas
de dados usando C. São Paulo: Pearson Makron Books, c1995. xx, 884 p. ISBN
8534603480
\item LOUDON, Kyle. Mastering algorithms with C. 1st ed. Sebastopol: O’Reilly, 1999. xvii, 540 p. ISBN 9781565924536.
\item PEREIRA, Silvio do Lago. Estruturas de dados fundamentais: conceitos e aplicações. 12. ed., rev. e atual. São Paulo: Érica, c2008. 264 p. ISBN 9788571943704.
\item WIRTH, Niklaus. Algoritmos e estruturas de dados. Rio de Janeiro: LTC, c1999. 255 p. ISBN 9788521611905.


%
\end{enumerate}
 \\ \hline
\end{longtable}


\input aprovacao.tex


\end{document}
