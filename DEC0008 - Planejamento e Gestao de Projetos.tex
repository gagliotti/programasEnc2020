\documentclass[12pt]{article}
\usepackage[brazil]{babel}
\usepackage{graphicx,t1enc,wrapfig,amsmath,float}
\usepackage{framed,fancyhdr}
\usepackage{multirow}
\usepackage{longtable}
\usepackage{array}
\newcolumntype{L}[1]{>{\raggedright\let\newline\\\arraybackslash\hspace{0pt}}m{#1}}
\newcolumntype{C}[1]{>{\centering\let\newline\\\arraybackslash\hspace{0pt}}m{#1}}
\newcolumntype{R}[1]{>{\raggedleft\let\newline\\\arraybackslash\hspace{0pt}}m{#1}}
%%%%%%%%%%%%%
\oddsidemargin -0.5cm
\evensidemargin -0.5cm
\textwidth 17.5cm
\topmargin -1.5cm
\textheight 22cm
%%%%%%%%%%%% 

%\pagestyle{empty}

\newcommand{\semestre}{2018.2}

\newcommand{\disciplina}{Planejamento e Gestão de Projetos}
\newcommand{\codigo}{DEC0008}


%%%%%%%%%%%%%%%%%%%%%%%%%%%%%%%%%%%%%%%%%%%%%%%%%%%%%%%
%%%%%%%%%%%%% CRETIDOS
\newcommand{\creditosT}{1}
\newcommand{\creditosP}{1}

%%%%%%%%%%%%%%%%%%%%%%%%%%%%%%%%%%%%%%%%%%%%%%%%%%%%%%%
%%%%%%%%%%%%%% REQUISITOS
\newcommand{\requisitoA}{DEC0009 & ENGENHARIA DE SOFTWARE}
\newcommand{\requisitoB}{}
\newcommand{\requisitoC}{}

%%%%%%%%%%%%%%%%%%%%%%%%%%%%%%%%%%%%%%%%%%%%%%%%%%%%%%%
%%%%%%%%%%%%%%% Atende aos Cursos
\newcommand{\cursoA}{Graduação em Engenharia de Computação. \\ \hline}
%\newcommand{\cursoB}{Graduação em Tecnologias da Informação e Comunicação \\ \hline}
\newcommand{\cursoC}{}

%%%%%%%%%%%%%%%%%%%%%%%%%%%%%%%%%%%%%%%%%%%%%%%%%%%%%%%%
%%%%%%%%%% EMENTA
\newcommand{\ementa}{

Fundamentos da Gestão de Projetos: Introdução e Histórico; Conceitos Básicos; Benefícios do Gerenciamento de Projetos. O Contexto da Gestão de Projetos: Fases e Ciclo de Vida de Projetos; Os Processos de Gestão de Projetos: Conceitos de Processos de Gerenciamento de Projetos; Processos e ciclo de vida de projetos. Início de um Projeto; Gestão do Escopo; Gestão de Prazos; Gestão de Custos; Gestão de Qualidade; Gestão de Recursos Humanos; Gestão de Comunicação; Gestão de Riscos. Avaliação dos resultados e impactos do projeto. Software de gerenciamento de projetos.
 \\ \hline
}




\begin{document}


%%%%%%%%%%%%%%%%%%%%%%%%%%%%%%%%%%%%%%%%%%%%%%%%%%%%%%%%%%%%%
\begin{longtable}{|C{0.2\textwidth}|C{0.8\textwidth}|} \hline
%
\multirow{6}*{\includegraphics[scale=0.5]{UFSC-foto.jpg}} &\\
%
&{\bf UNIVERSIDADE FEDERAL DE SANTA CATARINA}\hfill\\
%
&{\bf Centro de Ciências, Tecnologias e Saúde} \\
%
&{\bf Departamento de Computação}\\
%
&{\bf PROGRAMA DE ENSINO}\\
%
& \\ \hline

%\multicolumn{2}{|c|}{{\bf SEMESTRE \semestre}}\\ \hline
\end{longtable}


%%%%%%%%%%%%%%%%%%%%%%%%%%%%%%%%%%%%%%%%%%%%%%%%%%%%%%%%%%%%%
\begin{longtable}{|C{0.11\textwidth}|C{0.29\textwidth}|C{0.09\textwidth}|C{0.09\textwidth}|C{0.15\textwidth}|C{0.158\textwidth}|} \hline
%
\multicolumn{6}{|l|}{{\bf I. IDENTIFICAÇÃO DA DISCIPLINA}} \\ \hline
%
\multirow{3}*{{\small CÓDIGO}} & \multirow{3}*{NOME DA DISCIPLINA} &\multicolumn{2}{c|}{{\small N$^\circ$ DE HORAS-AULA }} & {{\small TOTAL DE}} & \multirow{3}*{{\small MODALIDADE}} \\ 
%
& & \multicolumn{2}{c|}{\small SEMANAIS}  & {\small HORAS-AULA} & \\ \cline{3-4}
%
& & {\tiny TEÓRICAS} & {\tiny PRÁTICAS} & {\small SEMESTRAIS} & \\ \hline
% codigo da disciplina carga horaria: teorica - pratica e total
{\bf \small \codigo} & {\bf \small \disciplina } & {\bf \creditosT} & {\bf \creditosP} & {\bf 36} & Presencial\\ \hline
\end{longtable}


%%%%%%%%%%%%%%%%%%%%%%%%%%%%%%%%%%%%%%%%%%%%%%%%%%%%%%%%%%%%%%
\begin{longtable}{|C{0.12\textwidth}|L{0.736\textwidth}|C{0.12\textwidth}|} \hline
%
\multicolumn{3}{|l|}{{\bf II. PRÉ-REQUISITO(S)}} \\ \hline
%
CÓDIGO & NOME DA DISCIPLINA \\ \hline	
%
\requisitoA \\ \hline
\requisitoB
\requisitoC
\end{longtable}


%%%%%%%%%%%%%%%%%%%%%%%%%%%%%%%%%%%%%%%%%%%%%%%%%%%%%%%%%%%%%%
\begin{longtable}{|L{1.025\textwidth}|} \hline
%
{\bf III. CURSO(S) PARA O(S) QUAL(IS) A DISCIPLINA É OFERECIDA } \\ \hline
%
\cursoA 
\cursoB
\cursoC

\end{longtable}

%%%%%%%%%%%%%%%%%%%%%%%%%%%%%%%%%%%%%%%%%%%%%%%%%%%%%%%%%%%%%%
\begin{longtable}{|L{1.025\textwidth}|} \hline
%
{\bf IV. EMENTA } \\ \hline
%
\ementa
\end{longtable}

\newpage



%%%%%%%%%%%%%%%%%%%%%%%%%%%%%%%%%%%%%%%%%%%%%%%%%%%%%%%%%%%%%%%
\begin{longtable}{|L{1.025\textwidth}|} \hline
%
{\bf V. OBJETIVOS } \\ \hline
%
\begin{itemize}
\item \textbf{Objetivo Geral}: Desenvolver no futuro profissional de Engenharia de Computação uma visão abrangente e estratégica dos negócios na área de Tecnologias da Informação. Noções de planejamento, técnicas, habilidades necessárias para a gestão de serviços de
tecnologia.
\item \textbf{Objetivos Específicos:}
\begin{itemize}
\item Fornecer uma visão ampla da aplicação e dos benefícios da gestão de projetos;
\item Expor o futuro profissional as técnicas, padrões e métodos com o intuito de traçar
objetivos, estimar custos e estabelecer cronogramas viáveis e realistas.
\end{itemize}
\end{itemize}
\\ \hline
\end{longtable}


%%%%%%%%%%%%%%%%%%%%%%%%%%%%%%%%%%%%%%%%%%%%%%%%%%%%%%%%%%%%%%%
\begin{longtable}{|L{1.025\textwidth}|} \hline
%
{\bf VI. CONTEÚDO PROGRAMÁTICO } \\ \hline
%% LUCIANA: REVISAR
Unidade 1: Introdução ao Gerenciamento de Projetos\\
Definições: o que é projeto; o que é programa; o que é portfólio;\\
Características de um projeto;\\
Diferença entre projeto e atividade funcional;\\
O que é gestão de projeto;\\ 
Fases e ciclo de vida dos projetos.\\
\\
Unidade 2: Planejamento do Projeto\\
Iniciação;\\ Planejamento;\\ Tarefas;\\ Escopo;\\ Diagrama de Precedência;\\ Cronograma;\\ Custos;\\ Riscos;\\Recursos Humanos;\\ Comunicação;\\ Qualidade;\\ Ferramentas de gestão de projetos.\\
\\
Unidade 3: Execução e Controle\\ Gerenciamento do cronograma;\\ Gerenciamento do Custo;\\ Gerenciamento de Mudanças;\\Resultados e impactos;\\ Encerramento.\\
\\
 \hline
\end{longtable} 





%%%%%%%%%%%%%%%%%%%%%%%%%%%%%%%%%%%%%%%%%%%%%%%%%%%%%%%%%%%%%%%
\begin{longtable}{|L{1.025\textwidth}|} \hline
%
{\bf VII. BIBLIOGRAFIA BÁSICA} \\ \hline
\begin{enumerate}
\item XAVIER, Carlos Magno da Silva. Gerenciamento de projetos: como definir e controlar o escopo do projeto. 2. ed. São Paulo: Saraiva, 2009. 259 p. ISBN 9788502061958.
\item VARGAS, Ricardo Viana. Manual prático do plano de projeto: utilizando o PMBOK guide. 4.ed. Rio de Janeiro: Brasport, 2009. 230p. ISBN 9788574524306.
\item MENEZES, Luís César de Moura. Gestão de projetos. 3. ed. São Paulo: Atlas, 2009. 242p. ISBN 9788522440405.
\end{enumerate}
 \\ \hline
\end{longtable}


%\newpage

%%%%%%%%%%%%%%%%%%%%%%%%%%%%%%%%%%%%%%%%%%%%%%%%%%%%%%%%%%%%%%%
\begin{longtable}{|L{1.025\textwidth}|} \hline
%
{\bf VIII. BIBLIOGRAFIA COMPLEMENTAR} \\ \hline
\begin{enumerate}

\item VIEIRA, Marconi Fábio. Gerenciamento de projetos de tecnologia da informação. 2. ed. total. rev. e atual. Rio de Janeiro: Elsevier, c2007. 1 CD-ROM
\item VERZUH, Eric. MBA compacto: gestão de projetos. Rio de Janeiro: Elsevier, 2000. 398p. ISBN 853520637X.
\item SOTILLE, Mauro Afonso. Gerenciamento do escopo em projetos. 2.ed. Rio de Janeiro: Ed. da FGV, 2010. 171p. ISBN 8522505799 (broch.).
\item BORDEAUX-RÊGO, Ricardo. Viabilidade econômico-financeira de projetos. 3.ed. Rio de Janeiro: FGV, 2010. 161p. ISBN 9788522507788


%
\end{enumerate}
 \\ \hline
\end{longtable}


\input aprovacao.tex


\end{document}
