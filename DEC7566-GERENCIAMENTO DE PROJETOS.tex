\documentclass[12pt]{article}
\usepackage[brazil]{babel}
\usepackage{graphicx,t1enc,wrapfig,amsmath,float}
\usepackage{framed,fancyhdr}
\usepackage{multirow}
\usepackage{longtable}
\usepackage{array}
\newcolumntype{L}[1]{>{\raggedright\let\newline\\\arraybackslash\hspace{0pt}}m{#1}}
\newcolumntype{C}[1]{>{\centering\let\newline\\\arraybackslash\hspace{0pt}}m{#1}}
\newcolumntype{R}[1]{>{\raggedleft\let\newline\\\arraybackslash\hspace{0pt}}m{#1}}
%%%%%%%%%%%%%
\oddsidemargin -0.5cm
\evensidemargin -0.5cm
\textwidth 17.5cm
\topmargin -1.5cm
\textheight 22cm
%%%%%%%%%%%% 

%\pagestyle{empty}

\newcommand{\semestre}{2018.2}

\newcommand{\disciplina}{GERENCIAMENTO DE PROJETOS}
\newcommand{\codigo}{DEC7566}


%%%%%%%%%%%%%%%%%%%%%%%%%%%%%%%%%%%%%%%%%%%%%%%%%%%%%%%
%%%%%%%%%%%%% CRETIDOS
\newcommand{\creditosT}{2}
\newcommand{\creditosP}{2}

%%%%%%%%%%%%%%%%%%%%%%%%%%%%%%%%%%%%%%%%%%%%%%%%%%%%%%%
%%%%%%%%%%%%%% REQUISITOS
\newcommand{\requisitoA}{}
\newcommand{\requisitoB}{}
\newcommand{\requisitoC}{}

%%%%%%%%%%%%%%%%%%%%%%%%%%%%%%%%%%%%%%%%%%%%%%%%%%%%%%%
%%%%%%%%%%%%%%% Atende aos Cursos
\newcommand{\cursoA}{Graduação em Engenharia de Computação \\ \hline}
\newcommand{\cursoB}{}%Graduação em Tecnologias da Informação e Comunicação \\ \hline}
\newcommand{\cursoC}{}

%%%%%%%%%%%%%%%%%%%%%%%%%%%%%%%%%%%%%%%%%%%%%%%%%%%%%%%%
%%%%%%%%%% EMENTA
\newcommand{\ementa}{
Fundamentos da Gestão de Projetos: Introdução e Histórico; Conceitos Básicos; Benefícios do Gerenciamento de Projetos. O Contexto da Gestão de Projetos: Fases e Ciclo de Vida de Projetos; Os Processos de Gestão de Projetos: Conceitos de Processos de Gerenciamento de Projetos; Processos e ciclo de vida de projetos. Início de um Projeto; Gestão do Escopo; Gestão de Prazos; Gestão de Custos; Gestão de Qualidade; Gestão de Recursos Humanos; Gestão de Comunicação; Gestão de Riscos. Introdução ao MSProject.
\\ \hline
}


\begin{document}

%%%%%%%%%%%%%%%%%%%%%%%%%%%%%%%%%%%%%%%%%%%%%%%%%%%%%%%%%%%%%

\input cabecalho.tex


%%%%%%%%%%%%%%%%%%%%%%%%%%%%%%%%%%%%%%%%%%%%%%%%%%%%%%%%%%%%%
\begin{longtable}{|C{0.11\textwidth}|C{0.29\textwidth}|C{0.09\textwidth}|C{0.09\textwidth}|C{0.15\textwidth}|C{0.158\textwidth}|} \hline
%
\multicolumn{6}{|l|}{{\bf I. IDENTIFICAÇÃO DA DISCIPLINA}} \\ \hline
%
\multirow{3}*{{\small CÓDIGO}} & \multirow{3}*{NOME DA DISCIPLINA} &\multicolumn{2}{c|}{{\small N$^\circ$ DE HORAS-AULA }} & {{\small TOTAL DE}} & \multirow{3}*{{\small MODALIDADE}} \\ 
%
& & \multicolumn{2}{c|}{\small SEMANAIS}  & {\small HORAS-AULA} & \\ \cline{3-4}
%
& & {\tiny TEÓRICAS} & {\tiny PRÁTICAS} & {\small SEMESTRAIS} & \\ \hline
% codigo da disciplina carga horaria: teorica - pratica e total
{\bf \small \codigo} & {\bf \small \disciplina } & {\bf \creditosT} & {\bf \creditosP} & {\bf 72} & Presencial\\ \hline
\end{longtable}


%%%%%%%%%%%%%%%%%%%%%%%%%%%%%%%%%%%%%%%%%%%%%%%%%%%%%%%%%%%%%%
\begin{longtable}{|C{0.12\textwidth}|L{0.736\textwidth}|C{0.12\textwidth}|} \hline
%
\multicolumn{3}{|l|}{{\bf II. PRÉ-REQUISITO(S)}} \\ \hline
%
CÓDIGO & NOME DA DISCIPLINA & CURSO \\ \hline	
%
\requisitoA
\requisitoB
\requisitoC
\end{longtable}


%%%%%%%%%%%%%%%%%%%%%%%%%%%%%%%%%%%%%%%%%%%%%%%%%%%%%%%%%%%%%%
\begin{longtable}{|L{1.025\textwidth}|} \hline
%
{\bf III. CURSO(S) PARA O(S) QUAL(IS) A DISCIPLINA É OFERECIDA } \\ \hline
%
\cursoA 
\cursoB
\cursoC

\end{longtable}

%%%%%%%%%%%%%%%%%%%%%%%%%%%%%%%%%%%%%%%%%%%%%%%%%%%%%%%%%%%%%%
\begin{longtable}{|L{1.025\textwidth}|} \hline
%
{\bf IV. EMENTA } \\ \hline
%
\ementa
\end{longtable}

\newpage



%%%%%%%%%%%%%%%%%%%%%%%%%%%%%%%%%%%%%%%%%%%%%%%%%%%%%%%%%%%%%%%
\begin{longtable}{|L{1.025\textwidth}|} \hline
%
{\bf V. OBJETIVOS } \\ \hline
Objetivo Geral: \\

Desenvolver no futuro profissional de Engenharia de Computação uma visão abrangente e estratégica dos negócios na área de Tecnologias da Informação. Noções de planejamento, técnicas, habilidades necessárias para a gestão de serviços de tecnologia.\\
\\
Objetivos Específicos:\\

Fornecer uma visão ampla da aplicação e dos benefícios da gestão de projetos;
Expor o futuro profissional as técnicas, padrões e métodos com o intuito de traçar objetivos, estimar custos e estabelecer cronogramas viáveis e realistas. 
\\ \hline
\end{longtable}


%%%%%%%%%%%%%%%%%%%%%%%%%%%%%%%%%%%%%%%%%%%%%%%%%%%%%%%%%%%%%%%
\begin{longtable}{|L{1.025\textwidth}|} \hline
%
{\bf VI. CONTEÚDO PROGRAMÁTICO } \\ \hline
UNIDADE 1: Teoria\\
Apresentar os fundamentos de gestão de projeto;\\
Gestão da Integração;\\
Gestão do Escopo;\\
Gestão do Tempo;\\
Gestão de Custos;\\
Gestão da Qualidade;\\
Gestão de Recursos humanos;\\
Gestão da Comunicação;\\
Gestão de Riscos.\\
\\
UNIDADE 2: Atividades práticas\\
Usar softwares de auxílio de gestão de prazos, custos, materiais, mão de obra e demais recursos necessários para o planejamento, execução e acompanhamento de projetos.

\\ \hline
\end{longtable} 

%\newpage


%%%%%%%%%%%%%%%%%%%%%%%%%%%%%%%%%%%%%%%%%%%%%%%%%%%%%%%%%%%%%%%
\begin{longtable}{|L{1.025\textwidth}|} \hline
%
{\bf VII. BIBLIOGRAFIA BÁSICA} \\ \hline

\begin{enumerate}
\item XAVIER, Carlos Magno da Silva. Gerenciamento de projetos: como definir e controlar o escopo do 95 projeto. 2. ed. São Paulo: Saraiva, 2009. 259 p. ISBN 9788502061958. 
\item VARGAS, Ricardo Viana. Manual prático do plano de projeto: utilizando o PMBOK guide. 4.ed. Rio de Janeiro: Brasport, 2009. 230p. ISBN 9788574524306. 
\item  MENEZES, Luís César de Moura. Gestão de projetos. 3. ed. São Paulo: Atlas, 2009. 242p. ISBN 9788522440405.
\end{enumerate}

 \\ \hline
\end{longtable}


%\newpage

%%%%%%%%%%%%%%%%%%%%%%%%%%%%%%%%%%%%%%%%%%%%%%%%%%%%%%%%%%%%%%%
\begin{longtable}{|L{1.025\textwidth}|} \hline
%
{\bf VIII. BIBLIOGRAFIA COMPLEMENTAR} \\ \hline
\begin{enumerate}
\item VIEIRA, Marconi Fábio. Gerenciamento de projetos de tecnologia da informação. 2. ed. total. rev. e atual. Rio de Janeiro: Elsevier, c2007. 1 CD-ROM 
\item VERZUH, Eric. MBA compacto: gestão de projetos. Rio de Janeiro: Elsevier, 2000. 398p. ISBN 853520637X. 
\item SOTILLE, Mauro Afonso. Gerenciamento do escopo em projetos. 2.ed. Rio de Janeiro: Ed. da FGV, 2010. 171p. ISBN 8522505799 (broch.).
\item BORDEAUX-RÊGO, Ricardo. Viabilidade econômico-financeira de projetos. 3.ed. Rio de Janeiro: FGV, 2010. 161p. ISBN 9788522507788
\item HELDMAN, Kim. Gerência de projetos: guia para o exame oficial do PMI. 7. ed. atual. Rio de Janeiro: Elsevier, 2015. 603 p. ISBN 9788535276152. 
%
\end{enumerate}
 \\ \hline
\end{longtable}


\input aprovacao.tex


\end{document}
