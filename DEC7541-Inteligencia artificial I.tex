\documentclass[12pt]{article}
\usepackage[brazil]{babel}
\usepackage{graphicx,t1enc,wrapfig,amsmath,float}
\usepackage{framed,fancyhdr}
\usepackage{multirow}
\usepackage{longtable}
\usepackage{array}
\newcolumntype{L}[1]{>{\raggedright\let\newline\\\arraybackslash\hspace{0pt}}m{#1}}
\newcolumntype{C}[1]{>{\centering\let\newline\\\arraybackslash\hspace{0pt}}m{#1}}
\newcolumntype{R}[1]{>{\raggedleft\let\newline\\\arraybackslash\hspace{0pt}}m{#1}}
%%%%%%%%%%%%%
\oddsidemargin -0.5cm
\evensidemargin -0.5cm
\textwidth 17.5cm
\topmargin -1.5cm
\textheight 22cm
%%%%%%%%%%%% 

%\pagestyle{empty}

\newcommand{\semestre}{2018.2}

\newcommand{\disciplina}{INTELIGÊNCIA ARTIFICIAL I}
\newcommand{\codigo}{DEC7541}


%%%%%%%%%%%%%%%%%%%%%%%%%%%%%%%%%%%%%%%%%%%%%%%%%%%%%%%
%%%%%%%%%%%%% CRETIDOS
\newcommand{\creditosT}{4}
\newcommand{\creditosP}{0}

%%%%%%%%%%%%%%%%%%%%%%%%%%%%%%%%%%%%%%%%%%%%%%%%%%%%%%%
%%%%%%%%%%%%%% REQUISITOS
\newcommand{\requisitoA}{}
\newcommand{\requisitoB}{}
\newcommand{\requisitoC}{}

%%%%%%%%%%%%%%%%%%%%%%%%%%%%%%%%%%%%%%%%%%%%%%%%%%%%%%%
%%%%%%%%%%%%%%% Atende aos Cursos
\newcommand{\cursoA}{Graduação em Engenharia de Computação \\ \hline}
\newcommand{\cursoB}{}%Graduação em Tecnologias da Informação e Comunicação \\ \hline}
\newcommand{\cursoC}{}%Graduação em Engenharia de Energia \\ \hline}

%%%%%%%%%%%%%%%%%%%%%%%%%%%%%%%%%%%%%%%%%%%%%%%%%%%%%%%%
%%%%%%%%%% EMENTA
\newcommand{\ementa}{
Introdução à resolução de problemas. Notas Históricas. Métodos de Busca de informação e heurística. Representação e aquisição de Conhecimento. Introdução à Aprendizagem da Máquina e a algoritmos de aprendizagem simbólica. Sistemas Especialistas, Agentes Inteligentes e Sistemas Multiagentes.

\\ \hline
}


\begin{document}


%%%%%%%%%%%%%%%%%%%%%%%%%%%%%%%%%%%%%%%%%%%%%%%%%%%%%%%%%%%%%
\input cabecalho.tex


%%%%%%%%%%%%%%%%%%%%%%%%%%%%%%%%%%%%%%%%%%%%%%%%%%%%%%%%%%%%%
\begin{longtable}{|C{0.11\textwidth}|C{0.29\textwidth}|C{0.09\textwidth}|C{0.09\textwidth}|C{0.15\textwidth}|C{0.158\textwidth}|} \hline
%
\multicolumn{6}{|l|}{{\bf I. IDENTIFICAÇÃO DA DISCIPLINA}} \\ \hline
%
\multirow{3}*{{\small CÓDIGO}} & \multirow{3}*{NOME DA DISCIPLINA} &\multicolumn{2}{c|}{{\small N$^\circ$ DE HORAS-AULA }} & {{\small TOTAL DE}} & \multirow{3}*{{\small MODALIDADE}} \\ 
%
& & \multicolumn{2}{c|}{\small SEMANAIS}  & {\small HORAS-AULA} & \\ \cline{3-4}
%
& & {\tiny TEÓRICAS} & {\tiny PRÁTICAS} & {\small SEMESTRAIS} & \\ \hline
% codigo da disciplina carga horaria: teorica - pratica e total
{\bf \small \codigo} & {\bf \small \disciplina } & {\bf \creditosT} & {\bf \creditosP} & {\bf 72} & Presencial\\ \hline
\end{longtable}


%%%%%%%%%%%%%%%%%%%%%%%%%%%%%%%%%%%%%%%%%%%%%%%%%%%%%%%%%%%%%%
\begin{longtable}{|C{0.12\textwidth}|L{0.736\textwidth}|C{0.12\textwidth}|} \hline
%
\multicolumn{3}{|l|}{{\bf II. PRÉ-REQUISITO(S)}} \\ \hline
%
CÓDIGO & NOME DA DISCIPLINA & CURSO \\ \hline	
%
\requisitoA
\requisitoB
\requisitoC
\end{longtable}


%%%%%%%%%%%%%%%%%%%%%%%%%%%%%%%%%%%%%%%%%%%%%%%%%%%%%%%%%%%%%%
\begin{longtable}{|L{1.025\textwidth}|} \hline
%
{\bf III. CURSO(S) PARA O(S) QUAL(IS) A DISCIPLINA É OFERECIDA } \\ \hline
%
\cursoA 
\cursoB
\cursoC

\end{longtable}

%%%%%%%%%%%%%%%%%%%%%%%%%%%%%%%%%%%%%%%%%%%%%%%%%%%%%%%%%%%%%%
\begin{longtable}{|L{1.025\textwidth}|} \hline
%
{\bf IV. EMENTA } \\ \hline
%
\ementa
\end{longtable}

%\newpage



%%%%%%%%%%%%%%%%%%%%%%%%%%%%%%%%%%%%%%%%%%%%%%%%%%%%%%%%%%%%%%%
\begin{longtable}{|L{1.025\textwidth}|} \hline
%
{\bf V. OBJETIVOS } \\ \hline
%
Objetivos Gerais: \\
Capacitar os alunos à criação de soluções para problemas em computação usando técnicas da Inteligência Artificial.\\
 \\
Objetivos Específicos:
\begin{itemize}
\item Caracterizar a inteligência artificial na resolução de problemas;
\item Conhecer as técnicas da inteligência Artificial Simbólica;
\item Desenvolver uma aplicação utilizando as técnicas de Inteligência Artificial.
\end{itemize}
\\ \hline
\end{longtable}

\newpage
%%%%%%%%%%%%%%%%%%%%%%%%%%%%%%%%%%%%%%%%%%%%%%%%%%%%%%%%%%%%%%%
\begin{longtable}{|L{1.025\textwidth}|} \hline
%
{\bf VI. CONTEÚDO PROGRAMÁTICO } \\ \hline

Unidade 1. Introdução e histórico da Inteligência Artificial\\
O que é a inteligência artificial\\
 Histórico da inteligência artificial\\
 Problemas tratados em inteligência artificial\\
 Domínios de aplicação da inteligência artificial\\
\\
Unidade 2. Introdução à resolução de problemas.\\
 Teoria de problemas\\
 Características de problemas\\
 Complexidade de algoritmos na solução de problemas\\
              Estratégias para resolver problemas\\
             Exemplo de um problema clássico de IA\\
\\
Unidade 3. Métodos de Busca de informação e heurística.\\
Busca heurística \\
Método de busca cega\\
Método de busca competitiva em Jogos.\\
\\
Unidade 4. Aprendizado de máquina e a algoritmos de aprendizagem simbólica. Representação e aquisição de Conhecimento:\\
           Símbolos e representações\\
            Representação Lógica \\
            Engenharia ontológica\\
            Representações declarativas.\\
\\
Unidade 5. Sistemas Especialistas \\
            Definição de sistemas especialistas\\
            Estrutura de um  sistema especialista\\
            Técnicas de extração do conhecimento\\
            Raciocínio progressivo e regressivo\\
            Exemplos de sistemas especialistas desenvolvidos.\\
            Ferramenta para desenvolvimento de Sistemas Especialistas\\
\\
Unidade 6.  Agentes Inteligentes e Sistemas Multiagentes\\
 Definição de Agentes\\
 Tipos e Propriedades de Agentes\\
 Arquiteturas e Organizações SMA\\
 Comunicação, Coordenação, Cooperação e Colaboração\\
 Integração e Interoperação de SMA\\
             Modelagem e Implementação de Agentes
\\ \hline
\end{longtable} 

%\newpage

%%%%%%%%%%%%%%%%%%%%%%%%%%%%%%%%%%%%%%%%%%%%%%%%%%%%%%%%%%%%%%%
\begin{longtable}{|L{1.025\textwidth}|} \hline
%
{\bf VII. BIBLIOGRAFIA BÁSICA} \\ \hline
\begin{enumerate}
%
\item RUSSELL, S.; NORVIG, P. Inteligência Artificial. 2 ed. Editora Campus. 2004. 
\item NASCIMENTO JÚNIOR, Cairo Lúcio; YONEYAMA, Takashi. Inteligência artificial: em controle e automação. São Paulo: FAPESP, c2000. vii, 218 p. ISBN 9788521203100.
\item ROSA, João Luís Garcia. Fundamentos da inteligência artificial. Rio de Janeiro: LTC, c2011. xv, 212 p. ISBN 9788521605935.

\end{enumerate}
 \\ \hline
\end{longtable}


%\newpage

%%%%%%%%%%%%%%%%%%%%%%%%%%%%%%%%%%%%%%%%%%%%%%%%%%%%%%%%%%%%%%%
\begin{longtable}{|L{1.025\textwidth}|} \hline
%
{\bf VIII. BIBLIOGRAFIA COMPLEMENTAR} \\ \hline
\begin{enumerate}

\item COPPIN, Ben. Inteligência artificial. Rio de Janeiro: LTC, c2010. xxv, 636 p. ISBN 9788521617297.
\item COSTA E.; SIMÕES A., Inteligência Artificial: Fundamentos e Aplicações, 2a Edição, Editora FCA, 2008. 
\item FACELI, Katti et al. Inteligência artificial: uma abordagem de aprendizado por máquina. Rio de Janeiro: LTC, c2011. xvi, 378 p. ISBN 9788521618805.

\item BELLIFEMINE F, CAIRE, G. GREENWOOD, D, Developing multiagents system with JADE, Series Editor: Michael Wooldridge, Liverpool University, UK 2004.

\end{enumerate}
 \\ \hline
\end{longtable}


\input aprovacao.tex


\end{document}
