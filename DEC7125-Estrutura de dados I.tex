\documentclass[12pt]{article}
\usepackage[brazil]{babel}
\usepackage{graphicx,t1enc,wrapfig,amsmath,float}
\usepackage{framed,fancyhdr}
\usepackage{multirow}
\usepackage{longtable}
\usepackage{array}
\newcolumntype{L}[1]{>{\raggedright\let\newline\\\arraybackslash\hspace{0pt}}m{#1}}
\newcolumntype{C}[1]{>{\centering\let\newline\\\arraybackslash\hspace{0pt}}m{#1}}
\newcolumntype{R}[1]{>{\raggedleft\let\newline\\\arraybackslash\hspace{0pt}}m{#1}}
%%%%%%%%%%%%%
\oddsidemargin -0.5cm
\evensidemargin -0.5cm
\textwidth 17.5cm
\topmargin -1.5cm
\textheight 22cm
%%%%%%%%%%%% 

%\pagestyle{empty}

\newcommand{\semestre}{2018.2}

\newcommand{\disciplina}{ESTRUTURA DE DADOS I}
\newcommand{\codigo}{DEC7125}


%%%%%%%%%%%%%%%%%%%%%%%%%%%%%%%%%%%%%%%%%%%%%%%%%%%%%%%
%%%%%%%%%%%%% CRETIDOS
\newcommand{\creditosT}{2}
\newcommand{\creditosP}{2}

%%%%%%%%%%%%%%%%%%%%%%%%%%%%%%%%%%%%%%%%%%%%%%%%%%%%%%%
%%%%%%%%%%%%%% REQUISITOS
\newcommand{\requisitoA}{}
\newcommand{\requisitoB}{}
\newcommand{\requisitoC}{}

%%%%%%%%%%%%%%%%%%%%%%%%%%%%%%%%%%%%%%%%%%%%%%%%%%%%%%%
%%%%%%%%%%%%%%% Atende aos Cursos
\newcommand{\cursoA}{Graduação em Engenharia de Computação \\ \hline}
\newcommand{\cursoB}{}%{Graduação em Tecnologias da Informação e Comunicação \\ \hline}
\newcommand{\cursoC}{}

%%%%%%%%%%%%%%%%%%%%%%%%%%%%%%%%%%%%%%%%%%%%%%%%%%%%%%%%
%%%%%%%%%% EMENTA
\newcommand{\ementa}{
Listas lineares e suas generalizações: listas ordenadas, listas encadeadas, pilhas e filas. Aplicações de listas. Algoritmos de inserção, remoção e consulta. Tabelas de Espalhamento Árvores binária. Métodos de pesquisa. Técnicas de implementação iterativa e recursiva de estruturas de dados.
 \\ \hline
}




\begin{document}


%%%%%%%%%%%%%%%%%%%%%%%%%%%%%%%%%%%%%%%%%%%%%%%%%%%%%%%%%%%%%
\input cabecalho.tex


%%%%%%%%%%%%%%%%%%%%%%%%%%%%%%%%%%%%%%%%%%%%%%%%%%%%%%%%%%%%%
\begin{longtable}{|C{0.11\textwidth}|C{0.29\textwidth}|C{0.09\textwidth}|C{0.09\textwidth}|C{0.15\textwidth}|C{0.158\textwidth}|} \hline
%
\multicolumn{6}{|l|}{{\bf I. IDENTIFICAÇÃO DA DISCIPLINA}} \\ \hline
%
\multirow{3}*{{\small CÓDIGO}} & \multirow{3}*{NOME DA DISCIPLINA} &\multicolumn{2}{c|}{{\small N$^\circ$ DE HORAS-AULA }} & {{\small TOTAL DE}} & \multirow{3}*{{\small MODALIDADE}} \\ 
%
& & \multicolumn{2}{c|}{\small SEMANAIS}  & {\small HORAS-AULA} & \\ \cline{3-4}
%
& & {\tiny TEÓRICAS} & {\tiny PRÁTICAS} & {\small SEMESTRAIS} & \\ \hline
% codigo da disciplina carga horaria: teorica - pratica e total
{\bf \small \codigo} & {\bf \small \disciplina } & {\bf \creditosT} & {\bf \creditosP} & {\bf 72} & Presencial\\ \hline
\end{longtable}


%%%%%%%%%%%%%%%%%%%%%%%%%%%%%%%%%%%%%%%%%%%%%%%%%%%%%%%%%%%%%%
\begin{longtable}{|C{0.12\textwidth}|L{0.736\textwidth}|C{0.12\textwidth}|} \hline
%
\multicolumn{3}{|l|}{{\bf II. PRÉ-REQUISITO(S)}} \\ \hline
%
CÓDIGO & NOME DA DISCIPLINA & CURSO \\ \hline	
%
\requisitoA
\requisitoB
\requisitoC
\end{longtable}


%%%%%%%%%%%%%%%%%%%%%%%%%%%%%%%%%%%%%%%%%%%%%%%%%%%%%%%%%%%%%%
\begin{longtable}{|L{1.025\textwidth}|} \hline
%
{\bf III. CURSO(S) PARA O(S) QUAL(IS) A DISCIPLINA É OFERECIDA } \\ \hline
%
\cursoA 
\cursoB
\cursoC

\end{longtable}

%%%%%%%%%%%%%%%%%%%%%%%%%%%%%%%%%%%%%%%%%%%%%%%%%%%%%%%%%%%%%%
\begin{longtable}{|L{1.025\textwidth}|} \hline
%
{\bf IV. EMENTA } \\ \hline
%
\ementa
\end{longtable}

\newpage



%%%%%%%%%%%%%%%%%%%%%%%%%%%%%%%%%%%%%%%%%%%%%%%%%%%%%%%%%%%%%%%
\begin{longtable}{|L{1.025\textwidth}|} \hline
%
{\bf V. OBJETIVOS } \\ \hline
%

Objetivo Geral:\\
Abordar formalmente as estruturas de dados e as técnicas de manipulação destas estruturas, bem como analisar métodos de pesquisa, ordenação e representação de dados aplicando a estrutura de dados mais adequada para um dado sistema computacional.\\
\\
Objetivos Específicos:
\begin{itemize}
\item Estudar as técnicas para estruturação de dados;
\item Analisar e conhecer os principais algoritmos de ordenação de dados;
\item Estudar técnicas de busca de dados; e
\item Implementar estruturas de dados e algoritmos de ordenação e pesquisa de dados usando a linguagem de programação C
\end{itemize}

\\ \hline
\end{longtable}


%%%%%%%%%%%%%%%%%%%%%%%%%%%%%%%%%%%%%%%%%%%%%%%%%%%%%%%%%%%%%%%
\begin{longtable}{|L{1.025\textwidth}|} \hline
%
{\bf VI. CONTEÚDO PROGRAMÁTICO } \\ \hline

Conteúdo Teórico seguido de Conteúdo Prático com desenvolvimento de problemas em computador:\\
\\
Unidade 1: Introdução\\
Apresentação da disciplina (ementa, bibliografia, metodologia e avaliações)\\
Introdução às estruturas de dados\\
Tipo de dados abstrato \\
Lista encadeada, circular e duplamente encadeada\\
Implementação de listas encadeadas\\
Aplicação de listas encadeadas\\
\\
Unidade 2: Pilhas e Filas\\
Pilha\\
Fila\\
Implementação de pilha e fila\\
Aplicação de fila e fila\\
\\
Unidade 3: Algoritmos de Ordenação de Dados\\
Algoritmos de ordenação de dados\\
Algoritmos de inserção, remoção e pesquisa de dados\\
Técnicas de implementação iterativa e recursiva de estruturas de dados\\
Métodos de busca\\
\\
Unidade 4: Árvore\\
Árvore binária (conceitos e aplicações)\\
Implementação de árvore binária\\
Busca de dados em árvore binária\\
Árvore balanceada\\
Fundamentos de Grafos\
\\
Unidade 5: Tabela de Espalhamento\\
Tabela de espalhamento\\
Implementação de tabela de espalhamento\\
Tratamento de colisões
\\ \hline
\end{longtable} 





%%%%%%%%%%%%%%%%%%%%%%%%%%%%%%%%%%%%%%%%%%%%%%%%%%%%%%%%%%%%%%%
\begin{longtable}{|L{1.025\textwidth}|} \hline
%
{\bf VII. BIBLIOGRAFIA BÁSICA} \\ \hline
\begin{enumerate}
%
\item CORMEN, T. H. et al. Algoritmos - Teoria e Prática. Campus, 2002. 
\item FEOFILOFF, P. Algoritmos em Linguagem C. Campus 2008. 
\item ZIVIANI, N. Projeto de Algoritmos com Implementação em Pascal e C. 2 ed. São Paulo: Pioneira Thomson Learning, 2004.
%
\end{enumerate}
 \\ \hline
\end{longtable}


%\newpage

%%%%%%%%%%%%%%%%%%%%%%%%%%%%%%%%%%%%%%%%%%%%%%%%%%%%%%%%%%%%%%%
\begin{longtable}{|L{1.025\textwidth}|} \hline
%
{\bf VIII. BIBLIOGRAFIA COMPLEMENTAR} \\ \hline
\begin{enumerate}
%
\item CELES, Waldemar et al. Introdução a Estruturas de Dados. Campus, 2004. 
\item TENENBAUM, A. M. et al. Estruturas de Dados Usando C. São Paulo: Pearson Makron Books, 1995. 
\item LOUDON, K. Mastering Algorithms with C. O'Reilly Media, 1st edition, 1999. 
\item PEREIRA, Silvio do Lago. Estruturas de Dados Fundamentais. Érica, 2008. 
\item WIRTH, Nicolaus. Algoritmos e Estruturas de Dados. Rio de Janeiro: LTC Editora, 1989.
%
\end{enumerate}
 \\ \hline
\end{longtable}


\input aprovacao.tex


\end{document}
