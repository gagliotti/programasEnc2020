\documentclass[12pt]{article}
\usepackage[brazil]{babel}
\usepackage{graphicx,t1enc,wrapfig,amsmath,float}
\usepackage{framed,fancyhdr}
\usepackage{multirow}
\usepackage{longtable}
\usepackage{array}
\newcolumntype{L}[1]{>{\raggedright\let\newline\\\arraybackslash\hspace{0pt}}m{#1}}
\newcolumntype{C}[1]{>{\centering\let\newline\\\arraybackslash\hspace{0pt}}m{#1}}
\newcolumntype{R}[1]{>{\raggedleft\let\newline\\\arraybackslash\hspace{0pt}}m{#1}}
%%%%%%%%%%%%%
\oddsidemargin -0.5cm
\evensidemargin -0.5cm
\textwidth 17.5cm
\topmargin -1.5cm
\textheight 22cm
%%%%%%%%%%%% 

%\pagestyle{empty}

\newcommand{\semestre}{2018.2}

\newcommand{\disciplina}{FUNDAMENTOS MATEMÁTICOS PARA COMPUTAÇÃO}
\newcommand{\codigo}{DEC7121}


%%%%%%%%%%%%%%%%%%%%%%%%%%%%%%%%%%%%%%%%%%%%%%%%%%%%%%%
%%%%%%%%%%%%% CRETIDOS
\newcommand{\creditosT}{4}
\newcommand{\creditosP}{0}

%%%%%%%%%%%%%%%%%%%%%%%%%%%%%%%%%%%%%%%%%%%%%%%%%%%%%%%
%%%%%%%%%%%%%% REQUISITOS
\newcommand{\requisitoA}{}
\newcommand{\requisitoB}{}
\newcommand{\requisitoC}{}

%%%%%%%%%%%%%%%%%%%%%%%%%%%%%%%%%%%%%%%%%%%%%%%%%%%%%%%
%%%%%%%%%%%%%%% Atende aos Cursos
\newcommand{\cursoA}{Graduação em Engenharia de Computação. \\ \hline}
\newcommand{\cursoB}{Graduação em Tecnologias da Informação e Comunicação \\ \hline}
\newcommand{\cursoC}{}

%%%%%%%%%%%%%%%%%%%%%%%%%%%%%%%%%%%%%%%%%%%%%%%%%%%%%%%%
%%%%%%%%%% EMENTA
\newcommand{\ementa}{

Lógica matemática. Indução finita. Conjuntos. Relações e funções. Contagem. Álgebra booleana. Recursão. Fundamentos de grafos.
 \\ \hline
}




\begin{document}


%%%%%%%%%%%%%%%%%%%%%%%%%%%%%%%%%%%%%%%%%%%%%%%%%%%%%%%%%%%%%
\input cabecalho.tex

%%%%%%%%%%%%%%%%%%%%%%%%%%%%%%%%%%%%%%%%%%%%%%%%%%%%%%%%%%%%%
\begin{longtable}{|C{0.11\textwidth}|C{0.29\textwidth}|C{0.09\textwidth}|C{0.09\textwidth}|C{0.15\textwidth}|C{0.158\textwidth}|} \hline
%
\multicolumn{6}{|l|}{{\bf I. IDENTIFICAÇÃO DA DISCIPLINA}} \\ \hline
%
\multirow{3}*{{\small CÓDIGO}} & \multirow{3}*{NOME DA DISCIPLINA} &\multicolumn{2}{c|}{{\small N$^\circ$ DE HORAS-AULA }} & {{\small TOTAL DE}} & \multirow{3}*{{\small MODALIDADE}} \\ 
%
& & \multicolumn{2}{c|}{\small SEMANAIS}  & {\small HORAS-AULA} & \\ \cline{3-4}
%
& & {\tiny TEÓRICAS} & {\tiny PRÁTICAS} & {\small SEMESTRAIS} & \\ \hline
% codigo da disciplina carga horaria: teorica - pratica e total
{\bf \small \codigo} & {\bf \small \disciplina :} & {\bf \creditosT} & {\bf \creditosP} & {\bf 72} & Presencial\\ \hline
\end{longtable}


%%%%%%%%%%%%%%%%%%%%%%%%%%%%%%%%%%%%%%%%%%%%%%%%%%%%%%%%%%%%%%
\begin{longtable}{|C{0.12\textwidth}|L{0.736\textwidth}|C{0.12\textwidth}|} \hline
%
\multicolumn{3}{|l|}{{\bf II. PRÉ-REQUISITO(S)}} \\ \hline
%
CÓDIGO & NOME DA DISCIPLINA & CURSO \\ \hline	
%
\requisitoA
\requisitoB
\requisitoC
\end{longtable}


%%%%%%%%%%%%%%%%%%%%%%%%%%%%%%%%%%%%%%%%%%%%%%%%%%%%%%%%%%%%%%
\begin{longtable}{|L{1.025\textwidth}|} \hline
%
{\bf III. CURSO(S) PARA O(S) QUAL(IS) A DISCIPLINA É OFERECIDA } \\ \hline
%
\cursoA 
\cursoB
\cursoC

\end{longtable}

%%%%%%%%%%%%%%%%%%%%%%%%%%%%%%%%%%%%%%%%%%%%%%%%%%%%%%%%%%%%%%
\begin{longtable}{|L{1.025\textwidth}|} \hline
%
{\bf IV. EMENTA } \\ \hline
%
\ementa
\end{longtable}

\newpage



%%%%%%%%%%%%%%%%%%%%%%%%%%%%%%%%%%%%%%%%%%%%%%%%%%%%%%%%%%%%%%%
\begin{longtable}{|L{1.025\textwidth}|} \hline
%
{\bf V. OBJETIVOS } \\ \hline
%
Objetivo Geral:\\

Permitir a construção e desenvolvimento de um raciocínio lógico a partir da teoria dos conjuntos, da lógica matemática, das provas matemáticas e dos conceitos de funções e relações. Trabalhar com ferramentas de contagem para permitir a contagem de estruturas discretas. Apresentar os fundamentos de uma estrutura discreta (grafos). \\
\\
Objetivos Específicos:
\begin{itemize}
\item Dominar os conceitos básicos da teoria dos conjuntos, da lógica matemática, e de funções e relações.
\item Dominar ferramentas para contagem (permutação, combinação, coeficiente binomial e triângulo de pascal).
\item Dominar os princípios da demonstração matemática.
\item Dominar os princípios da indução matemática.
\item Dominar os fundamentos de grafos.
\end{itemize}
\\ \hline
\end{longtable}


%%%%%%%%%%%%%%%%%%%%%%%%%%%%%%%%%%%%%%%%%%%%%%%%%%%%%%%%%%%%%%%
\begin{longtable}{|L{1.025\textwidth}|} \hline
%
{\bf VI. CONTEÚDO PROGRAMÁTICO } \\ \hline

UNIDADE 1: Conjuntos, Relações e Funções \\

Conjuntos\\
Relações\\
Relações equivalentes e partições\\
Funções\\
\\
UNIDADE 2: Lógica Computacional \\

Operadores AND, OR, NOT, Tabelas Verdade\\
Implicação e bicondicional\\
Tautologias\\
Argumentos e Princípios da Demonstração\\
Quantificadores \\
Métodos de prova\\
\\
UNIDADE 3: Contagem \\

Permutação\\
Combinação\\
Princípio da casa dos pombos\\
Triângulo de Pascal\\
\\
UNIDADE 4: Teoria de Grafos \\

Definição e conceitos preliminares\\
Diferentes tipos de grafos\\
Representações de grafos\\
Conexidade e distância\\
Caminho\\
Problemas do menor caminho\\
Árvores
\\ \hline
\end{longtable} 





%%%%%%%%%%%%%%%%%%%%%%%%%%%%%%%%%%%%%%%%%%%%%%%%%%%%%%%%%%%%%%%
\begin{longtable}{|L{1.025\textwidth}|} \hline
%
{\bf VII. BIBLIOGRAFIA BÁSICA} \\ \hline
\begin{enumerate}
%
\item GERSTING, J. L. Fundamentos Matemáticos para a Ciência da Computação. 5 ed.. Rio de Janeiro: LTC, 2004. 
\item FILHO, Alencar E. Iniciação a Lógica Matemática. 21. ed. São Paulo: Nobel, 2008. 
\item MENEZES, P. B. Matemática Discreta para Computação e Informática. 2 Ed. Porto Alegre: Bookman, 2008.

%

\end{enumerate}
 \\ \hline
\end{longtable}


%\newpage

%%%%%%%%%%%%%%%%%%%%%%%%%%%%%%%%%%%%%%%%%%%%%%%%%%%%%%%%%%%%%%%
\begin{longtable}{|L{1.025\textwidth}|} \hline
%
{\bf VIII. BIBLIOGRAFIA COMPLEMENTAR} \\ \hline
\begin{enumerate}

\item SCHEINERMAN, E. R. Matemática Discreta Uma Introdução. São Paulo: Pioneira Thomson Learning, 2003. 
\item LIPSCHUTZ, Seymour; LIPSON, Marc. Matemática discreta. Porto Alegre: Bookman, 2004. (Coleção Schaum). 
\item CORMEN, Thomas H. et al. Algoritmos: teoria e prática. Rio de Janeiro: Elsevier, 2002.
\item GRAHAM, R. L., D. E. Knuth, et al. Concrete mathematics: a foundation for computer science. Reading: Addison-Wesley, 1994. 
\item FILHO, Edgard de Alencar. Iniciação a lógica matemática. São Paulo: Nobel, 2002. 203p.5.
%
\end{enumerate}
 \\ \hline
\end{longtable}


\input aprovacao.tex


\end{document}
