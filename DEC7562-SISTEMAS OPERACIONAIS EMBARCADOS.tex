\documentclass[12pt]{article}
\usepackage[brazil]{babel}
\usepackage{graphicx,t1enc,wrapfig,amsmath,float}
\usepackage{framed,fancyhdr}
\usepackage{multirow}
\usepackage{longtable}
\usepackage{array}
\newcolumntype{L}[1]{>{\raggedright\let\newline\\\arraybackslash\hspace{0pt}}m{#1}}
\newcolumntype{C}[1]{>{\centering\let\newline\\\arraybackslash\hspace{0pt}}m{#1}}
\newcolumntype{R}[1]{>{\raggedleft\let\newline\\\arraybackslash\hspace{0pt}}m{#1}}
%%%%%%%%%%%%%
\oddsidemargin -0.5cm
\evensidemargin -0.5cm
\textwidth 17.5cm
\topmargin -1.5cm
\textheight 22cm
%%%%%%%%%%%% 

%\pagestyle{empty}

\newcommand{\semestre}{2018.2}

\newcommand{\disciplina}{SISTEMAS OPERACIONAIS EMBARCADOS}
\newcommand{\codigo}{DEC7562}


%%%%%%%%%%%%%%%%%%%%%%%%%%%%%%%%%%%%%%%%%%%%%%%%%%%%%%%
%%%%%%%%%%%%% CRETIDOS
\newcommand{\creditosT}{0}
\newcommand{\creditosP}{4}

%%%%%%%%%%%%%%%%%%%%%%%%%%%%%%%%%%%%%%%%%%%%%%%%%%%%%%%
%%%%%%%%%%%%%% REQUISITOS
\newcommand{\requisitoA}{}
\newcommand{\requisitoB}{}
\newcommand{\requisitoC}{}

%%%%%%%%%%%%%%%%%%%%%%%%%%%%%%%%%%%%%%%%%%%%%%%%%%%%%%%
%%%%%%%%%%%%%%% Atende aos Cursos
\newcommand{\cursoA}{Graduação em Engenharia de Computação \\ \hline}
\newcommand{\cursoB}{}%Graduação em Tecnologias da Informação e Comunicação \\ \hline}
\newcommand{\cursoC}{}

%%%%%%%%%%%%%%%%%%%%%%%%%%%%%%%%%%%%%%%%%%%%%%%%%%%%%%%%
%%%%%%%%%% EMENTA
\newcommand{\ementa}{
Conceitos de sistemas embarcados e sistemas operacionais embarcados. Projeto de sistemas operacionais embarcados. Sistemas operacionais embarcados de tempo real. Implementação de sistemas operacionais embarcados.
\\ \hline
}


\begin{document}

%%%%%%%%%%%%%%%%%%%%%%%%%%%%%%%%%%%%%%%%%%%%%%%%%%%%%%%%%%%%%

\input cabecalho.tex


%%%%%%%%%%%%%%%%%%%%%%%%%%%%%%%%%%%%%%%%%%%%%%%%%%%%%%%%%%%%%
\begin{longtable}{|C{0.11\textwidth}|C{0.29\textwidth}|C{0.09\textwidth}|C{0.09\textwidth}|C{0.15\textwidth}|C{0.158\textwidth}|} \hline
%
\multicolumn{6}{|l|}{{\bf I. IDENTIFICAÇÃO DA DISCIPLINA}} \\ \hline
%
\multirow{3}*{{\small CÓDIGO}} & \multirow{3}*{NOME DA DISCIPLINA} &\multicolumn{2}{c|}{{\small N$^\circ$ DE HORAS-AULA }} & {{\small TOTAL DE}} & \multirow{3}*{{\small MODALIDADE}} \\ 
%
& & \multicolumn{2}{c|}{\small SEMANAIS}  & {\small HORAS-AULA} & \\ \cline{3-4}
%
& & {\tiny TEÓRICAS} & {\tiny PRÁTICAS} & {\small SEMESTRAIS} & \\ \hline
% codigo da disciplina carga horaria: teorica - pratica e total
{\bf \small \codigo} & {\bf \small \disciplina } & {\bf \creditosT} & {\bf \creditosP} & {\bf 72} & Presencial\\ \hline
\end{longtable}


%%%%%%%%%%%%%%%%%%%%%%%%%%%%%%%%%%%%%%%%%%%%%%%%%%%%%%%%%%%%%%
\begin{longtable}{|C{0.12\textwidth}|L{0.736\textwidth}|C{0.12\textwidth}|} \hline
%
\multicolumn{3}{|l|}{{\bf II. PRÉ-REQUISITO(S)}} \\ \hline
%
CÓDIGO & NOME DA DISCIPLINA & CURSO \\ \hline	
%
\requisitoA
\requisitoB
\requisitoC
\end{longtable}


%%%%%%%%%%%%%%%%%%%%%%%%%%%%%%%%%%%%%%%%%%%%%%%%%%%%%%%%%%%%%%
\begin{longtable}{|L{1.025\textwidth}|} \hline
%
{\bf III. CURSO(S) PARA O(S) QUAL(IS) A DISCIPLINA É OFERECIDA } \\ \hline
%
\cursoA 
\cursoB
\cursoC

\end{longtable}

%%%%%%%%%%%%%%%%%%%%%%%%%%%%%%%%%%%%%%%%%%%%%%%%%%%%%%%%%%%%%%
\begin{longtable}{|L{1.025\textwidth}|} \hline
%
{\bf IV. EMENTA } \\ \hline
%
\ementa
\end{longtable}

\newpage



%%%%%%%%%%%%%%%%%%%%%%%%%%%%%%%%%%%%%%%%%%%%%%%%%%%%%%%%%%%%%%%
\begin{longtable}{|L{1.025\textwidth}|} \hline
%
{\bf V. OBJETIVOS } \\ \hline

Objetivo Geral: 
\\
Esta disciplina tem por objetivo apresentar os conceitos, problemas e soluções típicas no desenvolvimento de sistemas operacionais embarcados incluindo aqueles com restrições temporais. \\
\\
Objetivos Específicos:

\begin{itemize}
\item Definir e fundamentar os sistemas operacionais embarcados;
\item Estudar os principais aspectos envolvidos no projeto e no desenvolvimento de sistemas operacionais embarcados, tais como gerência de tarefas, memória e de entrada e saída de dados;
\item Projetar e implementar sistemas operacionais embarcados;
\item Estudar e utilizar sistemas operacionais embarcados existentes.
\end{itemize}
\\ \hline
\end{longtable}


%%%%%%%%%%%%%%%%%%%%%%%%%%%%%%%%%%%%%%%%%%%%%%%%%%%%%%%%%%%%%%%
\begin{longtable}{|L{1.025\textwidth}|} \hline
%
{\bf VI. CONTEÚDO PROGRAMÁTICO } \\ \hline
Conteúdo Teórico seguido de Conteúdo Prático com desenvolvimento de problemas em computador: \\
\\
UNIDADE 1: Introdução [4 horas-aula]\\
Definição e Características de um Sistema Embarcado\\
Sistemas operacionais embarcados (sistemas operacionais de tempo real)\\
Exemplos de sistemas operacionais embarcados e sistemas embarcados\\
Hardware para sistemas operacionais embarcados\\
Revisão de conceitos sobre Sistemas Operacionais\\
\\
UNIDADE 2: Gerência de Processos [30 horas-aula]\\
Definição\\
Definição de tarefa\\
Geração e tratamento de interrupções\\
Controle de timers\\
Escalonamento de tarefas\\
Escalonamento cooperativo versus preemptivo\\
Escalonamento de tarefas de tempo real\\
Sincronização de tarefas\\
Seção crítica\\
Semáforos e mutex\\
Controle de deadlock\\
Inversão de prioridade\\
Comunicação entre tarefas\\
Filas de mensagens\\
Estudo de caso: FreeRTOS\\
\\
UNIDADE 3: Gerência de Memória [08 horas-aula]\\
Definição\\
Alocação estática\\
Alocação dinâmica\\
Estudo de caso: FreeRTOS\\
\\
UNIDADE 4: Gerência de Entrada e Saída [06 horas-aula]\\
Definição\\
Funções de entrada e saída\\
Controle de concorrência\\
Estudo de caso: FreeRTOS\\
\\
UNIDADE 5: Estudo de Sistemas Operacionais Embarcados [16 horas-aula]\\
OSA Operating System\\
Linux Embarcado\\
\\ \hline
\end{longtable} 

%\newpage


%%%%%%%%%%%%%%%%%%%%%%%%%%%%%%%%%%%%%%%%%%%%%%%%%%%%%%%%%%%%%%%
\begin{longtable}{|L{1.025\textwidth}|} \hline
%
{\bf VII. BIBLIOGRAFIA BÁSICA} \\ \hline
\begin{enumerate}
\item LI, Qing. Real-Time Concepts for Embedded Systems. CRC Press, 2010. 
\item WHITE, Elecia. Making Embedded Systems. O'Reilly, 2012. 
\item OLIVEIRA, André Schneider; ANDRADA, Fernando Souza. Sistemas Embarcados - hardware, firmware na prática. 2a ed. Editora Érica, 2013.
\end{enumerate}
 \\ \hline
\end{longtable}


\newpage

%%%%%%%%%%%%%%%%%%%%%%%%%%%%%%%%%%%%%%%%%%%%%%%%%%%%%%%%%%%%%%%
\begin{longtable}{|L{1.025\textwidth}|} \hline
%
{\bf VIII. BIBLIOGRAFIA COMPLEMENTAR} \\ \hline
\begin{enumerate}
\item  IBRAHIM, Dogan. Microcontroller Based Applied Digital Control. John Wiley \& Sons Ltd, 2006. 
\item LABROUSE, Jean J. Embedded Systems Building Blocks. 2a ed. CRC Press, 2002. 
\item Son Sang H., Lee I., and Leung J. Handbook of Real-Time and Embedded Systems. Boca Raton: Chapman and Hall, 2008. 
\item SILBERSCHATZ, Abraham; GALVIN, Peter Baer; GAGNE; Greg. Fundamentos de Sistemas Operacionais. 8a ed. LTC, 2011. 
\item WOLF, Wayne. Computers as components: principles of embedded computing system design. San Francisco: Morgan Kaufmann, 2001. 662p.

%
\end{enumerate}
 \\ \hline
\end{longtable}


\input aprovacao.tex


\end{document}
