\documentclass[12pt]{article}
\usepackage[brazil]{babel}
\usepackage{graphicx,t1enc,wrapfig,amsmath,float}
\usepackage{framed,fancyhdr}
\usepackage{multirow}
\usepackage{longtable}
\usepackage{array}
\newcolumntype{L}[1]{>{\raggedright\let\newline\\\arraybackslash\hspace{0pt}}m{#1}}
\newcolumntype{C}[1]{>{\centering\let\newline\\\arraybackslash\hspace{0pt}}m{#1}}
\newcolumntype{R}[1]{>{\raggedleft\let\newline\\\arraybackslash\hspace{0pt}}m{#1}}
%%%%%%%%%%%%%
\oddsidemargin -0.5cm
\evensidemargin -0.5cm
\textwidth 17.5cm
\topmargin -1.5cm
\textheight 22cm
%%%%%%%%%%%% 

%\pagestyle{empty}

\newcommand{\semestre}{2018.2}

\newcommand{\disciplina}{LÓGICA DE PROGRAMAÇÃO}
\newcommand{\codigo}{DEC7143}


%%%%%%%%%%%%%%%%%%%%%%%%%%%%%%%%%%%%%%%%%%%%%%%%%%%%%%%
%%%%%%%%%%%%% CRETIDOS
\newcommand{\creditosT}{2}
\newcommand{\creditosP}{2}

%%%%%%%%%%%%%%%%%%%%%%%%%%%%%%%%%%%%%%%%%%%%%%%%%%%%%%%
%%%%%%%%%%%%%% REQUISITOS
\newcommand{\requisitoA}{}
\newcommand{\requisitoB}{}
\newcommand{\requisitoC}{}

%%%%%%%%%%%%%%%%%%%%%%%%%%%%%%%%%%%%%%%%%%%%%%%%%%%%%%%
%%%%%%%%%%%%%%% Atende aos Cursos
\newcommand{\cursoA}{Graduação em Engenharia de Computação \\ \hline}
\newcommand{\cursoB}{}%Graduação em Tecnologias da Informação e Comunicação \\ \hline}
\newcommand{\cursoC}{Graduação em Engenharia de Energia \\ \hline}

%%%%%%%%%%%%%%%%%%%%%%%%%%%%%%%%%%%%%%%%%%%%%%%%%%%%%%%%
%%%%%%%%%% EMENTA
\newcommand{\ementa}{
Lógica de Programação. Sequências lógicas, pseudocódigo, fluxograma, diagrama de chapin. Variáveis: nomeação, declaração, inicialização, tipos de dados. Expressões aritméticas, expressões literais, expressões lógicas, expressões relacionais. Estruturas de Dados Simples: vetores, matrizes, registros. Estruturas de 48 Controle de Fluxo: Linear, condicional, repetição. Entrada e Saída de Dados. Aplicação dos conceitos de lógica de programação em uma linguagem de programação.
 \\ \hline
}


\begin{document}


%%%%%%%%%%%%%%%%%%%%%%%%%%%%%%%%%%%%%%%%%%%%%%%%%%%%%%%%%%%%%
\input cabecalho.tex


%%%%%%%%%%%%%%%%%%%%%%%%%%%%%%%%%%%%%%%%%%%%%%%%%%%%%%%%%%%%%
\begin{longtable}{|C{0.11\textwidth}|C{0.29\textwidth}|C{0.09\textwidth}|C{0.09\textwidth}|C{0.15\textwidth}|C{0.158\textwidth}|} \hline
%
\multicolumn{6}{|l|}{{\bf I. IDENTIFICAÇÃO DA DISCIPLINA}} \\ \hline
%
\multirow{3}*{{\small CÓDIGO}} & \multirow{3}*{NOME DA DISCIPLINA} &\multicolumn{2}{c|}{{\small N$^\circ$ DE HORAS-AULA }} & {{\small TOTAL DE}} & \multirow{3}*{{\small MODALIDADE}} \\ 
%
& & \multicolumn{2}{c|}{\small SEMANAIS}  & {\small HORAS-AULA} & \\ \cline{3-4}
%
& & {\tiny TEÓRICAS} & {\tiny PRÁTICAS} & {\small SEMESTRAIS} & \\ \hline
% codigo da disciplina carga horaria: teorica - pratica e total
{\bf \small \codigo} & {\bf \small \disciplina } & {\bf \creditosT} & {\bf \creditosP} & {\bf 72} & Presencial\\ \hline
\end{longtable}


%%%%%%%%%%%%%%%%%%%%%%%%%%%%%%%%%%%%%%%%%%%%%%%%%%%%%%%%%%%%%%
\begin{longtable}{|C{0.12\textwidth}|L{0.736\textwidth}|C{0.12\textwidth}|} \hline
%
\multicolumn{3}{|l|}{{\bf II. PRÉ-REQUISITO(S)}} \\ \hline
%
CÓDIGO & NOME DA DISCIPLINA & CURSO \\ \hline	
%
\requisitoA
\requisitoB
\requisitoC
\end{longtable}


%%%%%%%%%%%%%%%%%%%%%%%%%%%%%%%%%%%%%%%%%%%%%%%%%%%%%%%%%%%%%%
\begin{longtable}{|L{1.025\textwidth}|} \hline
%
{\bf III. CURSO(S) PARA O(S) QUAL(IS) A DISCIPLINA É OFERECIDA } \\ \hline
%
\cursoA 
\cursoB
\cursoC

\end{longtable}

%%%%%%%%%%%%%%%%%%%%%%%%%%%%%%%%%%%%%%%%%%%%%%%%%%%%%%%%%%%%%%
\begin{longtable}{|L{1.025\textwidth}|} \hline
%
{\bf IV. EMENTA } \\ \hline
%
\ementa
\end{longtable}

\newpage



%%%%%%%%%%%%%%%%%%%%%%%%%%%%%%%%%%%%%%%%%%%%%%%%%%%%%%%%%%%%%%%
\begin{longtable}{|L{1.025\textwidth}|} \hline
%
{\bf V. OBJETIVOS } \\ \hline
%
Objetivo Geral:\\

Tornar o aluno apto a transpor para a forma algorítmica , soluções de problemas utilizando-se de notações formais de representação de algoritmos, tais como, pseudo-linguagens e diagramas de fluxo. \\
\\
Objetivos Específicos: 
\begin{itemize}
\item  Estudar os principais elementos de construção de algoritmos;
\item  Estudar e exercitar as principais formas de representação de algoritmos;
\item  Estudar e exercitar as estruturas de seleção e repetição;
\item  Estudar e exercitar as estruturas de dados simples: vetores, matrizes e registros;
\item  Estudar e exercitar os conceitos de modularização de algoritmos.
\end{itemize}
\\ \hline
\end{longtable}


%%%%%%%%%%%%%%%%%%%%%%%%%%%%%%%%%%%%%%%%%%%%%%%%%%%%%%%%%%%%%%%
\begin{longtable}{|L{1.025\textwidth}|} \hline
%
{\bf VI. CONTEÚDO PROGRAMÁTICO } \\ \hline
Conteúdo Teórico seguido de Conteúdo Prático com desenvolvimento de algoritmos.\\
\\
UNIDADE 1: Introdução à Lógica de Programação\\
Conceituação.\\
Histórico.\\
Instruções.\\
Algoritmos.\\
Formas de representar algoritmos.\\
\\
UNIDADE 2: Estruturas básicas\\
Tipos de dados: numéricos, literais e lógicos.\\
Declaração e atribuição.\\
Entrada e Saída de dados.\\
\\
UNIDADE 3: Estruturas de controle\\
Seleção: simples e composta.\\
Repetição.\\
Aninhamento e identação.\\
\\
UNIDADE 4: Estruturas de dados\\
Vetores.\\
Matrizes.\\
Registros.\\
\\
UNIDADE 5: Funções\\
Introdução.\\
Uso de funções em programas.\\
\\ \hline
\end{longtable} 

%\newpage

%%%%%%%%%%%%%%%%%%%%%%%%%%%%%%%%%%%%%%%%%%%%%%%%%%%%%%%%%%%%%%%
\begin{longtable}{|L{1.025\textwidth}|} \hline
%
{\bf VII. BIBLIOGRAFIA BÁSICA} \\ \hline
\begin{enumerate}
%
\item FORBELLONE, André L. V.; EBERSPÄCHER, Henri F. Lógica de Programação - a construção de algoritmos e estruturas de dados. 3a ed. Pearson Prentice Hall, 2005. 
\item XAVIER, Gley Fabiano Cardoso. Lógica de Programação. 13a ed. Senac, 2014. 
\item ASCENCIO, Ana Fernanda; CAMPOS, Edilene A. V. C. Fundamentos de Programação - algoritmos, Pascal, C/C++ e Java. 2a ed. Pearson Prentice Hall, 2008. 

\end{enumerate}
 \\ \hline
\end{longtable}


%\newpage

%%%%%%%%%%%%%%%%%%%%%%%%%%%%%%%%%%%%%%%%%%%%%%%%%%%%%%%%%%%%%%%
\begin{longtable}{|L{1.025\textwidth}|} \hline
%
{\bf VIII. BIBLIOGRAFIA COMPLEMENTAR} \\ \hline
\begin{enumerate}
\item MEDINA, Marcos; FERTIG, Cristina. Algoritmo e Programação - teoria e prática. Novatec, 2006. 
\item MANZANO, José A.; OLIVEIRA, Jayr Fiqueiredo de. Algoritmos - lógica para o desenvolvimento de programas de computador. 27a ed. Érica, 2014. 
\item FEOFILOFF, Paulo. Algoritmos em Linguagem C. Campus, 2009. 
\item GUIMARÃES, Ângelo de M.; LAGES, Newton A. de C. Algoritmos e Estruturas de Dados. 33a ed. Gen LTC, 2008. 
\item SEBESTA, Robert. Conceitos de Linguagens de Programação. 9a ed. Bookman, 2010.
\end{enumerate}
 \\ \hline
\end{longtable}


\input aprovacao.tex


\end{document}
