\documentclass[12pt]{article}
\usepackage[brazil]{babel}
\usepackage{graphicx,t1enc,wrapfig,amsmath,float}
\usepackage{framed,fancyhdr}
\usepackage{multirow}
\usepackage{longtable}
\usepackage{array}
\newcolumntype{L}[1]{>{\raggedright\let\newline\\\arraybackslash\hspace{0pt}}m{#1}}
\newcolumntype{C}[1]{>{\centering\let\newline\\\arraybackslash\hspace{0pt}}m{#1}}
\newcolumntype{R}[1]{>{\raggedleft\let\newline\\\arraybackslash\hspace{0pt}}m{#1}}
%%%%%%%%%%%%%
\oddsidemargin -0.5cm
\evensidemargin -0.5cm
\textwidth 17.5cm
\topmargin -1.5cm
\textheight 22cm
%%%%%%%%%%%% 

%\pagestyle{empty}

\newcommand{\semestre}{2018.2}

\newcommand{\disciplina}{TEORIA GERAL DE SISTEMAS}
\newcommand{\codigo}{DEC7535}


%%%%%%%%%%%%%%%%%%%%%%%%%%%%%%%%%%%%%%%%%%%%%%%%%%%%%%%
%%%%%%%%%%%%% CRETIDOS
\newcommand{\creditosT}{4}
\newcommand{\creditosP}{0}

%%%%%%%%%%%%%%%%%%%%%%%%%%%%%%%%%%%%%%%%%%%%%%%%%%%%%%%
%%%%%%%%%%%%%% REQUISITOS
\newcommand{\requisitoA}{}
\newcommand{\requisitoB}{}
\newcommand{\requisitoC}{}

%%%%%%%%%%%%%%%%%%%%%%%%%%%%%%%%%%%%%%%%%%%%%%%%%%%%%%%
%%%%%%%%%%%%%%% Atende aos Cursos
\newcommand{\cursoA}{Graduação em Engenharia de Computação \\ \hline}
\newcommand{\cursoB}{}%Graduação em Tecnologias da Informação e Comunicação \\ \hline}
\newcommand{\cursoC}{}%Graduação em Engenharia de Energia \\ \hline}

%%%%%%%%%%%%%%%%%%%%%%%%%%%%%%%%%%%%%%%%%%%%%%%%%%%%%%%%
%%%%%%%%%% EMENTA
\newcommand{\ementa}{
A origem e o conceito da Teoria Geral de Sistemas. O conceito de sistema. Componentes genéricos de um sistema. As relações entre sistema e ambiente. Hierarquia de sistemas. Classificações dos sistemas. Enfoque sistêmico. O pensamento sistêmico aplicado na resolução de problemas. O pensamento sistêmico aplicado às organizações. Modelagem de Sistemas.
 \\ \hline
}


\begin{document}


%%%%%%%%%%%%%%%%%%%%%%%%%%%%%%%%%%%%%%%%%%%%%%%%%%%%%%%%%%%%%
\input cabecalho.tex

%%%%%%%%%%%%%%%%%%%%%%%%%%%%%%%%%%%%%%%%%%%%%%%%%%%%%%%%%%%%%
\begin{longtable}{|C{0.11\textwidth}|C{0.29\textwidth}|C{0.09\textwidth}|C{0.09\textwidth}|C{0.15\textwidth}|C{0.158\textwidth}|} \hline
%
\multicolumn{6}{|l|}{{\bf I. IDENTIFICAÇÃO DA DISCIPLINA}} \\ \hline
%
\multirow{3}*{{\small CÓDIGO}} & \multirow{3}*{NOME DA DISCIPLINA} &\multicolumn{2}{c|}{{\small N$^\circ$ DE HORAS-AULA }} & {{\small TOTAL DE}} & \multirow{3}*{{\small MODALIDADE}} \\ 
%
& & \multicolumn{2}{c|}{\small SEMANAIS}  & {\small HORAS-AULA} & \\ \cline{3-4}
%
& & {\tiny TEÓRICAS} & {\tiny PRÁTICAS} & {\small SEMESTRAIS} & \\ \hline
% codigo da disciplina carga horaria: teorica - pratica e total
{\bf \small \codigo} & {\bf \small \disciplina } & {\bf \creditosT} & {\bf \creditosP} & {\bf 72} & Presencial\\ \hline
\end{longtable}


%%%%%%%%%%%%%%%%%%%%%%%%%%%%%%%%%%%%%%%%%%%%%%%%%%%%%%%%%%%%%%
\begin{longtable}{|C{0.12\textwidth}|L{0.736\textwidth}|C{0.12\textwidth}|} \hline
%
\multicolumn{3}{|l|}{{\bf II. PRÉ-REQUISITO(S)}} \\ \hline
%
CÓDIGO & NOME DA DISCIPLINA & CURSO \\ \hline	
%
\requisitoA
\requisitoB
\requisitoC
\end{longtable}


%%%%%%%%%%%%%%%%%%%%%%%%%%%%%%%%%%%%%%%%%%%%%%%%%%%%%%%%%%%%%%
\begin{longtable}{|L{1.025\textwidth}|} \hline
%
{\bf III. CURSO(S) PARA O(S) QUAL(IS) A DISCIPLINA É OFERECIDA } \\ \hline
%
\cursoA 
\cursoB
\cursoC

\end{longtable}

%%%%%%%%%%%%%%%%%%%%%%%%%%%%%%%%%%%%%%%%%%%%%%%%%%%%%%%%%%%%%%
\begin{longtable}{|L{1.025\textwidth}|} \hline
%
{\bf IV. EMENTA } \\ \hline
%
\ementa
\end{longtable}

%\newpage



%%%%%%%%%%%%%%%%%%%%%%%%%%%%%%%%%%%%%%%%%%%%%%%%%%%%%%%%%%%%%%%
\begin{longtable}{|L{1.025\textwidth}|} \hline
%
{\bf V. OBJETIVOS } \\ \hline
%
Objetivo Geral:\\
Esta disciplina contribui para o desenvolvimento de um raciocínio de análise e modelagem sistêmica de problemas, em contraposição ao modelo reducionista. Está baseada na metodologia e técnicas decorrentes do trabalho de Ludwig von Bertalanffy, um biologista considerado o pai da Teoria Geral de Sistemas.\\
\\
\\
Objetivos Específicos:
\begin{itemize}
\item Caracterizar o pensamento sistêmico;
\item  Conceituar a teoria geral de sistemas no âmbito da Engenharia de Computação;
\item  Aplicar a dinâmica dos sistemas e a sua modelagem na compreensão e na intervenção do homem com relação aos sistemas/organizações;
\item  Relacionar a TGS come outras áreas do conhecimento.
\end{itemize}
\\ \hline
\end{longtable}


%%%%%%%%%%%%%%%%%%%%%%%%%%%%%%%%%%%%%%%%%%%%%%%%%%%%%%%%%%%%%%%
\begin{longtable}{|L{1.025\textwidth}|} \hline
%
{\bf VI. CONTEÚDO PROGRAMÁTICO } \\ \hline
UNIDADE 1: Conceitos da Teoria Geral de Sistemas\\
 Apresentação da disciplina (ementa, bibliografia, metodologia e avaliações)\\
 Origem da Teoria Geral de Sistemas\\
 Abordagem clássica versus abordagem sistêmica\\
 Definições e visão geral de sistemas\\
 Classificações dos sistemas (hierárquico, emergente e teleólogos)\\
 Características dos sistemas.\\
 Holismo e mecanicismo\\
 Indução e dedução\\
\\
UNIDADE 2: . O conceito de sistema e os componentes genéricos de um sistema\\
 Conceito gerais de sistemas\\
 Componentes\\
 Sistemas abertos e fechados\\
 Objetivos e escopo\\
 Relações\\
 Entradas e saídas\\
 Limites\\
 Ambiente\\
 Hierarquia\\
 Entropia e Negentropia\\
 Isomorfismo e Homomorfismo\\
 Retroalimentação\\
 Sinergia\\
 Fragmentação\\
 Controle\\
 Homeostase\\
\\
UNIDADE 3: As relações entre sistema e ambiente. \\
Sistemas e aplicações nas diversas áreas. Hierarquia e classificações dos sistemas. O pensamento sistêmico aplicado na resolução de problemas.\\
\\
UNIDADE 4: Sistemas de Informação\\
 Conceito de Informação\\
 Conceitos, características e componentes\\
 Taxonomias dos sistemas de informação\\
 Relação entre a Teoria Geral de Sistemas e os Sistemas de Informação\\
\\
UNIDADE 5: Cibernética\\
 Cibernética\\
 Origens da Cibernética\\
 Definições para Cibernética\\
 Propriedades dos Sistemas Cibernéticos\\
\\
UNIDADE 6: Modelagem de Sistemas\\
 Noções básicas sobre modelagem de sistemas



\\ \hline
\end{longtable} 

%\newpage

%%%%%%%%%%%%%%%%%%%%%%%%%%%%%%%%%%%%%%%%%%%%%%%%%%%%%%%%%%%%%%%
\begin{longtable}{|L{1.025\textwidth}|} \hline
%
{\bf VII. BIBLIOGRAFIA BÁSICA} \\ \hline
\begin{enumerate}
%
\item BERTALANFFY, Ludwig von. Teoria geral dos sistemas: fundamentos, desenvolvimento e aplicações. 7.ed. Petropolis: Vozes, 2013. 360 p. ISBN 9788532636904.
\item ALVES, João Bosco da Mota. Teoria Geral de Sistemas. Florianópolis:Instituto Stela, 2012. 
\item O'BRIEN, James A. Sistema de informação e as decisões gerenciais na era da internet. 2. ed. São Paulo : Saraiva, 2004.
\end{enumerate}
 \\ \hline
\end{longtable}


%\newpage

%%%%%%%%%%%%%%%%%%%%%%%%%%%%%%%%%%%%%%%%%%%%%%%%%%%%%%%%%%%%%%%
\begin{longtable}{|L{1.025\textwidth}|} \hline
%
{\bf VIII. BIBLIOGRAFIA COMPLEMENTAR} \\ \hline
\begin{enumerate}
\item DAMASIO, Antonio R. O Erro de Descartes : emoção, razão e o cérebro humano. São Paulo: Companhia das letras, 1996. 
\item HOFFMAN, Donald D. Inteligência visual: como criamos o que vemos. Rio de Janeiro: Campus, 2001. 
\item BLILIE, Charles. The Promise and Limits of Computer Modeling. Singapore: World Scientific Publishing, 2007. 
\item VASCONCELLOS, Maria José E. Pensamento Sistêmico: o novo paradigma da Ciência. 2.ed. Campinas-SP: Papirus, 2002. 
\item FLAKE, Gary William. The computational beauty of nature: computer explorations of fractals, chaos, complex systems, and adaptation. Cambridge, MA: MIT Press, c1998 xviii, 493 p.

\end{enumerate}
 \\ \hline
\end{longtable}


\input aprovacao.tex


\end{document}
