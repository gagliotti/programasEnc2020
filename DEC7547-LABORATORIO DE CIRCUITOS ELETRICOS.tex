\documentclass[12pt]{article}
\usepackage[brazil]{babel}
\usepackage{graphicx,t1enc,wrapfig,amsmath,float}
\usepackage{framed,fancyhdr}
\usepackage{multirow}
\usepackage{longtable}
\usepackage{array}
\newcolumntype{L}[1]{>{\raggedright\let\newline\\\arraybackslash\hspace{0pt}}m{#1}}
\newcolumntype{C}[1]{>{\centering\let\newline\\\arraybackslash\hspace{0pt}}m{#1}}
\newcolumntype{R}[1]{>{\raggedleft\let\newline\\\arraybackslash\hspace{0pt}}m{#1}}
%%%%%%%%%%%%%
\oddsidemargin -0.5cm
\evensidemargin -0.5cm
\textwidth 17.5cm
\topmargin -1.5cm
\textheight 22cm
%%%%%%%%%%%% 

%\pagestyle{empty}

\newcommand{\semestre}{2018.2}

\newcommand{\disciplina}{LABORATÓRIO DE CIRCUITOS ELÉTRICOS}
\newcommand{\codigo}{DEC7547}


%%%%%%%%%%%%%%%%%%%%%%%%%%%%%%%%%%%%%%%%%%%%%%%%%%%%%%%
%%%%%%%%%%%%% CRETIDOS
\newcommand{\creditosT}{0}
\newcommand{\creditosP}{4}

%%%%%%%%%%%%%%%%%%%%%%%%%%%%%%%%%%%%%%%%%%%%%%%%%%%%%%%
%%%%%%%%%%%%%% REQUISITOS
\newcommand{\requisitoA}{EES7378 & Eletrônica de Potência & ENE \\ \hline}
\newcommand{\requisitoB}{}
\newcommand{\requisitoC}{}

%%%%%%%%%%%%%%%%%%%%%%%%%%%%%%%%%%%%%%%%%%%%%%%%%%%%%%%
%%%%%%%%%%%%%%% Atende aos Cursos
\newcommand{\cursoA}{Graduação em Engenharia de Computação \\ \hline}
\newcommand{\cursoB}{Graduação em Engenharia de Energia \\ \hline}
\newcommand{\cursoC}{}

%%%%%%%%%%%%%%%%%%%%%%%%%%%%%%%%%%%%%%%%%%%%%%%%%%%%%%%%
%%%%%%%%%% EMENTA
\newcommand{\ementa}{
Desenvolvimento de atividades práticas que permitam explorar os fundamentos, conceitos e técnicas relativas em circuitos elétricos e eletrônicos.
\\ \hline
}


\begin{document}

%%%%%%%%%%%%%%%%%%%%%%%%%%%%%%%%%%%%%%%%%%%%%%%%%%%%%%%%%%%%%

\input cabecalho.tex


%%%%%%%%%%%%%%%%%%%%%%%%%%%%%%%%%%%%%%%%%%%%%%%%%%%%%%%%%%%%%
\begin{longtable}{|C{0.11\textwidth}|C{0.29\textwidth}|C{0.09\textwidth}|C{0.09\textwidth}|C{0.15\textwidth}|C{0.158\textwidth}|} \hline
%
\multicolumn{6}{|l|}{{\bf I. IDENTIFICAÇÃO DA DISCIPLINA}} \\ \hline
%
\multirow{3}*{{\small CÓDIGO}} & \multirow{3}*{NOME DA DISCIPLINA} &\multicolumn{2}{c|}{{\small N$^\circ$ DE HORAS-AULA }} & {{\small TOTAL DE}} & \multirow{3}*{{\small MODALIDADE}} \\ 
%
& & \multicolumn{2}{c|}{\small SEMANAIS}  & {\small HORAS-AULA} & \\ \cline{3-4}
%
& & {\tiny TEÓRICAS} & {\tiny PRÁTICAS} & {\small SEMESTRAIS} & \\ \hline
% codigo da disciplina carga horaria: teorica - pratica e total
{\bf \small \codigo} & {\bf \small \disciplina } & {\bf \creditosT} & {\bf \creditosP} & {\bf 72} & Presencial\\ \hline
\end{longtable}


%%%%%%%%%%%%%%%%%%%%%%%%%%%%%%%%%%%%%%%%%%%%%%%%%%%%%%%%%%%%%%
\begin{longtable}{|C{0.12\textwidth}|L{0.736\textwidth}|C{0.12\textwidth}|} \hline
%
\multicolumn{3}{|l|}{{\bf II. PRÉ-REQUISITO(S)}} \\ \hline
%
CÓDIGO & NOME DA DISCIPLINA & CURSO \\ \hline	
%
\requisitoA
\requisitoB
\requisitoC
\end{longtable}


%%%%%%%%%%%%%%%%%%%%%%%%%%%%%%%%%%%%%%%%%%%%%%%%%%%%%%%%%%%%%%
\begin{longtable}{|L{1.025\textwidth}|} \hline
%
{\bf III. CURSO(S) PARA O(S) QUAL(IS) A DISCIPLINA É OFERECIDA } \\ \hline
%
\cursoA 
\cursoB
\cursoC

\end{longtable}

%%%%%%%%%%%%%%%%%%%%%%%%%%%%%%%%%%%%%%%%%%%%%%%%%%%%%%%%%%%%%%
\begin{longtable}{|L{1.025\textwidth}|} \hline
%
{\bf IV. EMENTA } \\ \hline
%
\ementa
\end{longtable}

\newpage



%%%%%%%%%%%%%%%%%%%%%%%%%%%%%%%%%%%%%%%%%%%%%%%%%%%%%%%%%%%%%%%
\begin{longtable}{|L{1.025\textwidth}|} \hline
%
{\bf V. OBJETIVOS } \\ \hline
Objetivos Gerais:\\
Esta disciplina deverá abordar aspectos práticos, em laboratório, de circuitos elétricos e eletrônicos.\\
\\
Objetivos Específicos:
\begin{itemize}
\item Introduzir conceitos básicos de circuitos elétricos;
\item Discutir o conceito de fontes ideais independentes e dependentes em redes resistivas;
\item Discutir o conceito de amplificador operacional ideal;
\item Discutir técnicas de análise e características de circuitos em corrente contínua;
\item Discutir técnicas de análise e características de circuitos de corrente alternada;
\item Discutir dispositivos eletrônicos como diodo, transistor de efeito de campo e junção bipolar.
\end{itemize}
\\ \hline
\end{longtable}


%%%%%%%%%%%%%%%%%%%%%%%%%%%%%%%%%%%%%%%%%%%%%%%%%%%%%%%%%%%%%%%
\begin{longtable}{|L{1.025\textwidth}|} \hline
%
{\bf VI. CONTEÚDO PROGRAMÁTICO } \\ \hline
Instrumentos de medição\\
Lei de Ohm e Circuitos em Série\\
Circuitos em paralelo e série/paralelo\\
Teoria de Redes: Equivalente de Thevenin\\
Circuito RC: análise DC\\
Osciloscópio\\
Circuito RL: análise AC\\
Circuitos RC: análise AC\\
Filtros RL e RC\\
Amplificador Operacional\\
Diodo\\
Transistor de Junção Bipolar\\
Transistor de Efeito de Campo\\
\\ \hline
\end{longtable} 

\newpage


%%%%%%%%%%%%%%%%%%%%%%%%%%%%%%%%%%%%%%%%%%%%%%%%%%%%%%%%%%%%%%%
\begin{longtable}{|L{1.025\textwidth}|} \hline
%
{\bf VII. BIBLIOGRAFIA BÁSICA} \\ \hline

\begin{enumerate}
\item NILSSON, James William; RIEDEL, Susan A. Circuitos elétricos. 6. ed Rio de Janeiro (RJ): LTC, c2003. 656p. 
\item ALEXANDER, CHARLES K.; SADIKU, MATTHEW - FUNDAMENTOS DE CIRCUITOS  ELETRICOS - MCGRAW HILL - ARTMED, 2008, ISBN: 8586804975, ISBN-13: 9788586804977 
\item SEDRA, Adel S.; SMITH, Kenneth C. Microeletrônica. 5. ed. São Paulo: Pearson Prentice Hall, 2007. xiv, 848 p. ISBN 9788576050223.
\end{enumerate}

 \\ \hline
\end{longtable}


%\newpage

%%%%%%%%%%%%%%%%%%%%%%%%%%%%%%%%%%%%%%%%%%%%%%%%%%%%%%%%%%%%%%%
\begin{longtable}{|L{1.025\textwidth}|} \hline
%
{\bf VIII. BIBLIOGRAFIA COMPLEMENTAR} \\ \hline
\begin{enumerate}
\item NAHVI, Mahmood; EDMINISTER, Joseph A. Teoria e problemas de circuitos elétricos. 4. ed. Porto Alegre: Bookman, 2005. 478 p. (Schaum). ISBN 9788536305516 (broch.).
\item JOHNSON, David E.; HILBURN, John L.; JOHNSON, Johnny Ray. Fundamentos de análise de circuitos elétricos. 4. ed. Rio de Janeiro: LTC, c1994. 539 p. ISBN 9788521612384.
\item RAZAVI, BEHZAD, - FUNDAMENTOS DE MICROELETRONICA - LTC, 2010, ISBN: 8521617321, ISBN-13: 9788521617327 
\item DORF, RICHARD; SVOBODA, JAMES A. - INTRODUÇAO AOS CIRCUITOS ELETRICOS - LTC, 2008, ISBN: 8521615825, ISBN-13: 9788521615828 
\item PEDRONI, Volnei A. Eletrônica Digital Moderna e VHDL: Princípios Digitais, Eletrônica Digital, Projeto Digital, Microeletrônica e VHDL. 1 ed. [S.l.]:Elsevier, 2010. 648 p. ISBN 978-8535234657.

%
\end{enumerate}
 \\ \hline
\end{longtable}


\input aprovacao.tex


\end{document}
