\documentclass[12pt]{article}
\usepackage[brazil]{babel}
\usepackage{graphicx,t1enc,wrapfig,amsmath,float}
\usepackage{framed,fancyhdr}
\usepackage{multirow}
\usepackage{longtable}
\usepackage{array}
\newcolumntype{L}[1]{>{\raggedright\let\newline\\\arraybackslash\hspace{0pt}}m{#1}}
\newcolumntype{C}[1]{>{\centering\let\newline\\\arraybackslash\hspace{0pt}}m{#1}}
\newcolumntype{R}[1]{>{\raggedleft\let\newline\\\arraybackslash\hspace{0pt}}m{#1}}
%%%%%%%%%%%%%
\oddsidemargin -0.5cm
\evensidemargin -0.5cm
\textwidth 17.5cm
\topmargin -1.5cm
\textheight 22cm
%%%%%%%%%%%% 

%\pagestyle{empty}

\newcommand{\semestre}{2018.2}

\newcommand{\disciplina}{REDES DE COMPUTADORES I}
\newcommand{\codigo}{DEC7126}


%%%%%%%%%%%%%%%%%%%%%%%%%%%%%%%%%%%%%%%%%%%%%%%%%%%%%%%
%%%%%%%%%%%%% CRETIDOS
\newcommand{\creditosT}{3}
\newcommand{\creditosP}{1}

%%%%%%%%%%%%%%%%%%%%%%%%%%%%%%%%%%%%%%%%%%%%%%%%%%%%%%%
%%%%%%%%%%%%%% REQUISITOS
\newcommand{\requisitoA}{DEC7121 & FUNDAMENTOS MATEMÁTICOS PARA COMPUTAÇÃO & TIC\\ \hline}
\newcommand{\requisitoB}{}
\newcommand{\requisitoC}{}

%%%%%%%%%%%%%%%%%%%%%%%%%%%%%%%%%%%%%%%%%%%%%%%%%%%%%%%
%%%%%%%%%%%%%%% Atende aos Cursos
\newcommand{\cursoB}{}%Graduação em Engenharia de Computação. \\ \hline}
\newcommand{\cursoA}{Graduação em Tecnologias da Informação e Comunicação \\ \hline}
\newcommand{\cursoC}{}

%%%%%%%%%%%%%%%%%%%%%%%%%%%%%%%%%%%%%%%%%%%%%%%%%%%%%%%%
%%%%%%%%%% EMENTA
\newcommand{\ementa}{
Redes de computadores e a Internet. Camada de aplicação. Camada de transporte. Camada de rede. A camada de enlace e redes locais.
 \\ \hline
}




\begin{document}


%%%%%%%%%%%%%%%%%%%%%%%%%%%%%%%%%%%%%%%%%%%%%%%%%%%%%%%%%%%%%
\input cabecalho.tex

%%%%%%%%%%%%%%%%%%%%%%%%%%%%%%%%%%%%%%%%%%%%%%%%%%%%%%%%%%%%%
\begin{longtable}{|C{0.11\textwidth}|C{0.29\textwidth}|C{0.09\textwidth}|C{0.09\textwidth}|C{0.15\textwidth}|C{0.158\textwidth}|} \hline
%
\multicolumn{6}{|l|}{{\bf I. IDENTIFICAÇÃO DA DISCIPLINA}} \\ \hline
%
\multirow{3}*{{\small CÓDIGO}} & \multirow{3}*{NOME DA DISCIPLINA} &\multicolumn{2}{c|}{{\small N$^\circ$ DE HORAS-AULA }} & {{\small TOTAL DE}} & \multirow{3}*{{\small MODALIDADE}} \\ 
%
& & \multicolumn{2}{c|}{\small SEMANAIS}  & {\small HORAS-AULA} & \\ \cline{3-4}
%
& & {\tiny TEÓRICAS} & {\tiny PRÁTICAS} & {\small SEMESTRAIS} & \\ \hline
% codigo da disciplina carga horaria: teorica - pratica e total
{\bf \small \codigo} & {\bf \small \disciplina } & {\bf \creditosT} & {\bf \creditosP} & {\bf 72} & Presencial\\ \hline
\end{longtable}


%%%%%%%%%%%%%%%%%%%%%%%%%%%%%%%%%%%%%%%%%%%%%%%%%%%%%%%%%%%%%%
\begin{longtable}{|C{0.12\textwidth}|L{0.736\textwidth}|C{0.12\textwidth}|} \hline
%
\multicolumn{3}{|l|}{{\bf II. PRÉ-REQUISITO(S)}} \\ \hline
%
CÓDIGO & NOME DA DISCIPLINA & CURSO \\ \hline	
%
\requisitoA
\requisitoB
\requisitoC
\end{longtable}


%%%%%%%%%%%%%%%%%%%%%%%%%%%%%%%%%%%%%%%%%%%%%%%%%%%%%%%%%%%%%%
\begin{longtable}{|L{1.025\textwidth}|} \hline
%
{\bf III. CURSO(S) PARA O(S) QUAL(IS) A DISCIPLINA É OFERECIDA } \\ \hline
%
\cursoA 
\cursoB
\cursoC

\end{longtable}

%%%%%%%%%%%%%%%%%%%%%%%%%%%%%%%%%%%%%%%%%%%%%%%%%%%%%%%%%%%%%%
\begin{longtable}{|L{1.025\textwidth}|} \hline
%
{\bf IV. EMENTA } \\ \hline
%
\ementa
\end{longtable}

%\newpage



%%%%%%%%%%%%%%%%%%%%%%%%%%%%%%%%%%%%%%%%%%%%%%%%%%%%%%%%%%%%%%%
\begin{longtable}{|L{1.025\textwidth}|} \hline
%
{\bf V. OBJETIVOS } \\ \hline
%
Objetivo Geral:\\

O principal objetivo é apresentar os principais conceitos relacionados às Arquiteturas, Serviços e Protocolos das Redes de Computadores. \\
\\
Objetivos Específicos:

\begin{itemize}
\item Apresentar um histórico, as características e as classes de Redes de Computadores;
\item Introduzir o conceito de Arquitetura Multicamadas e os princípios básicos de operação;
\item Descrever a organização da arquitetura e os conceitos associados ao Modelo de Referência  OSI e da arquitetura de protocolos TCP/IP;
\item Apresentar as noções básicas da arquitetura Internet e seus principais protocolos de comunicação;
\item Apresentar as principais técnicas associadas à transmissão de dados em meios de transmissão (modos de transmissão, técnicas de codificação, modulação, multiplexação etc);
\item Apresentar as características associadas aos Meios de Transmissão mais utilizados para transferência de dados em Redes de Computadores;
\item Introduzir os conceitos relativos às arquiteturas de Redes Locais de Computadores e os padrões associados.
\end{itemize}


\\ \hline
\end{longtable}


%%%%%%%%%%%%%%%%%%%%%%%%%%%%%%%%%%%%%%%%%%%%%%%%%%%%%%%%%%%%%%%
\begin{longtable}{|L{1.025\textwidth}|} \hline
%
{\bf VI. CONTEÚDO PROGRAMÁTICO } \\ \hline
Unidade 1: Introdução às Redes de Computadores [8ha]\\
Conceitos Gerais	\\
Medidas de Desempenho\\
Camadas de protocolos e serviços\\
Histórico das redes de computadores e Internet\\
Topologias de redes\\
\\
Unidade 2: Camada de Aplicação [12ha]\\
Fundamentos das aplicações de rede\\
Principais protocolos da camada de aplicação (HTTP, FTP, SMTP)\\
Serviço de diretório da Internet (DNS)\\
\\
Unidade 3: Camada de Transporte [20ha]\\
Introdução e Serviços da camada de transporte\\
Protocolos TCP e UDP\\
Princípios do controle de congestionamento\\
\\
Unidade 4: Camada de Rede [24ha]\\
Introdução\\
Endereçamento IP\\
O protocolo IP\\
Alocação dinâmica de IPs\\
Tradução e Mapeamento de IPs\
\\
Unidade 5: Camada de enlace e redes locais [8ha]\\
Serviços oferecidos pela camada de enlace\\
Protocolos de acesso múltiplo\\
Endereçamento na camada de enlace\\
Redes Ethernet
\\ \hline
\end{longtable} 





%%%%%%%%%%%%%%%%%%%%%%%%%%%%%%%%%%%%%%%%%%%%%%%%%%%%%%%%%%%%%%%
\begin{longtable}{|L{1.025\textwidth}|} \hline
%
{\bf VII. BIBLIOGRAFIA BÁSICA} \\ \hline
\begin{enumerate}
%
\item CARISSIMI, A. S.; ROCHOL, J.; GRANVILLE, L. Z. Redes de Computadores. Porto Alegre: Bookman, 2009.
\item KUROSE, James F; ROSS, Keith W. Redes de computadores e a Internet: uma abordagem topdow.
5. ed. São Paulo: Addison Wesley, 2010.
\item TANENBAUM, Andrew S. Redes de computadores. 4. ed. Rio de Janeiro: Elsevier, 2003.

%

\end{enumerate}
 \\ \hline
\end{longtable}


%\newpage

%%%%%%%%%%%%%%%%%%%%%%%%%%%%%%%%%%%%%%%%%%%%%%%%%%%%%%%%%%%%%%%
\begin{longtable}{|L{1.025\textwidth}|} \hline
%
{\bf VIII. BIBLIOGRAFIA COMPLEMENTAR} \\ \hline
\begin{enumerate}
\item COMER, Douglas. Interligação em rede com TCP/IP. Rio de Janeiro: Campus, 2006.
\item SOARES, Luiz Fernando Gomes; LEMOS, Guido; COLCHER, Sergio. Redes de Computadores: Das LANs, MANs e WANs, às Redes ATM. Rio de Janeiro: Editora Campus, 1995.
\item STALLINGS, W. Redes e Sistemas de Comunicação de Dados, Rio de Janeiro: Elsevier. 5. Edicao, 2005.
\item TORRES, Gabriel. Redes de Computadores. Rio de Janeiro: Nova Terra, 2009. 
\item FOROUZAN, Behrouz A.; FEGAN, Sophia Chung; GRIESI, Ariovaldo. Comunicação de dados e redes de computadores. 4. ed. São Paulo: McGraw Hill, 2008. 1134 p. ISBN 9788586804885
%
\end{enumerate}
 \\ \hline
\end{longtable}


\input aprovacao.tex


\end{document}
