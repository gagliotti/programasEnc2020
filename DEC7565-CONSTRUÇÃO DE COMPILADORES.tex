\documentclass[12pt]{article}
\usepackage[brazil]{babel}
\usepackage{graphicx,t1enc,wrapfig,amsmath,float}
\usepackage{framed,fancyhdr}
\usepackage{multirow}
\usepackage{longtable}
\usepackage{array}
\newcolumntype{L}[1]{>{\raggedright\let\newline\\\arraybackslash\hspace{0pt}}m{#1}}
\newcolumntype{C}[1]{>{\centering\let\newline\\\arraybackslash\hspace{0pt}}m{#1}}
\newcolumntype{R}[1]{>{\raggedleft\let\newline\\\arraybackslash\hspace{0pt}}m{#1}}
%%%%%%%%%%%%%
\oddsidemargin -0.5cm
\evensidemargin -0.5cm
\textwidth 17.5cm
\topmargin -1.5cm
\textheight 22cm
%%%%%%%%%%%% 

%\pagestyle{empty}

\newcommand{\semestre}{2018.2}

\newcommand{\disciplina}{CONSTRUÇÃO DE COMPILADORES}
\newcommand{\codigo}{DEC7565}


%%%%%%%%%%%%%%%%%%%%%%%%%%%%%%%%%%%%%%%%%%%%%%%%%%%%%%%
%%%%%%%%%%%%% CRETIDOS
\newcommand{\creditosT}{2}
\newcommand{\creditosP}{2}

%%%%%%%%%%%%%%%%%%%%%%%%%%%%%%%%%%%%%%%%%%%%%%%%%%%%%%%
%%%%%%%%%%%%%% REQUISITOS
\newcommand{\requisitoA}{}
\newcommand{\requisitoB}{}
\newcommand{\requisitoC}{}

%%%%%%%%%%%%%%%%%%%%%%%%%%%%%%%%%%%%%%%%%%%%%%%%%%%%%%%
%%%%%%%%%%%%%%% Atende aos Cursos
\newcommand{\cursoA}{Graduação em Engenharia de Computação \\ \hline}
\newcommand{\cursoB}{}%Graduação em Tecnologias da Informação e Comunicação \\ \hline}
\newcommand{\cursoC}{}

%%%%%%%%%%%%%%%%%%%%%%%%%%%%%%%%%%%%%%%%%%%%%%%%%%%%%%%%
%%%%%%%%%% EMENTA
\newcommand{\ementa}{
Linguagens-fonte, objeto, de alto-nível e de baixo-nível. Especificação de linguagens de programação. Compilação e interpretação. Processadores de linguagens de programação. Máquinas reais e virtuais. Bootstrapping. Análise sintática. Análise de contexto. Ambientes de execução. Geração de código. 90 Otimização de código independente de máquina. Otimização de código dependente de máquina.
\\ \hline
}


\begin{document}

%%%%%%%%%%%%%%%%%%%%%%%%%%%%%%%%%%%%%%%%%%%%%%%%%%%%%%%%%%%%%

\input cabecalho.tex


%%%%%%%%%%%%%%%%%%%%%%%%%%%%%%%%%%%%%%%%%%%%%%%%%%%%%%%%%%%%%
\begin{longtable}{|C{0.11\textwidth}|C{0.29\textwidth}|C{0.09\textwidth}|C{0.09\textwidth}|C{0.15\textwidth}|C{0.158\textwidth}|} \hline
%
\multicolumn{6}{|l|}{{\bf I. IDENTIFICAÇÃO DA DISCIPLINA}} \\ \hline
%
\multirow{3}*{{\small CÓDIGO}} & \multirow{3}*{NOME DA DISCIPLINA} &\multicolumn{2}{c|}{{\small N$^\circ$ DE HORAS-AULA }} & {{\small TOTAL DE}} & \multirow{3}*{{\small MODALIDADE}} \\ 
%
& & \multicolumn{2}{c|}{\small SEMANAIS}  & {\small HORAS-AULA} & \\ \cline{3-4}
%
& & {\tiny TEÓRICAS} & {\tiny PRÁTICAS} & {\small SEMESTRAIS} & \\ \hline
% codigo da disciplina carga horaria: teorica - pratica e total
{\bf \small \codigo} & {\bf \small \disciplina } & {\bf \creditosT} & {\bf \creditosP} & {\bf 72} & Presencial\\ \hline
\end{longtable}


%%%%%%%%%%%%%%%%%%%%%%%%%%%%%%%%%%%%%%%%%%%%%%%%%%%%%%%%%%%%%%
\begin{longtable}{|C{0.12\textwidth}|L{0.736\textwidth}|C{0.12\textwidth}|} \hline
%
\multicolumn{3}{|l|}{{\bf II. PRÉ-REQUISITO(S)}} \\ \hline
%
CÓDIGO & NOME DA DISCIPLINA & CURSO \\ \hline	
%
\requisitoA
\requisitoB
\requisitoC
\end{longtable}


%%%%%%%%%%%%%%%%%%%%%%%%%%%%%%%%%%%%%%%%%%%%%%%%%%%%%%%%%%%%%%
\begin{longtable}{|L{1.025\textwidth}|} \hline
%
{\bf III. CURSO(S) PARA O(S) QUAL(IS) A DISCIPLINA É OFERECIDA } \\ \hline
%
\cursoA 
\cursoB
\cursoC

\end{longtable}

%%%%%%%%%%%%%%%%%%%%%%%%%%%%%%%%%%%%%%%%%%%%%%%%%%%%%%%%%%%%%%
\begin{longtable}{|L{1.025\textwidth}|} \hline
%
{\bf IV. EMENTA } \\ \hline
%
\ementa
\end{longtable}

\newpage



%%%%%%%%%%%%%%%%%%%%%%%%%%%%%%%%%%%%%%%%%%%%%%%%%%%%%%%%%%%%%%%
\begin{longtable}{|L{1.025\textwidth}|} \hline
%
{\bf V. OBJETIVOS } \\ \hline
Objetivo Geral:\\

Capacitar o aluno na síntese, análise e manipulação de especificações de linguagens de programação de alto nível, assim como no emprego de técnicas de implementação de processadores de linguagens.\\
\\
Objetivos Específicos:
\begin{itemize}
\item Estudar e conhecer os princípios de um compilador;
\item Estudar o processo de análise léxica e semântica em um compilador;
\item Estudar o processo de geração de código intermediário e código objeto final;
\item Estudar o processo de otimização de código intermediário e código objeto final.
\end{itemize}
\\ \hline
\end{longtable}


%%%%%%%%%%%%%%%%%%%%%%%%%%%%%%%%%%%%%%%%%%%%%%%%%%%%%%%%%%%%%%%
\begin{longtable}{|L{1.025\textwidth}|} \hline
%
{\bf VI. CONTEÚDO PROGRAMÁTICO } \\ \hline
Conteúdo Teórico seguido de Conteúdo Prático com desenvolvimento de problemas em computador:\\
\\
Unidade 1: Introdução\\
Apresentação da disciplina (ementa, bibliografia, metodologia e avaliações)\\
Introdução aos compiladores\\
Fases de um compilador\\
\\
Unidade 2: Análise Léxica\\
Expressão regular\\
Reconhecedores\\
Autômatos finitos para análise léxica\\
\\
Unidade 3: Análise Sintática\\
Gramáticas livres de contexto\\
Análise sintática top-down e bottom-up\\
Conjuntos FIRST e FOLLOW\\
Analisador sintático LR\\
Reconhecedores\\
\\
Unidade 4: Análise Semântica\\
Atributos semânticos herdados e sintetizados\\
Esquemas S e L atributos\\
\\
Unidade 5: Geração e Otimização de Código\\
Geração de código intermediário\\
Otimização de código intermediário\\
Geração de código objeto\\
Otimização de código objeto\\
\\ \hline
\end{longtable} 

%\newpage


%%%%%%%%%%%%%%%%%%%%%%%%%%%%%%%%%%%%%%%%%%%%%%%%%%%%%%%%%%%%%%%
\begin{longtable}{|L{1.025\textwidth}|} \hline
%
{\bf VII. BIBLIOGRAFIA BÁSICA} \\ \hline

\begin{enumerate}
\item AHO, Alfred V. et al. Compiladores: Princípios, Técnicas e Ferramentas. 2a ed. São Paulo: Pearson Addison Wesley, 2008. 
\item PRICE, Ana Maria de A.; TOSCANI, Simão S. Implementação de Linguagens de Programação: Compiladores. 3a ed. Porto Alegre. Bookman, 2008. 
\item ASCENCIO, Ana Fernanda Gomes; CAMPOS, Edilene Aparecida Veneruchi de. Fundamentos da programação de computadores: algoritmos, Pascal, C/C++ e Java. 2. ed. São Paulo: Pearson Prentice Hall, 2008.
\end{enumerate}

 \\ \hline
\end{longtable}


%\newpage

%%%%%%%%%%%%%%%%%%%%%%%%%%%%%%%%%%%%%%%%%%%%%%%%%%%%%%%%%%%%%%%
\begin{longtable}{|L{1.025\textwidth}|} \hline
%
{\bf VIII. BIBLIOGRAFIA COMPLEMENTAR} \\ \hline
\begin{enumerate}
\item SEBESTA, Robert W. Conceitos de linguagens de programação. 9. ed. Porto Alegre: Bookman, 2011. 
\item  PATTERSON, David A.; HENNESSY, John L. Organização e projeto de computadores: a interface hardware/sofware. 3. ed. Rio de Janeiro: Elsevier, 2005. 
\item  SANTOS, Rafael. Introdução à programação orientada a objetos usando JAVA. Rio de Janeiro: Campus, 2003
%
\item MANZANO, José Augusto N. G; OLIVEIRA, Jayr Figueiredo de. Algoritmos: lógica para desenvolvimento de programação de computadores. 27. ed. rev. São Paulo: Érica, 2014. 328 p. ISBN 9788536502212.
\item MEDINA, Marco; FERTING, Cristina. Algoritmos e programação: teoria e prática. 2.ed. São Paulo: Novatec, 2006. 384 p. ISBN 857522073X (broch.).
\end{enumerate}
 \\ \hline
\end{longtable}


\input aprovacao.tex


\end{document}
