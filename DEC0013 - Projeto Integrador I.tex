\documentclass[12pt]{article}
\usepackage[brazil]{babel}
\usepackage{graphicx,t1enc,wrapfig,amsmath,float}
\usepackage{framed,fancyhdr}
\usepackage{multirow}
\usepackage{longtable}
\usepackage{array}
\usepackage{colortbl}
\usepackage[table]{xcolor}
\newcolumntype{L}[1]{>{\raggedright\let\newline\\\arraybackslash\hspace{0pt}}m{#1}}
\newcolumntype{C}[1]{>{\centering\let\newline\\\arraybackslash\hspace{0pt}}m{#1}}
\newcolumntype{R}[1]{>{\raggedleft\let\newline\\\arraybackslash\hspace{0pt}}m{#1}}
%%%%%%%%%%%%%
\oddsidemargin -0.5cm
\evensidemargin -0.5cm
\textwidth 17.5cm
\topmargin -1.5cm
\textheight 22cm
%%%%%%%%%%%% 

%\pagestyle{empty}

\newcommand{\semestre}{2021.2}

\newcommand{\disciplina}{PROJETO INTEGRADOR I}
\newcommand{\codigo}{DEC0013}


%%%%%%%%%%%%%%%%%%%%%%%%%%%%%%%%%%%%%%%%%%%%%%%%%%%%%%%
%%%%%%%%%%%%% CRETIDOS
\newcommand{\creditosT}{0}
\newcommand{\creditosP}{2}

%%%%%%%%%%%%%%%%%%%%%%%%%%%%%%%%%%%%%%%%%%%%%%%%%%%%%%%
%%%%%%%%%%%%%% REQUISITOS
\newcommand{\requisitoA}{}
\newcommand{\requisitoB}{}
\newcommand{\requisitoC}{}

%%%%%%%%%%%%%%%%%%%%%%%%%%%%%%%%%%%%%%%%%%%%%%%%%%%%%%%
%%%%%%%%%%%%%%% Atende aos Cursos
\newcommand{\cursoA}{Graduação em Engenharia de Computação. \\ \hline}
\newcommand{\cursoB}{Graduação em Tecnologias da Informação e Comunicação \\ \hline}
\newcommand{\cursoC}{}

%%%%%%%%%%%%%%%%%%%%%%%%%%%%%%%%%%%%%%%%%%%%%%%%%%%%%%%%
%%%%%%%%%% EMENTA
\newcommand{\ementa}{
Serão atividades ligadas a conteúdos de outras disciplinas da matriz curricular, nas quais os estudantes vivenciam situações de aprendizagem diferenciadas e ações que contribuam para o
desenvolvimento de práticas de temas que já foram apresentados em sala de aula. O tema do
Projeto Integrador I será determinado pelos órgãos colegiados do curso (NDE e Colegiado do
Curso) anualmente com previsão para os dois semestres subsequentes, de forma que se possam
ser organizadas com antecedência o conjunto de atividades que serão necessárias para a
avaliação dos projetos pelo professor supervisor da disciplina.
 \\ \hline
}




\begin{document}


%%%%%%%%%%%%%%%%%%%%%%%%%%%%%%%%%%%%%%%%%%%%%%%%%%%%%%%%%%%%%
\begin{longtable}{|C{0.2\textwidth}|C{0.8\textwidth}|} \hline
%
\multirow{6}*{\includegraphics[scale=0.5]{UFSC-foto.jpg}} &\\
%
&{\bf UNIVERSIDADE FEDERAL DE SANTA CATARINA}\hfill\\
%
&{\bf Centro de Ciências, Tecnologias e Saúde} \\
%
&{\bf Departamento de Computação}\\
%
&{\bf PROGRAMA DE ENSINO}\\
%
& \\ \hline

%\multicolumn{2}{|c|}{{\bf SEMESTRE \semestre}}\\ \hline
\end{longtable}


%%%%%%%%%%%%%%%%%%%%%%%%%%%%%%%%%%%%%%%%%%%%%%%%%%%%%%%%%%%%%
\begin{longtable}{|C{0.11\textwidth}|C{0.29\textwidth}|C{0.09\textwidth}|C{0.09\textwidth}|C{0.15\textwidth}|C{0.158\textwidth}|} \hline
%
\multicolumn{6}{|l|}{{\bf I. IDENTIFICAÇÃO DA DISCIPLINA}} \\ \hline
%
\multirow{3}*{{\small CÓDIGO}} & \multirow{3}*{NOME DA DISCIPLINA} &\multicolumn{2}{c|}{{\small N$^\circ$ DE HORAS-AULA }} & {{\small TOTAL DE}} & \multirow{3}*{{\small MODALIDADE}} \\ 
%
& & \multicolumn{2}{c|}{\small SEMANAIS}  & {\small HORAS-AULA} & \\ \cline{3-4}
%
& & {\tiny TEÓRICAS} & {\tiny PRÁTICAS} & {\small SEMESTRAIS} & \\ \hline
% codigo da disciplina carga horaria: teorica - pratica e total
{\bf \small \codigo} & {\bf \small \disciplina } & {\bf \creditosT} & {\bf \creditosP} & {\bf 36} & Presencial\\ \hline
\end{longtable}


%%%%%%%%%%%%%%%%%%%%%%%%%%%%%%%%%%%%%%%%%%%%%%%%%%%%%%%%%%%%%%
\begin{longtable}{|L{1.025\textwidth}|} \hline
%
{\bf II. PRÉ-REQUISITO(S) } \\ \hline
%
O aluno precisa ter concluído no mínimo 60 créditos. \\ \hline

\end{longtable}


%%%%%%%%%%%%%%%%%%%%%%%%%%%%%%%%%%%%%%%%%%%%%%%%%%%%%%%%%%%%%%
\begin{longtable}{|L{1.025\textwidth}|} \hline
%
{\bf III. CURSO(S) PARA O(S) QUAL(IS) A DISCIPLINA É OFERECIDA } \\ \hline
%
\cursoA 
\cursoB
\cursoC

\end{longtable}

%%%%%%%%%%%%%%%%%%%%%%%%%%%%%%%%%%%%%%%%%%%%%%%%%%%%%%%%%%%%%%
\begin{longtable}{|L{1.025\textwidth}|} \hline
%
{\bf IV. EMENTA } \\ \hline
%
\ementa
\end{longtable}

\newpage



%%%%%%%%%%%%%%%%%%%%%%%%%%%%%%%%%%%%%%%%%%%%%%%%%%%%%%%%%%%%%%%
\begin{longtable}{|L{1.025\textwidth}|} \hline
%
{\bf V. OBJETIVOS } \\ \hline
%
O objetivo do Projeto Integrador I é estimular o aluno através de atividades práticas e desafiadoras, buscando evitar a evasão de alunos no início do curso. Além disso, o Projeto Integrador I deverá articular as competências do perfil profissional do curso desenvolvidas pelas disciplinas cursadas até a terceira/quarta fase.

Objetivos Específicos: 
\begin{itemize}
    \item Integrar o conteúdo das componentes curriculares dos semestres anteriores;
    \item Capacitar o aluno a desenvolver projetos e soluções para problemas;
    \item Desenvolver as habilidades do aluno com o manuseio e aplicação de ferramentas, instrumentos de medidas e equipamentos de laboratório;
    \item Incentivar o trabalho em grupo;
    \item Desenvolver habilidades de apresentação em público;
    \end{itemize}

\\ \hline
\end{longtable}


%%%%%%%%%%%%%%%%%%%%%%%%%%%%%%%%%%%%%%%%%%%%%%%%%%%%%%%%%%%%%%%
\begin{longtable}{|L{1.025\textwidth}|} \hline
%
{\bf VI. CONTEÚDO PROGRAMÁTICO } \\ \hline

{\bf UNIDADE 1:} Apresentação da disciplina, plano de ensino, divisão em grupos de trabalho. \\



{\bf UNIDADE 2:}  Metodologia para elaboração do projeto: Definição do problema a ser resolvido e seu escopo e relevância. Identificação dos atores envolvidos. Estudo da literatura relacionada. Vantagens/desvantagens da solução proposta. Formas de avaliação da solução. Capacidade de escala da solução. Viabilidade física/financeira do projeto. Modelagem do projeto. Definição de cronograma para desenvolvimento do projeto.  Documentação inicial do projeto,  materiais/ferramentas utilizadas. \\



{\bf UNIDADE 3:}   Execução do projeto: particionamento do projeto em tarefas. Criação de um MPV (Mínimo Produto Viável). Acompanhamento das equipes e apresentação pública das etapas de desenvolvimento, Avaliação e testes parciais. Relatórios parciais de execução.\\



{\bf UNIDADE 4:} Documentação final do projeto: relatório acadêmico/científico, apresentação pública, demonstração experimental.\\



\\ \hline
\end{longtable} 





%%%%%%%%%%%%%%%%%%%%%%%%%%%%%%%%%%%%%%%%%%%%%%%%%%%%%%%%%%%%%%%
\begin{longtable}{|L{1.025\textwidth}|} \hline
%
{\bf VII. BIBLIOGRAFIA BÁSICA} \\ \hline
1. SCHWABER, Ken. Agile project management with Scrum. Redmond: Microsoft Press, 2004. xix, 163 p. ISBN 9780735619937.

2. XAVIER, Carlos Magno da Silva. Gerenciamento de projetos: como definir e controlar o escopo do projeto. 2. ed. São Paulo: Saraiva, 2009. 259 p. ISBN 9788502061958.

3. THIEL, Peter; MASTERS, Blake. Zero to One: Notes on Startups, or How to Build the Future, Currency, 2014. 224 p. ISBN-10 : 9780804139298


 \\ \hline
\end{longtable}


%\newpage

%%%%%%%%%%%%%%%%%%%%%%%%%%%%%%%%%%%%%%%%%%%%%%%%%%%%%%%%%%%%%%%
\begin{longtable}{|L{1.025\textwidth}|} \hline
%
{\bf VIII. BIBLIOGRAFIA COMPLEMENTAR} \\ \hline

 4. RIES, Eric. The Lean Startup: How Today's Entrepreneurs Use Continuous Innovation to Create Radically Successful Businesses, Currency, 2011,  336 p. ISBN-10 : 9780307887894
 
 5. Daychouw, Merhi. 40 ferramentas e técnicas de gerenciamento. Ed. Basport, 2018. ISBN: 9788574528687
 
 6. BROWN, Tim.  Design Thinking, Alta Books, 2017 , 272 p. ISBN-10 : 8550801348
 
 7. Reis, Dálcio Roberto. GESTÃO DA INOVAÇÃO TECNOLÓGICA. Saraiva, 2005.Ed. Manole. ISBN: 9788520426784

 8. KEELING, Ralph. GESTÃO DE PROJETOS – UMA ABORDAGEM GLOBAL. São Paulo: Saraiva, 2009. ISBN: 9788553131631






 \\ \hline
\end{longtable}


\input aprovacao.tex


\end{document}
