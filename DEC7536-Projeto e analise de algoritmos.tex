\documentclass[12pt]{article}
\usepackage[brazil]{babel}
\usepackage{graphicx,t1enc,wrapfig,amsmath,float}
\usepackage{framed,fancyhdr}
\usepackage{multirow}
\usepackage{longtable}
\usepackage{array}
\newcolumntype{L}[1]{>{\raggedright\let\newline\\\arraybackslash\hspace{0pt}}m{#1}}
\newcolumntype{C}[1]{>{\centering\let\newline\\\arraybackslash\hspace{0pt}}m{#1}}
\newcolumntype{R}[1]{>{\raggedleft\let\newline\\\arraybackslash\hspace{0pt}}m{#1}}
%%%%%%%%%%%%%
\oddsidemargin -0.5cm
\evensidemargin -0.5cm
\textwidth 17.5cm
\topmargin -1.5cm
\textheight 22cm
%%%%%%%%%%%% 

%\pagestyle{empty}

\newcommand{\semestre}{2018.2}

\newcommand{\disciplina}{PROJETO E ANÁLISE DE ALGORITMOS}
\newcommand{\codigo}{DEC7536}


%%%%%%%%%%%%%%%%%%%%%%%%%%%%%%%%%%%%%%%%%%%%%%%%%%%%%%%
%%%%%%%%%%%%% CRETIDOS
\newcommand{\creditosT}{2}
\newcommand{\creditosP}{2}

%%%%%%%%%%%%%%%%%%%%%%%%%%%%%%%%%%%%%%%%%%%%%%%%%%%%%%%
%%%%%%%%%%%%%% REQUISITOS
\newcommand{\requisitoA}{}
\newcommand{\requisitoB}{}
\newcommand{\requisitoC}{}

%%%%%%%%%%%%%%%%%%%%%%%%%%%%%%%%%%%%%%%%%%%%%%%%%%%%%%%
%%%%%%%%%%%%%%% Atende aos Cursos
\newcommand{\cursoA}{Graduação em Engenharia de Computação \\ \hline}
\newcommand{\cursoB}{}%Graduação em Tecnologias da Informação e Comunicação \\ \hline}
\newcommand{\cursoC}{}%Graduação em Engenharia de Energia \\ \hline}

%%%%%%%%%%%%%%%%%%%%%%%%%%%%%%%%%%%%%%%%%%%%%%%%%%%%%%%%
%%%%%%%%%% EMENTA
\newcommand{\ementa}{
Técnicas de análise de algoritmos identificando classes de problemas e soluções eficientes. Algoritmos clássicos. Complexidade de Algoritmos. Projetos de algoritmos. Estruturas de dados Avançadas. Teoria dos Grafos: conceitos básicos, algorítmos e aplicações.
 \\ \hline
}


\begin{document}


%%%%%%%%%%%%%%%%%%%%%%%%%%%%%%%%%%%%%%%%%%%%%%%%%%%%%%%%%%%%%
\input cabecalho.tex

%%%%%%%%%%%%%%%%%%%%%%%%%%%%%%%%%%%%%%%%%%%%%%%%%%%%%%%%%%%%%
\begin{longtable}{|C{0.11\textwidth}|C{0.29\textwidth}|C{0.09\textwidth}|C{0.09\textwidth}|C{0.15\textwidth}|C{0.158\textwidth}|} \hline
%
\multicolumn{6}{|l|}{{\bf I. IDENTIFICAÇÃO DA DISCIPLINA}} \\ \hline
%
\multirow{3}*{{\small CÓDIGO}} & \multirow{3}*{NOME DA DISCIPLINA} &\multicolumn{2}{c|}{{\small N$^\circ$ DE HORAS-AULA }} & {{\small TOTAL DE}} & \multirow{3}*{{\small MODALIDADE}} \\ 
%
& & \multicolumn{2}{c|}{\small SEMANAIS}  & {\small HORAS-AULA} & \\ \cline{3-4}
%
& & {\tiny TEÓRICAS} & {\tiny PRÁTICAS} & {\small SEMESTRAIS} & \\ \hline
% codigo da disciplina carga horaria: teorica - pratica e total
{\bf \small \codigo} & {\bf \small \disciplina } & {\bf \creditosT} & {\bf \creditosP} & {\bf 72} & Presencial\\ \hline
\end{longtable}


%%%%%%%%%%%%%%%%%%%%%%%%%%%%%%%%%%%%%%%%%%%%%%%%%%%%%%%%%%%%%%
\begin{longtable}{|C{0.12\textwidth}|L{0.736\textwidth}|C{0.12\textwidth}|} \hline
%
\multicolumn{3}{|l|}{{\bf II. PRÉ-REQUISITO(S)}} \\ \hline
%
CÓDIGO & NOME DA DISCIPLINA & CURSO \\ \hline	
%
\requisitoA
\requisitoB
\requisitoC
\end{longtable}


%%%%%%%%%%%%%%%%%%%%%%%%%%%%%%%%%%%%%%%%%%%%%%%%%%%%%%%%%%%%%%
\begin{longtable}{|L{1.025\textwidth}|} \hline
%
{\bf III. CURSO(S) PARA O(S) QUAL(IS) A DISCIPLINA É OFERECIDA } \\ \hline
%
\cursoA 
\cursoB
\cursoC

\end{longtable}

%%%%%%%%%%%%%%%%%%%%%%%%%%%%%%%%%%%%%%%%%%%%%%%%%%%%%%%%%%%%%%
\begin{longtable}{|L{1.025\textwidth}|} \hline
%
{\bf IV. EMENTA } \\ \hline
%
\ementa
\end{longtable}

%\newpage



%%%%%%%%%%%%%%%%%%%%%%%%%%%%%%%%%%%%%%%%%%%%%%%%%%%%%%%%%%%%%%%
\begin{longtable}{|L{1.025\textwidth}|} \hline
%
{\bf V. OBJETIVOS } \\ \hline
%
Preparar os alunos para desenvolver, projetar e analisar algoritmos sobre aplicações reais das mais variadas áreas da Computação. 
\\ \hline
\end{longtable}


%%%%%%%%%%%%%%%%%%%%%%%%%%%%%%%%%%%%%%%%%%%%%%%%%%%%%%%%%%%%%%%
\begin{longtable}{|L{1.025\textwidth}|} \hline
%
{\bf VI. CONTEÚDO PROGRAMÁTICO } \\ \hline

UNIDADE 1. Técnicas de análise de algoritmos: Modelo de computação, notação assintótica, invariantes, recorrência.\\
\\
UNIDADE 2. Algoritmos clássicos, e projetos e análise de algoritmos: \\
Algoritmos sobre números (Fibonacci, Algoritmo de Euclides, Multiplicação de Números Grandes, Multiplicação de Matrizes)\\
Algoritmos de ordenação (Inserção, HeapSort, MergeSort, QuickSort) \\
Divisão e conquista\\
Programação dinâmica\\
Algoritmos Gulosos\\
\\
UNIDADE 3. Estruturas de dados avançadas:\\
Árvores AVL\\
Estruturas de dados para conjuntos disjuntos\\
\\

UNIDADE 4. Teoria dos Grafos: conceitos básicos, algoritmos e aplicações\\
\\
UNIDADE 5. Complexidade de algoritmos: Classes P e NP\\



\\ \hline
\end{longtable} 

%\newpage

%%%%%%%%%%%%%%%%%%%%%%%%%%%%%%%%%%%%%%%%%%%%%%%%%%%%%%%%%%%%%%%
\begin{longtable}{|L{1.025\textwidth}|} \hline
%
{\bf VII. BIBLIOGRAFIA BÁSICA} \\ \hline
\begin{enumerate}
%
\item CORMEN, Thomas H. et al. Introduction to algorithms. 3rd ed. Cambridge: MIT Press; 2009. xix 1292 p. 
\item SEDGEWICK, Robert; WAYNE, Kevin Daniel. Algorithms. 4th ed. Upper Saddle River: Addison Wesley, c2011. xii, 955 p. 
\item ZIVIANI, Nivio. Projeto de algoritmos: com implementações em Pascal e C. 3. ed. rev. e ampl. São Paulo: Cengage Learning, c2011. xx, 639 p.
\end{enumerate}
 \\ \hline
\end{longtable}

%\newpage

%%%%%%%%%%%%%%%%%%%%%%%%%%%%%%%%%%%%%%%%%%%%%%%%%%%%%%%%%%%%%%%
\begin{longtable}{|L{1.025\textwidth}|} \hline
%
{\bf VIII. BIBLIOGRAFIA COMPLEMENTAR} \\ \hline
\begin{enumerate}
\item KNUTH, Donald E. The art of computer programming. 3rd ed. Reading: Addison Wesley, [c1997- c1998]. 4 v. 
\item ARAÚJO, Everton Coimbra de. Algoritmos: fundamento e prática. 3. ed. ampl. e atual. Florianópolis: Visual Books, 2007. 414 p. 
\item GUIMARÃES, Angelo de Moura; LAGES, Newton Alberto de Castilho. Algoritmos e estruturas de dados. Rio de Janeiro: Livros Técnicos e Científicos, c1994. (28.a tiragem) 216p.
\item FEOFILOFF, Paulo. Algoritmos em linguagem C. Rio de Janeiro: Elsevier, c2009. xv, 208 p.
\item SZWARCFITER, Jayme Luiz; MARKENZON, Lilian. Estruturas de dados e seus algoritmos. 3. ed. Rio de Janeiro; LTC, 2010. xv, 302 p.
\end{enumerate}
 \\ \hline
\end{longtable}


\input aprovacao.tex


\end{document}
