\documentclass[12pt]{article}
\usepackage[brazil]{babel}
\usepackage{graphicx,t1enc,wrapfig,amsmath,float}
\usepackage{framed,fancyhdr}
\usepackage{multirow}
\usepackage{longtable}
\usepackage{array}
\newcolumntype{L}[1]{>{\raggedright\let\newline\\\arraybackslash\hspace{0pt}}m{#1}}
\newcolumntype{C}[1]{>{\centering\let\newline\\\arraybackslash\hspace{0pt}}m{#1}}
\newcolumntype{R}[1]{>{\raggedleft\let\newline\\\arraybackslash\hspace{0pt}}m{#1}}
%%%%%%%%%%%%%
\oddsidemargin -0.5cm
\evensidemargin -0.5cm
\textwidth 17.5cm
\topmargin -1.5cm
\textheight 22cm
%%%%%%%%%%%% 

%\pagestyle{empty}

\newcommand{\semestre}{2018.2}

\newcommand{\disciplina}{BANCO DE DADOS}
\newcommand{\codigo}{DEC7129}


%%%%%%%%%%%%%%%%%%%%%%%%%%%%%%%%%%%%%%%%%%%%%%%%%%%%%%%
%%%%%%%%%%%%% CRETIDOS
\newcommand{\creditosT}{2}
\newcommand{\creditosP}{2}

%%%%%%%%%%%%%%%%%%%%%%%%%%%%%%%%%%%%%%%%%%%%%%%%%%%%%%%
%%%%%%%%%%%%%% REQUISITOS
\newcommand{\requisitoA}{DEC0006 & ESTRUTURA DE DADOS & ENC\\ \hline}
\newcommand{\requisitoB}{}
\newcommand{\requisitoC}{}

%%%%%%%%%%%%%%%%%%%%%%%%%%%%%%%%%%%%%%%%%%%%%%%%%%%%%%%
%%%%%%%%%%%%%%% Atende aos Cursos
\newcommand{\cursoB}{Graduação em Engenharia de Computação - ENC \\ \hline}
\newcommand{\cursoC}{}

%%%%%%%%%%%%%%%%%%%%%%%%%%%%%%%%%%%%%%%%%%%%%%%%%%%%%%%%
%%%%%%%%%% EMENTA
\newcommand{\ementa}{
Introdução aos sistemas de gerência de bancos de dados. Projeto de banco de dados: conceitual, lógico e físico. Projeto conceitual de dados: entidades, relacionamentos, atributos, generalização e especialização. Projeto lógico para o modelo de dados relacional. Dependências funcionais e normalização. Linguagens de definição e de manipulação de dados. Restrições de integridade. Visões. Tópicos avançados de banco de dados. Desenvolvimento de aplicação de banco de dados.

 \\ \hline
}




\begin{document}


%%%%%%%%%%%%%%%%%%%%%%%%%%%%%%%%%%%%%%%%%%%%%%%%%%%%%%%%%%%%%
\input cabecalho.tex

%%%%%%%%%%%%%%%%%%%%%%%%%%%%%%%%%%%%%%%%%%%%%%%%%%%%%%%%%%%%%
\begin{longtable}{|C{0.11\textwidth}|C{0.29\textwidth}|C{0.09\textwidth}|C{0.09\textwidth}|C{0.15\textwidth}|C{0.158\textwidth}|} \hline
%
\multicolumn{6}{|l|}{{\bf I. IDENTIFICAÇÃO DA DISCIPLINA}} \\ \hline
%
\multirow{3}*{{\small CÓDIGO}} & \multirow{3}*{NOME DA DISCIPLINA} &\multicolumn{2}{c|}{{\small N$^\circ$ DE HORAS-AULA }} & {{\small TOTAL DE}} & \multirow{3}*{{\small MODALIDADE}} \\ 
%
& & \multicolumn{2}{c|}{\small SEMANAIS}  & {\small HORAS-AULA} & \\ \cline{3-4}
%
& & {\tiny TEÓRICAS} & {\tiny PRÁTICAS} & {\small SEMESTRAIS} & \\ \hline

% codigo da disciplina carga horaria: teorica - pratica e total
{\bf \small \codigo } & {\bf \small \disciplina } & {\bf \creditosT} & {\bf \creditosP} & {\bf 72} & Presencial\\ \hline
\end{longtable}


%%%%%%%%%%%%%%%%%%%%%%%%%%%%%%%%%%%%%%%%%%%%%%%%%%%%%%%%%%%%%%
\begin{longtable}{|C{0.12\textwidth}|L{0.736\textwidth}|C{0.12\textwidth}|} \hline
%
\multicolumn{3}{|l|}{{\bf II. PRÉ-REQUISITO(S) }}\\ \hline
%
CÓDIGO & NOME DA DISCIPLINA & CURSO \\ \hline	
%
\requisitoA
\requisitoB
\requisitoC
\end{longtable}





%%%%%%%%%%%%%%%%%%%%%%%%%%%%%%%%%%%%%%%%%%%%%%%%%%%%%%%%%%%%%%
\begin{longtable}{|L{1.025\textwidth}|} \hline
%
{\bf III. CURSO(S) PARA O(S) QUAL(IS) A DISCIPLINA É OFERECIDA } \\ \hline
%
\cursoA 
\cursoB
\cursoC

\end{longtable}

%%%%%%%%%%%%%%%%%%%%%%%%%%%%%%%%%%%%%%%%%%%%%%%%%%%%%%%%%%%%%%
\begin{longtable}{|L{1.025\textwidth}|} \hline
%
{\bf IV. EMENTA } \\ \hline
%
\ementa
\end{longtable}

\newpage



%%%%%%%%%%%%%%%%%%%%%%%%%%%%%%%%%%%%%%%%%%%%%%%%%%%%%%%%%%%%%%%
\begin{longtable}{|L{1.025\textwidth}|} \hline
%
{\bf V. OBJETIVOS } \\ \hline
%
Objetivo Geral:\\

Prover ao aluno conhecimentos que possibilitem um entendimento sólido sobre banco de dados permitindo a elaboração de projetos e aplicações na área de banco de dados.\\
\\
Objetivos Específicos \\
\begin{itemize} 
\item Apresentar os principais conceitos de banco de dados;
\item Aplicar os conceitos de banco de dados em uma ferramenta de modelagem;
\item Desenvolver aplicações que manipulem informações disponíveis em um banco de dados.
\end{itemize}
\\ \hline
\end{longtable}


%%%%%%%%%%%%%%%%%%%%%%%%%%%%%%%%%%%%%%%%%%%%%%%%%%%%%%%%%%%%%%%
\begin{longtable}{|L{1.025\textwidth}|} \hline
%
{\bf VI. CONTEÚDO PROGRAMÁTICO } \\ \hline
Conteúdo Teórico seguido de Conteúdo Prático com elaboração de modelagem e implementação de projeto de banco de dados em computador: \\
\\
UNIDADE 1: Introdução [4 horas-aula]\\
Sistema de gerência de banco de dados\\
Conceitos básicos (modelo, objeto, modelagem)\\
Modelos de banco de dados (conceitual, lógico, físico)\\
Projeto de banco de dados\\
\\
UNIDADE 2: Projeto conceitual [12 horas-aula]\\
 Entidades\\
 Relacionamentos\\
 Atributos\\
 Generalização\\
 Especialização\\
\\
UNIDADE 3: Projeto lógico [8 horas-aula]\\
 Tabelas\\
 Chaves\\
 Domínios\\
Restrições de integridade\\
\\
UNIDADE 4: Normalização e Dependência funcional [4 horas-aula]\\
Formas normais\\
\\
UNIDADE 5: Linguagens, Restrições e Visões [12 horas-aula]\\
 Linguagem de definição de dados e manipulação de dados (Álgebra Relacional e SQL)\\
 Restrições de integridade\\
 Visões\\
\\
UNIDADE 6: Desenvolvimento de aplicação de banco de dados [8 horas-aula]\\
\\
UNIDADE 7: Tópicos avançados de banco de dados [8 horas-aula]\\
\\
UNIDADE 8: Apresentação de Trabalhos [8 horas-aula]

\\ \hline
\end{longtable} 





%%%%%%%%%%%%%%%%%%%%%%%%%%%%%%%%%%%%%%%%%%%%%%%%%%%%%%%%%%%%%%%
\begin{longtable}{|L{1.025\textwidth}|} \hline
%
{\bf VII. BIBLIOGRAFIA BÁSICA} \\ \hline
\begin{enumerate}
%
\item HEUSER, Carlos Alberto. Projeto de banco de dados. 6. ed. Porto Alegre: Bookman, 2009. xii, 282 p. (Livros didáticos informática UFRGS ; 4). ISBN 9788577803828.
\item SILBERSCHATZ, Abraham; KORTH, Henry F.; SUDARSHAN, S. Sistema de banco de dados. Rio de Janeiro: Elsevier, 2006. 781 p. ISBN 9788535211078.
\item ELMASRI, Ramez; NAVATHE, Sham. Sistemas de banco de dados. 6. ed. São Paulo: Pearson Addison Wesley, 2011. xviii, 788 p. ISBN 9788579360855.

\end{enumerate}
 \\ \hline
\end{longtable}


%\newpage

%%%%%%%%%%%%%%%%%%%%%%%%%%%%%%%%%%%%%%%%%%%%%%%%%%%%%%%%%%%%%%%
\begin{longtable}{|L{1.025\textwidth}|} \hline
%
{\bf VIII. BIBLIOGRAFIA COMPLEMENTAR} \\ \hline
\begin{enumerate}

\item COUGO, Paulo. Modelagem conceitual e projeto de bancos de dados. São Paulo: Elsevier, Campus, 1997. 284 p. ISBN 8535201580.

\item DATE, C. J. Introdução a sistemas de bancos de dados. Rio de Janeiro: Campus, 2004. 865 p. ISBN 8535212736.

\item GARCIA-MOLINA, Hector; ULLMAN, Jeffrey D.; WIDOM, Jennifer. Database systems: the complete book. 2nd ed. New Jersey: Prentice-Hall, 2009. xxvi, 1203 p. ISBN 9780131873254.

\item RAMAKRISHNAN, Raghu; GEHRKE, Johannes. Sistemas de gerenciamento de banco de dados. São Paulo: McGraw Hill, c2008. xxvii, 884 p. ISBN 9788577260270.

\item ULLMAN, Jeffrey D.; WIDOM, Jennifer. A first course in database systems. 3th. ed. United States of America: Pearson Prentice Hall, 2008. xxi, 565 p. ISBN 9780136006374 (enc.).

%
\end{enumerate}
 \\ \hline
\end{longtable}


\input aprovacao.tex


\end{document}
